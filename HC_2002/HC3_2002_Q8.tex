\documentclass[a4paper,12pt]{article}
%%%%%%%%%%%%%%%%%%%%%%%%%%%%%%%%%%%%%%%%%%%%%%%%%%%%%%%%%%%%%%%%%%%%%%%%%%%%%%%%%%%%%%%%%%%%%%%%%%%%%%%%%%%%%%%%%%%%%%%%%%%%%%%%%%%%%%%%%%%%%%%%%%%%%%%%%%%%%%%%%%%%%%%%%%%%%%%%%%%%%%%%%%%%%%%%%%%%%%%%%%%%%%%%%%%%%%%%%%%%%%%%%%%%%%%%%%%%%%%%%%%%%%%%%%%%
\usepackage{eurosym}
\usepackage{vmargin}
\usepackage{amsmath}
\usepackage{graphics}
\usepackage{epsfig}
\usepackage{enumerate}
\usepackage{multicol}
\usepackage{subfigure}
\usepackage{fancyhdr}
\usepackage{listings}
\usepackage{framed}
\usepackage{graphicx}
\usepackage{amsmath}
\usepackage{chngpage}
%\usepackage{bigints}

\usepackage{vmargin}
% left top textwidth textheight headheight
% headsep footheight footskip
\setmargins{2.0cm}{2.5cm}{16 cm}{22cm}{0.5cm}{0cm}{1cm}{1cm}
\renewcommand{\baselinestretch}{1.3}

\setcounter{MaxMatrixCols}{10}
\begin{document}

Higher Certificate, Paper III, 2002. Question 8
\begin{enumerate}

\item  Completely randomised design is analysed according to the linear model
yij = m+ ti + eij , where i = 1 to 4, j = 1 to 10,
yij is the number of weeds on the jth plot that received treatment i, ti is the effect
(departure from overall mean m) due to treatment i, and eij is a random (natural
variation) term, Normally distributed with mean 0 and variance σ
2 which is constant
for all observations.
The variances in the four treatments do not appear constant in the untransformed data.
We assume that the model is additive (a sum of terms) but cannot check this without
computing the values of the residuals.
\item  The transformation exp
100
 y 
 
 
also gives very unequal variances. The range
of variances in y is largest/smallest ≈ 8.7 whereas for ( ) 10 log y it is ≈ 4.0 ; thus
we should choose ( ) 10 log y . However, a ratio 4:1 among variances is still rather
high, though not unusual with small samples of data. A better transformation could
probably be found, provided it made physical sense.
\item  The herbicide totals using ( ) 10 log y are 19.857, 18.943, 21.260, 22.502, each
based on 10 observations; these add to 82.562. Herbicides SS is therefore
1 (19.8572 ... 22.5022 ) 1 (82.5622 ) 171.1465302 170.4120961 0.7344341
10 40
+ + − = − = .
Analysis of Variance of ( ) 10 log y :
Source df Sum of Squares Mean Square F value
Herbicides 3 0.73443 0.2448 19.45 (very highly sig)
Residual 36 0.45300 0.01258
Total 39 1.18743
There is strong evidence of the presence of differences among the herbicide means,
since the value 19.45 is very highly significant when referred to F3,36.
\item  A residual is the difference between an observed yij and its fitted value using
the linear model. In a completely randomised design, fitted values for each treatment
are the treatment mean for that treatment. If residuals are plotted against fitted values,
for each observation, the resulting pattern should show a set of values randomly
scattered about 0, with concentration near 0 and no outliers so that the Normality
assumption is acceptable. Variability should show no pattern depending on the size of
yij, or which treatment it received.
\end{enumerate}

\end{document}
