\documentclass[a4paper,12pt]{article}
%%%%%%%%%%%%%%%%%%%%%%%%%%%%%%%%%%%%%%%%%%%%%%%%%%%%%%%%%%%%%%%%%%%%%%%%%%%%%%%%%%%%%%%%%%%%%%%%%%%%%%%%%%%%%%%%%%%%%%%%%%%%%%%%%%%%%%%%%%%%%%%%%%%%%%%%%%%%%%%%%%%%%%%%%%%%%%%%%%%%%%%%%%%%%%%%%%%%%%%%%%%%%%%%%%%%%%%%%%%%%%%%%%%%%%%%%%%%%%%%%%%%%%%%%%%%
\usepackage{eurosym}
\usepackage{vmargin}
\usepackage{amsmath}
\usepackage{graphics}
\usepackage{epsfig}
\usepackage{enumerate}
\usepackage{multicol}
\usepackage{subfigure}
\usepackage{fancyhdr}
\usepackage{listings}
\usepackage{framed}
\usepackage{graphicx}
\usepackage{amsmath}
\usepackage{chngpage}
%\usepackage{bigints}

\usepackage{vmargin}
% left top textwidth textheight headheight
% headsep footheight footskip
\setmargins{2.0cm}{2.5cm}{16 cm}{22cm}{0.5cm}{0cm}{1cm}{1cm}
\renewcommand{\baselinestretch}{1.3}

\setcounter{MaxMatrixCols}{10}
\begin{document}

Higher Certificate, Paper III, 2002. Question 7
f (x) =λ 2xe−λ x , x ≥ 0 . Mean is 2/λ. Cdf is F (x) =1− (1+λ x)e−λ x , x ≥ 0 .
\begin{enumerate}
\item  ( ) 2 { x x} df x
e xe
dx
=λ −λ −λ −λ which is 0 for (1−λ x) = 0 , i.e. x 1
λ
= . This is the
mode. Check that 2 ( )
2
d f x
dx
is negative here, so that we do indeed have a maximum:
We have ( ) 2 x (1 ) df x
e x
dx
=λ −λ −λ , so ( ) ( ) ( )( ) 2
2 2
2 x 1 x d f x
e x e
dx
=λ −λ −λ +λ −λ −λ −λ
and at x 1
λ
= the second term is 0 and the first is < 0, so
2
2 d f 0
dx
< .
\item  When λ =1, f (x) = xe−x and the mode is at x = 1. f (1) = 0.3679 .
[Note. The diagram is not drawn to scale.]
\item  In general, the mode is at 1/λ and the mean is 2/λ. The distribution is strongly
skew to the right. It is not very likely that this distribution would work well in a
supermarket that channels customers with only few ("less than 10", say) purchases
through special cash outlets. But if there are none of these and everyone must go
through the same channels, then there will be some very small values of x to combine
with the higher ones, and with a few very high service times. This distribution could
work if applied to pooled data from all the outlets together.
Continued on next page
f(x)
x
0 1 2
\item  ( ) 2 2 ( )
1 2
1
, , ..., i i
n
x n x
n i i
i
L x x x xe e x λ λ λ λ − −
=
= Π = Σ Π
and ln ( ) 2 ln ln ( ) i i l = L = n λ −λΣx + Πx .
So we have 2
i
dl n x
dλ λ
= −Σ ; dl 0
dλ
= gives 2
ˆ i
n x
λ
=Σ or ˆ 2
x
λ = .
2
2 2
d l 2n 0
dλ λ
= − < , so this is a maximum.
Setting x equal to E[X ] 2
λ
= , we find that the moments estimator of λ is 2
x
. In this
case, both estimators are the same.
\item  F (x) =1− (1+ x)e−x , and n = 200 service times are sampled.
F (1.0) =1− 2e−1 = 0.2642 ; and F (1.5) =1− 2.5e−1.5 = 0.4422 .
So P(1.0 < X ≤1.5) = 0.4422 − 0.2642 = 0.1780 , and the corresponding expected
frequency is 35.59.
Using the fact that 200 observations were made, the final value for x > 5.0 is 8.09.
The usual chi-squared test here will have 10 degrees of freedom; no parameters had
to be estimated, and all 11 intervals can be used since only one has an expected value
that is just below 5. The test statistic is
( ) ( ) ( ) ( ) ( ) 2 2 2 2 2 21 18.04 30 34.81 36 35.59 29 30.36 27 23.74
18.04 34.81 35.59 30.36 23.74
− − − − −
+ + + +
( ) ( ) ( ) ( ) ( ) ( ) 2 2 2 2 2 2 19 17.63 17 12.65 4 8.86 8 6.10 2 4.13 7 8.09
17.63 12.65 8.86 6.10 4.13 8.09
− − − − − −
+ + + + + +
= 7.769.
This value is nowhere near to being significant when compared with 2
10 χ , so there is
no evidence to reject the given null hypothesis.
\end{enumerate}

\end{document}
