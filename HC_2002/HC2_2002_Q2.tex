\documentclass[a4paper,12pt]{article}
%%%%%%%%%%%%%%%%%%%%%%%%%%%%%%%%%%%%%%%%%%%%%%%%%%%%%%%%%%%%%%%%%%%%%%%%%%%%%%%%%%%%%%%%%%%%%%%%%%%%%%%%%%%%%%%%%%%%%%%%%%%%%%%%%%%%%%%%%%%%%%%%%%%%%%%%%%%%%%%%%%%%%%%%%%%%%%%%%%%%%%%%%%%%%%%%%%%%%%%%%%%%%%%%%%%%%%%%%%%%%%%%%%%%%%%%%%%%%%%%%%%%%%%%%%%%
\usepackage{eurosym}
\usepackage{vmargin}
\usepackage{amsmath}
\usepackage{graphics}
\usepackage{epsfig}
\usepackage{enumerate}
\usepackage{multicol}
\usepackage{subfigure}
\usepackage{fancyhdr}
\usepackage{listings}
\usepackage{framed}
\usepackage{graphicx}
\usepackage{amsmath}
\usepackage{chngpage}
%\usepackage{bigints}

\usepackage{vmargin}
% left top textwidth textheight headheight
% headsep footheight footskip
\setmargins{2.0cm}{2.5cm}{16 cm}{22cm}{0.5cm}{0cm}{1cm}{1cm}
\renewcommand{\baselinestretch}{1.3}

\setcounter{MaxMatrixCols}{10}
\begin{document}

% Higher Certificate, Paper II, 2002.  Question 2 
%%%%%%%%%%%%%%%%%%%%%%%%%%%%%%%%%%%%%%%%%%%%%%%%%%%%%%%%%%%%%%%%%%%%%%%%%%%%%%%%%%%%%%%%%%%%%%%%%%%%%%%%%%%%%%%%%%%%%%%%%%%%%%%%%%%%%%%%%%% 
\begin{table}[ht!]
 
\centering
 
\begin{tabular}{|p{15cm}|}
 
\hline  


2. (a) Explain the meaning of the following statistical terms used in hypothesis tests. 
 
(i) Type I error. 
(2) 
 
 (ii) Type II error. 
(2) 
 
 (iii) Level of significance. 
(2) 
 
 (iv) Power. 
\\ \hline
  
\end{tabular}

\end{table}


%%%%%%%%%%%%%%%%%%%%%%%%%%%%%%%%%%%%%%%%%%%%%%%%%%%%%%%%%%%%%%%%%%%%%%%%%%%%%%%%%%%%%%%%%%%%%%%%%%%%%%%%%%%%%%%%%%%%%%%%%%%%%%%%%%%%%%%%%%%
\begin{enumerate} 
\item In a simple significance test, there is a null hypothesis (NH or H0) which is the basis for calculations and an alternative hypothesis (AH or H1) which is accepted when the NH is rejected.  For example, NH may be that data come from () 2 1 N, µ σ
 and AH that they come from () 2 2 N, µ σ , with 21 µµ > . 
 
\begin{enumerate}[(i)] 

\item  Type I error = P(reject H0 when H0 is true). 
\item Type II error = P(not reject H0 when H1 is true). 
\item Level of significance = P(Type I error) = α . 
\item  Power = 1 − β , where β = P(Type II error). 
\end{enumerate} 

\newpage
\begin{table}[ht!]
 
\centering
 
\begin{tabular}{|p{15cm}|}
 
\hline  
 
 
(b) According to medical experts, high sodium intake may be related to ulcers and stomach cancer.  
As a consequence, experts state that the human requirement for salt is only 220 milligrams per day.  
It is believed that this level of salt is exceeded by the salt in an average 20g packet of a popular snack food.  A random sample of 20g packets of the snack food was selected, and each packet was analysed in the laboratory with the following results. 
 
\[ \{220, 218, 222, 215, 227, 229, 224, 217, 226. \}\]
 
 
(i) Perform a suitable statistical test to investigate whether, on average, the salt content of a single serving of 
this snack food exceeds the recommended daily intake. (6) 
 

\\ \hline
  
\end{tabular}

\end{table} 
\item  9 n= .   222.0 x = .   2 23.50 s = .   0H : 220 µ = ;   1H : 220 µ > .  Test statistic is 222.0 220.0 2.0 1.238 1.61623.50/9 − ==, refer to t8. 
 A one-tail test is required, so the 5\% point of t8 is 1.860.  The result is not significant.  There is no evidence that the recommended intake is exceeded, on average. 
 
%%%%%%%%%%%%%%%%%%%%%%%%%%%%%%%%%%%%%%5
\newpage
\begin{table}[ht!]
 
\centering
 
\begin{tabular}{|p{15cm}|}
 
\hline  

(ii) Suppose that the sample had consisted of 25 packets and the values of the sample mean and standard deviation had been found to be the same as the actual sample of 9 packets.  Perform a second analysis, using this sample of 25 packets, and compare and comment on the conclusions of the two analyses. (6) 

\\ \hline
  
\end{tabular}

\end{table}





 
\item  If n = 25, with the same values of x and 2 s as in (i), the test statistic is 2.0 2.063 23.50/25 = which is referred to t24.  This is significant as a one-tail test (5\% point 1.711).  Therefore we may reject the null hypothesis and accept that the recommended intake is exceeded.  A larger sample size has given a more powerful test. \end{enumerate}
\end{document}
