\documentclass[a4paper,12pt]{article}
%%%%%%%%%%%%%%%%%%%%%%%%%%%%%%%%%%%%%%%%%%%%%%%%%%%%%%%%%%%%%%%%%%%%%%%%%%%%%%%%%%%%%%%%%%%%%%%%%%%%%%%%%%%%%%%%%%%%%%%%%%%%%%%%%%%%%%%%%%%%%%%%%%%%%%%%%%%%%%%%%%%%%%%%%%%%%%%%%%%%%%%%%%%%%%%%%%%%%%%%%%%%%%%%%%%%%%%%%%%%%%%%%%%%%%%%%%%%%%%%%%%%%%%%%%%%
\usepackage{eurosym}
\usepackage{vmargin}
\usepackage{amsmath}
\usepackage{graphics}
\usepackage{epsfig}
\usepackage{enumerate}
\usepackage{multicol}
\usepackage{subfigure}
\usepackage{fancyhdr}
\usepackage{listings}
\usepackage{framed}
\usepackage{graphicx}
\usepackage{amsmath}
\usepackage{chngpage}
%\usepackage{bigints}

\usepackage{vmargin}
% left top textwidth textheight headheight
% headsep footheight footskip
\setmargins{2.0cm}{2.5cm}{16 cm}{22cm}{0.5cm}{0cm}{1cm}{1cm}
\renewcommand{\baselinestretch}{1.3}

\setcounter{MaxMatrixCols}{10}
\begin{document}

%%%%%%%%%%%%%%%%%%%%%%%%%%%%%%%%%%%%%%%%%%%%%%%%%%%%%%%%%%%%%% 
\begin{framed}
2. (a) As part of an investigation of student expenditure patterns at a UK university, a random sample of students was taken.  
It was found that a high proportion of them had mobile telephones.  The following summary data, in £ per week, refer to the expenditure on mobile telephone calls of all students.  (Note that the sample includes students who did not possess mobile telephones.) 
 
 Sample size Sum of expenditures 
Sum of squares of expenditures Males 87 1098.60 32234.71 Females 63   887.75 25810.04 
 
 
(i) Test whether there is any difference between the mean expenditures of males and females, stating your null and alternative hypotheses clearly. (5) 
 
 
(ii) On what assumptions, if any, is your test based?  Comment on whether you feel the test result to be reliable, and on any further information which should be obtained in order to throw greater light on the impact of mobile telephones on student expenditure. (5) 
 
 
 
 
 
 
 
 
Part (b) of question 2 is on the next page 
4 
%%%%%%%%%%%%%%%%%%%%%%%%%%%%%%%%%%%%%%%%%%%%%%%%%%%%%%%%%%%%%% 
This is part (b) of question 2 
 
 
(b) The table below shows, for each of 15 companies randomly selected from the largest 1000 companies in a particular country, 
the dividends announced in the years 1999 (x1) and 2000 (x2), together with the difference x (x = x2 − x1).  All figures for 1999 
and 2000 are expressed as percentages of the company's dividend in 1995. 
 
Company x1   (1999) x2    (2000) x 
A 104.8 106.1 1.3 B 106.2 107.7 1.5 
C 108.7 111.3 2.6 D 108.5 108.9 0.4 
E 105.0 106.3 1.3 F 105.1 106.6 1.5 
G 122.1 125.2 3.1 H 112.9 117.4 4.5 
I 111.3 113.8 2.5 J 106.7 108.8 2.1 
K 107.8 110.4 2.6 L 110.8 113.3 2.5 
M 110.7 113.5 2.8 N 105.4 106.9 1.5 
O 110.4 112.8 2.4 

 Note:   Σx1 = 1636.4,    Σx2 = 1669.0,    Σx = 32.6,             Σx12 = 178797.12,    Σx22 = 186071.08,    Σx2 = 84.18. 
 
 
(i) Carry out a test based on the differences between the 1999 and 2000 figures in which you take account of the fact that each pair of observations refers to one company in two successive years. (5) 
 
(ii) A commentator reported on the data by comparing the difference between the sample means for 1999 and 2000 with the standard error calculated as 
 
 ()() 22 1 1 2 2 28 x x x x − + − ∑∑ , 
 giving a value for t of 1.241.  What conclusion would the commentator reach? (2) 
 
(iii) Explain and discuss any difference between the commentator's conclusion and that from your test in part (i). (3) 

\end{framed}
%%%%%%%%%%%%%%%%%%%%%%%%%%%%%%%%%%%%%%%%%%%%%%%%%%%%%%%%%%%%%% 
Higher Certificate, Paper III, 2002. Question 2
\begin{enumerate} 
\item  On the null hypothesis that males and females have equal mean
expenditures (μ M = μ F ) , against the alternative hypothesis that they do not,
and with large enough sample sizes to assume that the difference ( ) M F X − X
between the observed means is approximately Normally distributed, an
appropriate test uses ( ) M F
M F
Z X X
SE X X
= −
−
. The estimated variances of the two
means are
2M
M
s
n
and
2F
F
s
n
, and so ( ) 2 2
M F
M F
M F
SE X X s s
n n
− = + .
2
2 1 32234.71 1098.60 18362.044 213.512
86 87 86 M s
 
=  −  = =
 
.
2
2 1 25810.04 887.75 13300.515 214.524
62 63 62 F s
 
=  −  = =
 
.
1098.60 12.6276
87 M x= = ; 887.75 14.0913
63 F x= = ; 1.464 M F x − x = − .
( ) 213.512 214.524 2.4542 3.4051 2.421
87 63 M F SE X − X = + = + = .
Hence the value of Z is 1.464
2.421
− = −0.605, which (compare with N(0,1)) is not
significant.
There is no evidence of a difference between M μ and F μ .
\item  The assumptions stated in (i) are all that are theoretically necessary.
The underlying populations do not need to be Normally distributed nor to have
equal variances. The samples are assumed random. The main practical doubt
about validity is the existence of zeros in the data. It would be best to base the
test on the non-zero items, and additionally compare the proportions of zeros
in the two samples.
Continued on next page
(b) \item  The null hypothesis will be 0 X μ = , and we assume X follows a
Normal distribution. n =15, 32.6 2.173
15
x= = .
2
2 1 84.18 32.6 0.9521
14 15
s
 
=  −  =
 
.
Test statistic is 0
0.9521
15
x − = 8.63, which we refer to t14 − very highly
significant.
This is very strong evidence against the null hypothesis, which we shall reject.
\item  The commentator's result is not significant and the null hypothesis
1 2 "μ = μ " cannot be rejected ( i
μ is the mean in year i).
\item  There is substantial systematic variation from company to company: if
x1 is below average, so is x2 in most cases. If this between-company variation
is removed, by using the differences x = x1 − x2, the values x should (on the
null hypothesis) represent only random variation and give a valid basis for
comparison. Clearly in this case the between-company variation was very
large, and removing it gave a much more precise comparison of the two years.
\end{enumerate}

\end{document}
