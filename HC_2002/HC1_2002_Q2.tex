\documentclass[a4paper,12pt]{article}
%%%%%%%%%%%%%%%%%%%%%%%%%%%%%%%%%%%%%%%%%%%%%%%%%%%%%%%%%%%%%%%%%%%%%%%%%%%%%%%%%%%%%%%%%%%%%%%%%%%%%%%%%%%%%%%%%%%%%%%%%%%%%%%%%%%%%%%%%%%%%%%%%%%%%%%%%%%%%%%%%%%%%%%%%%%%%%%%%%%%%%%%%%%%%%%%%%%%%%%%%%%%%%%%%%%%%%%%%%%%%%%%%%%%%%%%%%%%%%%%%%%%%%%%%%%%
\usepackage{eurosym}
\usepackage{vmargin}
\usepackage{amsmath}
\usepackage{graphics}
\usepackage{epsfig}
\usepackage{enumerate}
\usepackage{multicol}
\usepackage{subfigure}
\usepackage{fancyhdr}
\usepackage{listings}
\usepackage{framed}
\usepackage{graphicx}
\usepackage{amsmath}
\usepackage{chngpage}
%\usepackage{bigints}

\usepackage{vmargin}
% left top textwidth textheight headheight
% headsep footheight footskip
\setmargins{2.0cm}{2.5cm}{16 cm}{22cm}{0.5cm}{0cm}{1cm}{1cm}
\renewcommand{\baselinestretch}{1.3}

\setcounter{MaxMatrixCols}{10}
\begin{document}
Higher Certificate, Paper I, 2002. Question 2
H ∼ N(160,16)
\begin{enumerate}[(a)]
\item  P(156 < H ≤ 164) = 156 160 164 160
4 4
P − < Z ≤ − 
 
[where Z ∼ N(0,1) ]
= P(−1< Z ≤1) = Φ(1) −Φ(−1) = Φ(1) −{1−Φ(1)} by symmetry, i.e. 2(Φ(1)) −1
= 0.6826.
P(H >168) = P(Z > 2) =1−Φ(2) = 0.0228 .
\item  (a) The relevant portion of the Normal distribution of H is that beginning
at 168, which corresponds to the value Z = 2. The median value m within this
portion has 1 ( )
2 0.0228 probability above it, i.e. 0.0114, so Φ(m) =
1 − 0.0114 = 0.9886, corresponding to Z = 2.277. The corresponding value of
H is μ +σ Z which is 160 + (4 × 2.277) = 169.1 cm.
\item H = 170 corresponds to 10 2.5 ; we have (2.5) 0.9938
4
Z= = Φ = , so
1−Φ(2.5) = 0.0062 . Conditional on H > 168, the probability is ( )
( )
170
168
P H
P H
>
>
= 0.0062
0.0228
= 0.272.
\item Mean height of 25 members
1.3522 N 169.5,
25
 
∼  
 
.
P(mean > 170) = ( ) ( ) 170 169.5 1 0.5 5 1 1.849
1.352 / 5 1.352
P Z
 −   ×   >  = −Φ  = −Φ    
= 1− 0.9677 = 0.0323.
\end{enumerate}
\end{document}