\documentclass[a4paper,12pt]{article}
%%%%%%%%%%%%%%%%%%%%%%%%%%%%%%%%%%%%%%%%%%%%%%%%%%%%%%%%%%%%%%%%%%%%%%%%%%%%%%%%%%%%%%%%%%%%%%%%%%%%%%%%%%%%%%%%%%%%%%%%%%%%%%%%%%%%%%%%%%%%%%%%%%%%%%%%%%%%%%%%%%%%%%%%%%%%%%%%%%%%%%%%%%%%%%%%%%%%%%%%%%%%%%%%%%%%%%%%%%%%%%%%%%%%%%%%%%%%%%%%%%%%%%%%%%%%
\usepackage{eurosym}
\usepackage{vmargin}
\usepackage{amsmath}
\usepackage{graphics}
\usepackage{epsfig}
\usepackage{enumerate}
\usepackage{multicol}
\usepackage{subfigure}
\usepackage{fancyhdr}
\usepackage{listings}
\usepackage{framed}
\usepackage{graphicx}
\usepackage{amsmath}
\usepackage{chngpage}
%\usepackage{bigints}

\usepackage{vmargin}
% left top textwidth textheight headheight
% headsep footheight footskip
\setmargins{2.0cm}{2.5cm}{16 cm}{22cm}{0.5cm}{0cm}{1cm}{1cm}
\renewcommand{\baselinestretch}{1.3}

\setcounter{MaxMatrixCols}{10}
\begin{document}
Higher Certificate, Paper I, 2002. Question 2
\begin{framed}

2. Among the inhabitants of Altamania, height, H say, is distributed Normally with
mean 160 cm and standard deviation 4 cm, i.e. N(160, 16).
(i) Find the proportion of the population whose heights are within one
standard deviation of the mean. Find also the proportion of the population
who are more than 168 cm tall.


\end{framed}
%%%%%%%%%%%%%%%%%%%%%%%%%%%%%%%%%%%

$H \sim N(160,16)$


\begin{enumerate}[(a)]
\item  $P(156 < H \leq 164)$ = 156 160 164 160
4 4
P − < Z ≤ − 
 
[where $Z \sim N(0,1^2)$ ]
\begin{eqnarray*} 
P(−1< Z \leq 1) &=& \Phi(1) -\Phi(-1) \\&=& \Phi(1) -[1-\Phi(1)] \quad \mbox{by symmetry}\\
&=& 2(\Phi(1)) -1\\
&=& 0.6826.\\
\end{eqnarray*}
\[P(H >168) = P(Z > 2) =1-\Phi(2) = 0.0228\]
\item  The relevant portion of the Normal distribution of H is that beginning
at 168, which corresponds to the value $Z = 2$.
\begin{itemize}
    \item The median value m within this
portion has 1 ( )
2 0.0228 probability above it, i.e. 0.0114, so \[\Phi(m) =
1 − 0.0114 = 0.9886,\] corresponding to Z = 2.277. \item The corresponding value of
H is $μ +\sigma Z$ which is $160 + (4 \times 2.277) = 169.1 \mbox{cm}$.
\end{itemize}

\newpage

\begin{framed}

(ii) The Altamanian Police Force (APF) is restricted to persons who are more
than 168 cm tall, and may be assumed to consist of a random sample of
Altamanians satisfying this condition.
Find
(a) the median height of members of the APF,
(b) the proportion of members of the APF who are more than 170 cm
tall.

(iii) Assume that the mean and standard deviation of height among members of
the APF are 169.5 cm and 1.352 cm respectively. Find an approximate
value for the probability that the mean height of a random sample of 25
members of the APF is more than 170 cm.
\end{framed}

\item H = 170 corresponds to 10 2.5 ; we have (2.5) 0.9938
4
Z= = Φ = , so
\[1−\Phi(2.5) = 0.0062 . \]Conditional on H > 168, the probability is ( )
( )
170
168
P H
P H
>
>
= 0.0062/0.0228
= 0.272.
\item Mean height of 25 members
$  \sim N (169.5,\frac{1.3522}{25})


\begin{eqnarray*}
P(mean > 170) &=& ( ) ( ) 170 169.5 1 0.5 5 1 
1.352 / 5 1.352
P Z
 −   ×   >  = −Φ
&=& \Phi() - \Phi(1.849)\\
  
&=& 1- 0.9677 \\
&=& 0.0323.\\
\end{enumerate}
\end{document}
