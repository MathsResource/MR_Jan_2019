\documentclass[a4paper,12pt]{article}
%%%%%%%%%%%%%%%%%%%%%%%%%%%%%%%%%%%%%%%%%%%%%%%%%%%%%%%%%%%%%%%%%%%%%%%%%%%%%%%%%%%%%%%%%%%%%%%%%%%%%%%%%%%%%%%%%%%%%%%%%%%%%%%%%%%%%%%%%%%%%%%%%%%%%%%%%%%%%%%%%%%%%%%%%%%%%%%%%%%%%%%%%%%%%%%%%%%%%%%%%%%%%%%%%%%%%%%%%%%%%%%%%%%%%%%%%%%%%%%%%%%%%%%%%%%%
\usepackage{eurosym}
\usepackage{vmargin}
\usepackage{amsmath}
\usepackage{graphics}
\usepackage{epsfig}
\usepackage{enumerate}
\usepackage{multicol}
\usepackage{subfigure}
\usepackage{fancyhdr}
\usepackage{listings}
\usepackage{framed}
\usepackage{graphicx}
\usepackage{amsmath}
\usepackage{chngpage}
%\usepackage{bigints}

\usepackage{vmargin}
% left top textwidth textheight headheight
% headsep footheight footskip
\setmargins{2.0cm}{2.5cm}{16 cm}{22cm}{0.5cm}{0cm}{1cm}{1cm}
\renewcommand{\baselinestretch}{1.3}

\setcounter{MaxMatrixCols}{10}
\begin{document}
Higher Certificate, Paper I, 2002. Question 4
(i) ( )
4 4
!
e r P X r
r
−
= = , r = 0, 1, 2, … .
( )
( )
( ) ( )
( ) ( )
( ) ( )
( ) ( )
( ) ( )
( ) ( )
( ) ( )
( ) ( )
4
4
0 0.0183
1 4 0.0733
2 2 1 0.1465
3 4 2 0.1954
3
4 3 0.1954
5 4 4 0.1563
5
6 2 5 0.1042
3
7 4 6 0.0595
7
8 1 7 0.0298
2
9 4 8 0.0132
9
P e
P e
P P
P P
P P
P P
P P
P P
P P
P P
−
−
= =
= =
= =
= =
= =
= =
= =
= =
= =
= =
0
0.05
0.1
0.15
0.2
0.25
0 1 2 3 4 5 6 7 8 9
No. of accidents
Probability
and so on (probabilities beyond r = 9 have not been shown on the diagram).
(a) P(at most 1 fatal accident) = P(0) + P(1) = 0.0916.
(b) In half-year, mean = 2 giving P(0) = e−2 = 0.1353 .
(c) In 1
2 1 years, mean = 6 giving P(0) = e−6 = 0.00248 .
(ii) Given that X and Y are independent Poissons, V = X +Y ~ Poisson(16).
E[V ] = Var (V ) =16 .
( ) ( )
( )
4 5 12 15
16 20
5, 15 4 12 20! 5 | 20 . .
20 5! 15! 16
P X Y e e P X V
P V e
− −
−
= =
= = = =
=
=
5 15 20 15 15
20 20 20
4 12 . 20! 4 3 . 20 19 18 17 16 3 .19
16 5!15! 16 5 4 3 2 4
= × × × × = ×17× 48
× × ×
= 0.2023.
(iii) W is Poisson with mean 16 × 4 = 64. Use Normal approximation N(64, 64).
( 70) 70.5 64
8
P W > = P Z > − 
 
, where Z ∼ N(0,1) and a continuity correction is
used. This is P(Z > 0.8125), which is 1−Φ(0.8125) =1− 0.7917 = 0.2083.
Higher Certificate, Paper I, 2002. Question 5
Let A1, …, Ak be a set of mutually exclusive and exhaustive events, and let B be any
other event.
Then ( ) ( ) ( )
( )
( ) ( )
( ) ( )
1
| |
|
|
i i i i
i k
j j
j
P B A P A P B A P A
P A B
P B P B A P A
=
= =
Σ
.
It is useful in inference about an Ai which cannot be observed directly but is related to
an observable event B.
P(011100|A) = P(5 right, 1 wrong) = p(1− p)5 .
P(011100|R) = P(3 right, 3 wrong) = ( )p3 1− p 3 .
P(011100|S) = P(2 right, 4 wrong) = ( )p4 1− p 2 .
This assumes all errors are independent.
Given that P(A) = 0.1, P(R) = 0.4, P(S) = 0.5, we have P(A|011100) = ( )
( )
5 0.1 p 1 p
P B
× −
,
where ( ) ( ) ( ) ( ) P B = 0.1p 1− p 5 + 0.4 p3 1− p 3 + 0.5 p4 1− p 2 .
So P(A|011100) = ( )
( ( ) ( ) ( ) )
5
5 3 3 4 2
1
1 4 1 5 1
p p
p p p p p p
−
− + − + −
= ( )
( ) ( )
3
3 2 3
1
1 4 1 5
p
p p p p
−
− + − +
.
Similarly, P(R|011100) = ( )
( ( ) ( ) ( ) )
3 3
5 3 3 4 2
4 1
1 4 1 5 1
p p
p p p p p p
−
− + − + −
= ( )
( ) ( )
2
3 2 3
4 1
1 4 1 5
p p
p p p p
−
− + − +
.
Also, P(S|011100) = ( )
( ( ) ( ) ( ) )
4 2
5 3 3 4 2
5 1
1 4 1 5 1
p p
p p p p p p
−
− + − + −
=
( ) ( )
3
3 2 3
5
1 4 1 5
p
− p + p − p + p
.
Continued on next page
As p → 0, P(A|011100) → 1, while the others do not.
In general, P(A|011100) > P(R|011100) if (1− p)3 > 4 p2 (1− p) , or ( )1− p 2 > 4 p2 ,
which requires 1− p > 2 p , i.e. 1 > 3p or 1
3 p < .
Likewise P(A|011100) > P(S|011100) if ( )1− p 3 > 5p3 , or (1− p) > p 3 5 , which
requires 1 > p(1+ 51/3 ) = 2.71p , or 1
2.71
p < .
When p ≤ 0.1, both these conditions are satisfied so choose A.
[The probabilities are in the ratio A:R:S ≡ 0.729 : 0.036 : 0.005.]
Higher Certificate, Paper I, 2002. Question 6
f (x) =λ e−λ x x > 0, λ > 0
(i) ( ) ( )
0 0
tX x tx t x
X M t E e λ e λ dx λ e λ dx =   = ∞ − + = ∞ − −   ∫ ∫
( ) 1
0
1 1
e t x t
t t
λ λ
λ λ λ
 − − ∞ − = − = =  −   −  −      
[| t |<λ ] .
( )
2
2 1 ... x
M t t t
λ λ
= + + + .
E[Xk] = coefficient of
!
xk
k
in the expansion.
Hence E[X ] 1
λ
= ; also, 2
2
E X 2
λ
  = , so ( )
2
2 2
Var X 2 1 1
λ λ λ
= −   =  
 
.
(ii) ( ) ( ) 1
1 1
| , ..., i exp
n n
x n
n i
i i
L λ x x λ e−λ λ λ x
= =
  = = − 
 
Π Σ .
ln ln (ln ) i L = n λ −λΣx = n λ −λ x
(ln )
0
d L n nx
dλ λ
= − = for ˆ 1
x
λ = .
2 ( )
2 2
d ln L n
dλ λ
= − which confirms maximum of L.
(iii) The asymptotic variance of ˆλ [Cramér-Rao lower bound for variance] is
( )
2
2
2
1
d ln L n
E
d
λ
λ
=
 
− 
 
.
We have
2 ˆ approx N ,
n
λ λ λ
 
∼  
 
, i.e. the estimate of SE(λˆ) is
ˆ
n
λ , so that
ˆ ˆ ˆ ˆ 1.96 1.96
n n
λ − λ <λ <λ + λ is an approximate 95% confidence interval for λ when
n is large.
This is 1 1.96 1 1.96
x x n x x n
− <λ < + .
Higher Certificate, Paper I, 2002. Question 7
Y
0 1 2 3
0 k 6k 9k 4k
X 1 8k 18k 12k 2k
2 k 6k 9k 4k
(i) The sum of all the entries in the table is 80k. Hence 1
80
k = .
(ii) Row and column totals give the marginal distributions of X and Y:
X 0 1 2
P(X) 1/4 1/2 1/4
Y 0 1 2 3
P(Y) 1/8 3/8 3/8 1/8
(iii) For P( X = x |Y = 2) , use ( )
( )
and 2
2
P X x Y
P Y
= =
=
:
X 0 1 2
Probability 9 /80
3/8
= 9/30 = 0.3 12 / 80
3/8
= 0.4 9 /80
3/8
= 0.3
(iv) E[X] = 1, E[Y] = 1.5, by symmetry. The distribution of XY is:
XY 0 1 2 3 4 6
P(XY) 29/80 18/80 18/80 2/80 9/80 4/80
[ ] 120 1.5
80
E XY = = .
So E[XY ] = E[X ]E[Y ]. Therefore Cov( X,Y ) = 0 , so correlation = 0.
(iv) Zero correlation is not sufficient. Every individual P( X = x, Y = y) in the
table must be the product of its two marginal probabilities.
Consider x = y = 0 . We have (0,0) 1
80
P = k = . But (0) (0) 1 1 1
4 8 32 X Y P P = × = . So
there is not independence.
Higher Certificate, Paper I, 2002. Question 8
45
50
55
60
65
70
16 18 20 22 24 26
Age (years)
y
There appears to be an outlier (25, 69), without which correlation (and the regression
gradient) would be near 0. (Rank correlation may be more appropriate.)
Σx =171 Σy = 477 n = 9 x =19 y = 53
1712 3309 60
9 XX S = − = .
4772 25633 352
9 YY S = − = . 9183 171 477 120
9 XY S = − × = .
y − y = bˆ (x − x ) , and ˆ 120 2
60
XY
XX
b S
S
= = = .
So we have y −53 = 2(x −19) or y = 2x +15
(a) For x = 18, yˆ = 51. (b) For x = 26, yˆ = 67 .
Including the data for I, correlation
2
XY
xy
XX YY
r S
S SQ
= = 0.826. (Significant at 5% level.)
[Hence the null hypothesis "b = 0" is rejected.]
However, without I, Σx =146 , x =18.25; Σy = 408 , y = 51.00 .
Σx2 = 3309 − 252 = 2684 ; Σy2 = 25633− 692 = 20872 ;
Σxy = 9183− (25×69) = 7458; ( )
2
ˆ 7458 146 408 / 8 12 0.615
2684 146 /8 19.5
b
− ×
= = =
−
.
19.5 XX S = , 12 XY S = ,
4082 20872 64
8 YY S = − = .
12
64 19.5 XY ∴r =
×
= 0.340 which does not approach significance. The line still has
positive gradient, but not significantly different from 0.