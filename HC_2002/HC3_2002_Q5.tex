\documentclass[a4paper,12pt]{article}
%%%%%%%%%%%%%%%%%%%%%%%%%%%%%%%%%%%%%%%%%%%%%%%%%%%%%%%%%%%%%%%%%%%%%%%%%%%%%%%%%%%%%%%%%%%%%%%%%%%%%%%%%%%%%%%%%%%%%%%%%%%%%%%%%%%%%%%%%%%%%%%%%%%%%%%%%%%%%%%%%%%%%%%%%%%%%%%%%%%%%%%%%%%%%%%%%%%%%%%%%%%%%%%%%%%%%%%%%%%%%%%%%%%%%%%%%%%%%%%%%%%%%%%%%%%%
\usepackage{eurosym}
\usepackage{vmargin}
\usepackage{amsmath}
\usepackage{graphics}
\usepackage{epsfig}
\usepackage{enumerate}
\usepackage{multicol}
\usepackage{subfigure}
\usepackage{fancyhdr}
\usepackage{listings}
\usepackage{framed}
\usepackage{graphicx}
\usepackage{amsmath}
\usepackage{chngpage}
%\usepackage{bigints}

\usepackage{vmargin}
% left top textwidth textheight headheight
% headsep footheight footskip
\setmargins{2.0cm}{2.5cm}{16 cm}{22cm}{0.5cm}{0cm}{1cm}{1cm}
\renewcommand{\baselinestretch}{1.3}

\setcounter{MaxMatrixCols}{10}
\begin{document}

Higher Certificate, Paper III, 2002. Question 5
\begin{enumerate} 
\item  Boxplots require median, quartiles, minimum and maximum values. As each
set of data has been arranged in increasing order of size, it is easy to check whether
any outliers have been included when making these calculations.
For Banks, the listing shows no outliers. Since N = 36, the median M is between the
18th and 19th observations, both of which are 3.0. Q1 is between the 9th and 10th,
which are 2.4 and 2.5. The program calculates 2.425; 2.45 is also acceptable. Q3 is
found in a similar way.
For E&EEq there are three obvious outliers. N = 46, so M is between the 23rd and
24th observations, at 3.25. Clearly the full set of data have been used in the program's
calculation (not the N = 43 when outliers are omitted). Q1, Q3 are 0.3, 5.325, or near
to these depending on the method of calculation.
The * points are outliers, and the upper whisker does not include them. [Since we use
median and quartiles, not mean and standard deviation, the difference between values
with and without outliers would not be great.]
For SS, there may be one outlier at 11.9 (although there is also considerable skewness
at the upper end). M = 2.2, Q1 = 0.325, Q3 = 4.775.
[Note. The plots drawn above might not appear exactly correct, due to screen and/or printer
resolution.]
Continued on next page
0 1 2 3 4 5
x
6
x
0 3 6 9 12 15 18
* **
x
0 3 6 9 12
*
\item  The column of numbers 001122… is the "stem", which is the number before
the decimal point. To the right of this, listed in increasing order, are the "leaves",
which are the decimal parts; those with decimal parts 0 – 4 are listed on one row, and
5 – 9 on the next row below; for example,
0 0
0 789
shows that the first four data in
'Banks' were 0.0 0.7 0.8 0.9.
Frequencies are cumulated, row by row, from each end, so that they meet in the
middle, where the bracketed number, e.g. (7) for Banks, shows the actual frequency in
the interval containing the median.
The dotplots show where each observation is located on the scale of measurement,
with one dot for each item. (On the scales used here, spacing forces some adjacent
numbers to come together.)
\item  Banks: There are 4 very low figures, 1% or less; apart from these, there is a
reasonably symmetrical pattern with 3% as its approximate centre. The spread of this
set of data is not great; there are no upper outliers and the range of the 32 items
excluding the 4 low ones is from 2.1 to 5.1.
E&EEq: The DESCRIBE program results are rather distorted by the three very large
observations, which are clearly outliers, and the general skewness of the whole
pattern. There is a substantial number of zeros. An "exponential decay" pattern (an
exponential distribution) might explain all but the last three.
SS: This is rather similar to E&EEq, having several zeros and an exponential pattern.
However, this pattern gradually tails off and does not have outliers far above their
neighbours – it is probably wise to treat 11.9 as an outlier, although it would fit an
exponential distribution.
General: Stem and leaf diagrams seem to give the interpretation more easily than
dotplots (at least with the program used) but the numerical descriptive summaries
seem less satisfactory than either of these graphical methods. The question of a subgroup
of zeros arises in two of the data sets.
\end{enumerate}

\end{document}
