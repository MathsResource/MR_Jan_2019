\documentclass[a4paper,12pt]{article}
%%%%%%%%%%%%%%%%%%%%%%%%%%%%%%%%%%%%%%%%%%%%%%%%%%%%%%%%%%%%%%%%%%%%%%%%%%%%%%%%%%%%%%%%%%%%%%%%%%%%%%%%%%%%%%%%%%%%%%%%%%%%%%%%%%%%%%%%%%%%%%%%%%%%%%%%%%%%%%%%%%%%%%%%%%%%%%%%%%%%%%%%%%%%%%%%%%%%%%%%%%%%%%%%%%%%%%%%%%%%%%%%%%%%%%%%%%%%%%%%%%%%%%%%%%%%
\usepackage{eurosym}
\usepackage{vmargin}
\usepackage{amsmath}
\usepackage{graphics}
\usepackage{epsfig}
\usepackage{enumerate}
\usepackage{multicol}
\usepackage{subfigure}
\usepackage{fancyhdr}
\usepackage{listings}
\usepackage{framed}
\usepackage{graphicx}
\usepackage{amsmath}
\usepackage{chngpage}
%\usepackage{bigints}

\usepackage{vmargin}
% left top textwidth textheight headheight
% headsep footheight footskip
\setmargins{2.0cm}{2.5cm}{16 cm}{22cm}{0.5cm}{0cm}{1cm}{1cm}
\renewcommand{\baselinestretch}{1.3}

\setcounter{MaxMatrixCols}{10}
\begin{document}Higher Certificate, Paper II, 2002.  Question 1 
%%%%%%%%%%%%%%%%%%%%%%%%%%%%%%%%%%%%%%%%%%%%%%%%%%%%%%%%%%%%%%%%%%%%%%%%%%%%%%%%%%%%%%%%%%%%%%%%%%%%%%%%%%%%%%%%%%%%%%%%%%%%%%%%%%%%%%%%%%%
\begin{table}[ht!]
 
\centering
 
\begin{tabular}{|p{15cm}|}
 
\hline  



1. (i) Compare the uses of the two-sample t test and the paired t test. 
 Using examples to illustrate your answer, clearly explain why each method should be used in preference to the other in particular situations. (6) 

\\ \hline
  
\end{tabular}

\end{table}

\begin{table}[ht!]
 
\centering
 
\begin{tabular}{|p{15cm}|}
 
\hline  
 
(ii) A farmer wishes to investigate whether the inclusion of a chicken food additive would affect the number of eggs laid by his chickens.  


To examine the impact of the food additive, the farmer selected a random sample of 24 chickens, all of similar age, and randomly allocated 12 to receive the normal food for three weeks and 12 to receive the normal food together with the food additive for three weeks.  


The number of eggs laid by each chicken during this period was recorded as follows. 
 
\begin{center} 
\begin{tabular}{|c|c|} \hline 
Normal food only & \{9 18 16 17 13 15 11 14 14 17 
18 16\} \\ \hline 
Normal food and additive & \{14 14 12 11 12 13 15 18 14 16 12 15 \} \\ \hline 
\end{tabular}
\end{centre}
 
Using an appropriate statistical test, investigate whether the inclusion of the food additive in the diet has any effect on the number of eggs laid by the chickens. (14) 
\\ \hline
  
\end{tabular}

\end{table} 
%%%%%%%%%%%%%%%%%%%%%%%%%%%%%%%%%%%%%%%%%%%%%%%%%%%%%%%%%%%%%%%%%%%%%%%%%%%%%%%%%%%%%%%%%%%%%%%%%%%%%%%%%%%%%%%%%%%%%%%%%%%%%%%%%%%%%%%%%%% 
\begin{enumerate} 
\item If two independent sets of data are available from Normally distributed populations of measurements, with possibly different means $\mu_1$, $\mu_2$ but the same variance $\sigma^2$, a two-sample t test is used.  It compares the sample means, 1 x and 2 x , based on $n_1$ and $n_2$ observations, against the null hypothesis $\mu_1 = \mu_2$. 
 
The test statistic is 
\[ TS = \frac{\bar{x}_1 - \bar{x}_2}{ \sqrt{ s^2 \left(  \frac{1}{n_1} + \frac{1}{n_2}  \right)} }, \]
o be referred to the $t_(n_1+n_2-2)$ distribution, and where 
 ()() 12 22 1 1 2 2 112
12 2
nn
ij ij x x x x
s
nn == − + −
=
+− ∑∑  is a "pooled" estimate of $\sigma^2$. 
 \begin{itemize}
     \item  This is not valid when the data sets have been collected from the same units (and so are not independent). 
     \item In this case there will be pairs of data ($x1i, x2i$);  for example in a medical trial these could be blood pressures before and after a standard exercise programme, or levels of a chemical before and after treatment with a drug. 
     \item Each patient is now acting as his or her own "control" and systematic patient-to-patient variation is removed. 
     \item The measurement for analysis is $di = x2i − x1i$ for each pair ($i = 1, 2, \ldots , n$).  \item A one-sample test of "$\mu_d = 0$" is now appropriate, using tn−1.  
     
     \item This procedure is called a paired test. 
 \end{itemize}

 
\item  A two-sample test is required, and since the chickens were all of similar age a common value of $\sigma^2$ may be assumed. 
$n_1 = n_2 = 12$

\begin{tabular}{cccc}
 $\sum x_1 = 160 $ ,&  $\sum x^2_1 = 2196 $ ,&  $\bar{x}_1= 13.33  $,&   $s^2_1 = 5.6970 $. \\
 $\sum x_2 =184  $,&  $\sum x^2_2 = 2870  $,& $\bar{x}_2 = 15.33  $,& $ s^2_2 = 4.4242 .$\\ 
\end{tabular} 

\[s^2 = \frac{1}{22} \left( 11 s_1^2 + 11 s^2_2 \right) = 5.0606\] 

The test statistic is 

\[ TS =  \frac{13.33 - 15.33}{5.0606 \left( \frac{1}{12} + \frac{1}{12} \right) } = - \frac{2}{0.918} = -2.178\]
 
This is significant at the 5\% level (the two-tailed 5\% point of t22 is 2.074). 
 
The null hypothesis \mu_1 = \mu_2 is rejected at the 5\% level, and so there is evidence that the two food regimes differ in effect. 

\end{enumerate}
\end{document}
