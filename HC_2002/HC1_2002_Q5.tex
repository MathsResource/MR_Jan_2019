\documentclass[a4paper,12pt]{article}
%%%%%%%%%%%%%%%%%%%%%%%%%%%%%%%%%%%%%%%%%%%%%%%%%%%%%%%%%%%%%%%%%%%%%%%%%%%%%%%%%%%%%%%%%%%%%%%%%%%%%%%%%%%%%%%%%%%%%%%%%%%%%%%%%%%%%%%%%%%%%%%%%%%%%%%%%%%%%%%%%%%%%%%%%%%%%%%%%%%%%%%%%%%%%%%%%%%%%%%%%%%%%%%%%%%%%%%%%%%%%%%%%%%%%%%%%%%%%%%%%%%%%%%%%%%%
\usepackage{eurosym}
\usepackage{vmargin}
\usepackage{amsmath}
\usepackage{graphics}
\usepackage{epsfig}
\usepackage{enumerate}
\usepackage{multicol}
\usepackage{subfigure}
\usepackage{fancyhdr}
\usepackage{listings}
\usepackage{framed}
\usepackage{graphicx}
\usepackage{amsmath}
\usepackage{chngpage}
%\usepackage{bigints}

\usepackage{vmargin}
% left top textwidth textheight headheight
% headsep footheight footskip
\setmargins{2.0cm}{2.5cm}{16 cm}{22cm}{0.5cm}{0cm}{1cm}{1cm}
\renewcommand{\baselinestretch}{1.3}

\setcounter{MaxMatrixCols}{10}
\begin{document}


Higher Certificate, Paper I, 2002. Question 5
\begin{framed}
5. State Bayes' Theorem and explain its use in practice.
\end{framed}
%%%%%%%%%%%%%%%%%%%%%%%%%%%%%%%%%%%
\begin{enumerate}[(a)]
    \item 
\begin{itemize}
\item Let A1, …, Ak be a set of mutually exclusive and exhaustive events, and let B be any
other event.

Bayes' theorem is stated mathematically as the following equation:
\[ {\displaystyle P(A\mid B)={\frac {P(B\mid A)\,P(A)}{P(B)}}} \]
 
where 
${\displaystyle A}$ 
 and 
${\displaystyle B}$ 
 are events and 
${\displaystyle P(B)\neq 0}$ 
. 
${\displaystyle P(A\mid B)}$ 
 is a conditional probability: the likelihood of event 
${\displaystyle A}$ 
 occurring given that 
${\displaystyle B}$ 
 is true.
$ {\displaystyle P(B\mid A)} $
 is also a conditional probability: the likelihood of event 
 $ {\displaystyle B} $ occurring given that 
 ${\displaystyle A}$ 
 is true.
${\displaystyle P(A)} $
 and 
${\displaystyle P(B)} $
 are the probabilities of observing 
$ {\displaystyle A} $
 and 
 ${\displaystyle B}$ 
 independently of each other; this is known as the marginal probability.

\item It is useful in inference about an Ai which cannot be observed directly but is related to
an observable event B.
\end{itemize}


\begin{framed}
A signal corps has to use an information channel, which is faulty. All messages
are sent as a sequence of six binary digits. The receiver knows that the message is
one of the following: '\textbf{\textit{Advance}}' (A), or '\textbf{\textit{Retreat}}' (R), or '\textbf{\textit{Stay where you are}}' (S).
From past experience he expects these messages in the respective ratios 1:4:5.
The three messages are sent as
\begin{description}
\item[A:] 0 1 0 1 0 0; 
\item[R:] 0 1 1 0 1 1; 
\item[S:] 1 0 1 0 0 1.
\end{description}

Independently for each character in the message, the fault causes '0' to be sent in
place of '1' with probability p, and '1' to be sent in place of '0' with equal
probability$ $p, the probability that any given character is transmitted correctly
being $1 – p$. The message is received as \[0 1 1 1 0 0\]. Show that
$P(011100 \mbox{ is received } | A: 010100 \mbox{ is sent } ) = p(1− p)^5$
and obtain similar expressions for
$P(011100 \mbox{ is received } | R: 011011 \mbox{ is sent })$
and
$P(011100 \mbox{ is received } | S: 101001 \mbox{ is sent })$.
\end{framed}



\item 
\[P(011100|A) = P(\mbox{5 right, 1 wrong}) = p(1− p)^5 .\]
\[P(011100|R) = P(\mbox{3 right, 3 wrong}) = (p^3)(1− p)^3 .\]
\[P(011100|S) = P(\mbox{2 right, 4 wrong}) = (p^4)(1− p)^2 .\]
This assumes all errors are independent.
\item Given that $P(A) = 0.1$, $P(R) = 0.4$, $P(S) = 0.5$, we have 
\[P(A|011100) = ( )
( )
5 0.1 p 1 p
P B
× −
,\]
where ( ) ( ) ( ) ( ) P B = 0.1p 1− p 5 + 0.4 p3 1− p 3 + 0.5 p4 1− p 2 .
So
\begin{eqnarray*}
P(A|011100) &=& ( )
( ( ) ( ) ( ) )
5
5 3 3 4 2
1
1 4 1 5 1
p p
p p p p p p
−
− + − + −
\\ &=&  ( )
( ) ( )
3
3 2 3
1
1 4 1 5
p
p p p p
−
− + − +
\end{eqnarray*}
\item Similarly, 

\begin{eqnarray*}
P(R|011100) &=& ( )
( ( ) ( ) ( ) )
3 3
5 3 3 4 2
4 1
1 4 1 5 1
p p
p p p p p p
−
− + − + −
\\ &=& ( )
( ) ( )
2
3 2 3
4 1
1 4 1 5
p p
p p p p
−
− + − +
.\\
\end{eqnarray*}
\item Also, 

\begin{eqnarray*}
P(S|011100) &=& ( )
( ( ) ( ) ( ) )
4 2
5 3 3 4 2
5 1
1 4 1 5 1
p p
p p p p p p
−
− + − + −
\\&=&
( ) ( )
3
3 2 3
5
1 4 1 5
p
− p + p − p + p
\end{eqnarray*}
\begin{framed}
Deduce that
P(A: 010100 \mbox{ is sent } | 011100 is received) = ( )
( ) ( )
3
3 2 3
1
,
1 4 1 5
p
p p p p
−
− + − +
and write down similar expressions for
\begin{itemize}
    \item $P(R: 011011 \mbox{ is sent } | 011100 is received)$

\item $P(S: 101001 \mbox{ is sent } | 011100 is received)$.
\end{itemize}


If it is assumed that p is at most 0.1, which interpretation of the message is most
likely to be correct?

\end{framed}

\begin{itemize}
\item As p → 0, $P(A|011100) \rightarrow 1$, while the others do not.
\item In general, $P(A|011100) > P(R|011100)$ 

% 2002 HC1 Q5


if $(1-p)^3 > 4p^2(1-p)$ or $(1-p)^2 >4p^2$, which requires $1-p>2p$, i.e. $1>3p$ or ${\displaystyle p<\frac{1}{3}}$.




\item Likewise $P(A|011100) > P(S|011100)$ if $(1-p)^3 > 4p^2(1-p)$ or $(1-p)^2 >5p^3$,  or ${\displaystyle (1-p) > p \sqrt{3}{5} }$, which requires $1 > p(1+\sqrt{3}{5}$, i.e. $1> 2.71p$, therefore
\[p < \frac{1}{2.17}\]

\item When $p \leq 0.1$, both these conditions are satisfied so choose A.
\item The probabilities are in the ratio $A:R:S \equiv≡ 0.729 : 0.036 : 0.005.$
\end{itemize}
%%%%%%%%%%%%%%%%%%%%%%%%%%%%%%%%
\end{enumerate}
\end{document}
