\documentclass[a4paper,12pt]{article}
%%%%%%%%%%%%%%%%%%%%%%%%%%%%%%%%%%%%%%%%%%%%%%%%%%%%%%%%%%%%%%%%%%%%%%%%%%%%%%%%%%%%%%%%%%%%%%%%%%%%%%%%%%%%%%%%%%%%%%%%%%%%%%%%%%%%%%%%%%%%%%%%%%%%%%%%%%%%%%%%%%%%%%%%%%%%%%%%%%%%%%%%%%%%%%%%%%%%%%%%%%%%%%%%%%%%%%%%%%%%%%%%%%%%%%%%%%%%%%%%%%%%%%%%%%%%
\usepackage{eurosym}
\usepackage{vmargin}
\usepackage{amsmath}
\usepackage{graphics}
\usepackage{epsfig}
\usepackage{enumerate}
\usepackage{multicol}
\usepackage{subfigure}
\usepackage{fancyhdr}
\usepackage{listings}
\usepackage{framed}
\usepackage{graphicx}
\usepackage{amsmath}
\usepackage{chngpage}
%\usepackage{bigints}

\usepackage{vmargin}
% left top textwidth textheight headheight
% headsep footheight footskip
\setmargins{2.0cm}{2.5cm}{16 cm}{22cm}{0.5cm}{0cm}{1cm}{1cm}
\renewcommand{\baselinestretch}{1.3}

\setcounter{MaxMatrixCols}{10}
\begin{document}


Higher Certificate, Paper III, 2002. Question 4

%%%%%%%%%%%%%%%%%%%%%%%%%%%%%%%%%%%%%%%%%%%%%%%%%%%%%%%%%%%%%% 
\begin{framed}
4. 
Quarterly data relating to passenger traffic on United Kingdom railways, published in the Monthly Digest of 
Statistics August 1999 and August 2001, are fitted by a centred four-quarter arithmetic average.  
The differences between the actual data and the moving average are obtained.  The results are as follows. 
 
  National Rail, passenger kilometres, millions Period  Actual Trend Difference 1996 

\begin{center}
\begin{tabular}{|c|c|c|c|c|}
1996	&	Q1	&	7554	&	na	&	na	\\ \hline
	&	Q2	&	7817	&	na	&	na	\\ \hline
	&	Q3	&	8101	&	8005.5	&	95.5	\\ \hline
	&	Q4	&	8330	&	8125.625	&	204.375	\\ \hline \hline
1997	&	Q1	&	7994	&	8277.5	&	−283.500	\\ \hline
	&	Q2	&	8338	&	8443.875	&	−105.875	\\ \hline
	&	Q3	&	8795	&	8594.125	&	200.875	\\ \hline
	&	Q4	&	8967	&	8713.375	&	253.625	\\ \hline \hline
1998	&	Q1	&	8559	&	8801.5	&	−242.500	\\ \hline
	&	Q2	&	8727	&	8895.125	&	−168.125	\\ \hline
	&	Q3	&	9111	&	9009.5	&	101.5	\\ \hline
	&	Q4	&	9400	&	9134.875	&	265.125	\\ \hline \hline 
1999	&	Q1	&	9041	&	9277.5	&	−236.500	\\ \hline
	&	Q2	&	9248	&	9397.75	&	−149.750	\\ \hline
	&	Q3	&	9731	&	9512.375	&	218.625	\\ \hline
	&	Q4	&	9742	&	9664.625	&	77.735	\\ \hline \hline
2000	&	Q1	&	9616	&	9885.75	&	−269.750	\\ \hline
	&	Q2	&	9891	&	9969.5	&	−78.500	\\ \hline
	&	Q3	&	10857	&	na	&	na	\\ \hline
	&	Q4	&	9286	&	na	&	na	\\ \hline
\end{tabular}
\end{center}
 
     Note:  "na" means "not available" 
 
 
(i) Using the method of differences from a moving arithmetic average, estimate the seasonal pattern in the observed data and hence correct that series for seasonal fluctuations. (12) 
 
(ii) Comment on the results. 
(4) 
 
(iii) Describe another method of seasonal correction which might have been preferable.  What advantages would it have had in this instance? (4) 
 
\end{framed}
%%%%%%%%%%%%%%%%%%%%%%%%%%%%%%%%%%%%%%%%%%%%%%%%%%%%%%%%%%%%%% 
\begin{enumerate} 
\item  A − T (Actual − Trend) figures:
Quarter
1 2 3 4
1996 . . 95.500 204.375
1997 –283.500 –105.875 200.875 253.625
1998 –242.500 –168.125 101.500 265.125
1999 –236.500 –149.750 218.625 77.735
Year
2000 –269.750 –78.500 . .
–1032.250 –502.250 616.500 800.860
−258.0625 −125.5625 154.1250 200.2150 –29.285 ÷ 4 =
–7.32125
Sum
Mean
Correction 7.3213 7.3213 7.3213 7.3213
–250.741 –118.241 161.446 207.536 0
Round to: –251 –118 161 208 0
To correct for seasonality:
Year Quarter A – S
1996 1 7554 –251 7805
2 7817 –118 7935
3 8101 161 7940
4 8330 208 8122
1997 1 7994 –251 8245
2 8338 –118 8456
3 8795 161 8634
4 8967 208 8759
1998 1 8559 –251 8810
2 8727 –118 8845
3 9111 161 8950
4 9400 208 9192
1999 1 9041 –251 9292
2 9248 –118 9366
3 9731 161 9570
4 9742 208 9534
2000 1 9616 –251 9867
2 9891 –118 10009
3 10857 161 10696
4 9286 208 9078
\item  There is a strong seasonal pattern, in addition to the trend; Q3 is the main
holiday period and Q4 includes pre-Christmas travel. The final figure, 2000 Q4, is an
outlier on the evidence of remaining data, so some attempt to find an explanation is
needed.
\item  With such a rapid increase in the actual figures, it may be that the seasonal use
has increased proportionately so that a multiplicative model would be better.
However, the A – T figures above do not really suggest this is the case. A
multiplicative model may be analysed following a log transformation to make it linear
(or in other ways also).

\end{enumerate}

\end{document}
