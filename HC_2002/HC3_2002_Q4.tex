\documentclass[a4paper,12pt]{article}
%%%%%%%%%%%%%%%%%%%%%%%%%%%%%%%%%%%%%%%%%%%%%%%%%%%%%%%%%%%%%%%%%%%%%%%%%%%%%%%%%%%%%%%%%%%%%%%%%%%%%%%%%%%%%%%%%%%%%%%%%%%%%%%%%%%%%%%%%%%%%%%%%%%%%%%%%%%%%%%%%%%%%%%%%%%%%%%%%%%%%%%%%%%%%%%%%%%%%%%%%%%%%%%%%%%%%%%%%%%%%%%%%%%%%%%%%%%%%%%%%%%%%%%%%%%%
\usepackage{eurosym}
\usepackage{vmargin}
\usepackage{amsmath}
\usepackage{graphics}
\usepackage{epsfig}
\usepackage{enumerate}
\usepackage{multicol}
\usepackage{subfigure}
\usepackage{fancyhdr}
\usepackage{listings}
\usepackage{framed}
\usepackage{graphicx}
\usepackage{amsmath}
\usepackage{chngpage}
%\usepackage{bigints}

\usepackage{vmargin}
% left top textwidth textheight headheight
% headsep footheight footskip
\setmargins{2.0cm}{2.5cm}{16 cm}{22cm}{0.5cm}{0cm}{1cm}{1cm}
\renewcommand{\baselinestretch}{1.3}

\setcounter{MaxMatrixCols}{10}
\begin{document}


Higher Certificate, Paper III, 2002. Question 4
\begin{enumerate} 
\item  A − T (Actual − Trend) figures:
Quarter
1 2 3 4
1996 . . 95.500 204.375
1997 –283.500 –105.875 200.875 253.625
1998 –242.500 –168.125 101.500 265.125
1999 –236.500 –149.750 218.625 77.735
Year
2000 –269.750 –78.500 . .
–1032.250 –502.250 616.500 800.860
−258.0625 −125.5625 154.1250 200.2150 –29.285 ÷ 4 =
–7.32125
Sum
Mean
Correction 7.3213 7.3213 7.3213 7.3213
–250.741 –118.241 161.446 207.536 0
Round to: –251 –118 161 208 0
To correct for seasonality:
Year Quarter A – S
1996 1 7554 –251 7805
2 7817 –118 7935
3 8101 161 7940
4 8330 208 8122
1997 1 7994 –251 8245
2 8338 –118 8456
3 8795 161 8634
4 8967 208 8759
1998 1 8559 –251 8810
2 8727 –118 8845
3 9111 161 8950
4 9400 208 9192
1999 1 9041 –251 9292
2 9248 –118 9366
3 9731 161 9570
4 9742 208 9534
2000 1 9616 –251 9867
2 9891 –118 10009
3 10857 161 10696
4 9286 208 9078
\item  There is a strong seasonal pattern, in addition to the trend; Q3 is the main
holiday period and Q4 includes pre-Christmas travel. The final figure, 2000 Q4, is an
outlier on the evidence of remaining data, so some attempt to find an explanation is
needed.
\item  With such a rapid increase in the actual figures, it may be that the seasonal use
has increased proportionately so that a multiplicative model would be better.
However, the A – T figures above do not really suggest this is the case. A
multiplicative model may be analysed following a log transformation to make it linear
(or in other ways also).

\end{enumerate}

\end{document}
