\documentclass[a4paper,12pt]{article}

%%%%%%%%%%%%%%%%%%%%%%%%%%%%%%%%%%%%%%%%%%%%%%%%%%%%%%%%%%%%%%%%%%%%%%%%%%%%%%%%%%%%%%%%%%%%%%%%%%%%%%%%%%%%%%%%%%%%%%%%%%%%%%%%%%%%%%%%%%%%%%%%%%%

\usepackage{eurosym}
\usepackage{vmargin}
\usepackage{amsmath}
\usepackage{graphics}
\usepackage{epsfig}
\usepackage{enumerate}
\usepackage{multicol}
\usepackage{subfigure}
\usepackage{fancyhdr}
\usepackage{listings}
\usepackage{framed}
\usepackage{graphicx}
\usepackage{amsmath}
\usepackage{chngpage}

%\usepackage{bigints}



\usepackage{vmargin}

% left top textwidth textheight headheight

% headsep footheight footskip

\setmargins{2.0cm}{2.5cm}{16 cm}{22cm}{0.5cm}{0cm}{1cm}{1cm}

\renewcommand{\baselinestretch}{1.3}

\setcounter{MaxMatrixCols}{10}

\begin{document}Higher Certificate, Paper II, 2004. Question 6
Part (i)
(a) If a set of data can be assumed to be a sample from a Normal distribution
(with mean μ and variance σ 2), the sample mean ( x ) from a sample of size n itself
has an underlying Normal distribution (with mean μ and variance σ 2/n). Thus the null
hypothesis μ = μ 0, where μ 0 is a specified value, is tested using the test statistic
0
/
z x
n
μ
σ
= −
and referring this to the N(0, 1) distribution.
In practice it is unusual to know σ
2. It might be known from past experience
("historical data"), and this can for example arise in some industrial statistical work, in
which case this method can be used.
In large samples from any distribution that is reasonably symmetrical, the method
works to a very good level of approximation using the estimated variance s2 instead of
σ 2. As an example, economic data (sales etc) are often so treated. How large the
sample needs to be depends on the symmetry of the underlying distribution.
Many "Normal approximations" of other measurements exist. An example is the
observed proportion in a binomial situation, such as the proportion of households
owning more than one car; the size of sample required for the approximation to be
good depends on the value of the proportion.
(b) If data are sampled from a Normal distribution where σ 2 is not known (so that
s2 is used in its place) and where the sample size is small – typically smaller than 30
– the t distribution must be used instead of N(0, 1) as above. Most sets of biological
and agricultural data are like this; the samples are not large, and the underlying
variability needs to be estimated from the data in each new experiment or study.
Part (ii)
2 2 10; 2159, 22360.44. 10; 2830, 19310.22. S S S N N N n = x = s = n = x = s =
The "pooled estimate" of variance is s2 = 20835.33.
The test statistic for testing the null hypothesis 600 N S μ −μ = is
1 1
10 10
600 71 1.10
64.55
N S x x
s
− − = =
+
,
which is referred to t18. This is not significant, so the null hypothesis cannot be
rejected. (The alternative hypothesis would be "difference > 600". Presumably the
result of the test would lead to the change not being made, on economic grounds.)
The time to recharging in each case is assumed to be Normally distributed, and the
variances of these Normal distributions are assumed to be equal.\end{enumerate}
\end{document}
