\documentclass[a4paper,12pt]{article}

%%%%%%%%%%%%%%%%%%%%%%%%%%%%%%%%%%%%%%%%%%%%%%%%%%%%%%%%%%%%%%%%%%%%%%%%%%%%%%%%%%%%%%%%%%%%%%%%%%%%%%%%%%%%%%%%%%%%%%%%%%%%%%%%%%%%%%%%%%%%%%%%%%%%%%%%%%%%%%%%%%%%%%%%%%%%%%%%%%%%%%%%%%%%%%%%%%%%%%%%%%%%%%%%%%%%%%%%%%%%%%%%%%%%%%%%%%%%%%%%%%%%%%%%%%%%

\usepackage{eurosym}
\usepackage{vmargin}
\usepackage{amsmath}
\usepackage{graphics}
\usepackage{epsfig}
\usepackage{enumerate}
\usepackage{multicol}
\usepackage{subfigure}
\usepackage{fancyhdr}
\usepackage{listings}
\usepackage{framed}
\usepackage{graphicx}
\usepackage{amsmath}
\usepackage{chngpage}

%\usepackage{bigints}
\usepackage{vmargin}

% left top textwidth textheight headheight

% headsep footheight footskip

\setmargins{2.0cm}{2.5cm}{16 cm}{22cm}{0.5cm}{0cm}{1cm}{1cm}

\renewcommand{\baselinestretch}{1.3}

\setcounter{MaxMatrixCols}{10}

\begin{document}
Higher Certificate, Paper I, 2004. Question 5

\begin{enumerate}
\item  \[f (t ) =\lambda e^{-\lambda t}  , t > 0; \lambda > 0\]
(a) Sketch of f (t).
[NOTE. The curve should of course appear as a smooth decaying exponential;
it might not do so, due to the limits of electronic reproduction.]
\item C.d.f. is ( ) ( ) 0
0
1 1
t

The cumulative distribution function is given by 
\[ {\displaystyle F(x;\lambda )={\begin{cases}1-e^{-\lambda x}&x\geq 0,\\0&x<0.\end{cases}}} \]


F t P T t t\lambdae \lambdavdv \lambdae \lambdav e \lambdat
λ
= ≤ = - = - -  = - -   ∫ .

\item 
\begin{eqnarray*}
P(a < T \leq b) &=& F (b) - F (a) \\ 
&=& e^{-\lambda a}  - e^{-\lambda b} 
\end{eqnarray*}
%%%%%%%%%%%%%%%%%%%%%%%%5
\item Assume all settlements of invoices are independent.
\begin{eqnarray*}
[P(50 in first week) &=& F{ (1)}^{50}  \\
&=& \left(1- e^{-\lambda}\right)^{50}  ,
\end{eqnarray*} 
because $T \leq 1$ for all these 50.

\begin{itemize}
\item Likewise, $1 < T \leq 2$ for the 35 in the second week, so we have P(35 in second week) =
\[{ ( ) ( )}35 F 2 - F 1 = ( )e^{-\lambda} - e^{-2\lambda} 35 .\]
\item The remaining 15 have T > 2, which has probability $1 – P(T \leq 2) = e^{-2\lambda}$ , and thus
\[P(\mbox{15 after week 2}) = ( e^{-2\lambda})^{15}\] .
\item The likelihood is therefore the product
\[ L(\lambda) = k (1- e^{-\lambda} )^{50} (e^{-\lambda} - (e^{-2\lambda})^{35} (e^{-2\lambda})^{15}\]
where k is a constant of proportionality.
\end{itemize}

f (t)
t
Taking logarithms (base e),
\begin{eqnarray*}
log L(\lambda) &=& log k + 50log(1- e^{-\lambda} )+ 35log{e^{-\lambda} (1- e^{-\lambda} )}+15log (e^{-2\lambda} )\\
&=& log k +85log (1- e^{-\lambda} )- (35 + 30)\lambda\\
&=& log k + 85log (1- e^{-\lambda} )- 65\lambda.
log 85 65 85 65
\end{eqnarray*}
\[ 1 1
d L e
d e e
λ
\lambda\lambdaλ
-
- ∴ = - = -
- -
.\]
Equating to zero, $85 = 65(e\lambda-1)$ or $e^\lambda=150 / 65$, so that $\hat{\lambda} = log (150 / 65) = 0.836 $$.


It is easy to check that this is indeed a maximum; e.g. ( )

\[2
2 2
log 85 0
1
d L
d\lambdaeλ
= - <
-
.\]
\item  \[1- e-0.836 = 0.5666; e-0.836 - e-1.672 = 0.43344 - 0.18787 = 0.2456 . \]

\begin{itemize}
\item Hence, out
of 100 invoices, 56.66, 24.56 and 18.78 would be expected to be paid, on this model,
in weeks 1, 2 and later. 
\item The actual numbers were 50, 35 and 15. The prediction for
the second week is a long way from what happened, balanced by smaller
discrepancies in the other two periods. 
\item This does not seem very satisfactory.
\end{itemize}
\end{enumerate}
\end{document}
