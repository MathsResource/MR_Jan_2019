Higher Certificate, Paper I, 2004. Question 2
(i) (a) The moment generating function is
( ) ( ) ( ) ( ) ( ( ))
0 0
exp exp 1
! !
t x x
tX tx t t
X
x x
e e M t E e e e e e
x x
λ
λ λ λ λ λ λ
∞ − ∞
−
= =
= =Σ =Σ = − = −
(b) ( ) ( ) ( )1
0
' 0
et t
X
t
E X M eeλ
λ λ −
=
= =   =  
.
( 2 ) ( ) ( 1)
0
'' 0
et t
X
t
E X M d e e
dt
λ λ −
=
= =         
( 1) ( 1) 2
0
. .
et et t t t
t
e e e e e λ λ λ λ λ λ λ − −
=
=  +  = +  
.
Hence ( ) ( ) ( ) Var X = E X 2 − E X 2 =λ   .
(Alternatively, the moments could be obtained from the power series
expansion of MX(t).)
(Alternatively, though with comparatively lengthy algebra, the
moments could be obtained directly by E(X) = ΣxP(X = x) and E(X2) =
Σx2P(X = x); or, somewhat easier, use E[X(X – 1)] = Σx(x – 1)P(X = x)
(this is λ2) and then Var(X) = E[X(X – 1)] + E(X) – {E(X)}2.)
(c) The binomial distribution with parameters n and p may be
approximated by the Poisson distribution with parameter np if n is
large and p is small. As a "rule of thumb", ½ ≤ np ≤ 10 gives an
indication of how large n should be and how small p should be. (If
np > 10, a Normal approximation to the binomial may be better.)
(ii) Let X = number of wrong calculations. We have X ~ B(200, 0.0075).
( ) ( )( )199 199 200
1 0.0075 0.9925 200 0.0075 0.9925 0.3353(2)
1
P X
 
= =   = × × =
 
.
( ) ( )( ) 200 4 196
4 0.0075 0.9925
4
P X
 
= = 
 
200 199 198 197 0.00754 0.9925196 0.0468(0)
4 3 2 1
= × × × × × =
× × ×
.
Continued on next page
(iii) We approximate using X ~ Poisson(200 × 0.0075 = 1.5). With this,
P( X =1) =1.5e−1.5 = 0.3347(0)
giving a percentage error of 100(0.33532 0.33470)
0.18%
0.33532
−
= , and
( ) ( ) ( )
1.5 1.5 4
4 0.0470 7
4!
e
P X
−
= = =
giving a percentage error of 100(0.04707 0.04680)
0.58%
0.04680
−
= .
[Note. These percentage errors might come out slightly differently if more
accuracy is kept in the binomial and Poisson probabilities.]
Both approximations are remarkably accurate, with percentage errors well
below 1%. The approximation for X = 1 (one wrong calculation) is the more
accurate of the two. That approximation is an underestimate; the other is an
overestimate.
