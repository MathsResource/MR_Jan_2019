\documentclass[a4paper,12pt]{article}

%%%%%%%%%%%%%%%%%%%%%%%%%%%%%%%%%%%%%%%%%%%%%%%%%%%%%%%%%%%%%%%%%%%%%%%%%%%%%%%%%%%%%%%%%%%%%%%%%%%%%%%%%%%%%%%%%%%%%%%%%%%%%%%%%%%%%%%%%%%%%%%%%%%%%%%%%%%%%%%%%%%%%%%%%%%%%%%%%%%%%%%%%%%%%%%%%%%%%%%%%%%%%%%%%%%%%%%%%%%%%%%%%%%%%%%%%%%%%%%%%%%%%%%%%%%%

\usepackage{eurosym}

\usepackage{vmargin}

\usepackage{amsmath}

\usepackage{graphics}

\usepackage{epsfig}

\usepackage{enumerate}

\usepackage{multicol}

\usepackage{subfigure}

\usepackage{fancyhdr}

\usepackage{listings}

\usepackage{framed}

\usepackage{graphicx}

\usepackage{amsmath}

\usepackage{chngpage}

%\usepackage{bigints}



\usepackage{vmargin}

% left top textwidth textheight headheight

% headsep footheight footskip

\setmargins{2.0cm}{2.5cm}{16 cm}{22cm}{0.5cm}{0cm}{1cm}{1cm}

\renewcommand{\baselinestretch}{1.3}



\setcounter{MaxMatrixCols}{10}

\begin{document}Higher Certificate, Paper I, 2004. Question 3
Actual volume X ~ N(1010, 82). Let Z ~ N(0,1).
\begin{enumerate}
\item  ( 1000) 1000 1010 ( 1.25) 0.1056
8
P X < = P Z < −  = P Z < − =
 
.
\item  Let Y be the total volume in a 6-pack.
We have Y ~ N(6 × 1010, 64 + 64 + 64 + 64 + 64 + 64), i.e. Y ~ N(6060, 384).
( 6000) 6000 6060 ( 3.06) 0.0011
384
P Y P Z −  P Z < =  <  = < − =
 
.
(Alternatively, could use X ~ N(1010, 64/6) and calculate P(X <1000).)
\begin{itemize}
    \item This probability is considerably smaller than that in part (i).
    \item In practical terms, this is
because there will be a tendency for heavier and lighter cartons in a 6-pack to balance
each other out. 
\item Alternatively, in terms of probability distributions, consider X and X :
X has the same mean as X but only one-sixth of the variance, so less of the lower tail
of the distribution of X is below the nominal volume of 1000.
\end{itemize}

\item The new volume W ~ N(μ, 42), where μ is the new mean. So we have that
( 1000) 1000
4
P W < = P Z < −μ 
 
. 

\begin{itemize}
\item 
We require that this probability must be no greater
than 0.1056.
\item Thus the cut-off point for Z is to be z = –1.25 (as before).
\item Hence
1000 1.25
4
−μ = − , giving $\mu = 1005$.
\item This means that 5 ml per carton could be saved, i.e. a cost saving per carton of
5
1000
× £1. 
\item To recover the £200, the number of cartons required is 200
5/1000
= 40000.
\end{itemize}

\end{enumerate}
\end{document}
