Higher Certificate, Paper I, 2004. Question 8
(i)
Trainee's time (y)
0
10
20
30
40
50
60
0 10 20 30 40 50
Benchmark time (x)
Simple linear regression analysis seems quite suitable.
(ii) The model is yi = α + β xi + ei, where {ei} are uncorrelated with zero mean and
(constant) variance σ 2 (independent identically distributed N(0, σ 2) for the purpose of
undertaking statistical tests, as in part (iii)). Estimating by the method of least squares
gives
ˆ xy
xx
S
S
β = , αˆ = y −βˆ x ,
where (standard notation)
( )( ) i i
xy i i i i
x y
S x x y y xy
n
= Σ − − = − Σ Σ Σ ,
( ) ( )2
2 2 i
xx i i
x
S x x x
n
= − = − Σ Σ Σ .
We have
( )
( 2 )
ˆ 4440 150 220 /10 1140 1.20
3200 150 /10 950
xy
xx
S
S
β
− ×
= = = =
−
and αˆ = 22 − (1.20×15) = 4 ,
so the line is
y = 4 + 1.2x.
Continued on next page
The total sum of squares is ( ) ( )2
2 2 1440
10
i
yy i i
y
S = y − y = y − = Σ Σ Σ .
The sum of squares for regression is ˆ
xy β S (or 2 / xy xx S S ) = 1368.
Therefore the residual sum of squares is 1440 – 1368 = 72.
This has 8 degrees of freedom, so the residual mean square (σˆ 2 ) is 72/8 = 9.
The coefficient of determination R2 = 1368/1440 = 0.95 (usually given as 95%).
(iii) The estimated variance of ˆβ is 9/950 = 0.009474. So the test statistic for
testing the null hypothesis β = 1 is 1.2 1
0.009474
− = 2.05, which we refer to t8.
This is not significant at the 5% level, so the null hypothesis β = 1 cannot be rejected.
(iv) The model here is yi = bxi + ei.
Estimating b by least squares, we minimise ( )2
1
n
i i
i
y bx
=
Ω =Σ − .
Differentiating with respect to b, we have 2 ( ) i i i
d y bx x
db
Ω = − Σ − .
Setting this equal to zero gives ˆ 2
i i i Σx y = bΣx , i.e. ˆ / 2 i i i b = Σx y Σx .
(Note that
2
2
2 2 0 i
d x
db
Ω = Σ > , so this is a minimum.)
Thus we have bˆ = 4440/3200 = 1.3875.
\end{document}
