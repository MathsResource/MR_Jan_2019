\documentclass{article}
\usepackage[utf8]{inputenc}

\title{RSS_Jan_2019_HC_2004}
\author{kobriendublin }
\date{December 2018}

\begin{document}

%-Higher Certificate, Paper I, 2004. Question 1
\begin{enumerate}

\item (a) P(all four favour the complex) = (0.6)4.
P(all four oppose the complex) = (0.3)4.
P(all four are indifferent) = (0.1)4.
So P(all four think alike) = (0.6)4 + (0.3)4 + (0.1)4 = 0.1378.
(b) P(an individual is not opposed) = 0.6 + 0.1 = 0.7.
So P(none of the four is opposed) = (0.7)4 = 0.2401.
(c) Possible favourable results are FFOI, FOOI, FOII, in any order.
P(FFOI) = (0.6)2(0.3)(0.1) = 0.0108
P(FOOI) = (0.6)(0.3)2(0.1) = 0.0054
P(FOII) = (0.6)(0.3)(0.1)2 = 0.0018
Each result can be arranged in 4!
2!1!1!
= 12 ways.

So overall probability is 12(0.0108 + 0.0054 + 0.0018) = 0.216.
\item \begin{itemize}
    \item From (a), P(all four in favour) = (0.6)4 = 0.1296.
    \item From (b), P(none
opposed) = 0.2401. 
\item So the required conditional probability is
0.1296/0.2401 = 0.5398.
\end{itemize} 
\item The number in favour is binomially distributed with n = 4 and p = 0.6. So the
expectation (mean) is 4 × 0.6 = 2.4 and the variance is 4 × 0.6 × 0.4 = 0.96.
\item P(opposed) = P(opposedyoung)P(young) + P(opposedolder)P(older)
= (0.12 × 0.25) + (p × 0.75)
where p = P(opposedolder). But we are given that P(opposed) = 0.3. Hence
p = 0.36.
\item In samples of one "young" and three "olders",
P(exactly one opposes) = P("young" opposes, "olders" do not)
+ P("young" does not oppose, one "older" opposes)
= {(0.12)(0.64)3} + {3(0.88)(0.36)(0.64)2} = 0.03146 + 0.38928 = 0.4207.
\end{enumerate}
\end{document}