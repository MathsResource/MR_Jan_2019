\documentclass[a4paper,12pt]{article}

%%%%%%%%%%%%%%%%%%%%%%%%%%%%%%%%%%%%%%%%%%%%%%%%%%%%%%%%%%%%%%%%%%%%%%%%%%%%%%%%%%%%%%%%%%%%%%%%%%%%%%%%%%%%%%%%%%%%%%%%%%%%%%%%%%%%%%%%%%%%%%%%%%%

\usepackage{eurosym}
\usepackage{vmargin}
\usepackage{amsmath}
\usepackage{graphics}
\usepackage{epsfig}
\usepackage{enumerate}
\usepackage{multicol}
\usepackage{subfigure}
\usepackage{fancyhdr}
\usepackage{listings}
\usepackage{framed}
\usepackage{graphicx}
\usepackage{amsmath}
\usepackage{chngpage}

%\usepackage{bigints}



\usepackage{vmargin}

% left top textwidth textheight headheight

% headsep footheight footskip

\setmargins{2.0cm}{2.5cm}{16 cm}{22cm}{0.5cm}{0cm}{1cm}{1cm}

\renewcommand{\baselinestretch}{1.3}

\setcounter{MaxMatrixCols}{10}

\begin{document}

Higher Certificate, Paper III, 2004.  Question 3 
 
 
(i) By using the same subjects on both occasions, experiment 1 should give more precise results than experiment 2;  subject-to-subject variation has been designed out.  This assumes, of course, that any effect of the drug would indeed have worn off within the week. 
 
 
(ii) n = 10.  Differences di (drug – placebo) are 4, 3, 6, –1, 7, 0, –5, 8, 5, 5.  So we have 23.2, 16.40 d ds ==.  The required 95% confidence interval is given by () 3.2 2.262 16.40/10 ±× where 2.262 is the double-tailed 5% point of t9, i.e. the interval is (0.30, 6.10). We must assume that the differences are Normally distributed. 
 
 
(iii) Let x refer to the drug and y to the placebo.  We have nx = ny = 10.  Sample means and variances are 2197.1, 816.322 xsx == and 2187.3, 770.233 ysy ==.  We must assume that the two samples are from Normal distributions with the same variance. 
 
The pooled estimate of this common variance is 793.278.  The required 95% confidence interval is given by () () () 11 10 10197.1 187.3 2.101 793.278 − ± × + where 2.101 is the double-tailed 5% point of t18, i.e. the interval is (–16.66, 36.26). 
 
 
(iv) The interval in part (ii) does not contain zero;  both its end-points are positive.  This gives some evidence that there is an increase due to the drug.  The interval in part (iii) is uninformative, being very wide and well spread in both directions about zero;  subject-to-subject variation has not been designed out and thus inflates the estimate of variance. 
 
