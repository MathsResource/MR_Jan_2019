\documentclass[a4paper,12pt]{article}

%%%%%%%%%%%%%%%%%%%%%%%%%%%%%%%%%%%%%%%%%%%%%%%%%%%%%%%%%%%%%%%%%%%%%%%%%%%%%%%%%%%%%%%%%%%%%%%%%%%%%%%%%%%%%%%%%%%%%%%%%%%%%%%%%%%%%%%%%%%%%%%%%%%%%%%%%%%%%%%%%%%%%%%%%%%%%%%%%%%%%%%%%%%%%%%%%%%%%%%%%%%%%%%%%%%%%%%%%%%%%%%%%%%%%%%%%%%%%%%%%%%%%%%%%%%%

\usepackage{eurosym}
\usepackage{vmargin}
\usepackage{amsmath}
\usepackage{graphics}
\usepackage{epsfig}
\usepackage{enumerate}
\usepackage{multicol}
\usepackage{subfigure}
\usepackage{fancyhdr}
\usepackage{listings}
\usepackage{framed}
\usepackage{graphicx}
\usepackage{amsmath}
\usepackage{chngpage}

%\usepackage{bigints}
\usepackage{vmargin}

% left top textwidth textheight headheight

% headsep footheight footskip

\setmargins{2.0cm}{2.5cm}{16 cm}{22cm}{0.5cm}{0cm}{1cm}{1cm}

\renewcommand{\baselinestretch}{1.3}

\setcounter{MaxMatrixCols}{10}

\begin{document}
Higher Certificate, Paper I, 2004. Question 7

\begin{enumerate}
\item (i) ( ) 2
0
0
1 1 1
2 2
E X x dx x
θ
θ θ
θ θ
= =   =   ∫ .
( ) 2
2 3 2
0
0
1 1 1
3 3
E X x dx x
θ
θ θ
θ θ
= =   =   ∫ .
( ) ( ) { ( )}
2
Var 2 2 1 2 1 1 2
3 2 12
∴ X = E X − E X = θ −  θ  = θ
 
.
\item  P(longest offcut is ≤ x) = P(all n offcuts are ≤ x).
The c.d.f. for each Xi is ( ) ( ) 0
0
x
F x P X x x du u x
θ θ θ
= ≤ = =   =   ∫ , and the Xi are all
independent. Therefore P(all n offcuts are ≤ x) = { ( )}
n
F x n x
θ
=    
 
, and this is also
P(longest offcut is ≤ x), i.e. the c.d.f. of the sample maximum (n) X . Thus the p.d.f. of
(n) X is the derivative of this, i.e. nxn–1/θ n. This is for the interval (0, θ ).
( ) ( ) 1
0
0 1 1
n n
n n n
E X nx dx n x n
n n
θ
θ θ
θ θ
 + 
∴ = =   =  +  + ∫ .
( ) ( ) 1 2 2
2
0
0 2 2
n n
n n n
E X nx dx n x n
n n
θ
θ θ
θ θ
+  + 
= =   =  +  + ∫ .
( ) ( ) ( ) ( ) ( ) { ( )} ( )
2 2 2 2
2
2 Var
2 1 n n n
X E X E X n n
n n
∴ = − = θ − θ
+ +
( ) ( )
( )( ) ( )( )
2 2
2
2 2
1 2
2 1 1 2
n n n n n
n n n n
θ θ
 + − + 
=   =
 + +  + +  
.
Immediately we have ( )
1
n
E n X
n
 +  =θ
 
, so ( )
1
n
n X
n
+ is an unbiased estimator of θ.
( ) ( )
( ) ( ) ( )
( )( ) ( )
2 2 2 2
2 2 2
1 1 1 Var Var
1 2 2 n
n n n n X n X
n n n n n n n
 +  = + = + θ = θ   +   + +
.
\item We have (see part (i)) that E(X) = θ /2. Thus the method of moments estimator
of θ /2 is X , and so the method of moments estimator of θ is 2X or 2
i X
n Σ as
required.
( ) ( ) ( )
2 4 4 2 2 Var Var 2 4Var Var .
12 3 iX X X X
n n n n
  = = = = θ =θ  
  Σ .
\end{enumerate}
\end{document}
