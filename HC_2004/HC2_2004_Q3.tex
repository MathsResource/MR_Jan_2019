\documentclass[a4paper,12pt]{article}

%%%%%%%%%%%%%%%%%%%%%%%%%%%%%%%%%%%%%%%%%%%%%%%%%%%%%%%%%%%%%%%%%%%%%%%%%%%%%%%%%%%%%%%%%%%%%%%%%%%%%%%%%%%%%%%%%%%%%%%%%%%%%%%%%%%%%%%%%%%%%%%%%%%

\usepackage{eurosym}
\usepackage{vmargin}
\usepackage{amsmath}
\usepackage{graphics}
\usepackage{epsfig}
\usepackage{enumerate}
\usepackage{multicol}
\usepackage{subfigure}
\usepackage{fancyhdr}
\usepackage{listings}
\usepackage{framed}
\usepackage{graphicx}
\usepackage{amsmath}
\usepackage{chngpage}

%\usepackage{bigints}



\usepackage{vmargin}

% left top textwidth textheight headheight

% headsep footheight footskip

\setmargins{2.0cm}{2.5cm}{16 cm}{22cm}{0.5cm}{0cm}{1cm}{1cm}

\renewcommand{\baselinestretch}{1.3}

\setcounter{MaxMatrixCols}{10}

\begin{document}
Higher Certificate, Paper II, 2004. Question 3


%%%%%%%%%%%%%%%%%%%%%%%%%%%%%%%%%%%%%%%%%%%%%%%%%%%%%%%%%%%%%%%%%%%%%%% 
%%-- Question 3
\begin{table}[ht!]
 
\centering
 
\begin{tabular}{|p{15cm}|}
 
\hline  

Discuss the advantages and disadvantages of using non-parametric rather than parametric methods in statistical analyses.
 
  

\\ \hline
  
\end{tabular}

\end{table}

 
\begin{table}[ht!]
 
\centering
 
\begin{tabular}{|p{15cm}|}
 
\hline  

 (ii) Ten boys and ten girls were selected at random from a large group of 16-yearold school children.  They were each asked to estimate how many hours each week they spent listening to music.  The results are shown in the following table. 
 
 Time in hours 
\begin{center}
\begin{tabular}{|c|cccccccccc|}
Boys & 21 & 6 & 3 & 18 & 28 & 16 & 5 & 30 &   2 & 0 \\ \hline 
Girls &   1 &  7 & 9 & 16 &   9 &   0 &  3 & 12&  21&  8\\  \hline 
\end{tabular}
\end{center} 
 
 (a) Draw separate stem and leaf diagrams for the data obtained from the boys and girls and hence comment on the distribution of the measurements in each group.  
Why would a t test be unsuitable for analysing these data?  
 
 
  (b) Using a suitable non-parametric test, investigate whether there is sufficient evidence to indicate a difference between the 
distributions of the number of hours reported listening to music per week for boys and for girls. (8) 
 

\\ \hline
  
\end{tabular}

\end{table} 
%%%%%%%%%%%%%%%%%%%%%%%%%%%%%%%%%%%%%%%%%%%%%%%%%%%%%%%%%%%%%%%%%%%%%%%  
\begin{enumerate}[(a)]
\item  Parametric methods need the assumption that the data come from a known
distribution, often the Normal distribution for continuous data or the binomial or
Poisson for discrete data. When these assumptions are satisfied, parametric tests of
hypotheses are the most powerful tests available. However, when these assumptions
are not satisfied, any tests or confidence intervals based on them are likely to give
wrong conclusions.
Non-parametric methods do not need distributional assumptions (even though the test
statistics actually used may have Normal approximations for adequate sizes of
sample). They are often based on ordering or ranking, and will serve for skewed data
and for ordered (and some categorical) data. They are often surprisingly powerful.
Nevertheless, when the conditions for a parametric test are satisfied, at least to a good
approximation, it should be used in preference to a non-parametric one to give greater
precision.
(ii) (a) Stems of 5 give the following.
BOYS 0 0 2 3 GIRLS 0 0 1 3
(5) 5 6 (5) 7 8 9 9
10 10 2
(15) 6 8 (15) 6
20 1 20 1
(25) 8
30 0
In each case the data appear to be skewed to the right. The non-Normality is
very pronounced, and a t test needs to make the assumption of Normality – so
it is not suitable.
\item A Mann-Whitney U test (or equivalently a Wilcoxon rank sum test)
may be used. The data and ranks are as follows, using average ranks for ties.
\begin{verbatim}
0 0 1 2 3 3 5 6 7 8 9 9 12 16 16 18 21 21 28 30
1½ 1½ 3 4 5½ 5½ 7 8 9 10 11½ 11½ 13 14½ 14½ 16 17½ 17½ 19 20
B G G B B G B B G G G G G B G B B G B B    
\end{verbatim}

n1 = 10, n2 = 10. Total rank for boys RB = 113; total rank for girls RG = 97.
Calculating the Mann-Whitney statistic via the ranks (note: it can also be
calculated directly, or the Wilcoxon rank-sum form could be used),
1 ( )
1 1 2 2 1 1 1 B U = n n + n n + − R = 100 + 55 – 113 = 42.
1 ( )
2 1 2 2 2 2 1 G U = n n + n n + − R = 100 + 55 – 97 = 58.
So Umin = 42. From tables, the critical value for a U test with n1 = n2 = 10 at
the 5% two-tailed level is 23. As 42 > 23, we accept the null hypothesis that
there is no difference between the distributions.
\end{enumerate}
\end{document}
