\documentclass[a4paper,12pt]{article}

%%%%%%%%%%%%%%%%%%%%%%%%%%%%%%%%%%%%%%%%%%%%%%%%%%%%%%%%%%%%%%%%%%%%%%%%%%%%%%%%%%%%%%%%%%%%%%%%%%%%%%%%%%%%%%%%%%%%%%%%%%%%%%%%%%%%%%%%%%%%%%%%%%%

\usepackage{eurosym}
\usepackage{vmargin}
\usepackage{amsmath}
\usepackage{graphics}
\usepackage{epsfig}
\usepackage{enumerate}
\usepackage{multicol}
\usepackage{subfigure}
\usepackage{fancyhdr}
\usepackage{listings}
\usepackage{framed}
\usepackage{graphicx}
\usepackage{amsmath}
\usepackage{chngpage}

%\usepackage{bigints}



\usepackage{vmargin}

% left top textwidth textheight headheight

% headsep footheight footskip

\setmargins{2.0cm}{2.5cm}{16 cm}{22cm}{0.5cm}{0cm}{1cm}{1cm}

\renewcommand{\baselinestretch}{1.3}

\setcounter{MaxMatrixCols}{10}

\begin{document}
Higher Certificate, Paper II, 2004. Question 3
(i) Parametric methods need the assumption that the data come from a known
distribution, often the Normal distribution for continuous data or the binomial or
Poisson for discrete data. When these assumptions are satisfied, parametric tests of
hypotheses are the most powerful tests available. However, when these assumptions
are not satisfied, any tests or confidence intervals based on them are likely to give
wrong conclusions.
Non-parametric methods do not need distributional assumptions (even though the test
statistics actually used may have Normal approximations for adequate sizes of
sample). They are often based on ordering or ranking, and will serve for skewed data
and for ordered (and some categorical) data. They are often surprisingly powerful.
Nevertheless, when the conditions for a parametric test are satisfied, at least to a good
approximation, it should be used in preference to a non-parametric one to give greater
precision.
(ii) (a) Stems of 5 give the following.
BOYS 0 0 2 3 GIRLS 0 0 1 3
(5) 5 6 (5) 7 8 9 9
10 10 2
(15) 6 8 (15) 6
20 1 20 1
(25) 8
30 0
In each case the data appear to be skewed to the right. The non-Normality is
very pronounced, and a t test needs to make the assumption of Normality – so
it is not suitable.
(b) A Mann-Whitney U test (or equivalently a Wilcoxon rank sum test)
may be used. The data and ranks are as follows, using average ranks for ties.
0 0 1 2 3 3 5 6 7 8 9 9 12 16 16 18 21 21 28 30
1½ 1½ 3 4 5½ 5½ 7 8 9 10 11½ 11½ 13 14½ 14½ 16 17½ 17½ 19 20
B G G B B G B B G G G G G B G B B G B B
n1 = 10, n2 = 10. Total rank for boys RB = 113; total rank for girls RG = 97.
Calculating the Mann-Whitney statistic via the ranks (note: it can also be
calculated directly, or the Wilcoxon rank-sum form could be used),
1 ( )
1 1 2 2 1 1 1 B U = n n + n n + − R = 100 + 55 – 113 = 42.
1 ( )
2 1 2 2 2 2 1 G U = n n + n n + − R = 100 + 55 – 97 = 58.
So Umin = 42. From tables, the critical value for a U test with n1 = n2 = 10 at
the 5% two-tailed level is 23. As 42 > 23, we accept the null hypothesis that
there is no difference between the distributions.\end{enumerate}
\end{document}
