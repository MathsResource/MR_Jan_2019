\documentclass[a4paper,12pt]{article}

%%%%%%%%%%%%%%%%%%%%%%%%%%%%%%%%%%%%%%%%%%%%%%%%%%%%%%%%%%%%%%%%%%%%%%%%%%%%%%%%%%%%%%%%%%%%%%%%%%%%%%%%%%%%%%%%%%%%%%%%%%%%%%%%%%%%%%%%%%%%%%%%%%%

\usepackage{eurosym}
\usepackage{vmargin}
\usepackage{amsmath}
\usepackage{graphics}
\usepackage{epsfig}
\usepackage{enumerate}
\usepackage{multicol}
\usepackage{subfigure}
\usepackage{fancyhdr}
\usepackage{listings}
\usepackage{framed}
\usepackage{graphicx}
\usepackage{amsmath}
\usepackage{chngpage}

%\usepackage{bigints}



\usepackage{vmargin}

% left top textwidth textheight headheight

% headsep footheight footskip

\setmargins{2.0cm}{2.5cm}{16 cm}{22cm}{0.5cm}{0cm}{1cm}{1cm}

\renewcommand{\baselinestretch}{1.3}

\setcounter{MaxMatrixCols}{10}

\begin{document}
Higher Certificate, Paper III, 2004.  Question 4 
 
 %%%%%%%%%%%%%%%%%%%%%%%%%%%%%%%%%%%%%%%%%%%%%%%%%%%%%%%%%%%%%%%%%%%%%%%%
\begin{framed}

The display and plot below show quarterly domestic sales (Sales) in terawatt hours by
the public electricity supply system in the UK from the first quarter of 1997 to the
third quarter of 2001. The display also shows centred 4-quarterly moving averages
(MA), and the values of the differences "Sales – MA".

\begin{center}
\begin{tabular}{ccccc}
Year	&	Quarter	&	Sales MA	&	MA	&	Sales – MA	\\ \hline
1997	&	1	&	31.54	&	*	&	*	\\ \hline
1997	&	2	&	22.33	&	*	&	*	\\ \hline
1997	&	3	&	20.29	&	26.22	&	–5.93	\\ \hline
1997	&	4	&	30.3	&	26.57	&	3.73	\\ \hline
1998	&	1	&	32.35	&	26.93	&	5.42	\\ \hline
1998	&	2	&	24.36	&	27.2	&	–2.84	\\ \hline
1998	&	3	&	21.16	&	27.54	&	–6.38	\\ \hline
1998	&	4	&	31.54	&	27.64	&	3.9	\\ \hline
1999	&	1	&	33.85	&	27.61	&	6.24	\\ \hline
1999	&	2	&	23.69	&	27.62	&	–3.93	\\ \hline
1999	&	3	&	21.55	&	27.43	&	–5.88	\\ \hline
1999	&	4	&	31.22	&	27.27	&	3.95	\\ \hline
2000	&	1	&	32.64	&	27.49	&	5.15	\\ \hline
2000	&	2	&	23.64	&	27.84	&	–4.20	\\ \hline
2000	&	3	&	23.37	&	28.15	&	–4.78	\\ \hline
2000	&	4	&	32.2	&	28.32	&	3.88	\\ \hline
2001	&	1	&	34.1	&	28.33	&	5.77	\\ \hline
2001	&	2	&	23.61	&	*	&	*	\\ \hline
2001	&	3	&	23.46	&	*	&	*	\\ \hline
\end{tabular}
\end{center}
35
30
4.
25
20
Q uarter
Year
1 2
1997
3 4 1 2 3
1998
4 1 2 3 4
1999
1 2 3 4 1
2000
2 3
2001
(i) Give the calculation leading to the value of 26.22 for the MA for 1997
quarter 3.
(4)
(ii) Using the differences given for "Sales – MA", estimate the pattern of the
seasonal variation in sales.
\end{framed}
\begin{framed}
(iii) Why is it good practice in the course of making such estimates to plot the data
and the moving average trend as time series?
(4)
(iv) Using the estimates found in part (ii), correct the 2000 Sales figures for
seasonal fluctuations.
Could you similarly correct the 2001 figures? Explain your answer.
(4)
5

\end{framed}
%%%%%%%%%%%%%%%%%%%%%%%%%%%%%%%%%%%%%%%%%%%%%%%%%%%%%%%%%%%%%%%%%%%%%%%%
\begin{enumerate}
\item 
Year Quarter Sales 4-quarter totals 8-quarter totals 
Moving average 
1997 1 31.54    1997 2 22.33   1997 3 20.29 209.73 26.216(25) 1997 4 30.30 
 104.46 105.27   1998 1 32.35    
 
 
\item  There is a sharp seasonal variation, "Sales – MA" being always substantially negative in quarters 2 and 3, always substantially positive in quarters 1 and 4.  To estimate the pattern of seasonal variation, we need the average of the "Sales – MA" figures for each quarter. 
 
 Q1 Q2 Q3 Q4   1997     –5.93   3.73   1998   5.42   –2.84   –6.38   3.90   1999   6.24   –3.93   –5.88   3.95   2000   5.15   –4.20   –4.78   3.88   2001   5.77      Seasonal totals 22.58 –10.97 –22.97 15.46   Seasonal averages     5.645     –3.657     –5.743     3.865 Sum:  0.110 Correction:  –0.028 Corrected seasonal averages     5.617     –3.685     –5.771     3.837 (–0.002)  
 
 
\item Using this case as an example, the visual pattern can show detail which is lost in the table of figures, such as in the year 2000 where the fluctuation was not so great as in other years, although the pattern was the same.  We can also see that, while it rises overall, the MA trend shows a slight dip from late 1998 onwards before a sharper rise in 2000.  We can visualise the trend from the table when it is fairly smooth like this, but not always so easily.  Trend and seasonal variation are important properties to observe, and a clear method of doing so is invaluable. 
 
\item We use observed sales minus estimated seasonal variation.  For year 2000: 
 
Q1 Q2 Q3 Q4 32.64 – 5.62 = 27.02 23.64 + 3.69 = 27.33 23.37 + 5.77 = 29.14 32.20 – 3.84 = 28.36 
 
It would not be a good idea to use this method on the 2001 sales because the estimated seasonal variation might have changed if we had had enough data to make "Sales – MA" up to the end of the year. 
\end{enumerate}
\end{document} 
