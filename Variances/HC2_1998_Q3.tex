\documentclass[a4paper,12pt]{article}
%%%%%%%%%%%%%%%%%%%%%%%%%%%%%%%%%%%%%%%%%%%%%%%%%%%%%%%%%%%%%%%%%%%%%%%%%%%%%%%%%%%%%%%%%%%%%%%%%%%%%%%%%%%%%%%%%%%%%%%%%%%%%%%%%%%%%%%%%%%%%%%%%%%%%%%%%%%%%%%%%%%%%%%%%%%%%%%%%%%%%%%%%%%%%%%%%%%%%%%%%%%%%%%%%%%%%%%%%%%%%%%%%%%%%%%%%%%%%%%%%%%%%%%%%%%%
\usepackage{eurosym}
\usepackage{vmargin}
\usepackage{amsmath}
\usepackage{graphics}
\usepackage{epsfig}
\usepackage{enumerate}
\usepackage{multicol}
\usepackage{subfigure}
\usepackage{fancyhdr}
\usepackage{listings}
\usepackage{framed}
\usepackage{graphicx}
\usepackage{amsmath}
\usepackage{chngpage}
%\usepackage{bigints}

\usepackage{vmargin}
% left top textwidth textheight headheight
% headsep footheight footskip
\setmargins{2.0cm}{2.5cm}{16 cm}{22cm}{0.5cm}{0cm}{1cm}{1cm}
\renewcommand{\baselinestretch}{1.3}

\setcounter{MaxMatrixCols}{10}
\begin{document}
%%%%%%%%%%%%%%%%%%%%%%%%%%%%%%%%%%%%%%%%%%%%%%%%%%%%%%%%%%%%%%%%%%%%%%%%%%%%%%%%%%%%%%%%%%%%%%%%%%%%%%%%%%%%%%%%%%%%%
\begin{table}[ht!]
 
\centering
 
\begin{tabular}{|p{15cm}|}
 
\hline   \large

%%-- 3. (a) 
\noindent A manufacturer of industrial light bulbs wishes to control the variability in the length of life of the bulbs so that its standard deviation $\sigma$ is less than or equal to 150 hours.  A random sample of 10 bulbs was taken and tested in the laboratory giving the following results in hours:
\[\{2100,  2302,  1951,  2415,  2067,  1911,  2149,  2489,  2083,  2124\}\]
Test the hypothesis $H_0: \sigma \leq 150$ hours against a suitable alternative.  Explain your results and state any assumptions which you made.

\\ \hline
  
\end{tabular}

\end{table}
%%%%%%%%%%%%%%%%%%%%%%%%%%%%%%%%%%%%%%%%%%%%%%%%%%%%%%%%%%%%%%%%%%%%%%%%%%%%%%%%%%%%%%%%%%%%%%%%%%%%%%%%%%%%%%%%%%%%%
\large 
\begin{itemize}
\item If we can assume that the lifetime distribution for the bulbs is normal with variance $\sigma^2$,
and all observations are independent of one another, then ${\displaystyle \frac{(n - 1)s^2}{\sigma^2} }$ will be distributed $\chi^2_{(n-1)}$.
\item Here n=10, and on $H_0$  we take $\sigma^2$ = $150^2$. 
\item Then effectively we test $H_0 : \sigma^2 \leq 150^2$ against
$H_1 : \sigma^2 > 150^2$.
\item For the data, $(n - 1)s^2 = 9 \times 35410.99$. 
\item Hence $\chi^2_{(9)} = 14.16$, which is not significant at the 5\%
level. (Critical Value is 23.684.)
\item Therefore we do not have enough evidence to reject $H_0$ which says $\sigma \leq 150$.
\end{itemize}

\newpage
\large
\begin{table}[ht!]
 
\centering
 
\begin{tabular}{|p{15cm}|}
 
\hline  \large

\noindent A chemical manufacturer using two production lines has made slight adjustments to the second in an attempt to reduce the variability in the levels of impurities in the chemical produced.  Twelve randomly selected batches of chemical from each process were analysed and the level of impurities found to be as follows:

\begin{center}
\begin{tabular}{|c|c|l|} \hline
Process 1 & 2.12 2.45 2.43 2.51 2.52 2.44  & $s_1 = 0.111966$ \\& 2.56 2.51 2.41 2.46 2.43 2.38 &\\ \hline
Process 2 & 2.46 2.45 2.46 2.44 2.55 2.56  & $s_2 = 0.0.061957$\\& 2.55 2.36 2.50 2.52 2.48 2.42 &\\ \hline
\end{tabular}
\end{center}


Using an appropriate statistical test, investigate whether the manufacturer has been successful in reducing the process variability for process.  
Explain your conclusions  stating any assumptions which you made.

\\ \hline
  
\end{tabular}

\end{table}
\begin{itemize}
%%%%%%%%%%%%%%%%
\item Since twelve randomly selected batches were used from each process we have independent estimates
of variances $\sigma^2_1$ and $\sigma^2_2$.
\item The Null Hypothesis will be $\sigma^2_1 = \sigma^2_2$, and $H_!: \sigma^2_1 > \sigma^2_2$


\item From the data, $s^2_1= 0.012536$ and $s^2_2= 0.003590$.
\item Assuming that the distributions of impurity levels are normal, $s^2_1=s^2_2$

is distributed as F(11,11).
\item The test statistic is \[ \frac{s^2_1}{s^2_2}
= 3.49,\]significant at the 5\% level (Critical Value = 2.817) so that $H_0$ is rejected in favor of $H_1$: there is evidence of a
reduction in process variability.
\end{itemize}

\end{document}


Sure, here's your document converted to Markdown:

```markdown
\setcounter{MaxMatrixCols}{10}
%%%%%%%%%%%%%%%%%%%%%%%%%%%%%%%%%%%%%%%%%%%%%%%%%%%%%%%%%%%%%%%%%%%%%%%%%%%%%%%%%%%%%%%%%%%%%%%%%%%%%%%%%%%%%%%%%%%%%
### Table

|                                                                                                                                                                                                                                                                         |
|---------------------------------------------------------------------------------------------------------------------------------------------------------------------------------------------------------------------------------|
| A manufacturer of industrial light bulbs wishes to control the variability in the length of life of the bulbs so that its standard deviation $\sigma$ is less than or equal to 150 hours.  A random sample of 10 bulbs was taken and tested in the laboratory giving the following results in hours: \[ \{2100,  2302,  1951,  2415,  2067,  1911,  2149,  2489,  2083,  2124\} \] Test the hypothesis $H_0: \sigma \leq 150$ hours against a suitable alternative.  Explain your results and state any assumptions which you made. |

\large

- If we can assume that the lifetime distribution for the bulbs is normal with variance $\sigma^2$, and all observations are independent of one another, then ${\displaystyle \frac{(n - 1)s^2}{\sigma^2} }$ will be distributed $\chi^2_{(n-1)}$.
- Here n=10, and on $H_0$ we take $\sigma^2$ = $150^2$.
- Then effectively we test $H_0 : \sigma^2 \leq 150^2$ against $H_1 : \sigma^2 > 150^2$.
- For the data, $(n - 1)s^2 = 9 \times 35410.99$.
- Hence $\chi^2_{(9)} = 14.16$, which is not significant at the 5% level. (Critical Value is 23.684.)
- Therefore we do not have enough evidence to reject $H_0$ which says $\sigma \leq 150$.

\newpage
### Table

|                                                                                                                                                                                                                                                 |
|-------------------------------------------------------------------------------------------------------------------------------------------------------------------------------------------------------------------------------------------------|
| A chemical manufacturer using two production lines has made slight adjustments to the second in an attempt to reduce the variability in the levels of impurities in the chemical produced. Twelve randomly selected batches of chemical from each process were analysed and the level of impurities found to be as follows: |

\begin{center}

| Process 1 & 2.12 2.45 2.43 2.51 2.52 2.44  & $s_1 = 0.111966$ |
| Process 1 & 2.56 2.51 2.41 2.46 2.43 2.38 &  |
| Process 2 & 2.46 2.45 2.46 2.44 2.55 2.56  & $s_2 = 0.061957$ |
| Process 2 & 2.55 2.36 2.50 2.52 2.48 2.42 & |

Using an appropriate statistical test, investigate whether the manufacturer has been successful in reducing the process variability for process.  
Explain your conclusions stating any assumptions which you made.

- Since twelve randomly selected batches were used from each process we have independent estimates of variances $\sigma^2_1$ and $\sigma^2_2$.
- The Null Hypothesis will be $\sigma^2_1 = \sigma^2_2$, and $H_1 : \sigma^2_1 > \sigma^2_2$
- From the data, $s^2_1 = 0.012536$ and $s^2_2 = 0.003590$.
- Assuming that the distributions of impurity levels are normal, $s^2_1 = s^2_2$ is distributed as F(11,11).
- The test statistic is \[ \frac{s^2_1}{s^2_2} = 3.49 \], significant at the 5% level (Critical Value = 2.817) so that $H_0$ is rejected in favor of $H_1$: there is evidence of a reduction in process variability.
```

Let me know if you need any further assistance!
