4. (i) Since we know that 2 2 2
1 1
( ) ,
n n
i i
i i
X X X nX
 
    then since 2 2
1 1
( ) ( ),
n n
i i
i i
E X E X
 
 
2 2 2
1 1
( ) ( ) ( ).
n n
i i
i i
E X X E X nE X
 
 
    
 
 
1 (alternative form), 1(written as expectations)
Now 2 2 2 2 2 2 ( ) ( ) i i   E X   E X   and so 2 2 2
1
( ) ( ).
n
i
i
E X n 

  
1 (E(X 2 )), 1 (multiply by n)
2 2
2 2 2 2 Var(X ) E(X ) E(X ) .
n n
 
      1 (var of mean), 1 (final expression)
We require
2
2 2 2 2 2 2
1
( ) ( ) ( 1) .
n
i
i
X nE X n n n
n

   

 
        
 
 So
2 2
1
1 1
( ) ( 1)
n
i
i
E X X n
n n


 
    
 
 as required. 1 (correctly combined)
For an unbiased estimator, we require the expectation to equal 2 . From the previous line,
2 2 2
1
1 ( 1)
( ) . ,
1 ( 1)
n
i
i
n n n
E X X
n n n n
 

  
        
 so 2
1
1
( )
1
n
i
i
X X
n 

  is an unbiased
estimator for 2 .
1 (require expectation = 2  ), 1 (final estimator correct)
TOTAL 9
(ii) 2 2 2 2
1 1
( ) ( ) ( ) ( 1) .
n n
a i i
i i
E S E a X X aE X X a n 
 
   
         
   
 
1 (constant out of sum), 1 (correct final expression)
So 2 2 2 2 ( ) ( 1) ( 1) . a Bias S  a n     an  a   1 (method), 1 (bias correct)
TOTAL 4
(iii) 2 2 2 2 2 4
1 1
Var( ) Var ( ) Var ( ) 2 ( 1) .
n n
a i i
i i
S a X X a X X a n 
 
   
         
   
 
1 (constant squared)
So 2 2 4 2 4 ( ) 2 ( 1) ( 1) . a MSE S  a n    an  a   1 (MSE correct)
For max/min, set
dM
da
equal to zero: 1 (method)
4 4 4 ( 1) 2( 1)( 1) 0.
dM
a n an a n
da
         1 (deriv correct)
Dividing through by 4 2(n 1) :
1
2 1 0 ( 1) 1 .
1
a an a a n a
n
        

1( 1/(n+1))
To show that this gives a minimum value, look at the second derivative:
2
4 2 4
2 4( 1) 2( 1) ,
d M
n n
da
      which is clearly positive, so the required value of a is
1
.
n 1