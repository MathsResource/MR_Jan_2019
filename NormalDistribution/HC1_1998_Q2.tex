\documentclass[a4paper,12pt]{article}
%%%%%%%%%%%%%%%%%%%%%%%%%%%%%%%%%%%%%%%%%%%%%%%%%%%%%%%%%%%%%%%%%%%%%%%%%%%%%%%%%%%%%%%%%%%%%%%%%%%%%%%%%%%%%%%%%%%%%%%%%%%%%%%%%%%%%%%%%%%%%%%%%%%%%%%%%%%%%%%%%%%%%%%%%%%%%%%%%%%%%%%%%%%%%%%%%%%%%%%%%%%%%%%%%%%%%%%%%%%%%%%%%%%%%%%%%%%%%%%%%%%%%%%%%%%%
\usepackage{eurosym}
\usepackage{vmargin}
\usepackage{amsmath}
\usepackage{graphics}
\usepackage{epsfig}
\usepackage{enumerate}
\usepackage{multicol}
\usepackage{subfigure}
\usepackage{fancyhdr}
\usepackage{listings}
\usepackage{framed}
\usepackage{graphicx}
\usepackage{amsmath}
\usepackage{chngpage}
%\usepackage{bigints}

\usepackage{vmargin}
% left top textwidth textheight headheight
% headsep footheight footskip
\setmargins{2.0cm}{2.5cm}{16 cm}{22cm}{0.5cm}{0cm}{1cm}{1cm}
\renewcommand{\baselinestretch}{1.3}

\setcounter{MaxMatrixCols}{10}

\begin{document}

\begin{table}[ht!]
     \centering
     \begin{tabular}{|p{15cm}|}
     \hline        
\noindent \large
Paget's disease is estimated to affect 10\% of persons over 65 years old.\smallskip

It is known that, among persons over 65 without the disease, a blood measure X in suitable units is distributed N(7,9) (i.e. Normally, with mean 7 and variance 9) but for persons over 65 suffering from the disease, X is distributed $N(19,36)$.  \smallskip
\\ \hline
 \end{tabular}
\end{table}

\begin{table}[ht!]
     \centering
     \begin{tabular}{|p{15cm}|}
     \hline        
 \noindent \large \textbf{Part (a)}\\ \large \smallskip
\noindent A value of $X \geq 10$ is taken as a cause for further investigation. \smallskip

(a) What proportion of non-sufferers over 65 will be investigated because of their value of $X$?



\\ \hline
 \end{tabular}
\end{table}
\large

%%%%%%%%%%%%%%%%%%%%%%%%%%%%%%%%%%%%%%%%%%%%%%%%%%%%%%%%%%%%%%%%%%%%%%%%%%%%%%%%%%%%%%
\begin{enumerate}[(a)]
\item For a non-sufferer, $X\sim N(7,9)$,

Compute $P(X \geq 10)$

%----------------%
\begin{framed}

\noindent \textbf{Z Score for $X = 10$}

\[z_{10}  = \frac{10-7}{ 3}  = \frac{3}{3} = 1\]
\end{framed}
%------------------%
\begin{eqnarray*}
P(x \geq 10 ) &=&  1- P(X \leq 10)
\\ & &
\\ &=&  1 - P(Z \leq 1)
\\ & & 
\\ &=& 1 - 0.8413
\\ & &
\\ &=& 0.1587 \\
\end{eqnarray*}
where $Z\sim N(0,1)$
\newpage
%%%%%%%%%%%%%%%%%%%%%%%%%%%%%%%%%
\begin{table}[ht!]
     \centering
     \begin{tabular}{|p{15cm}|}
     \hline        \large
 \noindent \textbf{Part (b)}\\ \large
\noindent What proportion of sufferers over 65 will go undiagnosed? \smallskip
\\ \hline
 \end{tabular}
\end{table}

\item For a sufferer, $X\sim N(19,36)$, so


%----------------%
\begin{framed}

\noindent \textbf{Z Score for $X = 10$}

\[z_{10}  = \frac{10\;-\;19}{ 6}  = -\frac{9}{6} = -1.50\]
\end{framed}
%------------------%

\begin{eqnarray*}
P(X<10) 
&=& P(Z < -1.50 )\\ &=& 0.0668
\end{eqnarray*}
where $Z\sim N(0,1)$
%[Calculate as 1-P(Z<+3/2) if using tables.]
%%%%%%%%%%%%%%%%%%%%%%%%%%%%

\newpage
%%%%%%%%%%%%%%%%%%%%%%%%%%%%%%%%%
\begin{table}[ht!]
     \centering
     \begin{tabular}{|p{15cm}|}
     \hline        \large
 \noindent \textbf{Part (c)}\\ \large
\noindent At what critical level of X (instead of 10) would the false positive rate be 5\% (i.e. 5\% of non-sufferers would have $X$ above this level)?

\\ \hline
 \end{tabular}
\end{table}
\item If critical level is $x_0$, $P(X¸\geq x_0 | x \sim N(7,9))=0.05$.

In $N(0,1)$ the upper 5\% point is $z_0=1.645$ ; hence 
\[z_0 = \frac{x_0\;-\;7}{3} = 1.645 \]
Re-arranging
\[x_0 = 7+(3\times1.645) =
11.935.\]
%%%%%%%%%%%%%%%%%%%%%%%%%%%%%%%%%%%%%%%%%%%%%%%%%%%%%%%%%%%%%%%%%%%%%%%%%5
\newpage

\begin{table}[ht!]
     \centering
     \begin{tabular}{|p{15cm}|}
     \hline        
 \noindent \textbf{Part (d)}\\
\noindent If all people over 65 are screened using $X \geq 10$ as the criterion for further investigation, what proportion will be investigated?
\\ \hline
 \end{tabular}
\end{table}


%%%%%%%%%%%%%%%%%%%%%%%%%%%%%%%%%%%%%%%%%%%%%%%
\item 

\begin{eqnarray*}
P(X\geq 10) &=& 
P(X \geq 10|\mbox{sufferer})P(\mbox{has disease}) +\\ & & P(X \geq 10|\mbox{non - sufferer})P(\mbox{does not have disease})\\
& & \\
&=& \left[(1 - 0.0668) \times 0.1\right] + \left[0.1587 \times 0.9\right]\\
& & \\
&=& 0.2362.\\
\end{eqnarray*}

% [use answers(i),(ii) and information that 10\% if population affected.]



%%%%%%%%%%%%%%%%%%%%%%%%%%%%%%%%%%%%%%%%%%%%%%%%%%%%%%%%%%%%%%%%%%%%%%%%%5
\newpage

\begin{table}[ht!]
     \centering
     \begin{tabular}{|p{15cm}|}
     \hline        
 \noindent \textbf{Part (e)}\\
\noindent Of those investigated further, what is the proportion of individuals that will actually have the disease?

\\ \hline
 \end{tabular}
\end{table}

\item \[P(\mbox{disease}|x \geq 10)=\frac{(0.9332\times0.1)}{0.2362}=0.3951 .\]
\end{enumerate}
\end{document}
