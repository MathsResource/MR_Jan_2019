\documentclass{article}
\usepackage[utf8]{inputenc}
\usepackage{enumerate}

\author{kobriendublin }
\date{December 2018}

\begin{document}

%- Higher Certificate, Module 5, 2008. Question 2
\section{Introduction}
Probability generating function, ()()XtEtπ=.
Moment generating function, ()()tXmtEe=.
Relationship: . ()()tmteπ=

\begin{enumerate}[(i)]
\item 
()()hhttPXh()01nnhhhhntpph −=⎛⎞=−⎜⎟⎝⎠Σ
()()01nhnh (1nptp=+− (using the binomial theorem).
%%%%%%%%%%%%%%%%%%%%%%%
\item 
()()()11.1,tnddEXnpptpEXnpdtdtππ=−==+−∴ .
()()()1222222(1).11tnddEXXnnpptpdtdtππ=−−==−+−,
()()2(1)1EXXnnp∴−=−.
()()()()2211 EXnnpEXnnpn ∴=−+=−+.
()()222222Var1Xnnpnpnpnpnpnpnp ∴=−+−=−+−= np(1 – p).

Alternatively, could directly use $Var(X) = π ''(1) + E(X)(1 – E(X)).]$
%%%%%%%%%%%%%%%%%%%%%%%
\item ()()()()1333333(1)(2).121tnddEXXXnnnpptpdtdtππ=−−−==−−+−,
()()()3(1)(2)12EXXXnnnp∴−−=−−.
()()()()()323212 EXEXEXnnn ∴−+=−−,
()()()()332123132EXnnnpnnpnpnp∴=−−+−+− ()()()321231nnnpnnpnp=−−+−+.
%%%%%%%%%%%%%%%%%%%%%%%
\item 
()()()11,2,...,iinXtptpimπ=+−=
()()11imnYitptpπ=∴=+−Π(1inptpΣ=+− , which is the pgf of B(),inpΣ.
∴ by the 1-1 correspondence between pgfs and distributions, . ()B,iYnΣ∼
\end{enumerate}
\end{document}