\documentclass{article}
\usepackage[utf8]{inputenc}
\usepackage{enumerate}

\author{kobriendublin }
\date{December 2018}

\begin{document}

%- Higher Certificate, Module 5, 2008. Question 4
\section{Introduction}
\begin{enumerate}[(i)]
\item 
 ()()21/2/22(for )ixiifxex\theta π \theta−−=−
Likelihood ()(){}21/2/212inxiLe\theta\thetaπ\theta−−==Π ()2/22inxe\thetaπ\theta−−Σ=.
%%%%%%%%%%%%%%%%%%%%%%%%%%%%%%%%%%%
\item  ()()()()2loglog2log22i xnnL\thetaπ\theta\thetaΣ=−−− 22log22ixdLnd\theta\theta\thetaΣ=−+ which on setting equal to zero gives solution 2ˆixn\thetaΣ=.
To investigate whether this is a maximum, consider 2222log2i xdLnd\theta\theta\thetaΣ=−.
Inserting ˆ\theta\theta= gives 22232ˆlog0ˆˆˆ22dLnnnd\theta\theta\theta\theta\theta=−=−<.
2ˆ/ixn\theta∴=Σ maximises ()logL\theta; thus 2/iXnΣ is the maximum likelihood estimator of \theta.
%%%%%%%%%%%%%%%%%%%%%%%%%%%%%%%%
\item  ()2222log2i EXdLnEd\theta\theta\thetaΣ⎛⎞−=−+⎜⎟⎝⎠.
As the mean is 0, we have \theta = Var(X) = E(X2). 2223log22dLnnnEd\theta \theta\theta\theta\theta⎛⎞∴−=−+=⎜⎟⎝⎠
∴For large n, 22ˆN,n\theta\theta\theta⎛⎜⎝⎠∼ , approximately.
%%%%%%%%%%%%%%%%%%%%%%%%%%%
\item  1000ˆ10100\theta==.
∴ approximate 95% confidence interval is given by 2210101.96100×±
i.e. it is 101.962±, i.e. $(7.23, 12.77)$.
\end{enumerate}
\end{document}