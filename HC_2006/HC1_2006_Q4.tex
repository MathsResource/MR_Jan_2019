\documentclass[a4paper,12pt]{article}

%%%%%%%%%%%%%%%%%%%%%%%%%%%%%%%%%%%%%%%%%%%%%%%%%%%%%%%%%%%%%%%%%%%%%%%%%%%%%%%%%%%%%%%%%%%%%%%%%%%%%%%%%%%%%%%%%%%%%%%%%%%%%%%%%%%%%%%%%%%%%%%%%%%%%%%%%%%%%%%%%%%%%%%%%%%%%%%%%%%%%%%%%%%%%%%%%%%%%%%%%%%%%%%%%%%%%%%%%%%%%%%%%%%%%%%%%%%%%%%%%%%%%%%%%%%%

\usepackage{eurosym}
\usepackage{vmargin}
\usepackage{amsmath}
\usepackage{graphics}
\usepackage{epsfig}
\usepackage{enumerate}
\usepackage{multicol}
\usepackage{subfigure}
\usepackage{fancyhdr}
\usepackage{listings}
\usepackage{framed}
\usepackage{graphicx}
\usepackage{amsmath}
\usepackage{chngpage}

%\usepackage{bigints}
\usepackage{vmargin}

% left top textwidth textheight headheight

% headsep footheight footskip

\setmargins{2.0cm}{2.5cm}{16 cm}{22cm}{0.5cm}{0cm}{1cm}{1cm}

\renewcommand{\baselinestretch}{1.3}

\setcounter{MaxMatrixCols}{10}

\begin{document}
% Higher Certificate, Paper I, 2006. Question 4
\begin{table}[ht!]
     \centering
     \begin{tabular}{|p{15cm}|}
     \hline        
\noindent 4. My cycle journey to work is 3 km, and my cycling time (in minutes) if there are no delays is distributed N(15, 1), i.e. Normally with mean $\mu = 15$ and variance $\sigma^2 = 1$.

(i) Find the probability that, if there are no delays, I get to work in at most 17 minutes.

\\ \hline
      \end{tabular}
    \end{table}







\begin{enumerate}[(a)]
\item 

Let X represent cycling time without delays: $X \sim N(15, 1)$.

Compute $P(X \leq 17)$
\begin{framed}
\noindent \textbf{Z-score}
\[Z_{17} = \frac{17-15}{1}= 2.00\]
\end{framed}
\begin{eqnarray*} 
P(X \leq 17)  &=& P(Z \leq 2.00) \\
 &=& \Phi(2.00) \qquad (\mbox{Equivalently})\\
 & & (\mbox{From Statistical Tables}) \\
 &=& 0.9772
\end{eqnarray*} 

Here $\Phi$ denotes the cdf of the standard Normal distribution.

\newpage
\begin{table}[ht!]
     \centering
     \begin{tabular}{|p{15cm}|}
     \hline        
\noindent \textbf{Part (b)}\\ On my route there are three sets of traffic lights.

Each time I meet a red traffic light, I am delayed by a random time that is distributed $N(0.7, 0.09)$. 

These lights operate independently. Find the probability of my getting to work in at most 17 minutes

(1) if just one light is set at red when I reach it,

(2) if just two lights are set at red when I reach them,

(3) if all three lights are set at red when I reach them.

\\ \hline
      \end{tabular}
    \end{table}
    



\item  Adding in the delay times, also Normally distributed [N(0.7, 0.09)], and letting $T$ denote the total time:
\begin{itemize}
\item $T \sim N(15.7, 1.09)$, so

\[\frac{17 - 15.7}{\sqrt{1.09}}\]
()()1715.7171.2450.89341.09PT−⎛⎞≤=Φ=Φ=⎜⎟⎝⎠;
\item $T \sim N(16.4, 1.18)$, so



\[\frac{17 - 16.4}{\sqrt{1.18}} = \frac{0.6}{1.0863}\]

()()1716.4170.5520.70961.18PT−⎛⎞≤=Φ=Φ=⎜⎟⎝⎠;
\item $T \sim N(17.1, 1.27)$, so

\[\frac{17 - 17.1}{\sqrt{1.27}}} = \frac{-0.1}{1.127} \]
()()1717.1170.08870.46461.27PT−⎛⎞≤=Φ=Φ−=⎜⎟⎝⎠.
\end{itemize}

\newpage


\begin{table}[ht!]
     \centering
     \begin{tabular}{|p{15cm}|}
     \hline        
\noindent (iii) Suppose that, for each set of lights, the chance of delay is 0.5. Deduce that the mean value of T, my total journey time, is 16.05 minutes.


\\ \hline
      \end{tabular}
    \end{table}
    
    \begin{table}[ht!]
     \centering
     \begin{tabular}{|p{15cm}|}
     \hline        
\noindent (iv) Given that Var(T ) = 1.5025, use a suitable approximation to calculate the probability that, over 10 journeys, my average journey time to work is at most 17 minutes.
\\ \hline
      \end{tabular}
    \end{table}
\item The number of delays is distributed as B(3, ½). Hence the situations in (i), (ii)(a), (ii)(b) and (ii)(c) arise with probabilities 1/8, 3/8, 3/8 and 1/8 respectively, so the (unconditional) mean of the total journey time is
\[1331128.4()1515.716.417.116.0588888ET=×+×+×+×==\] minutes.
\item Mean time 1.502516.05,10TN⎛⎞⎜⎟⎝⎠∼.
()()1716.05172.4510.99290.15025PT−⎛⎞≤=Φ=Φ=⎜⎟⎝⎠.
\end{enumerate}
\end{document}
