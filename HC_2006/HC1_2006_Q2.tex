\documentclass[a4paper,12pt]{article}

%%%%%%%%%%%%%%%%%%%%%%%%%%%%%%%%%%%%%%%%%%%%%%%%%%%%%%%%%%%%%%%%%%%%%%%%%%%%%%%%%%%%%%%%%%%%%%%%%%%%%%%%%%%%%%%%%%%%%%%%%%%%%%%%%%%%%%%%%%%%%%%%%%%%%%%%%%%%%%%%%%%%%%%%%%%%%%%%%%%%%%%%%%%%%%%%%%%%%%%%%%%%%%%%%%%%%%%%%%%%%%%%%%%%%%%%%%%%%%%%%%%%%%%%%%%%

\usepackage{eurosym}
\usepackage{vmargin}
\usepackage{amsmath}
\usepackage{graphics}
\usepackage{epsfig}
\usepackage{enumerate}
\usepackage{multicol}
\usepackage{subfigure}
\usepackage{fancyhdr}
\usepackage{listings}
\usepackage{framed}
\usepackage{graphicx}
\usepackage{amsmath}
\usepackage{chngpage}

%\usepackage{bigints}
\usepackage{vmargin}

% left top textwidth textheight headheight

% headsep footheight footskip

\setmargins{2.0cm}{2.5cm}{16 cm}{22cm}{0.5cm}{0cm}{1cm}{1cm}

\renewcommand{\baselinestretch}{1.3}

\setcounter{MaxMatrixCols}{10}

\begin{document}
\section{Introduction}
%- Higher Certificate, Paper I, 2006. Question 2
%- HC1 2006 Question 2


\begin{enumerate}
\item 
\begin{description}

\item[A:]  
\begin{itemize}
\item[$\bullet$] No Entries
\[ P( 0 \mbox{entries}) = \left(\frac{1}{2}\right)^2  = \left(\frac{1}{4}\right) = 0.25\]
\item[$\bullet$] One Entry
\[ P( 1 \mbox{entry}) = \left( 2 \times \frac{1}{2} \times \frac{1}{2}\right)  = \left(\frac{1}{2}\right) = 0.5\]
\end{itemize}

\item[B:]
\begin{itemize}
\item[$\bullet$] No Entries
\[ P( 0 \mbox{entries}) = \left(\frac{3}{4}\right)^3  = \left(\frac{27}{64}\right) = 0.4219\]
\item[$\bullet$] One Entry
\[ P( 1 \mbox{entry}) = \left( 3 \times \frac{1}{4} \times \frac{3}{4}\right)^2\right)  = \left(\frac{27}{64}\right) = 0.4219\]
\end{itemize}

\item[C:]
\begin{itemize}
\item[$\bullet$] No Entries
\[ P( 0 \mbox{ entries}) = \left(\frac{4}{5}\right)^5  = \left(\frac{1024}{3125}\right) = 0.327\]
\item[$\bullet$] One Entry
\[ P( 1 \mbox{ entry}) = \left( 5 \times \frac{1}{5} \times \frac{4}{5}\right)^4\right)  = \left(\frac{256}{625}\right) = 0.4096s\]
\end{itemize}
\end{description}

%%%%%%%%%%%%%%%%%%%%%%%%%%%%%%%%%%
\item  P(1 entry in total)
= P(1 from A, 0 from B and C) + P(1 from B, 0 from A and C)
+ P(1 from C, 0 from A and B)
12710242711024256127459264312564431256254643125=××+××+××=.
[If worked in decimals, this is 0.1469.]

\begin{eqnarray*}
P(1 from A | 1 in total) &=& \frac{P(\mbox{1 from A and 1 in total})}{ P(1 in total)}\\
&=& P(\mbox{1 from A, 0 from B and C}) / P(1 in total) \\
&=& 271024126431254593125817××=.\\
\end{eqnarray*}
%%%%%%%%%%%%%%%%%%%%%%%%%%%%%%%%%%


\item  Denote the numbers of entries from A, B, C as (0, 0, 0) etc. Then we need P(2, 0, 0) + P(0, 2, 0) + P(0, 0, 2) + P(1, 1, 0) + P(1, 0, 1) + P(0, 1, 1). Since entries from each group are independent, we have, as an example, P(1, 1, 0) = P(1 from A).P(1 from B).P(0 from C).
\end{enumerate}

\end{document}
