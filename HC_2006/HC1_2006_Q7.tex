\documentclass[a4paper,12pt]{article}

%%%%%%%%%%%%%%%%%%%%%%%%%%%%%%%%%%%%%%%%%%%%%%%%%%%%%%%%%%%%%%%%%%%%%%%%%%%%%%%%%%%%%%%%%%%%%%%%%%%%%%%%%%%%%%%%%%%%%%%%%%%%%%%%%%%%%%%%%%%%%%%%%%%%%%%%%%%%%%%%%%%%%%%%%%%%%%%%%%%%%%%%%%%%%%%%%%%%%%%%%%%%%%%%%%%%%%%%%%%%%%%%%%%%%%%%%%%%%%%%%%%%%%%%%%%%

\usepackage{eurosym}
\usepackage{vmargin}
\usepackage{amsmath}
\usepackage{graphics}
\usepackage{epsfig}
\usepackage{enumerate}
\usepackage{multicol}
\usepackage{subfigure}
\usepackage{fancyhdr}
\usepackage{listings}
\usepackage{framed}
\usepackage{graphicx}
\usepackage{amsmath}
\usepackage{chngpage}
\usepackage{multirow}

%\usepackage{bigints}
\usepackage{vmargin}

% left top textwidth textheight headheight

% headsep footheight footskip

\setmargins{2.0cm}{2.5cm}{16 cm}{22cm}{0.5cm}{0cm}{1cm}{1cm}

\renewcommand{\baselinestretch}{1.3}

\setcounter{MaxMatrixCols}{10}

\begin{document}Higher Certificate, Paper I, 2006. Question 7
%%%%%%%%%%%%%%%%%%%%%%%%%%%%%%%%%%%%%%%%%%%%%%%%%%%%%%%%%%%%%%%%%%%%%%%%%%%%
%-------------------------------------------------%

\begin{table}[ht!]
 
\centering
 
\begin{tabular}{|p{15cm}|}
 
\hline

\noindent
7. The table below shows the joint distribution of two random variables, X and Y.
Values of Y

\begin{center}
\begin{tabular}{|l|l|l|l|l|l|}
\hline
\multicolumn{2}{|l|}{\multirow{2}{*}{}}  & \multicolumn{4}{l|}{Values of X} \\ \cline{3-6} 
\multicolumn{2}{|l|}{}   & 1  & 2  & 3  & 4 \\ \hline
\multirow{3}{*}{\begin{tabular}[c]{@{}l@{}}Values of  X\end{tabular}} & 1 & 6c & 3c & 2c & 4c\\ \cline{2-6} 
 & 2 & 2c & 4c & 4c & 0 \\ \cline{2-6} 
 & 3 & 2c & c  & 0  & 2c\\ \hline
\end{tabular}
\end{center}
Values of X

(i) Find c.

(ii) Calculate the marginal distributions of X and Y.
\\ \hline
  
\end{tabular}

\end{table}




%%%%%%%%%%%%%%%%%%%%%%%%%%%%%%%%%%%%%%%%%%%%%%%%%%%%%%%%%%%%%%%%%%%%%%%%%%%%

\begin{enumerate}
\item  The sum of all 12 table entries is $30c$. These probabilities must add up to 1, so c = 1/30.

\begin{center}
\begin{tabular}{|l|l|l|l|l|l|l|}
\hline
\multicolumn{2}{|l|}{\multirow{2}{*}{}}  & \multicolumn{5}{l|}{Values of X} \\ \cline{3-7} 
\multicolumn{2}{|l|}{}   & 1 & 2   & 3   & 4   & Total  \\ \hline
\multirow{4}{*}{\begin{tabular}[c]{@{}l@{}}Values of \\  X\end{tabular}} & 1 & 6c& 3c  & 2c  & 4c  & 15c\\ \cline{2-7} 
 & 2 & 2c& 4c  & 4c  & 0   & 10c\\ \cline{2-7} 
 & 3 & 2c& c   & 0   & 2c  & 5c \\ \cline{2-7} 
 & Total & 10c   & 8c  & 6c  & 6c  & 30c\\ \hline
\end{tabular}
\end{center}




\begin{center}
\begin{tabular}{|l|l|l|l|l|l|l|}
\hline
\multicolumn{2}{|l|}{\multirow{2}{*}{}}  & \multicolumn{5}{l|}{Values of X} \\ \cline{3-7} 
\multicolumn{2}{|l|}{}   & 1 & 2   & 3   & 4   & Total  \\ \hline
\multirow{4}{*}{\begin{tabular}[c]{@{}l@{}}Values of \\  X\end{tabular}} & 1 & 6/30& 3c  & 2/30  & 4/30  & 15/30\\ \cline{2-7} 
 & 2 & 2/30& 4/30  & 4/30  & 0   & 2/30\\ \cline{2-7} 
 & 3 & 2/30& c   & 0   & 2/30  & 5/30 \\ \cline{2-7} 
 & Total & 2/30   & 8/30  & 6/30  & 6/30  & 30c\\ \hline
\end{tabular}
\end{center}
\item  The marginal distributions are given by the row and column totals.
Hence:  
\begin{itemize}
\item $P(X = 1) = 15c = 1/2$; 
\item $P(X = 2) = 10c = 1/3$; 
\item $P(X = 3) = 5c = 1/6$.
\end{itemize}
Similarly: 
\begin{itemize}
\item $P(Y = 1) = 12/30 = 2/5$; 
\item $P(Y = 2) = 6/30 = 1/5$; 
\item $P(Y = 3) = 6/30 = 1/5$; 
\item $P(Y = 4) = 6/30 = 1/5$.
\end{itemize}
%%%%%%%%%%%%%%%%%%%%%%%%%%%%%%%%
%-------------------------------------------------%

\begin{table}[ht!]
 
\centering
 
\begin{tabular}{|p{15cm}|}
 
\hline

\noindent

(iii) Calculate E(X) and Var(X), and show that the covariance Cov(X, Y) = 0.
\\ \hline
  
\end{tabular}

\end{table}

%-------------------------------------------------%
\item 111121()1232362323EX⎛⎞⎛⎞⎛⎞=×+×+×=++=⎜⎟⎜⎟⎜⎟⎝⎠⎝⎠⎝⎠ .
2111143()1492362323EX⎛⎞⎛⎞⎛⎞=×+×+×=++=⎜⎟⎜⎟⎜⎟⎝⎠⎝⎠⎝⎠ .
21055Var()33X⎛⎞ ∴=−=⎜⎟⎝⎠.
We also need E(Y) later: 223411()55555EY=+++=.


Distribution of XY:
\begin{center}
\begin{tabular}{|c|c|c|} \hline 
Values of XY & Probability & $XY \times P(XY)$\\ \hline
1 & 6/30 & 6/30 \\ \hline 
2 & 7/30 & 14/30 \\ \hline 
3 & 4/30 &  12/30 \\ \hline 
4 & 6/30 &  24/30 \\ \hline
6 & 5/30 &  30/30 \\ \hline
12 & 2/30 &  24/30 \\ \hline
& Total & 110/30 \\ \hline 
\end{tabular}
\end{center}


6c
7c
4c
6c
5c
2c
[c = 1/30, see above]
614465211011()1234612303030303030303EXY⎛⎞⎛⎞⎛⎞⎛⎞⎛⎞⎛⎞=×+×+×+×+×+×==⎜⎟⎜⎟⎜⎟⎜⎟⎜⎟⎜⎟⎝⎠⎝⎠⎝⎠⎝⎠⎝⎠⎝⎠
Also we have 51111()()353EXEY=×=.
Cov(,)()()()0XYEXYEXEY∴=− .
%%%%%%%
%-------------------------------------------------%
\begin{center}
\begin{tabular}{|l|l|l|l|l|l|}
\hline
\multicolumn{2}{|l|}{\multirow{2}{*}{Values of XY}} & \multicolumn{4}{l|}{Values of X} \\ \cline{3-6} 
\multicolumn{2}{|l|}{}   & 1  & 2  & 3  & 4 \\ \hline
\multirow{3}{*}{\begin{tabular}[c]{@{}l@{}}Values of  X\end{tabular}} & 1 & 1 & 2   & 3   & 4 \\ \cline{2-6} 
  & 2 & 2 & 4 & 6 & 8 \\ \cline{2-6} 
 & 3 & 3 & 6  & 9  & 12\\ \hline
\end{tabular}
\end{center}



\begin{table}[ht!]
 
\centering
 
\begin{tabular}{|p{15cm}|}
 
\hline

\noindent

(iv) State, with a reason, whether or not X and Y are independent.
\\ \hline
  
\end{tabular}

\end{table}


%-------------------------------------------------%
\item  X and Y are not independent [even though Cov(X, Y) = 0 and even though some cells have 
\[P(X = x, Y = y) = P(X = x).P(Y = y)].\] 
For example, we have P(X = 1, Y = 4) = 2/15, but 
\[P(X = 1).P(Y = 4) = 1/10.\]
%%%%%%%%%%%%%

%-------------------------------------------------%

\begin{table}[ht!]
 
\centering
 
\begin{tabular}{|p{15cm}|}
 
\hline

\noindent

(v) The random variables U and V are defined by
\begin{center}
\begin{tabular}{ll}
U = 1 if X = 1 or 3,& U = 0 if X = 2,\\
V = 1 if Y = 1 or 3,& V = 0 if Y = 2 or 4.\\
\end{tabular}
\end{center}
Tabulate the joint distribution of U and V and state with a reason whether or not U and V are independent.

\\ \hline
  
\end{tabular}

\end{table}


%-------------------------------------------------%
\item  U = 1 if X = 1 or 3 U = 0 if X = 2
V = 1 if Y = 1 or 3 V = 0 if Y = 2 or 4
Table of joint distribution of U and V, with margins.
% Please add the following required packages to your document preamble:
% \usepackage{multirow}
\begin{center}
\begin{tabular}{cc|c|c|c|}
\cline{3-5}
                                                   &   & \multicolumn{2}{l|}{Values of V} & \multirow{2}{*}{Total} \\ \cline{3-4}
                                                   &   & 0               & 1              &                        \\ \hline
\multicolumn{1}{|l|}{\multirow{2}{*}{Values of U}} & 0 & 2c  =  1/15     & 8c  =  4/15    & 10c  =  1/3            \\ \cline{2-5} 
\multicolumn{1}{|l|}{}                             & 1 & 10c = 1/3       & 10c = 1/3      & 20c = 2/3              \\ \hline
\multicolumn{2}{|l|}{Total}                            & 12c = 2/5       & 18c = 3/5      & 1                      \\ \hline
\end{tabular}
\end{center}


Consider the cell with $(U, V) = (0, 0)$. The cell probability is 1/15 but the product of the marginal probabilities is 2/15. So U and V are not independent.
\end{enumerate}
\end{document}
