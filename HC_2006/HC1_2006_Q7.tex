\documentclass[a4paper,12pt]{article}

%%%%%%%%%%%%%%%%%%%%%%%%%%%%%%%%%%%%%%%%%%%%%%%%%%%%%%%%%%%%%%%%%%%%%%%%%%%%%%%%%%%%%%%%%%%%%%%%%%%%%%%%%%%%%%%%%%%%%%%%%%%%%%%%%%%%%%%%%%%%%%%%%%%%%%%%%%%%%%%%%%%%%%%%%%%%%%%%%%%%%%%%%%%%%%%%%%%%%%%%%%%%%%%%%%%%%%%%%%%%%%%%%%%%%%%%%%%%%%%%%%%%%%%%%%%%

\usepackage{eurosym}
\usepackage{vmargin}
\usepackage{amsmath}
\usepackage{graphics}
\usepackage{epsfig}
\usepackage{enumerate}
\usepackage{multicol}
\usepackage{subfigure}
\usepackage{fancyhdr}
\usepackage{listings}
\usepackage{framed}
\usepackage{graphicx}
\usepackage{amsmath}
\usepackage{chngpage}

%\usepackage{bigints}
\usepackage{vmargin}

% left top textwidth textheight headheight

% headsep footheight footskip

\setmargins{2.0cm}{2.5cm}{16 cm}{22cm}{0.5cm}{0cm}{1cm}{1cm}

\renewcommand{\baselinestretch}{1.3}

\setcounter{MaxMatrixCols}{10}

\begin{document}Higher Certificate, Paper I, 2006. Question 7
%%%%%%%%%%%%%%%%%%%%%%%%%%%%%%%%%%%%%%%%%%%%%%%%%%%%%%%%%%%%%%%%%%%%%%%%%%%%
%-------------------------------------------------%

\begin{table}[ht!]
     
\centering
     
\begin{tabular}{|p{15cm}|}
     
\hline        

\noindent
7. The table below shows the joint distribution of two random variables, X and Y.
Values of Y
1
2
3
4
1
6c
3c
2c
4c
2
4c
2c
4c
0
Values of X
3
2c
c
0
2c
(i) Find c.

(ii) Calculate the marginal distributions of X and Y.
\\ \hline
      
\end{tabular}
    
\end{table}


%-------------------------------------------------%


%%%%%%%%%%%%%%%%%%%%%%%%%%%%%%%%%%%%%%%%%%%%%%%%%%%%%%%%%%%%%%%%%%%%%%%%%%%%

\begin{enumerate}
\item  The sum of all 12 table entries is 30c. These probabilities must add up to 1, so c = 1/30.
\item  The marginal distributions are given by the row and column totals.
Hence:  
\begin{itemize}
\item $P(X = 1) = 15c = 1/2$; 
\item $P(X = 2) = 10c = 1/3$; 
\item $P(X = 3) = 5c = 1/6$.
\end{itemize}
Similarly: 
\begin{itemize}
\item $P(Y = 1) = 12c = 2/5$; 
\item $P(Y = 2) = 6c = 1/5$; 
\item $P(Y = 3) = 6c = 1/5$; 
\item $P(Y = 4) = 6c = 1/5$.
\end{itemize}
%%%%%%%%%%%%%%%%%%%%%%%%%%%%%%%%
%-------------------------------------------------%

\begin{table}[ht!]
     
\centering
     
\begin{tabular}{|p{15cm}|}
     
\hline        

\noindent

(iii) Calculate E(X) and Var(X), and show that the covariance Cov(X, Y) = 0.
\\ \hline
      
\end{tabular}
    
\end{table}


%-------------------------------------------------%
\item 111121()1232362323EX⎛⎞⎛⎞⎛⎞=×+×+×=++=⎜⎟⎜⎟⎜⎟⎝⎠⎝⎠⎝⎠ .
2111143()1492362323EX⎛⎞⎛⎞⎛⎞=×+×+×=++=⎜⎟⎜⎟⎜⎟⎝⎠⎝⎠⎝⎠ .
21055Var()33X⎛⎞ ∴=−=⎜⎟⎝⎠.
We also need E(Y) later: 223411()55555EY=+++=.
Distribution of XY:
Values of xy
1
2
3
4
6
12
Probability
6c
7c
4c
6c
5c
2c
[c = 1/30, see above]
614465211011()1234612303030303030303EXY⎛⎞⎛⎞⎛⎞⎛⎞⎛⎞⎛⎞=×+×+×+×+×+×==⎜⎟⎜⎟⎜⎟⎜⎟⎜⎟⎜⎟⎝⎠⎝⎠⎝⎠⎝⎠⎝⎠⎝⎠
Also we have 51111()()353EXEY=×=.
Cov(,)()()()0XYEXYEXEY∴=− .
%%%%%%%
%-------------------------------------------------%

\begin{table}[ht!]
     
\centering
     
\begin{tabular}{|p{15cm}|}
     
\hline        

\noindent

(iv) State, with a reason, whether or not X and Y are independent.
\\ \hline
      
\end{tabular}
    
\end{table}


%-------------------------------------------------%
\item  X and Y are not independent [even though Cov(X, Y) = 0 and even though some cells have 
\[P(X = x, Y = y) = P(X = x).P(Y = y)].\] 
For example, we have P(X = 1, Y = 4) = 2/15, but 
\[P(X = 1).P(Y = 4) = 1/10.\]
%%%%%%%%%%%%%

%-------------------------------------------------%

\begin{table}[ht!]
     
\centering
     
\begin{tabular}{|p{15cm}|}
     
\hline        

\noindent

(v) The random variables U and V are defined by
U = 1 if X = 1 or 3, U = 0 if X = 2,
V = 1 if Y = 1 or 3, V = 0 if Y = 2 or 4.
Tabulate the joint distribution of U and V and state with a reason whether or not U and V are independent.

\\ \hline
      
\end{tabular}
    
\end{table}


%-------------------------------------------------%
\item  U = 1 if X = 1 or 3 U = 0 if X = 2
V = 1 if Y = 1 or 3 V = 0 if Y = 2 or 4
Table of joint distribution of U and V, with margins.
Values of V
0
1
0
2c = 1/15
8c = 4/15
10c = 1/3
Values of U
1
10c = 1/3
10c = 1/3
20c = 2/3
12c = 2/5
18c = 3/5
Consider the cell with $(U, V) = (0, 0)$. The cell probability is 1/15 but the product of the marginal probabilities is 2/15. So U and V are not independent.
\end{enumerate}
\end{document}
