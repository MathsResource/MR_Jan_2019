\documentclass[a4paper,12pt]{article}

%%%%%%%%%%%%%%%%%%%%%%%%%%%%%%%%%%%%%%%%%%%%%%%%%%%%%%%%%%%%%%%%%%%%%%%%%%%%%%%%%%%%%%%%%%%%%%%%%%%%%%%%%%%%%%%%%%%%%%%%%%%%%%%%%%%%%%%%%%%%%%%%%%%%%%%%%%%%%%%%%%%%%%%%%%%%%%%%%%%%%%%%%%%%%%%%%%%%%%%%%%%%%%%%%%%%%%%%%%%%%%%%%%%%%%%%%%%%%%%%%%%%%%%%%%%%

\usepackage{eurosym}
\usepackage{vmargin}
\usepackage{amsmath}
\usepackage{graphics}
\usepackage{epsfig}
\usepackage{enumerate}
\usepackage{multicol}
\usepackage{subfigure}
\usepackage{fancyhdr}
\usepackage{listings}
\usepackage{framed}
\usepackage{graphicx}
\usepackage{amsmath}
\usepackage{chngpage}

%\usepackage{bigints}
\usepackage{vmargin}

% left top textwidth textheight headheight

% headsep footheight footskip

\setmargins{2.0cm}{2.5cm}{16 cm}{22cm}{0.5cm}{0cm}{1cm}{1cm}

\renewcommand{\baselinestretch}{1.3}

\setcounter{MaxMatrixCols}{10}

\begin{document}Higher Certificate, Paper II, 2006.  Question 3 
 \begin{framed}
A sample of 10 sea bass was caught by a fisheries scientist who then measured their length x (in millimetres) and their weight y (in grams).  The data are given in the table below. 

\begin{center}
\begin{tabular}{cc|c|c|c|c|c|c|c|c|c} 
Length (x) &387 & 366& 329& 293& 273& 268& 294& 198 &185 & 169 \\ 
Weight (y) & 720& 680& 480& 330& 270& 220& 380& 108  & 89 &  68 \\
\end{tabular} 
\end{center}
(i) Plot the weights of the 10 sea bass (on the y or vertical axis) against the corresponding lengths (on the x or horizontal axis).  Does it appear appropriate to fit a straight line to these data? (8) 
 
(ii) (a) Calculate the least-squares estimates of the parameters $\beta_0$ and $\beta_1$ 1 of the regression line y x 0 1 ββ += . 
 
[Note: 
 
11 10, 2762, 3345, nn ii ii n x y == = = = ∑∑
 
 
22
1 1 1 812594, 1610009, 1075861. n n n i i i i i i i x y x y = = = = = = ∑ ∑ ∑ ] 
 
 
(b) Comment on the appropriateness of the regression line estimated in part (a) as a model for the relationship between the weights and lengths of sea bass. (8) 
 

 \end{framed}
\begin{enumerate} 
\item  The graph suggests that a linear fit will be reasonable, at least as a first approximation.  There may be curvature in the relation of weight and length. 
 
 
0
100
200
300
400
500
600
700
800
0 100 200 300 400 500 Length (mm)
Weight (g)
 
 
\item  (a) 

\begin{itemize}
    \item $\hat{beta}_1 = \frac{S_{xy} }{S_{xx}$
    
    \item $ \sum(x_iy_i) - \frac{\sum(x_i) \sum(y_i)}{n} = 1075861 - \frac{2762\times 3345}{10} = 151972.0$
    
    
    \item $S_{xx} = 812594 - \frac{(2761)^2}{10} = 49729.6$
    
    
    \item $\hat{\beta}_1 = 3.056$
    
    
    \item $\hat{\beta}_0 = \bar{y} - \hat{\beta}_1 \bar{x}  = 334.5 - (3.056\times 276.2) = -509.56$
\end{itemize}
 
%%%%%%%%%%%%%%%%%%%%%%%%%%%%%%%%%%%%%%%%%% 
 
 
 (b) The first three points are not fitted at all well. 
 
 \begin{itemize}
     \item Note that the intercept is –509.56, whereas it looks as if it should be much nearer 0 if these are to be fitted.  
     \item But the remaining points are fitted reasonably well. 
     \item It would however be sensible to examine also a quadratic relationship. 
 \end{itemize}
%%%%%%%%%%%%%%%%%%%%%%%%%%%%%%%%%%%%%%%%%%%%%%%%%%%%%%%% 
\newpage 
\begin{framed}
(iii) Calculate and interpret the coefficient of determination. 
\end{framed} 
\item  The coefficient of determination is given by 
\[R2 = \frac{Sxy2}{SxxSyy.}\] 


\[  S_{yy} = 1610009 - \frac{23345^2}{10} =  491106.5 \]
 
\[ \therefore R^2 = \frac{151972.0}{49729.6 \times 491106.5} =  0.9457  \]
 
Thus 94.6\% of the total variation in the weights of the sea bass is explained by a linear relationship with their lengths. 
\end{enumerate}
\end{document}
