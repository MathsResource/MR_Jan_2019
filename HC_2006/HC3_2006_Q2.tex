\documentclass[a4paper,12pt]{article}

%%%%%%%%%%%%%%%%%%%%%%%%%%%%%%%%%%%%%%%%%%%%%%%%%%%%%%%%%%%%%%%%%%%%%%%%%%%%%%%%%%%%%%%%%%%%%%%%%%%%%%%%%%%%%%%%%%%%%%%%%%%%%%%%%%%%%%%%%%%%%%%%%%%%%%%%%%%%%%%%%%%%%%%%%%%%%%%%%%%%%%%%%%%%%%%%%%%%%%%%%%%%%%%%%%%%%%%%%%%%%%%%%%%%%%%%%%%%%%%%%%%%%%%%%%%%

\usepackage{eurosym}
\usepackage{vmargin}
\usepackage{amsmath}
\usepackage{graphics}
\usepackage{epsfig}
\usepackage{enumerate}
\usepackage{multicol}
\usepackage{subfigure}
\usepackage{fancyhdr}
\usepackage{listings}
\usepackage{framed}
\usepackage{graphicx}
\usepackage{amsmath}
\usepackage{chngpage}

%\usepackage{bigints}
\usepackage{vmargin}

% left top textwidth textheight headheight

% headsep footheight footskip

\setmargins{2.0cm}{2.5cm}{16 cm}{22cm}{0.5cm}{0cm}{1cm}{1cm}

\renewcommand{\baselinestretch}{1.3}

\setcounter{MaxMatrixCols}{10}

\begin{document}Higher Certificate, Paper III, 2006. Question 2
(i) The analysis of variance table is as follows. Entries in italics are given in the question. The others need to be calculated.
SOURCE
DF
SS
MS
F value
Percentages
4
262.64
65.66
9.25 Compare F4,20
Residual
20
142.00
7.10
= 2ˆσ
TOTAL
24
404.64
Upper critical points of F4,20 are as follows:
5%
1%
0.1%
2.87
4.43
7.10
The F value for percentages is very highly significant; we have very strong evidence that not all the percentages of cotton are the same in terms of mean tensile strength of the synthetic fibre.
(ii) The fitted values are simply the sample means for the different percentages of cotton: (15%) 9.8; (20%) 16.6; (25%) 18.0; (30%) 15.4; (35%) 10.8. The graph suggests that where the mean is lower the variance tends to be slightly higher. The analysis is based on a model in which the residual term has constant variance, but apparent departures from this are not great and there is no reason to doubt the results seriously.
Scatterplot of standardised residuals against fitted values
Standardised residuals
Note. Some points on this graph represent two coincident values.
Note also the "false origin" on the fitted values axis.
Solution continued on next page
(iii) A 95% confidence interval for an individual mean is given by ˆ2.0865/xσ± where 2.086 is the double-tailed 5% point of t20.
Further, , the residual mean square in the above analysis of variance. Thus the interval for 20% cotton is given by 2ˆ7.10σ=16.62.0867.105/±, i.e. it is (14.1, 19.1).
Similarly, for 25% the interval is (15.5, 20.5) and for 30% it is (12.9, 17.9).
30%
25%
20%
20.0
16.0
12.0
Strength
On this evidence, 25% cotton should be used.
(iv) It is worth exploring just below 25%, and perhaps just above. Depending on how much work can be done, it may be possible to search for a maximum in this region.
 \end{enumerate}
 \end{document}
 
