\documentclass[a4paper,12pt]{article}

%%%%%%%%%%%%%%%%%%%%%%%%%%%%%%%%%%%%%%%%%%%%%%%%%%%%%%%%%%%%%%%%%%%%%%%%%%%%%%%%%%%%%%%%%%%%%%%%%%%%%%%%%%%%%%%%%%%%%%%%%%%%%%%%%%%%%%%%%%%%%%%%%%%%%%%%%%%%%%%%%%%%%%%%%%%%%%%%%%%%%%%%%%%%%%%%%%%%%%%%%%%%%%%%%%%%%%%%%%%%%%%%%%%%%%%%%%%%%%%%%%%%%%%%%%%%

\usepackage{eurosym}
\usepackage{vmargin}
\usepackage{amsmath}
\usepackage{graphics}
\usepackage{epsfig}
\usepackage{enumerate}
\usepackage{multicol}
\usepackage{subfigure}
\usepackage{fancyhdr}
\usepackage{listings}
\usepackage{framed}
\usepackage{graphicx}
\usepackage{amsmath}
\usepackage{chngpage}

%\usepackage{bigints}
\usepackage{vmargin}

% left top textwidth textheight headheight

% headsep footheight footskip

\setmargins{2.0cm}{2.5cm}{16 cm}{22cm}{0.5cm}{0cm}{1cm}{1cm}

\renewcommand{\baselinestretch}{1.3}

\setcounter{MaxMatrixCols}{10}

\begin{document}Higher Certificate, Paper II, 2006.  Question 1 
\begin{framed}
\large
%%1. 
The duration of a "normal" pregnancy (i.e. one without medical complications) can be modelled by a Normal distribution with mean 266 days and standard deviation 16 days.  In an urban hospital in a relatively deprived area of the USA, a random sample of 60 pregnancies was studied and the duration of each pregnancy determined.  The relevant statistics were 
 


\begin{itemize}
    \item $\sum x = 15,568$
    \item $\sum x^2 = 4,054,484$
\end{itemize}
 
(i) Test whether the variance of this sampled population differs from that of "normal" pregnancy durations. 
%%-- (10) 
 

\end{framed}

%%%%%%%%%%%%%%%%%%%%%%%%%%%%%%%%%%%%%


 \begin{enumerate}
\item The variance of this sample is 

\[ s^2 = \frac{1}{59}\left( 4054484 - \frac{(15568)^2}{60} \right) = \frac{15106.93}{59} =256.05\]
 
%%%%%%%%%%%%%%%%%%%%%%
 
The null hypothesis to be tested is "$\sigma^2= 256$".  It seems obvious that this null hypothesis is not likely to be rejected (even if the sample had been of considerably smaller size), but continuing with a formal test we use test statistic 
 
\[  \frac{(n-1)s^2}{\sigma^2} = \frac{59 \times 256.05}{256} =59.01 \]
 
 
which is referred to $\chi^2_{(59)}$.  The upper 5\% point is about 78.  Clearly we cannot reject the null hypothesis as the data give no evidence for doing so. 
%%%%%%%%%%%%%%%%%%%%%%%%%%%%%%%%%%%%%%%%%%%%%%%%%%%%%%%%%%%%%%%%%%%%%%%%%%%%%%
\newpage

\begin{framed}
\large
%%-- (ii) 
Test whether there is evidence that the mean pregnancy duration for this sampled population differs from 266 days.  Find an approximate p-value for your test, and state your conclusions clearly. 
\end{framed} 

\large
\item We have  $ \displaystyle {\bar{x} = \frac{15568}{60} = 259.46 }$  and we wish the test the null hypothesis $\mu=266$.
\begin{itemize}
    \item Taking the value of $\sigma$ as 16, which seems highly plausible from part (i), we use test statistic 
\[  \frac{\bar{x} - 266}{16 / \sqrt{60} }\]
 
and refer to N(0, 1). 
   \item Alternatively, we could continue to use the sample variance $s^2 (= 256.05)$ and refer 266 60 x s − to t59;  this makes hardly any difference in practice in this case. 
 
   \item This is well beyond the double-tailed 1\% point of $N(0, 1)$;  there is strong evidence against this null hypothesis.  
   
      \item It is reasonable to conclude that this population has a mean different from the "normal" one;  it appears to be less. 
 
   \item Using $N(0, 1)$, we have , giving a p-value of 0.0016. 
\end{itemize}
\end{enumerate}
\end{document}
