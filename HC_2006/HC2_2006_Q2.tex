\documentclass[a4paper,12pt]{article}

%%%%%%%%%%%%%%%%%%%%%%%%%%%%%%%%%%%%%%%%%%%%%%%%%%%%%%%%%%%%%%%%%%%%%%%%%%%%%%%%%%%%%%%%%%%%%%%%%%%%%%%%%%%%%%%%%%%%%%%%%%%%%%%%%%%%%%%%%%%%%%%%%%%%%%%%%%%%%%%%%%%%%%%%%%%%%%%%%%%%%%%%%%%%%%%%%%%%%%%%%%%%%%%%%%%%%%%%%%%%%%%%%%%%%%%%%%%%%%%%%%%%%%%%%%%%

\usepackage{eurosym}
\usepackage{vmargin}
\usepackage{amsmath}
\usepackage{graphics}
\usepackage{epsfig}
\usepackage{enumerate}
\usepackage{multicol}
\usepackage{subfigure}
\usepackage{fancyhdr}
\usepackage{listings}
\usepackage{framed}
\usepackage{graphicx}
\usepackage{amsmath}
\usepackage{chngpage}

%\usepackage{bigints}
\usepackage{vmargin}

% left top textwidth textheight headheight

% headsep footheight footskip

\setmargins{2.0cm}{2.5cm}{16 cm}{22cm}{0.5cm}{0cm}{1cm}{1cm}

\renewcommand{\baselinestretch}{1.3}

\setcounter{MaxMatrixCols}{10}

\begin{document}Higher Certificate, Paper II, 2006.  Question 2 
\begin{framed}

2. A randomised experiment was carried out to test how well Entonox, a gas made up of 50\% oxygen and 50\% nitrous oxide, performed in reducing pain during a minor surgical procedure.  Twenty patients were randomised to receive air (10 patients) or Entonox (10 patients).  Immediately after the procedure, the patients used a 0 – 100 scale to record the level of pain they suffered during the procedure.  Their scores are given in the table below. 
 
Air Entonox 37   9 33   0 50 14   8 47 16 15 38 17 55 10 28 22 31   3 39 80 
 
(i) In situations such as this when comparing two independent samples of subjective scores, briefly discuss the advantages and disadvantages of using non-parametric rather than parametric tests. 

\end{framed}

\begin{enumerate} 
\item Parametric tests need assumptions about the distribution underlying the data  –  often that it is Normal (if the situation is continuous).  
\begin{itemize}
    \item Data in the form of subjective scores are unlikely to follow any of the common distributions, and two sets of independent data may not even be of the same shape, location, scatter or skewness. 
    \item Non-parametric tests allow simple characteristics of distributions to be compared with few or no theoretical assumptions.  
    \item However, they have less power than corresponding parametric tests in cases where the parametric test is in fact valid.  
    \item They therefore need larger sample sizes. 
\end{itemize}

%%%%%%%%%%%%%%%%%%%%%%%%%%%%%%%%%%%%%%%%%%%%%%%%%%%%%5
\newpage
\begin{framed}
 (ii) Use an appropriate non-parametric test to determine whether there is evidence that pain scores differ between patients given Entonox and patients given air during this minor surgical procedure.  Explain your conclusions in a manner appropriate for a doctor to understand, including a comment on what implications the size of the samples may have for your conclusion. (14) 
2 
Turn over 
\end{framed}
 
\item  The Wilcoxon rank sum test (or, equivalently, the Mann Whitney U form of this test) is suitable for this comparison.  First rank all 20 data items, as follows. 
\begin{center}
\begin{tabular}{|c||c|c|c|c|c|c|c|c|c|c|}\hline
Data & 0 & 3 & 8 & 9 & 10&  14&  15&  16&  17&  22\\ \hline

Ranks & 1 & 2 & 3 & 4 & 5 & 6 & 7 & 8 & 9 & 10\\ \hline
Group &  E&  E&  A&  E & E&  E&  E&  A&  E&  E\\ \hline \hline
 
Data & 28&  31&  33&  37&  38 & 39&  47&  50&  55 & 80\\ \hline
Ranks & 11&  12&  13&  14&  15&  16&  17 & 18 & 19&  20\\ \hline 
Group& A& A& A& A&  A &A &E &A &A &E\\ \hline
\end{tabular}
 
 \end{center}
The rank sum for Entonox (E) is 1 + 2 + 4 + 5 + 6 + 7 + 9 + 10 + 17 + 20 = 81.  That for A is 129. 
 
\begin{itemize} 
\item  The required test is two-sided.  
\item For a 5\% test, we refer the smaller of these (81) to the lower $2\frac{1}{2}\%$ point for the $W(10,10)$ distribution as shown in the Society's statistical tables for use in examinations. 
\item This is 78 so, at the 5\% level of significance, we cannot reject the null hypothesis that pain scores do not differ.  However, we note that the result is (just) significant at the 10\% level (the lower 5\% point is 82), and the sample sizes are quite small.
\item So, overall, we do not really have sufficient evidence to say whether or not there is an advantage for Entonox. 
\item A more powerful test should be conducted using larger samples before coming to a firm decision. 
\end{itemize}
The data do appear to need a non-parametric testing procedure. 
 \end{enumerate}
 \end{document}
 
