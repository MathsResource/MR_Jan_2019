\documentclass[a4paper,12pt]{article}

%%%%%%%%%%%%%%%%%%%%%%%%%%%%%%%%%%%%%%%%%%%%%%%%%%%%%%%%%%%%%%%%%%%%%%%%%%%%%%%%%%%%%%%%%%%%%%%%%%%%%%%%%%%%%%%%%%%%%%%%%%%%%%%%%%%%%%%%%%%%%%%%%%%%%%%%%%%%%%%%%%%%%%%%%%%%%%%%%%%%%%%%%%%%%%%%%%%%%%%%%%%%%%%%%%%%%%%%%%%%%%%%%%%%%%%%%%%%%%%%%%%%%%%%%%%%

\usepackage{eurosym}
\usepackage{vmargin}
\usepackage{amsmath}
\usepackage{graphics}
\usepackage{epsfig}
\usepackage{enumerate}
\usepackage{multicol}
\usepackage{subfigure}
\usepackage{fancyhdr}
\usepackage{listings}
\usepackage{framed}
\usepackage{graphicx}
\usepackage{amsmath}
\usepackage{chngpage}

%\usepackage{bigints}
\usepackage{vmargin}

% left top textwidth textheight headheight

% headsep footheight footskip

\setmargins{2.0cm}{2.5cm}{16 cm}{22cm}{0.5cm}{0cm}{1cm}{1cm}

\renewcommand{\baselinestretch}{1.3}

\setcounter{MaxMatrixCols}{10}

\begin{document}
Higher Certificate, Paper I, 2006. Question 6
(),2xfxexλλ−=−∞<<∞
0.00.00xf(x)lambda/2

\begin{itemize}
    \item By symmetry, E(X) = 0.
Hence 22Var()()02xXEXxedxλλ∞−−∞=−=∫ {}02202xxxedxxedxλλλ∞−−∞=+∫∫.
\item Substituting u = –x in the first integral gives , which is the same as the second.
\item Hence we get, integrating by parts, 20uueduλ∞−∫
220()xEXxeλλ∞−=∫
200.2xxeexxdxλλλλλ∞−−∞⎧⎫⎡⎤⎪⎪=+⎨⎬⎢⎥−⎣⎦⎪⎪⎩⎭∫
0[00]2xxedxλ∞−=−+∫
002xxeexdxλλλλ∞−−∞⎧⎫⎡⎤⎪⎪=+⎨⎬⎢⎥−⎣⎦⎪⎪⎩⎭∫
[]0200xeλλλ∞−⎡⎤=−+⎢⎥−⎣⎦ 22λ=.

\item If Q, q are the upper and lower quartiles, we have 0124Qxedxλλ− =∫, and q will be the same distance below 0 by symmetry.
(01111422Qxeeλλ−−⎡⎤ ∴=−=−+⎢⎥⎣⎦, giving 112Qeλ−=−. \item Therefore λQ = log 2.
\item Hence the semi-interquartile range is (log 2)/λ.
\end{itemize}

122iinnxxiLeeλλλλ−−=⎛⎞⎛⎞Σ==⎜⎟⎜⎟⎝⎠⎝⎠Π, and hence log constant logiiLnλλ=+−Σ .
logiidLnxdλλ∴=−Σ which on setting equal to zero gives that the maximum likelihood estimate is ˆiinxλ=Σ. [Consideration of 22logddλ confirms that this is a maximum: 222log0dLndλλ−=<.]
\end{document}
