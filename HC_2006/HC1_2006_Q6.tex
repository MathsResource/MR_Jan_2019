\documentclass[a4paper,12pt]{article}

%%%%%%%%%%%%%%%%%%%%%%%%%%%%%%%%%%%%%%%%%%%%%%%%%%%%%%%%%%%%%%%%%%%%%%%%%%%%%%%%%%%%%%%%%%%%%%%%%%%%%%%%%%%%%%%%%%%%%%%%%%%%%%%%%%%%%%%%%%%%%%%%%%%%%%%%%%%%%%%%%%%%%%%%%%%%%%%%%%%%%%%%%%%%%%%%%%%%%%%%%%%%%%%%%%%%%%%%%%%%%%%%%%%%%%%%%%%%%%%%%%%%%%%%%%%%

\usepackage{eurosym}
\usepackage{vmargin}
\usepackage{amsmath}
\usepackage{graphics}
\usepackage{epsfig}
\usepackage{enumerate}
\usepackage{multicol}
\usepackage{subfigure}
\usepackage{fancyhdr}
\usepackage{listings}
\usepackage{framed}
\usepackage{graphicx}
\usepackage{amsmath}
\usepackage{chngpage}

%\usepackage{bigints}
\usepackage{vmargin}

% left top textwidth textheight headheight

% headsep footheight footskip

\setmargins{2.0cm}{2.5cm}{16 cm}{22cm}{0.5cm}{0cm}{1cm}{1cm}

\renewcommand{\baselinestretch}{1.3}

\setcounter{MaxMatrixCols}{10}

\begin{document}
Higher Certificate, Paper I, 2006. Question 6
%%%%%%%%%%%%%%%%%%%%%%%%%%%%%%%%%%%%%%%%%%%%%%%%%%%%%%%%%%%%%%%%%%%%%%%%%%%%
%-------------------------------------------------%

\begin{table}[ht!]
     
\centering
     
\begin{tabular}{|p{15cm}|}
     
\hline        

\noindent
The random variable X has the distribution with probability density function
\[ f(x) =  \frac{\lambda}{2}  e^{-\lambda |x|},  \mbox{ where } –\infty < x < \infty .\]
Sketch a graph of this density function.
\\ \hline
      
\end{tabular}
    
\end{table}


%-------------------------------------------------%

%%%%%%%%%%%%%%%%%%%%%%%%%%%%%%%%%%%%%%%%%%%%%%%%%%%%%%%%%%%%%%%%%%%%%%%%%%%%
(),2xfxex$\lambda$$\lambda$−=−∞<<∞
0.00.00xf(
x)lambda/2

\newpage

%-------------------------------------------------%

\begin{table}[ht!]
     
\centering
     
\begin{tabular}{|p{15cm}|}
     
\hline        

\noindent

Write down $E(X)$ and show that $\operatorname{Var}(X) = \frac{2}{\lambda^2}$. Find also the semi-interquartile range of $X$.
\\ \hline
      
\end{tabular}
    
\end{table}


%-------------------------------------------------%
\begin{itemize}
    \item By symmetry, $E(X) = 0$.
Hence 
\begin{eqnarray}
E(X^2)  &=& \int^{\infty}_{\infty} x^2 \;\frac{\lambda}{2}\;  e^{-\lambda |x|} dx \\
&=& 2 \times  \int^{\infty}_{0} x^2 \;\frac{\lambda}{2}\;  e^{- \lambda x} dx \\ 
&=& \lambda \times  \int^{\infty}_{0} x^2 \;  e^{- \lambda x} dx \\ 
&  & \mbox{(Integration by Parts)}
\end{eqnarray}

\begin{framed}
\[I = uv - \int vdu\]
\begin{itemize}
    \item $u = x^2 $
    \item $dv = e^{- \lambda x}\;dx$ $\therefore$ $v = \int dv = \int  dxe^{- \lambda x} dx$
\end{itemize}
\end{framed}
\begin{eqnarray}
\operatorname{Var}(X) &=& E(X^2) - [0]^2 \\
&=&
22Var()()02xXEXxedx$\lambda$$\lambda$∞−−∞=−=∫ {}02202xxxedxxedx$\lambda$$\lambda$$\lambda$∞−−∞=+∫∫.
\end{eqnarray}


\item Substituting $u = –x$ in the first integral gives , which is the same as the second.
\item Hence we get, integrating by parts, 

\begin{eqnarray*}
\int^{\infty}_0 e^{\lambda} dx  &=&  \int^{\infty} e^{\lambda} dx        \\
  &=&  \lambda {e^{\lambda}      \\
  &=&  \int^{\infty}  dx        \\
\end{eqnarray*}

\[
200.2xxeexxdx$\lambda$$\lambda$$\lambda$$\lambda$$\lambda$∞−−∞⎧⎫⎡⎤⎪⎪=+⎨⎬⎢⎥−⎣⎦⎪⎪⎩⎭\]
\[0[00]2xxedx$\lambda$∞−=−+∫
002xxeexdx$\lambda$$\lambda

\begin{eqnarray*}
\int^{\infty}_0 e^{\lambda} dx  &=&  \int^{\infty} e^{\lambda} dx        \\
  &=&  \lambda {e^{\lambda}      \\
  &=&  \int^{\infty}  dx        \\
\end{eqnarray*}
 
%\lambda\lambda$∞−−∞⎧⎫⎡⎤⎪⎪=+⎨⎬⎢⎥−⎣⎦⎪⎪⎩⎭∫
% []0200xe$\lambda$$\lambda$$$∞−⎡⎤=−+⎢⎥−⎣⎦ 22$\lambda$=.\]

\item If Q, q are the upper and lower quartiles, we have 0124Qxedx$\lambda$$\lambda$− =∫, and q will be the same distance below 0 by symmetry.
\[(01111422Qxee$\lambda\lambda$−−⎡⎤ ∴=−=−+⎢⎥⎣⎦, giving 112Qe$\lambda$−=\]

−. 



%-------------------------------------------------%

\begin{table}[ht!]
     
\centering
     
\begin{tabular}{|p{15cm}|}
     
\hline        

\noindent

A random sample $x_1, x_2, \ldots, x_n$ is taken from this distribution. Show that the maximum likelihood estimate of $\lambda$ is given by
\[ \hat{\lambda}  = \frac{n}{\dislaystyle{\sum^{n}_{i=1} |x_i| }} \].
\\ \hline
      
\end{tabular}
    
\end{table}


%-------------------------------------------------%

\item Therefore $\lambda$Q = log 2.
\item Hence the semi-interquartile range is (log 2)/$\lambda$.
\end{itemize}

122iinnxxiLee$\lambda$$\lambda$$\lambda$$\lambda$−−=⎛⎞⎛⎞Σ==⎜⎟⎜⎟⎝⎠⎝⎠Π, and hence log constant logiiLn$\lambda$$\lambda$=+−Σ .
logiidLnxd$\lambda$$\lambda$∴=−Σ which on setting equal to zero gives that the maximum likelihood estimate is ˆiinx$\lambda$=Σ. [Consideration of 22logdd$\lambda$ confirms that this is a maximum: 222log0dLnd$\lambda$$\lambda$−=<.]
\end{document}
