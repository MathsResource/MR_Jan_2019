Higher Certificate, Paper I, 2006. Question 6
(),2xfxexλλ−=−∞<<∞
0.00.00xf(x)lambda/2
By symmetry, E(X) = 0.
Hence 22Var()()02xXEXxedxλλ∞−−∞=−=∫ {}02202xxxedxxedxλλλ∞−−∞=+∫∫.
Substituting u = –x in the first integral gives , which is the same as the second. Hence we get, integrating by parts, 20uueduλ∞−∫
220()xEXxeλλ∞−=∫
200.2xxeexxdxλλλλλ∞−−∞⎧⎫⎡⎤⎪⎪=+⎨⎬⎢⎥−⎣⎦⎪⎪⎩⎭∫
0[00]2xxedxλ∞−=−+∫
002xxeexdxλλλλ∞−−∞⎧⎫⎡⎤⎪⎪=+⎨⎬⎢⎥−⎣⎦⎪⎪⎩⎭∫
[]0200xeλλλ∞−⎡⎤=−+⎢⎥−⎣⎦ 22λ=.
Solution continued on next page
If Q, q are the upper and lower quartiles, we have 0124Qxedxλλ− =∫, and q will be the same distance below 0 by symmetry.
(01111422Qxeeλλ−−⎡⎤ ∴=−=−+⎢⎥⎣⎦, giving 112Qeλ−=−. Therefore λQ = log 2. Hence the semi-interquartile range is (log 2)/λ.
122iinnxxiLeeλλλλ−−=⎛⎞⎛⎞Σ==⎜⎟⎜⎟⎝⎠⎝⎠Π, and hence log constant logiiLnλλ=+−Σ .
logiidLnxdλλ∴=−Σ which on setting equal to zero gives that the maximum likelihood estimate is ˆiinxλ=Σ. [Consideration of 22logddλ confirms that this is a maximum: 222log0dLndλλ−=<.]
