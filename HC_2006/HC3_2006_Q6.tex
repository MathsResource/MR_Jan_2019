\documentclass[a4paper,12pt]{article}

%%%%%%%%%%%%%%%%%%%%%%%%%%%%%%%%%%%%%%%%%%%%%%%%%%%%%%%%%%%%%%%%%%%%%%%%%%%%%%%%%%%%%%%%%%%%%%%%%%%%%%%%%%%%%%%%%%%%%%%%%%%%%%%%%%%%%%%%%%%%%%%%%%%%%%%%%%%%%%%%%%%%%%%%%%%%%%%%%%%%%%%%%%%%%%%%%%%%%%%%%%%%%%%%%%%%%%%%%%%%%%%%%%%%%%%%%%%%%%%%%%%%%%%%%%%%

\usepackage{eurosym}
\usepackage{vmargin}
\usepackage{amsmath}
\usepackage{graphics}
\usepackage{epsfig}
\usepackage{enumerate}
\usepackage{multicol}
\usepackage{subfigure}
\usepackage{fancyhdr}
\usepackage{listings}
\usepackage{framed}
\usepackage{graphicx}
\usepackage{amsmath}
\usepackage{chngpage}

%\usepackage{bigints}
\usepackage{vmargin}

% left top textwidth textheight headheight

% headsep footheight footskip

\setmargins{2.0cm}{2.5cm}{16 cm}{22cm}{0.5cm}{0cm}{1cm}{1cm}

\renewcommand{\baselinestretch}{1.3}

\setcounter{MaxMatrixCols}{10}

\begin{document}
Higher Certificate, Paper III, 2006. Question 6
(i)
B
C
A
100
90
80
LOW
HIGH
Protein
Interaction is when factors (here, the level and type of protein) do not appear to function independently. Here all the types – A, B, C – give an increase in mean weight gain with increasing level (although the behaviour of B is rather different from A and C). There is unlikely to be a large interaction – if there is any.
(ii) The analysis of variance table is as follows. Entries in italics are given in the question. The others need to be calculated.
SOURCE
DF
SS
MS
F value
Level
1
3776.3
3776.30
17.60 Compare F1,54
Type
2
82.5
41.25
0.19 Compare F2,54
Level * Type (Interaction)
2
730.1
365.05
1.70 Compare F2,54
Error (Residual)
54
11586.0
214.56
= 2ˆσ
TOTAL
59
16174.9
Solution continued on next page
Upper critical points of F1,50 and F2,50 are taken from the Society's statistical tables for use in examinations. Values for (1, 54) and (2, 54) will be very similar.
5%
1%
0.1%
F1,50
4.03
7.17
12.22
F2,50
3.18
5.06
7.96
The F value for level is very highly significant; we have very strong evidence that the two levels of protein do not result in the same overall mean weight gain.
The F value for type is insignificant. We have no evidence to suggest that the three protein types are different in terms of the overall mean weight gain.
Similarly, we have no evidence that there is any interaction, i.e. that any protein type behaves differently as level is increased (even though the B responses are somewhat different from those of A and C). The graph in part (i) shows the pattern, and the analysis here confirms which are the significant sources of variation.
(iii) The 5 degrees of freedom for factors and interaction explain only 28.4% of the total variation (SS total). This is uncomfortably small.
We note also that the estimate of experimental error is 2ˆσ = 214.56 (ˆσ = 14.65), which is quite large compared with the values of the observations themselves (of order 100).
Perhaps it is simply the case that the weight gains are naturally very variable; or perhaps they are influenced by other covariates (e.g. initial weight).
The dependence of weight gain on protein level appears strong and is intuitively appealing. But there may be more to learn about the response variable.
 \end{enumerate}
 \end{document}
 
