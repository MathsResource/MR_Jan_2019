Higher Certificate, Paper I, 2006. Question 5
(i) E(X) = ()101!1!xxxxexeeexxλλλλλ λλλ−−∞∞−−====−ΣΣ .
E(X2) = ()()[]11EXXXEXXEX−+=−+⎡⎤⎡⎤⎣⎦⎣⎦.
()()()2220211!2!xxxxeEXXxxeeexxλλλλλ λλλ−−∞∞−−==−=−===⎡⎤⎣⎦−ΣΣ.
Hence 22()EXλλ=+, and {}22Var()()()XEXEXλ=−=.
(ii) 1!ixniieLxλλ−==Π, and hence . 1loglogconstantniiLnxλλ==−++Σ
logixdLndλλΣ∴=−+ which on setting equal to zero gives that the maximum likelihood estimate is ˆixxnλΣ==. [Consideration of 22logddλ confirms that this is a maximum: 222log0ixdLdλλ−Σ=<.]
(iii) ()()Var()ˆVarVarXXnnλλ== .
Thus the maximum likelihood estimator of Var(ˆ
λ) is ˆnλ.
By the central limit theorem, ˆ( Xλ= is approximately Normally distributed with mean λ and variance λ /n. We estimate the variance by ˆ
λ/n, so that we have ˆˆ~N,nλλλ⎛⎞⎜⎜⎝⎠, approximately.
Hence an approximate 95% confidence interval is given by
ˆ0.951.961.96ˆ/Pnλλλ⎛⎞−≈−<<⎜⎟⎜⎟⎝⎠,
leading to the interval ˆˆˆˆ1.96,1.96nnλλλλ⎛⎞⎜⎟−+⎜⎟⎝⎠.
Solution continued on next page
(iv) For the given sample, we have n = 12 and Σxi = 48, leading to ˆ4xλ==. The approximate confidence interval is therefore
441.96to41.961212⎛⎞−+⎜⎟⎜⎟⎝⎠, i.e. 2.87 to 5.13.
The sample also gives Σxi2 = 238; so the sample variance is
s2 = 2148462384.182111211⎛⎞−==⎜⎟⎝⎠.
This is close to the sample mean (4), supporting a Poisson hypothesis for the underlying model.
