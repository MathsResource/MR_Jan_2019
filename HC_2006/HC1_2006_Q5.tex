\documentclass[a4paper,12pt]{article}

%%%%%%%%%%%%%%%%%%%%%%%%%%%%%%%%%%%%%%%%%%%%%%%%%%%%%%%%%%%%%%%%%%%%%%%%%%%%%%%%%%%%%%%%%%%%%%%%%%%%%%%%%%%%%%%%%%%%%%%%%%%%%%%%%%%%%%%%%%%%%%%%%%%%%%%%%%%%%%%%%%%%%%%%%%%%%%%%%%%%%%%%%%%%%%%%%%%%%%%%%%%%%%%%%%%%%%%%%%%%%%%%%%%%%%%%%%%%%%%%%%%%%%%%%%%%

\usepackage{eurosym}
\usepackage{vmargin}
\usepackage{amsmath}
\usepackage{graphics}
\usepackage{epsfig}
\usepackage{enumerate}
\usepackage{multicol}
\usepackage{subfigure}
\usepackage{fancyhdr}
\usepackage{listings}
\usepackage{framed}
\usepackage{graphicx}
\usepackage{amsmath}
\usepackage{chngpage}

%\usepackage{bigints}
\usepackage{vmargin}

% left top textwidth textheight headheight

% headsep footheight footskip

\setmargins{2.0cm}{2.5cm}{16 cm}{22cm}{0.5cm}{0cm}{1cm}{1cm}

\renewcommand{\baselinestretch}{1.3}

\setcounter{MaxMatrixCols}{10}

\begin{document}
Higher Certificate, Paper I, 2006. Question 5
%%%%%%%%%%%%%%%%%%%%%%%%%%%%%%%%%%%%%%%%%%%%%%%%%%%%%%%%%%%%%%%%%%%%%%%%%%%%
%-------------------------------------------------%

\begin{table}[ht!]
     
\centering
     
\begin{tabular}{|p{15cm}|}
     
\hline        

\noindent
5. The random variable X follows a Poisson distribution with probability mass function
(),0,1,2,...!xPXxexxλλ−=== .
(i) Show that E(X) = Var(X) = λ.
\\ \hline
      
\end{tabular}
    
\end{table}


%-------------------------------------------------%
%-------------------------------------------------%

\begin{table}[ht!]
     
\centering
     
\begin{tabular}{|p{15cm}|}
     
\hline        

\noindent

(ii) Given a random sample of values x1, x2, …, xn from this distribution, obtain the maximum likelihood estimator (MLE) of λ, say. λˆ

(iii) Write down Var() as a function of λ, and hence find the MLE of Var(). Show that a large-sample approximate
95% confidence interval (CI) for λ is given by λˆλˆˆˆˆˆ 1.961.96nnλλλλλ−≤≤+.
\\ \hline
      
\end{tabular}
    
\end{table}


%-------------------------------------------------%
%-------------------------------------------------%

\begin{table}[ht!]
     
\centering
     
\begin{tabular}{|p{15cm}|}
     
\hline        

\noindent

(iv) Assume that the numbers of books, x1, x2, x3, …, that go missing each month from the local library follow a Poisson 
distribution with unknown mean λ. The monthly numbers of missing books in 2005 were
3 7 2 5 8 2 4 5 4 4 1 3.
Use these data to calculate and an approximate 95% CI for λ. Also compute the sample variance of the data; discuss 
briefly whether this computation supports, or throws doubt on, the Poisson model suggested (no formal test is required). λˆ

\\ \hline
      
\end{tabular}
    
\end{table}


%-------------------------------------------------%
%%%%%%%%%%%%%%%%%%%%%%%%%%%%%%%%%%%%%%%%%%%%%%%%%%%%%%%%%%%%%%%%%%%%%%%%%%%%

\begin{enumerate}
\item E(X) = ()101!1!xxxxexeeexx$\lambda$$\lambda$$\lambda$$\lambda$$\lambda$ $\lambda$$\lambda$$\lambda$−−∞∞−−====−ΣΣ .
E(X2) = ()()[]11EXXXEXXEX−+=−+⎡⎤⎡⎤⎣⎦⎣⎦.
()()()2220211!2!xxxxeEXXxxeeexx$\lambda$$\lambda$$\lambda$$\lambda$$\lambda$ $\lambda$$\lambda$$\lambda$−−∞∞−−==−=−===⎡⎤⎣⎦−ΣΣ.
Hence 22()EX$\lambda$$\lambda$=+, and {}22Var()()()XEXEX$\lambda$=−=.
\item  \[1!ixniieLx$\lambda$$\lambda$−==Π, \]and hence . 1loglogconstantniiLnx$\lambda$$\lambda$==−++Σ
logixdLnd$\lambda$$\lambda$Σ∴=−+ which on setting equal to zero gives that the maximum likelihood estimate is ˆixxn$\lambda$Σ==. [Consideration of 22logdd$\lambda$ confirms that this is a maximum: 222log0ixdLd$\lambda$$\lambda$−Σ=<.]
\item  ()()Var()ˆVarVarXXnn$\lambda$$\lambda$== .
Thus the maximum likelihood estimator of Var(ˆ
$\lambda$) is ˆn$\lambda$.
\begin{itemize}
    \item By the central limit theorem, ˆ( X$\lambda$= is approximately Normally distributed with mean $\lambda$ and variance $\lambda$ /n. 
    \item We estimate the variance by ˆ
$\lambda$/n, so that we have ˆˆ~N,n$\lambda$$\lambda$$\lambda$⎛⎞⎜⎜⎝⎠, approximately.
\item Hence an approximate 95\% confidence interval is given by
\[ˆ0.951.961.96ˆ/Pn$\lambda$$\lambda$$\lambda$⎛⎞−≈−<<⎜⎟⎜⎟⎝⎠,\]
leading to the interval \[ˆˆˆˆ1.96,1.96nn$\lambda$$\lambda$$\lambda$$\lambda$⎛⎞⎜⎟−+⎜⎟⎝⎠.\]
\end{itemize}

\item  For the given sample, we have n = 12 and Σxi = 48, leading to ˆ4x$\lambda$==. The approximate confidence interval is therefore
441.96to41.961212⎛⎞−+⎜⎟⎜⎟⎝⎠, i.e. 2.87 to 5.13.
The sample also gives Σxi2 = 238; so the sample variance is
\[s2 = 2148462384.182111211⎛⎞−==⎜⎟⎝⎠.\]
This is close to the sample mean (4), supporting a Poisson hypothesis for the underlying model.
\end{enumerate}
\end{document}
