\documentclass{article}
\usepackage[utf8]{inputenc}
\usepackage{enumerate}

\author{kobriendublin }
\date{December 2018}

\begin{document}

%- Higher Certificate, Module 5, 2010. Question 1
\section{Introduction}
\begin{enumerate}[(i)]
\item 


Higher Certificate, Module 5, 2010. Question 1
(i) P(X = 0, Y = 0) = P(Y = 0 | X = 0)P(X = 0) = 1 × 0.4 = 0.4
P(X = 1, Y = 0) = P(Y = 0 | X = 1)P(X = 1) = 0.6 × 0.3 = 0.18
P(X = 1, Y = 1) = P(Y = 1 | X = 1)P(X = 1) = 0.4 × 0.3 = 0.12
P(X = 2, Y = 0) = P(Y = 0 | X = 2)P(X = 2) = 0.62 × 0.2 = 0.072
P(X = 2, Y = 1) = P(Y = 1 | X = 2)P(X = 2) = 2 × 0.6 × 0.4 × 0.2 = 0.096
P(X = 2, Y = 2) = P(Y = 2 | X = 2)P(X = 2) = 0.42 × 0.2 = 0.032
P(X = 3, Y = 0) = P(Y = 0 | X = 3)P(X = 3) = 0.63 × 0.1 = 0.0216
P(X = 3, Y = 1) = P(Y = 1 | X = 3)P(X = 3) = 3 × 0.62 × 0.4 × 0.1 = 0.0432
P(X = 3, Y = 2) = P(Y = 2 | X = 3)P(X = 3) = 3 × 0.6 × 0.42 × 0.1 = 0.0288
P(X = 3, Y = 3) = P(Y = 3 | X = 3)P(X = 3) = 0.43 × 0.1 = 0.0064
All other entries in the two-way table are zero. Of course, it is not necessary to calculate all the above explicitly – many can be deduced by subtraction. The table is as follows.
Values of X
0
1
2
3
Total
Values of Y
0
0.4
0.18
0.072
0.0216
0.6736
1
0
0.12
0.096
0.0432
0.2592
2
0
0
0.032
0.0288
0.0608
3
0
0
0
0.0064
0.0064
Total
0.4
0.3
0.2
0.1
(ii) The marginal distribution of Y is in the Total column in the table in part (i):
P(Y = 0) = 0.6736 P(Y = 1) = 0.2592 P(Y = 2) = 0.0608 P(Y = 3) = 0.0064.
∴E(Y) = (0 × 0.6736) + (1 × 0.2592) + (2 × 0.0608) + (3 × 0.0064) = 0.4.
(iii) Cov(X, Y) = E(XY) – E(X)E(Y). From the table, we have
E(X) = (0 × 0.4) + (1 × 0.3) + (2 × 0.2) + (3 × 0.1) = 1.0,
E(XY) = (0 × 0 × 0.4) + ..... + (3 × 3 × 0.0064) = 0.8.
∴Cov(X, Y) = 0.8 – (1.0 × 0.4) = 0.4.
(iv) We need to find the probability distribution of X – Y, which we here call U.
P(U = 0) = P(X = Y) = 0.4 + 0.12 + 0.032 + 0.0064 = 0.5584
P(U = 1) = P(X – Y = 1) = 0.18 + 0.096 + 0.0288 = 0.3048
P(U = 2) = P(X – Y = 2) = 0.072 + 0.0432 = 0.1152
P(U = 3) = P(X – Y = 3) = 0.0216
[P(U = k) = 0 for all other values of k.]
\end{enumerate}

\end{document}