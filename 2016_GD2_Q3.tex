3. The continuous random variable U, which may take only non-negative values, has
cumulative distribution function F(u) and probability density function f (u). The
hazard function of U is defined to be
( ) ( ) , 0. 1 ()
f u hu u
F u
  
(i) The random variable Z has a Weibull distribution with probability density
function given by
  1 exp , 0, ( ) 0, otherwise,
z zz f z
         

where  > 0 and  > 0. Derive the hazard function of Z. Write down
conditions on  that give
(a) a hazard function that is constant for all z  0,
(b) a hazard function that decreases as z increases.
(6)
A system consists of two components, C1 and C2. Failures occur in the two
components independently. Let the random variable Xi (i = 1, 2) be the time to first
failure of Ci, where Xi has cumulative distribution function Fi (xi) and probability
density function fi(xi). Let Y be the time to first failure of the system.
(ii) C1 and C2 are connected in series. This means that the system fails as soon as
either of the two components fails. Show that Y has cumulative distribution
function
1 2 12 Gy F y F y F yF y y ( ) ( ) ( ) ( ) ( ), 0.  
Deduce that the hazard function of Y is the sum of the hazard functions of X1
and X2. Hence or otherwise show that, if Xi (i = 1, 2) follows a Weibull
distribution with parameters i and , then Y also follows a Weibull
distribution.
(8)
(iii) The components C1 and C2 are now connected in parallel to create a new
system. The time to first failure of this system is the maximum of the times to
first failure of the two components. Explain why the time to first failure of this
system, Y, has cumulative distribution function
1 2 Gy F yF y y ( ) ( ) ( ), 0.  
In the case where the two components are identical, and therefore have
identically distributed times to first failure, obtain an expression for the hazard
function of the system in terms of the cumulative distribution function and
probability density function of an individual component. Show that the hazard
for the system, h(y), is no greater than the hazard for an individual component,
for all values of y.
(6) 
