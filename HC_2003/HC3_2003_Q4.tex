\documentclass[a4paper,12pt]{article}
%%%%%%%%%%%%%%%%%%%%%%%%%%%%%%%%%%%%%%%%%%%%%%%%%%%%%%%%%%%%%%%%%%%%%%%%%%%%%%%%%%%%%%%%%%%%%%%%%%%%%%%%%%%%%%%%%%%%%%%%%%%%%%%%%%%%%%%%%%%%%%%%%%%%%%%%%%%%%%%%%%%%%%%%%%%%%%%%%%%%%%%%%%%%%%%%%%%%%%%%%%%%%%%%%%%%%%%%%%%%%%%%%%%%%%%%%%%%%%%%%%%%%%%%%%%%
\usepackage{eurosym}
\usepackage{vmargin}
\usepackage{amsmath}
\usepackage{graphics}
\usepackage{epsfig}
\usepackage{enumerate}
\usepackage{multicol}
\usepackage{subfigure}
\usepackage{fancyhdr}
\usepackage{listings}
\usepackage{framed}
\usepackage{graphicx}
\usepackage{amsmath}
\usepackage{chngpage}
%\usepackage{bigints}

\usepackage{vmargin}
% left top textwidth textheight headheight
% headsep footheight footskip
\setmargins{2.0cm}{2.5cm}{16 cm}{22cm}{0.5cm}{0cm}{1cm}{1cm}
\renewcommand{\baselinestretch}{1.3}

\setcounter{MaxMatrixCols}{10}
\begin{document}
Higher Certificate, Paper III, 2003. Question 4

%%%%%%%%%%%%%%%%%%%%%%%%%%%%%%%%%%%%%%%%%%%%%%%%%%%%%%%%%%%%%%%%%%%%%%%%
\begin{framed}
4.
The table below shows UK households' final consumption expenditure on alcohol and
tobacco, in £millions at 1995 prices, for the years 1990 to 2000 inclusive. (Source:
Economic Trends Annual Supplement.) The table also shows the forecasts obtained
from exponential smoothing, using 0.8 as the smoothing constant and taking the 1989
value as the forecast for 1990, and the errors in the forecasts.
Year
1990
1991
1992
1993
1994
1995
1996
1997
1998
1999
2000
Expenditure
20730
20148
19539
19255
19268
18776
19299
19459
19193
19863
19959
Forecast
20735
20731
20265
19684
19341
19283
18877
19215
19410
19236
19738
Error
− 5
− 583
− 726
− 429
− 73
− 507
422
244
− 217
627
221
(i) Explain, in a non-technical way, how to calculate the forecasts shown here.
Illustrate your answer by showing the calculation of the forecast for 2000 in
detail.
(6)
(ii) When, in general, is it appropriate to use a high value for the smoothing
constant?
(2)
(iii) Describe one method of measuring the accuracy of the fitted model. How
would this measure usually be used in practice?
(5)
(iv) Use the given results to forecast expenditure in 2001.
(2)
(v)
Plot the errors in the forecasts and comment.
(5)
\end{framed}
%%%%%%%%%%%%%%%%%%%%%%%%%%%%%%%%%%%%%%%%%%%%%%%%%%%%%%%%%%%%%%%%%%%%%%%%
\begin{enumerate}[(a)]
\item  Exponential smoothing forecasts xˆt as xˆt 1 α (xt 1 xˆt 1 ) − − − + − , i.e.
( ) 1 1 ˆ 1 ˆ t t t x α x α x − − = + − .
α is a value between 0 and 1, combining the previous forecast 1 ˆt x − and the actual
observed xt–1 as a weighted average.
For 2000, we use 1999 ˆx and 1999 x :
( )( ) ( )( ) 2000 xˆ = 0.8 19863 + 0.2 19236 =19737.6 ≈19738 .
Of course all earlier data are represented to some extent in this forecast.
%%%%%%%%%%%%%%%%%%%%%%%%%%%%%%%%%%%%%%%%%%%%%%%%
\item A high value of α, such as 0.8, is appropriate if there is little previous
experience or if there appears to have been some change in pattern of the data which
makes older data less relevant.
%%%%%%%%%%%%%%%%%%%%%%%%%%%%%%%%%%%%%%%%%%%%%%%%
\item Mean absolute deviation is a possible method. This is
1
1 ˆ
n
t t
t
MAD x x
n =
= Σ − .
It is a general method, not restricted to exponential smoothing, where various possible
models are being compared. The model with minimum MAD is usually preferred.
Minimum MSE (mean square error), and some others, have also been suggested; the
differences between t x and ˆt x for past data form the basic criterion.
(iv) ( )( ) ( )( ) 2001 2000 2000 xˆ = 0.8x + 0.2xˆ = 0.8 19959 + 0.2 19738 =19915.
(v) Plot of error (= expenditure – forecast) against time.
-800
-600
-400
-200
0
200
400
600
800
1988 1990 1992 1994 1996 1998 2000 2002
Initially, forecasts have proved to be too high; later, they are almost all too low.
From 1992 onwards, this method has shown an upward trend in the error. Also there
seems to be some tendency to a "cycling" pattern: … . A slightly
more complex model may be needed.
\end{enumerate}
\end{document}
