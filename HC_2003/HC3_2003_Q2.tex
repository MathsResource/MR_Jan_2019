\documentclass[a4paper,12pt]{article}
%%%%%%%%%%%%%%%%%%%%%%%%%%%%%%%%%%%%%%%%%%%%%%%%%%%%%%%%%%%%%%%%%%%%%%%%%%%%%%%%%%%%%%%%%%%%%%%%%%%%%%%%%%%%%%%%%%%%%%%%%%%%%%%%%%%%%%%%%%%%%%%%%%%%%%%%%%%%%%%%%%%%%%%%%%%%%%%%%%%%%%%%%%%%%%%%%%%%%%%%%%%%%%%%%%%%%%%%%%%%%%%%%%%%%%%%%%%%%%%%%%%%%%%%%%%%
\usepackage{eurosym}
\usepackage{vmargin}
\usepackage{amsmath}
\usepackage{graphics}
\usepackage{epsfig}
\usepackage{enumerate}
\usepackage{multicol}
\usepackage{subfigure}
\usepackage{fancyhdr}
\usepackage{listings}
\usepackage{framed}
\usepackage{graphicx}
\usepackage{amsmath}
\usepackage{chngpage}
%\usepackage{bigints}

\usepackage{vmargin}
% left top textwidth textheight headheight
% headsep footheight footskip
\setmargins{2.0cm}{2.5cm}{16 cm}{22cm}{0.5cm}{0cm}{1cm}{1cm}
\renewcommand{\baselinestretch}{1.3}

\setcounter{MaxMatrixCols}{10}
\begin{document}
Higher Certificate, Paper III, 2003. Question 2
\begin{enumerate}[(a)]
\item  If two (or more) "factors", qualitative or quantitative, are included in the same
experiment at various levels, it is often the case that the response to one factor
depends on the level at which the other factor is applied. For example, in this
experiment, the response to a given % antimony may be different according to which
cooling method has been used for a particular experimental unit. When this happens,
such factors are said to interact.
%%%%%%%%%%%%%%%%%%%%%%%%%%%%%%%%%%%%%%%%%%%%%%%%
\item The row in the analysis of variance which refers to the possible interaction
contains the p-value 0.152, which means that there is no real evidence of interaction
(since p > 0.05). A graph of the means for the different cooling methods can help to
illustrate this (the overall mean is also shown):-
mean shear strengths
15
20
25
0 5 10
AB
FC
OQ
WQ
Overall Mean
antimony (% weight)
(NB: For clarity in the diagram, and to avoid taking up excessive space, the vertical axis is shown with a
"false origin" at 15.)
The graph shows some departure from "parallelism" in the patterns for the cooling
methods, but this is not great when compared with (residual) natural variation.
%%%%%%%%%%%%%%%%%%%%%%%%%%%%%%%%%%%%%%%%%%%%%%%%
\item Since there is no interaction, the main effects of percentage antimony and cooling
method can be studied directly. Both are significant. The main characteristic for %
antimony is the drop at 10\% compared with all the others. This is very large and is
the reason why "p = 0.000". For cooling, AB and OQ give higher strengths than FC
and WQ. Cooling has p = 0.004, still clearly highly significant though the difference
is not so big as that given by the drop for 10% antimony.
As usual, the variances underlying all sets of data are assumed to be the same, and the
residual term in the appropriate linear model is assumed to be Normally distributed.
\end{enumerate}
\end{document}
