\documentclass[a4paper,12pt]{article}
%%%%%%%%%%%%%%%%%%%%%%%%%%%%%%%%%%%%%%%%%%%%%%%%%%%%%%%%%%%%%%%%%%%%%%%%%%%%%%%%%%%%%%%%%%%%%%%%%%%%%%%%%%%%%%%%%%%%%%%%%%%%%%%%%%%%%%%%%%%%%%%%%%%%%%%%%%%%%%%%%%%%%%%%%%%%%%%%%%%%%%%%%%%%%%%%%%%%%%%%%%%%%%%%%%%%%%%%%%%%%%%%%%%%%%%%%%%%%%%%%%%%%%%%%%%%
\usepackage{eurosym}
\usepackage{vmargin}
\usepackage{amsmath}
\usepackage{graphics}
\usepackage{epsfig}
\usepackage{enumerate}
\usepackage{multicol}
\usepackage{subfigure}
\usepackage{fancyhdr}
\usepackage{listings}
\usepackage{framed}
\usepackage{graphicx}
\usepackage{amsmath}
\usepackage{chngpage}
%\usepackage{bigints}

\usepackage{vmargin}
% left top textwidth textheight headheight
% headsep footheight footskip
\setmargins{2.0cm}{2.5cm}{16 cm}{22cm}{0.5cm}{0cm}{1cm}{1cm}
\renewcommand{\baselinestretch}{1.3}
%- Higher Certificate, Paper I, 2003. Question 2
\setcounter{MaxMatrixCols}{10}
\begin{document}
\begin{framed}
The events A, B and C have respective probabilities 2/3, 1/2 and 1/4
 and $A^c$ , $B^c$ and $C^c$
are, respectively, the complements of A, B and C.
(i) Given that A, B and C are mutually independent, find
(a) $P(A \cap B \cap C)$,
(b) $P(A\cap C A \cap B)$ .
\end{framed}

%%%%%%%%%%%%%%%%%%%%%%%%%%%%%%%%%%%%%%%%%%%%%%%%%%%%%%%%%%%%%%%%

( ) 2 ( ) 1 ( ) 1
3 2 4
P A = P B = P C =
\begin{enumerate}
\item (i) (a) By the given independence,
( ) ( ) ( ) ( ) 2 1 1 1 1 1
3 2 4 4
P A\capB \capC = P A P B P C =  −  −  =
  
.
\item ( ) (( ) ( ))
( )
( )
( )
P A C A B P A C B
P A C A B
P A B P A B
\cap \cap \cap \cap \cap
\cap \cap = =
\cap \cap
( )
1
4
2 1
3 2
3
1 4
= =
−
.

%%%%%%%%%%%%%%%%%%%%%%%%%%%%%%%%%%%%%%
\newpage
\begin{framed}
Now let A and B be independent, A and C be independent and B and C be
independent, so that A, B and C are pairwise independent, and let
$P( A \cap B \cap C) = x$. Find in terms of x
(a) $P(A \cap B \cap C)$,
(b) $P(A∩C A \cap B)$ ,
(c) $P( A \cup B \cup C)$ .
Also find the maximum and minimum possible values of x.
\end{framed}
\item 
Using pairwise independence, the value of $P( A\cap B) = P(A)P(B)$, etc, and
hence the values 1
3 x − , 18
− x and 1
6 − x are found. The others follow using
P(A), P(B) and P(C).
\item  
\begin{eqnarray*}
P \left(A\cap \bar{B} \cap \bar{C} \right) &=& \frac{1}{6}+ x
\end{eqnarray*} from the diagram.

\begin{eqnarray*}
P (A\cap \bar{C}| A \cap \bar{B}) &=& \frac{ P A  \cap C B}{P(A\cap \bar{B})} \\
&=& \frac{ P A  \cap C B}{P(A \cap \bar{B})} \\
&=& \frac{1/6 + x }{(1/6-x)} \\
&=& 3x+\frac{1}{2} \\
\end{eqnarray*}

\item 
\begin{eqnarray*}
P (A\cap C| A \cap B) &=& \frac{19}{24} + x
\end{eqnarray*}
Since all probabilities must lie in [0,1], we have 1
24 x ≥ and 18
x ≤ , i.e.
1 1
24 8
≤ x ≤ .
1
6 + x 1
3 − x 1
24 + x
x 18
− x 1
6 − x
1
24 x −
A B
C
\end{enumerate}
\end{document}
