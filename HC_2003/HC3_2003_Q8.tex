\documentclass[a4paper,12pt]{article}
%%%%%%%%%%%%%%%%%%%%%%%%%%%%%%%%%%%%%%%%%%%%%%%%%%%%%%%%%%%%%%%%%%%%%%%%%%%%%%%%%%%%%%%%%%%%%%%%%%%%%%%%%%%%%%%%%%%%%%%%%%%%%%%%%%%%%%%%%%%%%%%%%%%%%%%%%%%%%%%%%%%%%%%%%%%%%%%%%%%%%%%%%%%%%%%%%%%%%%%%%%%%%%%%%%%%%%%%%%%%%%%%%%%%%%%%%%%%%%%%%%%%%%%%%%%%
\usepackage{eurosym}
\usepackage{vmargin}
\usepackage{amsmath}
\usepackage{graphics}
\usepackage{epsfig}
\usepackage{enumerate}
\usepackage{multicol}
\usepackage{subfigure}
\usepackage{fancyhdr}
\usepackage{listings}
\usepackage{framed}
\usepackage{graphicx}
\usepackage{amsmath}
\usepackage{chngpage}
%\usepackage{bigints}

\usepackage{vmargin}
% left top textwidth textheight headheight
% headsep footheight footskip
\setmargins{2.0cm}{2.5cm}{16 cm}{22cm}{0.5cm}{0cm}{1cm}{1cm}
\renewcommand{\baselinestretch}{1.3}

\setcounter{MaxMatrixCols}{10}
\begin{document}


Higher Certificate, Paper III, 2003. Question 8

%%%%%%%%%%%%%%%%%%%%%%%%%%%%%%%%%%%%%%%%%%%%%%%%%%%%%%%%%%%%%%%%%%%%%%%%
\begin{framed}
8.
The table shows details of employees in the UK by sex and industry at June for four
selected years. Performing such calculations on the data as you think appropriate,
write a short report outlining the main features shown in this table, supporting your
answer by diagrams. Give particular attention to differences in trends between males
and females.
Note: You are advised to show only a selection of the data in diagrams.
Industry
A. Distribution, hotels, catering and repairs
B. Manufacturing
C. Financial and business services
D. Transport and communication
E. Construction
F. Agriculture
G. Energy and water supply
H. Other services 1978
15
35
9
9
8
2
5
16 Males
1981 1991
16
19
33
26
10
15
9
9
8
8
2
2
5
3
17
19
All employees (=100%) (millions) 13.4 12.6
11.5
1997
20
26
16
9
7
2
1
19 1978
24
22
11
3
1
1
1
38
11.5 9.4
Percentages
Females
1981 1991 1997
25
25
26
18
12
10
12
16
19
3
3
3
1
1
1
1
1
1
−
1
1
39
41
40
9.3
10.7
11.3
Source: Short-term Turnover and Employment Survey, Office for National Statistics. Reproduced in
Social Trends 28 Pocketbook
\end{framed}
%%%%%%%%%%%%%%%%%%%%%%%%%%%%%%%%%%%%%%%%%%%%%%%%%%%%%%%%%%%%%%%%%%%%%%%%
\begin{enumerate}[(a)]
\item Main points which could be made include the following.
Total number of employees remained fairly steady, around 22 million, but the
ratio of males to females decreased from 13.4/9.4 = 1.43 in 1978 (and 1.35 in
1981) to 11.5/11.3 = 1.02 in 1997 (1.07 in 1991).
Percentage of males in category H remained much the same, as did that of
females, but the average percentage for males was about 18 and for females
about 40 (presumably nursing and medical services are included in H, which
would provide an explanation).
Percentage of workers overall (both sexes) in category B fell sharply between
1978/1981 and 1991/1997.
Percentage of workers in category C increased for both sexes between
1978/1981 and 1991/1997.
In D, E and F the percentages over time remained similar, with more males
than females.
In G, the number of employees had dropped sharply by 1997.
Some calculations of actual numbers (bottom row × appropriate percentages) would
help to emphasise the drop in numbers of workers (both sexes) in manufacturing, and
the increases in numbers for financial and business services.
A combination of percentages and actual numbers would indicate a noticeable
increase in male employment in category A. For females, numbers increase though
not percentages.
Graphs of "time series" for the two sexes and four years, a single graph for each
category, would help to show the changes, in categories B and C particularly.
Bar charts could be used to show actual numbers or percentages over time, one for
each year. Because of the presence of several categories with small percentages,
annual pie-charts would be slightly less easy to appreciate (but quite valid).
Some of the categories are rather broad, and explanations of changes therefore not
always possible even for a UK commentator.
\end{enumerate}
\end{document}
