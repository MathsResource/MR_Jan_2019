\documentclass[a4paper,12pt]{article}
%%%%%%%%%%%%%%%%%%%%%%%%%%%%%%%%%%%%%%%%%%%%%%%%%%%%%%%%%%%%%%%%%%%%%%%%%%%%%%%%%%%%%%%%%%%%%%%%%%%%%%%%%%%%%%%%%%%%%%%%%%%%%%%%%%%%%%%%%%%%%%%%%%%%%%%%%%%%%%%%%%%%%%%%%%%%%%%%%%%%%%%%%%%%%%%%%%%%%%%%%%%%%%%%%%%%%%%%%%%%%%%%%%%%%%%%%%%%%%%%%%%%%%%%%%%%
\usepackage{eurosym}
\usepackage{vmargin}
\usepackage{amsmath}
\usepackage{graphics}
\usepackage{epsfig}
\usepackage{enumerate}
\usepackage{multicol}
\usepackage{subfigure}
\usepackage{fancyhdr}
\usepackage{listings}
\usepackage{framed}
\usepackage{graphicx}
\usepackage{amsmath}
\usepackage{chngpage}
%\usepackage{bigints}

\usepackage{vmargin}
% left top textwidth textheight headheight
% headsep footheight footskip
\setmargins{2.0cm}{2.5cm}{16 cm}{22cm}{0.5cm}{0cm}{1cm}{1cm}
\renewcommand{\baselinestretch}{1.3}

\setcounter{MaxMatrixCols}{10}
\begin{document}
Higher Certificate, Paper I, 2003. Question 5
Poisson distribution: ( ) !
e x f x
x
−λλ
= . Expectation = variance = λ.
(i) λ = 0.5 : f (0) = e-0.5 = 0.6065, f (1) = 0.3033, f (2) = 0.0758, … .
Expectation = variance = 0.5.
λ = 2 : f (0) = 0.1353, f (1) = 0.2707, f (2) = 0.2707, f (3) = 0.1804,
f (4) = 0.0902, … . Expectation = variance = 2.
Sketches are as shown.
(ii) Likelihood L =
1 ! !
n xi n xi
i i i
e e
x x
−λλ − λλ Σ
=
=
Π Π .
Taking logarithms to base e,
log ( )log log( !) i i L = − nλ + Σx λ − Π x .
Differentiating, log i d L n x
dλ λ
= − + Σ ; setting this equal to 0 gives the solution
1
ˆ 1 n
ML i
i
x x
n
λ
=
= Σ = . We have
2
2
log i 0 d L x
dλ λ 2
= − Σ < , confirming that this is
a maximum.
The central limit theorem gives X ∼ (approx) N(λ , λ /n ), so we have
(approximately)
P 1.96 X 1.96 0.95
n n
λ λ λ λ
 
 − ≤ ≤ +  =
 
Continued on next page
f (x) f (x)
0 x 0 x 2 … 2 4 …
or
P X 1.96 X 1.96 0.95
n n
λ λ λ
 
 − ≤ ≤ +  =
 
.
Hence, inserting the observed value x and, further, using ˆ
ML λ = x as an
estimate for the underlying variance, an approximate 95% confidence interval
for λ is
x 1.96 x , x 1.96 x
n n
− + .
(iii) 400, 2500; 6.25 i n = Σx = x = .
So the approximate 95% confidence interval is
6.25 1.96 6.25 , 6.25 1.96 6.25
400 400
− +
i.e. 6.005 , 6.495 .
Now using Σxi
2 = 25600, we have that the sample variance s2 is
( )2
2 1 2500 9975 25600 25.00
399 400 399
s
 
=  −  = =
 
 
.
Using s2 in the confidence interval gives the interval as
6.25 1.96 25.00 , 6.25 1.96 25.00
400 400
− +
i.e. 5.76 , 6.74 .
This interval is twice as wide – because s2 is four times the size of x – which
suggests that a Poisson assumption is not valid.
