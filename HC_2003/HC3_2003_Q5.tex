\documentclass[a4paper,12pt]{article}
%%%%%%%%%%%%%%%%%%%%%%%%%%%%%%%%%%%%%%%%%%%%%%%%%%%%%%%%%%%%%%%%%%%%%%%%%%%%%%%%%%%%%%%%%%%%%%%%%%%%%%%%%%%%%%%%%%%%%%%%%%%%%%%%%%%%%%%%%%%%%%%%%%%%%%%%%%%%%%%%%%%%%%%%%%%%%%%%%%%%%%%%%%%%%%%%%%%%%%%%%%%%%%%%%%%%%%%%%%%%%%%%%%%%%%%%%%%%%%%%%%%%%%%%%%%%
\usepackage{eurosym}
\usepackage{vmargin}
\usepackage{amsmath}
\usepackage{graphics}
\usepackage{epsfig}
\usepackage{enumerate}
\usepackage{multicol}
\usepackage{subfigure}
\usepackage{fancyhdr}
\usepackage{listings}
\usepackage{framed}
\usepackage{graphicx}
\usepackage{amsmath}
\usepackage{chngpage}
%\usepackage{bigints}

\usepackage{vmargin}
% left top textwidth textheight headheight
% headsep footheight footskip
\setmargins{2.0cm}{2.5cm}{16 cm}{22cm}{0.5cm}{0cm}{1cm}{1cm}
\renewcommand{\baselinestretch}{1.3}

\setcounter{MaxMatrixCols}{10}
\begin{document}
Higher Certificate, Paper III, 2003. Question 5

%%%%%%%%%%%%%%%%%%%%%%%%%%%%%%%%%%%%%%%%%%%%%%%%%%%%%%%%%%%%%%%%%%%%%%%%
\begin{framed}
5.
In World War II, a particular shipyard built numerous Liberty Ships. Orders for ships
were given serial numbers, so the first such order was given the order number (NBR)
1, the second such order was given NBR 2, and so on. The number of thousand man-
hours per ship built to meet each order (HRS) was noted. Table 1 below shows a
number of values of NBR together with the corresponding values of HRS. Table 2
gives some regression results.
It is conjectured that man-hours per ship depends on the order number.
(i) Draw a graph showing the data in Table 1. Is the relation between man-hours
per ship and order number linear? Justify your answer.
(8)
(ii) On the basis of the regression results, which of the three linear regressions
reported do you consider is best? Justify your answer.
(2)
(iii) Interpret the estimated constant and regression coefficient for the regression
you considered to be best in part (ii).
(8)
(iv) What other transformation of the data might you try to improve further on the
regression you considered to be best, and why?
(2)
Table 1
NBR
HRS
5
1094
10
894
25
647
30
659
75
529
100
424
125
376
150
395
175
395
200
388
225
376
250
353
285
280
Table 2
Regression Analysis: HRS versus NBR
The regression equation is
HRS = 790 - 2.09 NBR
Predictor
Constant
NBR
S = 138.8
Coef
789.54
-2.0871
SE Coef
65.77
0.4190
R-Sq = 69.3%
T
12.00
-4.98
P
0.000
0.000
R-Sq(adj) = 66.5%
\begin{verbatim}
Regression Analysis: HRS versus 1/NBR
The regression equation is
HRS = 391 + 3975 1/NBR
Predictor
Constant
1/NBR
S = 84.09
Coef
391.14
3975.2
SE Coef
27.34
427.4
R-Sq = 88.7%
T
14.31
9.30
P
0.000
0.000
R-Sq(adj) = 87.7%
\end{verbatim}
\begin{verbatim}
Regression Analysis: HRS versus logNBR
The regression equation is
HRS = 1306 - 181 logNBR
Predictor
Constant
logNBR
S = 48.17
Coef
1306.24
-180.51
SE Coef
48.14
10.67
R-Sq = 96.3%
T
27.13
-16.92
P
0.000
0.000
R-Sq(adj) = 96.0%
\end{verbatim}
\end{framed}
%%%%%%%%%%%%%%%%%%%%%%%%%%%%%%%%%%%%%%%%%%%%%%%%%%%%%%%%%%%%%%%%%%%%%%%%
\begin{enumerate}[(a)]
\item 
The relation is not linear. It began by showing an exponential-type fall over
the first 100 or so and then levelled off, with a suggestion of a further curved
downward trend after about 200.
%%%%%%%%%%%%%%%%%%%%%%%%%%%%%%%%%%%%%%%%%%%%%%%%
\item HRS versus logNBR gives the largest percentage of variation explained
(R2 = 96%). (HRS versus 1/NBR is also fairly good, but some way behind this
one.)
%%%%%%%%%%%%%%%%%%%%%%%%%%%%%%%%%%%%%%%%%%%%%%%%
\item HRS = 1306 – 181 logNBR. The constant is a baseline or average cost
estimated from these data. The coefficient of logNBR ("log" here implies to
base e) gives the reduction (in this case) in HRS (thousands of man-hours) for
every increase of 1 in logNBR, i.e. every 2.71828 along the number scale.
This is an average reduction.
If the residuals were available, we would look at them to see if they were
"random" or if they showed some pattern (for example, all the middle ones
were of one sign and the first and last were of the other sign). If so, this would
suggest that the model could still be improved, even given the high value of
R2.
(iv) Perhaps logHRS against logNBR might show improvement.
0
200
400
600
800
1000
1200
0 50 100 150 200 250 300
NBR
HRS
\end{enumerate}
\end{document}
