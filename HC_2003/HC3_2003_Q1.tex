\documentclass[a4paper,12pt]{article}
%%%%%%%%%%%%%%%%%%%%%%%%%%%%%%%%%%%%%%%%%%%%%%%%%%%%%%%%%%%%%%%%%%%%%%%%%%%%%%%%%%%%%%%%%%%%%%%%%%%%%%%%%%%%%%%%%%%%%%%%%%%%%%%%%%%%%%%%%%%%%%%%%%%%%%%%%%%%%%%%%%%%%%%%%%%%%%%%%%%%%%%%%%%%%%%%%%%%%%%%%%%%%%%%%%%%%%%%%%%%%%%%%%%%%%%%%%%%%%%%%%%%%%%%%%%%
\usepackage{eurosym}
\usepackage{vmargin}
\usepackage{amsmath}
\usepackage{graphics}
\usepackage{epsfig}
\usepackage{enumerate}
\usepackage{multicol}
\usepackage{subfigure}
\usepackage{fancyhdr}
\usepackage{listings}
\usepackage{framed}
\usepackage{graphicx}
\usepackage{amsmath}
\usepackage{chngpage}
%\usepackage{bigints}

\usepackage{vmargin}
% left top textwidth textheight headheight
% headsep footheight footskip
\setmargins{2.0cm}{2.5cm}{16 cm}{22cm}{0.5cm}{0cm}{1cm}{1cm}
\renewcommand{\baselinestretch}{1.3}

\setcounter{MaxMatrixCols}{10}
\begin{document}
Higher Certificate, Paper III, 2003. Question 2
%%%%%%%%%%%%%%%%%%%%%%%%%%%%%%%%%%%%%%%%%%%%%%%%%%%%%%%%%%%%%%%%%%%%%%%%
\begin{framed}
2.
An experiment, described in the Journal of Materials Science 1986, was conducted to
investigate the effect of antimony on the strength of tin-lead solder joints. Four
different amounts of antimony were considered (0%, 3%, 5% and 10% weight) and
four different cooling methods. Each of the four different amounts of antimony was
used with each of the cooling methods, giving a total of sixteen different treatments.
Three joints were assigned at random to each of these sixteen treatments and their
shear strengths were measured in MPa.
Table 1 shows the mean shear strength at each treatment. Table 2 shows some results
of an analysis of variance of the data.
Table 1. Mean shear strengths
Rows: amount of antimony (% weight)
Columns: cooling method
AB
FC
OQ
WQ
All
0
3
5
10 20.333
22.233
21.400
16.733 19.833
19.933
18.833
17.200 22.067
20.433
21.467
17.833 18.467
19.033
20.767
16.300 20.175
20.408
20.617
17.017
All 20.175 18.950 20.450 18.642 19.554
Key: AB
FC
OQ
WQ
=
=
=
=
air-blown
furnace-cooled
oil-quenched
water-quenched
Table 2
Two-way ANOVA: strength versus cooling, antimony
Analysis of Variance for strength
Source
Cooling
Antimony
Interaction
Error
Total
DF
3
3
9
32
47
SS
28.63
104.19
25.13
55.25
213.20
MS
9.54
34.73
2.79
1.73
F
5.53
20.12
1.62
P
0.004
0.000
0.152
(i) Explain what is meant by interaction, both generally and in the context of this
experiment.
(6)
(ii) By referring to the results of the analysis of variance, and by drawing a suitable
plot, comment on whether there appears to be interaction between the amount
of antimony and the cooling method.
(7)
(iii) Explain carefully how to interpret the F and P values for cooling and for
antimony in Table 2, stating any assumptions needed for your answer.
(7)
\end{framed}
%%%%%%%%%%%%%%%%%%%%%%%%%%%%%%%%%%%%%%%%%%%%%%%%%%%%%%%%%%%%%%%%%%%%%%%%
\begin{enumerate}[(a)]
\item  If two (or more) "factors", qualitative or quantitative, are included in the same
experiment at various levels, it is often the case that the response to one factor
depends on the level at which the other factor is applied. For example, in this
experiment, the response to a given % antimony may be different according to which
cooling method has been used for a particular experimental unit. When this happens,
such factors are said to interact.
%%%%%%%%%%%%%%%%%%%%%%%%%%%%%%%%%%%%%%%%%%%%%%%%
\item The row in the analysis of variance which refers to the possible interaction
contains the p-value 0.152, which means that there is no real evidence of interaction
(since p > 0.05). A graph of the means for the different cooling methods can help to
illustrate this (the overall mean is also shown):-
mean shear strengths
15
20
25
0 5 10
AB
FC
OQ
WQ
Overall Mean
antimony (% weight)
(NB: For clarity in the diagram, and to avoid taking up excessive space, the vertical axis is shown with a
"false origin" at 15.)
The graph shows some departure from "parallelism" in the patterns for the cooling
methods, but this is not great when compared with (residual) natural variation.
%%%%%%%%%%%%%%%%%%%%%%%%%%%%%%%%%%%%%%%%%%%%%%%%
\item Since there is no interaction, the main effects of percentage antimony and cooling
method can be studied directly. Both are significant. The main characteristic for %
antimony is the drop at 10\% compared with all the others. This is very large and is
the reason why "p = 0.000". For cooling, AB and OQ give higher strengths than FC
and WQ. Cooling has p = 0.004, still clearly highly significant though the difference
is not so big as that given by the drop for 10% antimony.
As usual, the variances underlying all sets of data are assumed to be the same, and the
residual term in the appropriate linear model is assumed to be Normally distributed.
\end{enumerate}
\end{document}
