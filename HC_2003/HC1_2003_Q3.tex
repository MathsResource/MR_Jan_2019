\documentclass[a4paper,12pt]{article}
%%%%%%%%%%%%%%%%%%%%%%%%%%%%%%%%%%%%%%%%%%%%%%%%%%%%%%%%%%%%%%%%%%%%%%%%%%%%%%%%%%%%%%%%%%%%%%%%%%%%%%%%%%%%%%%%%%%%%%%%%%%%%%%%%%%%%%%%%%%%%%%%%%%%%%%%%%%%%%%%%%%%%%%%%%%%%%%%%%%%%%%%%%%%%%%%%%%%%%%%%%%%%%%%%%%%%%%%%%%%%%%%%%%%%%%%%%%%%%%%%%%%%%%%%%%%
\usepackage{eurosym}
\usepackage{vmargin}
\usepackage{amsmath}
\usepackage{graphics}
\usepackage{epsfig}
\usepackage{enumerate}
\usepackage{multicol}
\usepackage{subfigure}
\usepackage{fancyhdr}
\usepackage{listings}
\usepackage{framed}
\usepackage{graphicx}
\usepackage{amsmath}
\usepackage{chngpage}
%\usepackage{bigints}

\usepackage{vmargin}
% left top textwidth textheight headheight
% headsep footheight footskip
\setmargins{2.0cm}{2.5cm}{16 cm}{22cm}{0.5cm}{0cm}{1cm}{1cm}
\renewcommand{\baselinestretch}{1.3}

\setcounter{MaxMatrixCols}{10}
\begin{document}
Higher Certificate, Paper I, 2003. Question 3
(i) (a) X + YA is N(10 + 15 , 12 + 16) i.e. N(25, 28).
(b) X + YB is N(10 + 12 , 12 + 9) i.e. N(22, 21).
(ii) If X is the same for both, we require P(YA < YB), i.e. P(YA – YB < 0).
YA – YB is N(15 – 12 , 16 + 9) i.e. N(3, 25).
( 0) 0 3
5 A B P Y −Y < = Φ − 
 
where (as usual) Φ denotes the cdf of the N(0, 1)
distribution. From tables Φ(–0.6) = 1 – Φ(0.6) = 0.2743.
(iii) Writing WA = X + YA and WB = X + YB, we require P(WA < WB), i.e.
P(WA – WB < 0).
WA – WB is N(25 – 22 , 28 + 21) i.e. N(3, 49).
( 0) 0 3
7 A B P W −W < = Φ − 
 
= 3
7
Φ −   
 
= 0.3341.
(iv) is N 25, 28 and is N 22, 21
16 16 A B W   W  
   
   
.
Let A B U =W −W ; then U is N 3, 49
16
 
 
 
, and we require
( ) ( ) 7
4
0 0 3 12 1.7143
7
P U
 −    < = Φ  = Φ−  = Φ −
   
= 0.0432.
