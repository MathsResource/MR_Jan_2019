\documentclass[a4paper,12pt]{article}
%%%%%%%%%%%%%%%%%%%%%%%%%%%%%%%%%%%%%%%%%%%%%%%%%%%%%%%%%%%%%%%%%%%%%%%%%%%%%%%%%%%%%%%%%%%%%%%%%%%%%%%%%%%%%%%%%%%%%%%%%%%%%%%%%%%%%%%%%%%%%%%%%%%%%%%%%%%%%%%%%%%%%%%%%%%%%%%%%%%%%%%%%%%%%%%%%%%%%%%%%%%%%%%%%%%%%%%%%%%%%%%%%%%%%%%%%%%%%%%%%%%%%%%%%%%%
\usepackage{eurosym}
\usepackage{vmargin}
\usepackage{amsmath}
\usepackage{graphics}
\usepackage{epsfig}
\usepackage{enumerate}
\usepackage{multicol}
\usepackage{subfigure}
\usepackage{fancyhdr}
\usepackage{listings}
\usepackage{framed}
\usepackage{graphicx}
\usepackage{amsmath}
\usepackage{chngpage}
%\usepackage{bigints}

\usepackage{vmargin}
% left top textwidth textheight headheight
% headsep footheight footskip
\setmargins{2.0cm}{2.5cm}{16 cm}{22cm}{0.5cm}{0cm}{1cm}{1cm}
\renewcommand{\baselinestretch}{1.3}

\setcounter{MaxMatrixCols}{10}
\begin{document}




Higher Certificate, Paper II, 2003. Question 7
\begin{enumerate}[(a)]
\item Suitable diagrams include the following:
pie charts, for chosen years between 1990 and 2001, showing the distribution
of spending between different categories of the household expenditure;
time series graphs for individual categories, expressing expenditure either as
an absolute figure or a percentage of the total;
bar charts, similar in purpose to pie charts, either as % bar charts or totals to
show overall expenditure as well as components.
Because prices are given in terms of 2000/2001 levels, it may be less easy to track the
effects of price changes on consumption of individual items, or to see what may have
altered in the expenditure of items within each category.
Some points suggested by the raw data are:
total was steady early on, then began to increase, more quickly at the end;
housing showed a fall, then a rise, sharp at the end;
fuel and power fell, particularly from 1998 on;
household and personal goods and services increased in absolute value, as did
travel and leisure;
These latter four categories could certainly be explored as percentages of total, as well
as absolute values.
A newspaper article might emphasise the largest and smallest areas of spending, in
which areas spending increased, decreased or remained constant, and any apparent
relations in behaviour of the various categories (e.g. the four noted above).
\end{enumerate}
\end{document}
