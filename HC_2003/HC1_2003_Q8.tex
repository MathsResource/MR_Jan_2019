\documentclass[a4paper,12pt]{article}
%%%%%%%%%%%%%%%%%%%%%%%%%%%%%%%%%%%%%%%%%%%%%%%%%%%%%%%%%%%%%%%%%%%%%%%%%%%%%%%%%%%%%%%%%%%%%%%%%%%%%%%%%%%%%%%%%%%%%%%%%%%%%%%%%%%%%%%%%%%%%%%%%%%%%%%%%%%%%%%%%%%%%%%%%%%%%%%%%%%%%%%%%%%%%%%%%%%%%%%%%%%%%%%%%%%%%%%%%%%%%%%%%%%%%%%%%%%%%%%%%%%%%%%%%%%%
\usepackage{eurosym}
\usepackage{vmargin}
\usepackage{amsmath}
\usepackage{graphics}
\usepackage{epsfig}
\usepackage{enumerate}
\usepackage{multicol}
\usepackage{subfigure}
\usepackage{fancyhdr}
\usepackage{listings}
\usepackage{framed}
\usepackage{graphicx}
\usepackage{amsmath}
\usepackage{chngpage}
%\usepackage{bigints}

\usepackage{vmargin}
% left top textwidth textheight headheight
% headsep footheight footskip
\setmargins{2.0cm}{2.5cm}{16 cm}{22cm}{0.5cm}{0cm}{1cm}{1cm}
\renewcommand{\baselinestretch}{1.3}

\setcounter{MaxMatrixCols}{10}
\begin{document}

Higher Certificate, Paper I, 2003. Question 8
\begin{framed}









8. (i) Given a random sample of paired data (x1, y1), (x2, y2), …, (xn, yn), write down
the formula for the sample product-moment correlation coefficient, r say, and
explain the meaning of this quantity. Also draw scatter diagrams to illustrate
(a) strong positive correlation,
(b) strong negative correlation,
(c) independent data,
(d) data that are uncorrelated but not independent.
\end{framed}

\begin{enumerate}
\item
Given paired data ${\displaystyle \left\{(x_{1},y_{1}),\ldots ,(x_{n},y_{n})\right\}} $ consisting of n  pairs, $ {\displaystyle r_{xy}}$ is defined as: 



\[  {\displaystyle r_{xy}={\frac {\sum _{i=1}^{n}(x_{i}-{\bar {x}})(y_{i}-{\bar {y}})}{{\sqrt {\sum _{i=1}^{n}(x_{i}-{\bar {x}})^{2}}}{\sqrt {\sum _{i=1}^{n}(y_{i}-{\bar {y}})^{2}}}}}}  \]



 
where:
\begin{itemize}
    \item ${\displaystyle n}$   is sample size
    \item $ {\displaystyle x_{i},y_{i}}  x_{i},y_{i}$ are the individual sample points indexed with i
    \item $ {\displaystyle {\bar {x}}={\frac {1}{n}}\sum _{i=1}^{n}x_{i}}$ (the sample mean); and analogously for $ {\displaystyle {\bar {y}}}  $
\end{itemize}


This explains the strength of linear relationship between the xi and yi, with
$r = \pm 1$ showing linearity and r = 0 showing no linear relationship. The
underlying X and Y are both random variables.
\begin{enumerate}[(i)]
\item r near to +1, small amount of scatter about an (increasing) linear
relationship
\item  r near to –1, y decreases as x increases, otherwise as in (a)
Continued on next page
\item  Independent data ($r \approx 0$)
\item Non-linear relationship, e.g. y = x2
\end{enumerate}

\newpage

\begin{framed}
(ii) (a) The following two pages (pages 9 and 10) show edited Minitab
analysis of the relationship between cholesterol level and age for a
sample of 9 males. Name the statistical model which is being fitted,
identify the independent and dependent variables and state the usual
assumptions made about the data.
(b) Use the output to calculate the correlation between "chol" and "age",
and the correlation between "newchol" and "newage", and give a
reason for the difference between these two correlations.
(c) Use the output to assess whether the constant term in the model is
statistically significant. What would omission of the constant term
imply for a new-born infant?
(d) Which of the ("chol", "age") and ("newchol", "newage") analyses do
you prefer as a summary of the data, and why?
(12)

\end{framed}
\item  (a) Simple linear regression of y = cholesterol on x = age. y is the
dependent variable, x the independent. Assume a linear relationship
underlying the data, $Yi = a + bxi + \epsilon$
i, where the $\{ \epsilon_i\}$
 are independent
identically distributed N(0, σ 2) random variables with σ 2 constant for
all i.
\begin{enumerate}[(i)]
\item $r = \sqrt(0.323)$ = 0.568 for 'chol' and 'age'.
$r = \sqrt(0.940)$ = 0.970 for 'newchol' and 'newage'.
\begin{itemize}
\item The latter consists of the 8 data points omitting the observation at
x = 27 which seems very far from the roughly linear pattern of the rest.
\item Omitting it has made a linear relationship seem much more plausible.
\item Subject number 2 has very high cholesterol for his age.
\end{itemize}


\item Using the "constant" row in either set of output, the constant term is
not significantly different from 0. 
\begin{itemize}
    \item A model omitting a could perhaps
be used.
   \item This would imply cholesterol 0 at age 0, which might not be very
sensible – but we do not actually have data in that region, so we
cannot claim that a linear relationship still holds.
   \item There is a tendency towards a curved relationship even when the very
"unusual" observation at age 27 is omitted. 
   \item The fit of a line without
that observation is however much better than with it, and the
diagnostic plots, of residuals and Normal probability, seem acceptable.
\end{itemize}

\end{enumerate}

\end{enumerate}
%%%%%%%%%%%%%%%%%%%%%%%%%%%%%%%%%%%%%%%

\begin{framed}
\begin{verbatim}
Cholesterol (mg/ml)(c1) 2.20 3.55 2.40 2.55 2.85 2.95 3.05 3.10 3.45
Age (years) (c2) 26 27 28 29 31 33 35 38 41
MTB > gstd
MTB > name c1 'chol' c2 'age'
MTB > plot c1 c2
- *
3.50+ *
-
chol -
-
- * *
3.00+ *
- *
-
-
- *
2.50+
- *
-
- *
-
------+---------+---------+---------+---------+---------+age
27.0 30.0 33.0 36.0 39.0 42.0
\end{verbatim}
\end{framed}
%%%%%%%%%%%%%%%%%%%%%%%%%%%%%%%%%%%%%%%

\begin{framed}
\begin{verbatim}

MTB > regress c1 1 c2
The regression equation is chol = 1.30 + 0.0500 age
Predictor Coef StDev T P
Constant 1.3000 0.8851 1.47 0.185
age 0.05000 0.02734 1.83 0.110
S = 0.4000 R-Sq = 32.3% R-Sq(adj) = 22.7%
Analysis of Variance
Source DF SS MS F P
Regression 1 0.5350 0.5350 3.34 0.110
Error 7 1.1200 0.1600
Total 8 1.6550
Unusual Observations
Obs age chol Fit StDev Fit Residual St Resid
2 27.0 3.550 2.650 0.191 0.900 2.56R
MTB > copy c1 c2 c3 c4; SUBC> omit row 2.
MTB > name c3 'newchol' c4 'newage'
MTB > plot c3 c4
3.50+ *
-
newchol -
-
- * *
3.00+ *
- *
-
-
- *
2.50+
- *
-
- *
-
2.00+
------+---------+---------+---------+---------+---------+newage
27.0 30.0 33.0 36.0 39.0 42.0
\end{verbatim}
\end{framed}
%%%%%%%%%%%%%%%%%%%%%%%%%%%%%%%%%%%%%%%
\newpage
\begin{framed}
\begin{verbatim}
MTB > regress c3 1 c4; SUBC> resi c5.
The regression equation is newchol = 0.299 + 0.0772 newage
Predictor Coef StDev T P
Constant 0.2989 0.2629 1.14 0.299
newage 0.077236 0.007971 9.69 0.000
S = 0.1087 R-Sq = 94.0% R-Sq(adj) = 93.0%
Analysis of Variance
Source DF SS MS F P
Regression 1 1.1088 1.1088 93.88 0.000
Error 6 0.0709 0.0118
Total 7 1.1797
MTB > name c5 'resnew'
MTB > plot c5 c4
resnew - *
-
-
0.10+ *
-
-
- *
- *
0.00+
- *
-
- *
-
-0.10+ *
-
- *
-
------+---------+---------+---------+---------+---------+newage
27.0 30.0 33.0 36.0 39.0 42.0
\end{verbatim}
\end{framed}
%%%%%%%%%%%%%%%%%%%%%%%%%%%%%%%%%%%%%%%
\newpage
\begin{framed}
\begin{verbatim}
MTB > nsco c5 c6
MTB > plot c5 c6; SUBC> title 'Normal probability plot'.
Normal probability plot
resnew - *
-
-
0.10+ *
-
-
- *
- *
0.00+
- *
-
- *
-
-0.10+ *
-
- *
-
--------+---------+---------+---------+---------+--------C6
-1.20 -0.60 0.00 0.60 1.20
\end{verbatim}
\end{framed}
\end{document}

