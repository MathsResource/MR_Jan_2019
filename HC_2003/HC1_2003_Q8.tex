\documentclass[a4paper,12pt]{article}
%%%%%%%%%%%%%%%%%%%%%%%%%%%%%%%%%%%%%%%%%%%%%%%%%%%%%%%%%%%%%%%%%%%%%%%%%%%%%%%%%%%%%%%%%%%%%%%%%%%%%%%%%%%%%%%%%%%%%%%%%%%%%%%%%%%%%%%%%%%%%%%%%%%%%%%%%%%%%%%%%%%%%%%%%%%%%%%%%%%%%%%%%%%%%%%%%%%%%%%%%%%%%%%%%%%%%%%%%%%%%%%%%%%%%%%%%%%%%%%%%%%%%%%%%%%%
\usepackage{eurosym}
\usepackage{vmargin}
\usepackage{amsmath}
\usepackage{graphics}
\usepackage{epsfig}
\usepackage{enumerate}
\usepackage{multicol}
\usepackage{subfigure}
\usepackage{fancyhdr}
\usepackage{listings}
\usepackage{framed}
\usepackage{graphicx}
\usepackage{amsmath}
\usepackage{chngpage}
%\usepackage{bigints}

\usepackage{vmargin}
% left top textwidth textheight headheight
% headsep footheight footskip
\setmargins{2.0cm}{2.5cm}{16 cm}{22cm}{0.5cm}{0cm}{1cm}{1cm}
\renewcommand{\baselinestretch}{1.3}

\setcounter{MaxMatrixCols}{10}
\begin{document}

Higher Certificate, Paper I, 2003. Question 8
(i)
( )( )
( ) ( )
1
2 2
1 1
n
i i
i
n n
i i
i i
x x y y
r
x x y y
=
= =
− −
=
− −
Σ
Σ Σ
.
This explains the strength of linear relationship between the xi and yi, with
r = ±1 showing linearity and r = 0 showing no linear relationship. The
underlying X and Y are both random variables.
(a) r near to +1, small amount of scatter about an (increasing) linear
relationship
(b) r near to –1, y decreases as x increases, otherwise as in (a)
Continued on next page
(c) Independent data (r ≈ 0)
(d) Non-linear relationship, e.g. y = x2
(ii) (a) Simple linear regression of y = cholesterol on x = age. y is the
dependent variable, x the independent. Assume a linear relationship
underlying the data, Yi = a + bxi + ε
i, where the { ε
i} are independent
identically distributed N(0, σ 2) random variables with σ 2 constant for
all i.
(b) r = √(0.323) = 0.568 for 'chol' and 'age'.
r = √(0.940) = 0.970 for 'newchol' and 'newage'.
The latter consists of the 8 data points omitting the observation at
x = 27 which seems very far from the roughly linear pattern of the rest.
Omitting it has made a linear relationship seem much more plausible.
Subject number 2 has very high cholesterol for his age.
Continued on next page
(c) Using the "constant" row in either set of output, the constant term is
not significantly different from 0. A model omitting a could perhaps
be used.
This would imply cholesterol 0 at age 0, which might not be very
sensible – but we do not actually have data in that region, so we
cannot claim that a linear relationship still holds.
(d) There is a tendency towards a curved relationship even when the very
"unusual" observation at age 27 is omitted. The fit of a line without
that observation is however much better than with it, and the
diagnostic plots, of residuals and Normal probability, seem acceptable.
\end{document}
