\documentclass[a4paper,12pt]{article}
%%%%%%%%%%%%%%%%%%%%%%%%%%%%%%%%%%%%%%%%%%%%%%%%%%%%%%%%%%%%%%%%%%%%%%%%%%%%%%%%%%%%%%%%%%%%%%%%%%%%%%%%%%%%%%%%%%%%%%%%%%%%%%%%%%%%%%%%%%%%%%%%%%%%%%%%%%%%%%%%%%%%%%%%%%%%%%%%%%%%%%%%%%%%%%%%%%%%%%%%%%%%%%%%%%%%%%%%%%%%%%%%%%%%%%%%%%%%%%%%%%%%%%%%%%%%
\usepackage{eurosym}
\usepackage{vmargin}
\usepackage{amsmath}
\usepackage{graphics}
\usepackage{epsfig}
\usepackage{enumerate}
\usepackage{multicol}
\usepackage{subfigure}
\usepackage{fancyhdr}
\usepackage{listings}
\usepackage{framed}
\usepackage{graphicx}
\usepackage{amsmath}
\usepackage{chngpage}
%\usepackage{bigints}

\usepackage{vmargin}
% left top textwidth textheight headheight
% headsep footheight footskip
\setmargins{2.0cm}{2.5cm}{16 cm}{22cm}{0.5cm}{0cm}{1cm}{1cm}
\renewcommand{\baselinestretch}{1.3}

\setcounter{MaxMatrixCols}{10}
\begin{document}

Higher Certificate, Paper III, 2003. Question 6


%%%%%%%%%%%%%%%%%%%%%%%%%%%%%%%%%%%%%%%%%%%%%%%%%%%%%%%%%%%%%%%%%%%%%%%%
\begin{framed}
6.
Consider the following three situations (I), (II) and (III) concerned with surveys of
subscribers to two journals A and B. Each journal had a very large number of
subscribers.
(I)
In a random sample of 150 subscribers to journal A, 84 agreed with the
statement "This journal is very useful to me in my work". In a random sample of 200
subscribers to journal B, 126 agreed with the same statement.
(II)
In the same random sample of 150 subscribers to journal A, 45 agreed with the
statement "This journal is quite useful to me in my work" and 21 agreed with the
statement "This journal is not at all useful to me in my work".
(III) In the same random sample of 200 subscribers to journal B, 43 agreed with the
statement "The current issue of this journal is not at all useful to me in my work" and
56 agreed with the statement "The previous issue of this journal is not at all useful to
me in my work".
(i) Obtain an approximate 95% confidence interval for the difference between the
proportion of subscribers to journal A agreeing with the statement "This
journal is very useful to me in my work" and the proportion of subscribers to
journal B agreeing with this statement (situation I). Comment.
(7)
(ii) Explain carefully why the method used in part (i) is not appropriate for
obtaining a confidence interval for the difference between the proportion of
subscribers to journal A agreeing with the statement "This journal is quite
useful to me in my work" and the proportion of subscribers to journal A
agreeing with the statement "This journal is not at all useful to me in my work"
(situation II).
(4)
(iii) Explain carefully why the method used in part (i) is not appropriate for
obtaining a confidence interval for the difference between the proportion of
subscribers to journal B agreeing with the statement "The current issue of this
journal is not at all useful to me in my work" and the proportion of subscribers
to journal B agreeing with the statement "The previous issue of this journal is
not at all useful to me in my work" (situation III).
(4)
(iv) On the basis of the sample result given in situation I above, how large a sample
of subscribers to journal B would be needed in order that the investigators
would have 95% confidence that the estimate of the proportion of subscribers
to journal B agreeing with the statement "This journal is very useful to me in
my work" is no more than 0.05 different from the true proportion?
(5)
7

\end{framed}
%%%%%%%%%%%%%%%%%%%%%%%%%%%%%%%%%%%%%%%%%%%%%%%%%%%%%%%%%%%%%%%%%%%%%%%%
\begin{enumerate}[(a)]
\item  84 126
ˆ 150 , 150. ˆ 200 , 200. A A B B p = n = p = n =
95% limits for the true value of (pA – pB) are estimated as
( ) ˆ (1 ˆ ) ˆ (1 ˆ )
ˆ ˆ 1.96 A A B B
A B
A B
p p p p
p p
n n
− −
− ± + ,
i.e. (0.56 0.63) 1.96 0.56 0.44 0.63 0.37
150 200
− ± × + ×
= −0.07 ±1.96 0.002808 = −0.07 ± 0.104 ,
i.e. (–0.174, 0.034).
The interval contains zero, so we should not claim that one journal is
significantly better than the other. However, with 95% confidence, we may
claim that the difference between them ranges from 3.4% in favour of A to
17.4% in favour of B.
%%%%%%%%%%%%%%%%%%%%%%%%%%%%%%%%%%%%%%%%%%%%%%%%
\item The two statements are alternatives, which together with (I) make up the
responses of the whole sample. Hence they are not independent, and the test
in part (i) assumes that they are.
%%%%%%%%%%%%%%%%%%%%%%%%%%%%%%%%%%%%%%%%%%%%%%%%
\item Although the statements relate to two different issues of the journal, they are
answered by (at least some of) the same people, and so once again the
responses do not come from independent samples. The two proportions will
most likely be correlated.
%%%%%%%%%%%%%%%%%%%%%%%%%%%%%%%%%%%%%%%%%%%%%%%%
\item Suppose n is the required sample size. Using ˆp = 0.63, as in part (i), Var( ˆp )
is estimated as (0.63)(0.37)/n. An approximate value for n is found by making
0.05 =1.96×SE( pˆ ) . This gives
( ) ( )( ) 0.05 2 0.63 0.37 Var ˆ
1.96
p
n
  = =  
 
,
( )( )
2 i.e. 0.63 0.37 1.96 358.2
0.05
n =   =
 
,
so at least 359 responses are needed.
Because this is a large sample, and p is not too far from ½, this approximation
will be satisfactory. Also we are told that there are a very large number of
subscribers, so that a "finite population correction" is unnecessary (even if we
knew N).
\end{enumerate}
\end{document}
