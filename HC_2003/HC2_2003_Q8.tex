\documentclass[a4paper,12pt]{article}
%%%%%%%%%%%%%%%%%%%%%%%%%%%%%%%%%%%%%%%%%%%%%%%%%%%%%%%%%%%%%%%%%%%%%%%%%%%%%%%%%%%%%%%%%%%%%%%%%%%%%%%%%%%%%%%%%%%%%%%%%%%%%%%%%%%%%%%%%%%%%%%%%%%%%%%%%%%%%%%%%%%%%%%%%%%%%%%%%%%%%%%%%%%%%%%%%%%%%%%%%%%%%%%%%%%%%%%%%%%%%%%%%%%%%%%%%%%%%%%%%%%%%%%%%%%%
\usepackage{eurosym}
\usepackage{vmargin}
\usepackage{amsmath}
\usepackage{graphics}
\usepackage{epsfig}
\usepackage{enumerate}
\usepackage{multicol}
\usepackage{subfigure}
\usepackage{fancyhdr}
\usepackage{listings}
\usepackage{framed}
\usepackage{graphicx}
\usepackage{amsmath}
\usepackage{chngpage}
%\usepackage{bigints}

\usepackage{vmargin}
% left top textwidth textheight headheight
% headsep footheight footskip
\setmargins{2.0cm}{2.5cm}{16 cm}{22cm}{0.5cm}{0cm}{1cm}{1cm}
\renewcommand{\baselinestretch}{1.3}

\setcounter{MaxMatrixCols}{10}
\begin{document}

Higher Certificate, Paper II, 2003. Question 8
\begin{enumerate}[(a)]
\item (i) The F distribution with m and n degrees of freedom is the ratio of two
independent chi-squared distributions divided by their numbers of degrees of
freedom:
2
, 2
χ
χ
m
m n
n
F m
n
= .
Thus if estimates of variance s1
2 and s2
2 are obtained from each of two
independent samples of Normally distributed data, (m+1) and (n+1) items in
the samples respectively, the null hypothesis that the underlying population
variances σ 1
2 and σ 2
2 are equal can be tested by referring s1
2/s2
2 to Fm,n. A
confidence interval for σ 1
2/σ 2
2 can also be found.
In analysis of variance, sums of squares will have χ2 distributions if Normality
is assumed. A suitable null hypothesis is set up, such as that a regression
coefficient is zero. The appropriate sum of squares is then compared with the
residual sum of squares, using the "mean squares" so as to deal with the
numbers of degrees of freedom, using an F distribution. If the null hypothesis
is true, both the mean squares will merely estimate experimental error so their
ratio has expected value 1, but if it is false the mean square corresponding to
the regression coefficient is expected to be larger. One-way and two-way (and
higher-way) analysis of variance for experimental designs uses the same
background theory.
\item  The null hypothesis is "σA
2 = σB
2", the same variability for both machines, and
the alternative hypothesis is "σA
2 ≠ σB
2".
Summary statistics for the two machines are:
2
1 1 1 A: n =16, x =1000.125, s =10.9167 (s = 3.304) ;
2
2 2 2 B : n = 20, x =1000.050, s = 2.9974 (s =1.731) .
s1
2/s2
2 = 3.64, refer to F15,19. This is significant at the 5% level so the null
hypothesis can be rejected and we can decide to use B because the evidence is
that it is less variable than A.
A report should mention that both machines dispense, on average, just over
1000 ml. However, the data from the trial period show that A was
significantly more variable than B (in fact, inspection of the data shows two
rather outlying values in A, at 993 and 1007), so B is the one to buy.
\end{enumerate}
\end{document}
