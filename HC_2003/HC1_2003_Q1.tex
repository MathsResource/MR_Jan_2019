\documentclass[a4paper,12pt]{article}
%%%%%%%%%%%%%%%%%%%%%%%%%%%%%%%%%%%%%%%%%%%%%%%%%%%%%%%%%%%%%%%%%%%%%%%%%%%%%%%%%%%%%%%%%%%%%%%%%%%%%%%%%%%%%%%%%%%%%%%%%%%%%%%%%%%%%%%%%%%%%%%%%%%%%%%%%%%%%%%%%%%%%%%%%%%%%%%%%%%%%%%%%%%%%%%%%%%%%%%%%%%%%%%%%%%%%%%%%%%%%%%%%%%%%%%%%%%%%%%%%%%%%%%%%%%%
\usepackage{eurosym}
\usepackage{vmargin}
\usepackage{amsmath}
\usepackage{graphics}
\usepackage{epsfig}
\usepackage{enumerate}
\usepackage{multicol}
\usepackage{subfigure}
\usepackage{fancyhdr}
\usepackage{listings}
\usepackage{framed}
\usepackage{graphicx}
\usepackage{amsmath}
\usepackage{chngpage}
%\usepackage{bigints}

\usepackage{vmargin}
% left top textwidth textheight headheight
% headsep footheight footskip
\setmargins{2.0cm}{2.5cm}{16 cm}{22cm}{0.5cm}{0cm}{1cm}{1cm}
\renewcommand{\baselinestretch}{1.3}

\setcounter{MaxMatrixCols}{10}
\begin{document}

Higher Certificate, Paper I, 2003. Question 1
\begin{enumerate}
\item Any of 0, 1, …, 9 can occur in each of the six positions, so the number
is 106 = 1000000.
(b) $10 × 9 × 8 × 7 × 6 × 5 = 151200$, since no repetition is allowed.
[Alternatively, ( )
10! 10!
10 6 ! 4!
=
−
as above.]
(c) There are
10
6
 
 
 
choices of six different digits from ten, each of which
can be used in only one of its possible orders. So the number is
10 10! 210
6 6! 4!
 
  = =
 
.
(d) Here there are
10
3
 
 
 
choices of digits, for each of which there are
6! 90
2!2! 2!
= orders, so the number is
10
90
3
 
 ×
 
= 10800.
\item  (a) There are 3! = 6 possible orders for the first three digits, and 1 order
(the reverse order of the first three digits) for the last three. So there
are 6 codes.
(b) If one digit is used 4 times in a palindromic code, another must be used
twice. These digits may be chosen in 3 and 2 ways respectively, i.e. in
6 ways for the pair. Once the digits have been chosen, only 3 patterns
are possible; for example, say the digits are 1 and 2, then the possible
patterns are 1 1 2 2 1 1, 1 2 1 1 2 1 and 2 1 1 1 1 2. So the total
number of codes is $6 × 3 = 18$.
(c) There are 3 choices of digit and only one possible pattern for each; so
there are 3 codes.
\item  Using the patterns from part (ii), there are
10
3
 
 
 
choices of digits for (a), each
giving 6 codes, i.e.
10
3
 
 
 
× 6 = 720 altogether. For (b), there are
10
2
 
 
 
choices
of digits, i.e. 45, with 3 patterns as above, in which the two digits can be used
in 2 ways (4 of 1 and 2 of 2 or vice versa), giving 45 × 3 × 2 = 270 ways. And
(c) can occur in 10 ways (because any digit can be used 6 times). Thus the
total is 720 + 270 + 10 = 1000 ways.
ALTERNATIVELY, the first three positions may each be filled in 10 ways,
and then the whole sequence is determined, so there are 103 = 1000 ways.
\end{enumerate}
\end{document}