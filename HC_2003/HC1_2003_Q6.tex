\documentclass[a4paper,12pt]{article}
%%%%%%%%%%%%%%%%%%%%%%%%%%%%%%%%%%%%%%%%%%%%%%%%%%%%%%%%%%%%%%%%%%%%%%%%%%%%%%%%%%%%%%%%%%%%%%%%%%%%%%%%%%%%%%%%%%%%%%%%%%%%%%%%%%%%%%%%%%%%%%%%%%%%%%%%%%%%%%%%%%%%%%%%%%%%%%%%%%%%%%%%%%%%%%%%%%%%%%%%%%%%%%%%%%%%%%%%%%%%%%%%%%%%%%%%%%%%%%%%%%%%%%%%%%%%
  \usepackage{eurosym}
\usepackage{vmargin}
\usepackage{amsmath}
\usepackage{graphics}
\usepackage{epsfig}
\usepackage{enumerate}
\usepackage{multicol}
\usepackage{subfigure}
\usepackage{fancyhdr}
\usepackage{listings}
\usepackage{framed}
\usepackage{graphicx}
\usepackage{amsmath}
\usepackage{chngpage}
%\usepackage{bigints}

\usepackage{vmargin}
% left top textwidth textheight headheight
% headsep footheight footskip
\setmargins{2.0cm}{2.5cm}{16 cm}{22cm}{0.5cm}{0cm}{1cm}{1cm}
\renewcommand{\baselinestretch}{1.3}

\setcounter{MaxMatrixCols}{10}
\begin{document}
Higher Certificate, Paper I, 2003. Question 6

\[E(Y) = np Var(Y) = np(1 - p)\]
\begin{enumerate}
\item Binomial with n = 48, p = 0.25.
\item Score is distributed 36 + B(12, 0.25).

\begin{description}
\item[(a)] Hence mean correct is 36 + (12/4) = 39 and variance is $12 \times 0.25 \times 0.75 = 9/4$.
\item[(b)] Number wrong is distributed $B(12, 0.75)$.
\item[(c)] The required probability is $1 - P(0) - P(1) - P(2)$ based on the B(12, 0.25)
distribution. 
\end{description}

This is
\begin{eqnarray*}
1 - P(0) - P(1) - P(2) &=& 1- \left[\left(\frac{3}{4}\right)^{12}\right] - \left[12\left(\frac{1}{4}\right) \left(\frac{3}{4}\right)^{11} \right] - \left[\frac{12 \times 11}{2} \left(\frac{1}{4}\right)^{2} \left(\frac{3}{4}\right)^{10}\right]\\ \\ \bigskip
&=& 1- 0.031676 - 0.126705 - 0.232293 \\ \\ \smallskip
&=& 0.6093 .
\end{eqnarray*}
\item 
\begin{itemize}
\item Number of correct answers for A is distributed as 27 + B(21, 0.25).
\item Number of correct answers for B is distributed as 28 + B(20, 0.25).
\item Number of correct answers for C is distributed as 30 + B(18, 0.25).
\end{itemize}
%%%%%%%%%%%%%%%%%%%%%%%%%%%%%%%%%%%%%%%%
\begin{itemize}
\item Means are 
\begin{itemize}
\item 27 + (21/4) = 32.25, 
\item 28 + (20/4) = 33, 
\item 30 + (18/4) = 34.5
\end{itemize}  
respectively.

%%%%%%%%%%%%%%%%%%%%%%%%%%%%%%%%%%%%%%%%
\item  Variances are 
\begin{itemize}
\item (21)(0.25)(0.75) = 63/16, 
\item (20)(0.25)(0.75) = 60/16 = 15/4, 
\item (18)(0.25)(0.75) = 54/16 = 27/8 
\end{itemize}  
  respectively.
 
%%%%%%%%%%%%%%%%%%%%%%%%%%%%%%%%%%%%%%%%
\item So overall mean is 
\[ \frac{1}{3}  (32.25 + 33  + 34.5) = 33.25,\]
and variance of overall mean is 
\[ \frac{1}{9}  (\frac{63}{16} + \frac{14}{4}  + \frac{27}{8}) = 1.2292\] 

%%%%%%%%%%%%%%%%%%%%%%%%%%%%%%%%%%%
( ) ( ) ( )
( ) ( ) ( )
, ,
29 29 ; 1 for , ,
29 3
i A B C
P APA
PA Pi i A B C
P iPi
=
  = = = Σ .
( ) ( )
19 2
1
4
29 B 21, 2 21 20 3 1
2 4 4
P A = P  =  = \times        
( ) ( )
19
1
4
29 B 20, 1 20 3 1
4 4
P B = P  =  =        
( ) ( 1 )
4 P 29 C = P B 18, = -1 = 0
\item  Note. C must get at least the 30 he knows, so it must follow that
  P(C 29) = 0 , which is true if P(29 C) = 0 .
So ( )
19 2
19 2 19
1 .21.20 3 1
29 2 4 4
1 .21.20 3 1 20 3 1
2 4 4 4 4
P A
   
   
=    
    +            
       
21
32 21 0.7241 21 1 29
32 4
= = =
  +
  and similarly
( )
1
29 4 8 0.2759 21 1 29
32 4
P B = = =
  +
  (and P(C 29) = 0 , see above).
\end{itemize}

%%%%%%%%%%%%%%%%%%%%%%
\end{enumerate}
\end{document}