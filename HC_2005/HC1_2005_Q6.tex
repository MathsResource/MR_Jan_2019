\documentclass[a4paper,12pt]{article}

%%%%%%%%%%%%%%%%%%%%%%%%%%%%%%%%%%%%%%%%%%%%%%%%%%%%%%%%%%%%%%%%%%%%%%%%%%%%%%%%%%%%%%%%%%%%%%%%%%%%%%%%%%%%%%%%%%%%%%%%%%%%%%%%%%%%%%%%%%%%%%%%%%%%%%%%%%%%%%%%%%%%%%%%%%%%%%%%%%%%%%%%%%%%%%%%%%%%%%%%%%%%%%%%%%%%%%%%%%%%%%%%%%%%%%%%%%%%%%%%%%%%%%%%%%%%

\usepackage{eurosym}
\usepackage{vmargin}
\usepackage{amsmath}
\usepackage{graphics}
\usepackage{epsfig}
\usepackage{enumerate}
\usepackage{multicol}
\usepackage{subfigure}
\usepackage{fancyhdr}
\usepackage{listings}
\usepackage{framed}
\usepackage{graphicx}
\usepackage{amsmath}
\usepackage{chngpage}

%\usepackage{bigints}
\usepackage{vmargin}

% left top textwidth textheight headheight

% headsep footheight footskip

\setmargins{2.0cm}{2.5cm}{16 cm}{22cm}{0.5cm}{0cm}{1cm}{1cm}

\renewcommand{\baselinestretch}{1.3}

\setcounter{MaxMatrixCols}{10}

\begin{document}Higher Certificate, Paper I, 2005. Question 6
%%%%%%%%%%%%%%%%%%%%%%%%%%%%%%%%%%%%%%%%%%%%%%%%%%%%%%%%%%%%%%%%%%
\begin{framed}
The random variable $X$ has the geometric probability mass function (pmf) given by
$f(x)$, where

\[ f(x)  = (1-p)^x p  \qquad \mbox{ for } x=0,1,2,3,\ldots, \mbox{ and }0\leq p \leq 1\]

(i) Sketch the graph of $f(x)$ for the case $p = 1/3$, for $0 \leq x \leq 5$.

(ii) Show that the probability generating function of X is given by $G(s)$ where
\[ G(s) = \frac{p}{1-(1-p)s} , \qquad |s| < \frac{1}{1-p} \]
and hence or otherwise obtain the mean and variance of $X$.
\end{framed}
%----------------------------------------------------------------%

%%%%%%%%%%%%%%%%%%%%%%%%%%%%%%%%%%%%%%%%%%%%%%%%%%%%%%%%%%%%%%%%%%

\begin{enumerate}
    \item Probability generating function G(s) is

\begin{eqnarray*}
G(S) &=& E[S^X] \\
  &=& \sum^{\infty}_{x=0} S^X(1-p)^xp \\
  &=& p \sum^{\infty}_{x=0} [(1-p)s]^x\\
   &=& \frac{1}{1-(1-p)s} \\
\end{eqnarray*}

This requires $ {\displaystyle|s| < \frac{1}{1-p} }$ for convergence.

The mean is given by $E(X) = G^{\prime}(1)$.
We have $ {\displaystyle G^{\prime}(s) = p\left[ \frac{1-p}{[1-(1-p)s]^2} \right] }$
and inserting $s=1$ gives 
\[ G^{\prime}(1) = \frac{p(1-p)}{p^2},\] the 
means is ${ \displaystyle \frac{1-p}{p} }.$

\begin{eqnarray*}
G^{\prime}(S)     
  &=& p \frac{1}{[1-(1-p)s]^2} \\
\end{eqnarray*}


\begin{eqnarray*}
G^{\prime\prime}(S)     
  &=&  \frac{2p(1-p)^2}{[1-(1-p)s]^3} \\
\end{eqnarray*}



The variance is given by $Var(X) = G^{\prime\prime}(1) + \mbox{mean} – \mbox{mean}^2$. ( ) ( )
\[ G^{\prime\prime}(s) = \frac{2p(1-p)^2}{[1-(1-p)s]^3},\]
so that


\[ G^{\prime\prime}(1) = \frac{2p(1-p)^2}{p^2},\]
The variance is given by
\begin{eqnarray*}
Var(X)    
&=&  \frac{2p(1-p)^2}{p^2}  + \frac{1-p}{p} - \left( \frac{1-p}{p}  \right)^2 \\
&=&  \frac{1}{p^2}\left( (1-p)^2 + p(1-p)  \right) \\
&=& \frac{1-p}{p^2} \\
\end{eqnarray*}

1/3
2/9
4/27
P(X = x)
0 1 2 3 4 5
x
%%%%%%%%%%%%%%%%%%%%%%%%%%%%%%%%%%%%%%%%%%%%%%%%%%%%%%
%----------------------------------------------------------------%
\begin{framed}
(iii) For any non-negative integer x, show that $P(X \geq x) = (1-p)^x$, and deduce
that for any non-negative integers $l$ and $m$
\[P( X \geq l + m| X \geq l ) = P( X \geq m).\] 
Interpret this result.
\end{framed}
%----------------------------------------------------------------%

\item  

$P(X \geq x) = \sum^{\infty}_{r=x}(1-p)^r$ (i.e. geometric series)

\begin{eqnarray*}
P(X \geq x) &=& \frac{p(1-p)^x}{1-(1-p)}\\
&=& (1-p)^x
\end{eqnarray*}
for $x=0,1,2,3,\ldots$.

We now use ${ \displaystyle P(A|B) = \frac{P(A \cap B)}{P(B)}  }$
and take the event $A$ as ``$X \geq l+m$`` and
the event $B$ as ``$X \geq l$`, so that $A \capB = A$.
Thus

\begin{eqnarray*}
P(X \geq l+m|X\geq l) &=& 
\frac{(1-p)^{l+m}}{(1-p)^l}\\
&=&(1-p)^m \\
&=&P(X \geq m)
\end{eqnarray*}

This is the ``lack of memory`` property of a geometric distribution.

%----------------------------------------------------------------%
\begin{framed}
(iv) The random variable Y has pmf g(y), where
\[ g(y) = (1- \theta)^y \theta, \qquad y={0,1,2,\ldots} , 0 < \theta < 1\]

X and Y are independent, and the random variable Z is defined as the minimum
of $X$ and $Y$, i.e. $Z = \mbox{min}(X, Y)$. 

By noting that $P(Z \geq z) = P(X \geq z and Y \geq z)$,
find an expression for $P(Z \geq z)$, where z is any non-negative integer. By
considering $P(Z \geq z) – P(Z \geq z + 1)$, or otherwise, show that

\[ P(Z = z) = [(1-p)(1-\theta)]^{z}(p+\theta - p\theta),\qquad z = 0,1,23,\ldots\]

Identify the form of this distribution and hence write down $E(Z)$ and $Var(Z)$.

\end{framed}

\item  By independence, 


\begin{eqnarray*}
P(Z \geq z) 
&=& P(X \geq z) P(Y \geq z) \\
&=& (1-p)^z(1-\theta)^z\\
\end{eqnarray*}

\begin{eqnarray*}
P(Z = z) 
&=& P(Z \geq z) + P(Z \geq z+1) \\
&=& [(1-p)(1-\theta)]^z + [(1-p)(1-\theta)]^{z+1}\\
&=& [(1-p)(1-\theta)]^z (1-1+p+\theta - p\theta)\\
&=& [(1-p)(1-\theta)]^z (p+\theta - p\theta)\\
\end{eqnarray*}

\begin{itemize}
\item This is a geometric distribution as given at the start of the question with p replaced by
$p + θ – p \theta$. 
\item Hence, from part (ii),
\[ E(X) = \frac{p+\theta - p\theta}{p+\theta-p \theta} \qquad \mbox{ and }\qquad Var(X) = \frac{1- (p+\theta - p\theta)}{(p+\theta-p \theta)^2}\]
\end{itemize}

\end{enumerate}

\end{document}
