\documentclass[a4paper,12pt]{article}

%%%%%%%%%%%%%%%%%%%%%%%%%%%%%%%%%%%%%%%%%%%%%%%%%%%%%%%%%%%%%%%%%%%%%%%%%%%%%%%%%%%%%%%%%%%%%%%%%%%%%%%%%%%%%%%%%%%%%%%%%%%%%%%%%%%%%%%%%%%%%%%%%%%%%%%%%%%%%%%%%%%%%%%%%%%%%%%%%%%%%%%%%%%%%%%%%%%%%%%%%%%%%%%%%%%%%%%%%%%%%%%%%%%%%%%%%%%%%%%%%%%%%%%%%%%%

\usepackage{eurosym}
\usepackage{vmargin}
\usepackage{amsmath}
\usepackage{graphics}
\usepackage{epsfig}
\usepackage{enumerate}
\usepackage{multicol}
\usepackage{subfigure}
\usepackage{fancyhdr}
\usepackage{listings}
\usepackage{framed}
\usepackage{graphicx}
\usepackage{amsmath}
\usepackage{chngpage}

%\usepackage{bigints}
\usepackage{vmargin}

% left top textwidth textheight headheight

% headsep footheight footskip

\setmargins{2.0cm}{2.5cm}{16 cm}{22cm}{0.5cm}{0cm}{1cm}{1cm}

\renewcommand{\baselinestretch}{1.3}

\setcounter{MaxMatrixCols}{10}

\begin{document}
Higher Certificate, Paper I, 2005. Question 3
(i) P(T > t) = P(no failure in time t) = P(X = 0) = e−λ t .
∴ 1 – F(t) = e−λ t and so the pdf of T is d F (t ) d (e t ) e t
dt dt
= − −λ =λ −λ .
(ii) P(all n systems still functioning at time t) = ( )
1 1
n n
t nt
i
i i
P T t e−λ e− λ
= =
Π > =Π = .
∴ the pdf of Tmin is d (e n t ) n e n t
dt
− − λ = λ − λ (for t > 0).
Thus Tmin is exponential with parameter nλ.
(iii) P(all n systems have failed by time t) = ( ) ( )
1 1
1
n n
t
i
i i
P T t e−λ
= =
Π ≤ =Π −
(1 )= − e−λ t n , and this is P(Tmax) ≤ t.
∴ the pdf of Tmax is {( ) } ( ) 1 1 1 d e t n n e t e t n
dt
λ λ λ λ − − = − − − − (for t > 0).
(Note that this is not exponential.)
We now have n = 10 and λ = 0.002. We require t1 and t2 such that 1− e−nλ t1 = 0.05
and (1 2 ) 0.95. − e−λ t n =
∴ 1− e−0.02t1 = 0.05 , giving 0.02t1 = –log(0.95) = 0.051293 so that t1 = 2.565.
Also, 2 ( )1− e−0.002t = 0.95 1/10 = 0.994884 , giving 0.002t2 = –log(0.005116) = –5.2754
so that t2 = 2638.
\end{document}
