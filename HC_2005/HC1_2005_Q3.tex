\documentclass[a4paper,12pt]{article}

%%%%%%%%%%%%%%%%%%%%%%%%%%%%%%%%%%%%%%%%%%%%%%%%%%%%%%%%%%%%%%%%%%%%%%%%%%%%%%%%%%%%%%%%%%%%%%%%%%%%%%%%%%%%%%%%%%%%%%%%%%%%%%%%%%%%%%%%%%%%%%%%%%%%%%%%%%%%%%%%%%%%%%%%%%%%%%%%%%%%%%%%%%%%%%%%%%%%%%%%%%%%%%%%%%%%%%%%%%%%%%%%%%%%%%%%%%%%%%%%%%%%%%%%%%%%

\usepackage{eurosym}
\usepackage{vmargin}
\usepackage{amsmath}
\usepackage{graphics}
\usepackage{epsfig}
\usepackage{enumerate}
\usepackage{multicol}
\usepackage{subfigure}
\usepackage{fancyhdr}
\usepackage{listings}
\usepackage{framed}
\usepackage{graphicx}
\usepackage{amsmath}
\usepackage{chngpage}

%\usepackage{bigints}
\usepackage{vmargin}

% left top textwidth textheight headheight

% headsep footheight footskip

\setmargins{2.0cm}{2.5cm}{16 cm}{22cm}{0.5cm}{0cm}{1cm}{1cm}
\renewcommand{\baselinestretch}{1.3}
\setcounter{MaxMatrixCols}{10}

\begin{document}

Higher Certificate, Paper I, 2005. Question 3

%%%%%%%%%%%%%%%%%%%%%%%%%%%%%%%%%%%%%%%%%%%%%%%%%%%%%%%%%%%%%%%%%%
\begin{framed}
3. The failures of a communications system occur in a Poisson process with rate
parameter $\lambda$, so that the random variable $X$ giving the number of failures in time $t$
satisfies

\[P(X=x) =  exp(-\lambda t) \frac{(\lambda t)^x}{x!}, \qquad x=0,1,2,3, \ldots\]

(i) By considering the probability that there is no failure in time $t$, or otherwise,
show that the probability distribution of the time T to the next failure is given
by
\[P(T > t ) = exp(−\lambda t ), t > 0 ,\]
and deduce the probability density function (pdf) of $T$..
\end{framed}


%%%%%%%%%%%%%%%%%%%%%%%%%%%%%%%%%%%%%%%%%%%%%%%%%%%%%%%%%%%%%%%%%%
\begin{enumerate}[(a)]
\item \[P(T > t) = P(\mbox{no failure in time t}) = P(X = 0) = e^{-\lambda} t .\]
\[\therefore 1 - F(t) = e^{-\lambda} t \] and so the pdf of T is 

\[ \frac{dF (t )}{dt}  =  \frac{dF (e^{-\lambda\,t} )}{dt}  = \lambda\,e^{-\lambda\,t}\]


%%%%%%%%%%%%%%%%%%%%%%%%%%%%%%%%%%%%%%%%%%%%%%%%%%%%%%%%%%%%%%%%%%
\newpage
\begin{framed}
(ii) A bank of n identical but independent communications systems of the above type is set in operation at time 0, and $T_i$ denotes the time to failure of the $i-$th system, $i = 1, 2, \ldots, n$. Obtain an expression for the probability that all $n$ systems continue to function without failure for at least a time $t$. Deduce that the pdf of $T_{min} = min(T1, …, Tn)$, the time to the first failure in the bank of $n$ systems, is of exponential form, with a rate parameter given by a function of $n$ and $\lambda$ which should be stated.
\end{framed}
%----------------------------------------------------------------%
\item 
\begin{eqnarray*}
P(\mbox{all n systems still functioning at time t}) &=& \prod^{n}_{i=1} P(T_i > t )\\
 &=& \prod^{n}_{i=1}  e^{-\lambda\,t}\\  
 &=& e^{-n\lambda\,t} 
\end{eqnarray*} 

Therefor the pdf of Tmin is 
\[ \frac{dF (e^{-\lambda\,t} )}{dt}  = \lambda\,e^{-\lambda\,t}\] (for $t > 0$).
Thus Tmin is exponential with parameter $n\,\lambda$.

%%%%%%%%%%%%%%%%%%%%%%%%%%%%%%%%%%%%%%%%%%%%%%%%%%%%%%%%%%%%%%%%%%
\newpage
\begin{framed}
(iii) Show that the probability that all n systems in the bank have failed by time $t$ is
given by $(1− e−\lambda\;t )^n$ , and deduce the pdf of $T_{max} = max(T_1,\ldots, T_n)$, the time to
the last failure in the bank of n systems.
Given that $n = 10$ and $\lambda = 0.002$, find the times $t_1$ and $t_2$ such that
\[P(T_{min} < t_1) = P(T_{max} > t_2) = 0.05.\]

\end{framed}
%----------------------------------------------------------------%
\item  P(all n systems have failed by time t) = ( ) ( )

\begin{eqnarray*}
\prod^{n}_{i=1} P(T_i > t )
 &=& \prod^{n}_{i=1}  e^{-\lambda\,t}\\  
 &=& e^{-n\lambda\,t} \\ 
 &=& - e^{-\lambda \,t}  n , \\
\end{eqnarray*}

%%%%%%%%%%%%%%%%%%%%%%%%%%%%%%%

and this is $P(T_{max}) \leq t$.

Therefore the pdf of $T_{max}$ is 

\[ \frac{d}{dt}\left\{ \left(1-e^{-\lambda\,t} \right)^n  \right\}  =  n \lambda\,e^{-\lambda\,t} (1-e^{-\lambda\,t})^{n-1}\]


(for $t > 0$).
(Note that this is not exponential.)


\begin{itemize}
\item We now have $n = 10$ and $\lambda = 0.002$. 
\item We require $t_1$ and $t_2$ such that $1- e^{-n\lambda t_1} = 0.05$
and $(1- e^{-\lambda t_2})^n = 0.95$
  \item \therefore $1- e^{-n\lambda t_1} = 0.05$ , giving $0.02t_1 = -log(0.95) = 0.051293$ 
\item so that $t_1 = 2.565$.
\item Also, $(1- e^{-\lambda t_2}) = 0.95^{1/10} = 0.994884$ , giving $0.002t_2 = -log(0.005116) = -5.2754$
\item so that $t_2 = 2638$.
\end{itemize}

\end{enumerate}

\end{document}
