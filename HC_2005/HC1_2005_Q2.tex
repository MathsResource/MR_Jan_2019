Higher Certificate, Paper I, 2005. Question 2
Cycle ~ N(27, 6.25)
Bus ~ N(13, 20) Walk1 ~ N(7, 4) Walk2 ~ N(5, 1)
Car ~ N(23, 36)
The sum of independent N( μ
i, σ
i 2) distributions is N(Σ μ
i, Σ σ
i 2).
(i) The distribution of total journey time by bus is N(7+13+5, 4+20+1), i.e. N(25,25).
(ii) Cycle: (N(27,6.25) 30) 30 27 (1.2) 0.8849
6.25
P  −  < = Φ  = Φ =
 
.
Bus: (N(25, 25) 30) 30 25 (1) 0.8413
25
P  −  < = Φ  = Φ =
 
.
Car: (N(23,36) 30) 30 23 (1.1667) 0.8783
36
P  −  < = Φ  = Φ =
 
.
Cycling is best, with a probability of 0.8849.
(iii) Cycle: (N(27,6.25) 35) 1 35 27 1 (3.2) 0.0007
6.25
P  −  > = −Φ  = −Φ =
 
.
Bus: (N(25, 25) 35) 1 35 25 1 (2) 0.0228
25
P  −  > = −Φ  = −Φ =
 
.
Car: (N(23,36) 35) 1 35 23 1 (2) 0.0228
36
P  −  > = −Φ  = −Φ =
 
.
Again cycling is best, with a probability of 0.0007.
(iv) P(cycle) = 0.3 P(bus) = 0.3 P(car) = 0.4
( ) ( ) ( )
( )
30 cycle cycle
cycle 30
30
P P
P
P
<
< =
<
, and similarly for the other modes of travel.
P(< 30) = P(< 30 cycle)P(cycle) + P(< 30 bus)P(bus) + P(< 30 car)P(car)
= (0.8849×0.3) + (0.8413×0.3) + (0.8783×0.4)
= 0.26547 + 0.25239 + 0.35132 = 0.86918.
Hence P(cycle < 30) = 0.26547 / 0.86918 = 0.3054
P(bus < 30) = 0.25239 / 0.86918 = 0.2904
P(car < 30) = 0.35132 / 0.86918 = 0.4042 .
