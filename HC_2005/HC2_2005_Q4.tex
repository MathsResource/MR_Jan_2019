\documentclass[a4paper,12pt]{article}

%%%%%%%%%%%%%%%%%%%%%%%%%%%%%%%%%%%%%%%%%%%%%%%%%%%%%%%%%%%%%%%%%%%%%%%%%%%%%%%%%%%%%%%%%%%%%%%%%%%%%%%%%%%%%%%%%%%%%%%%%%%%%%%%%%%%%%%%%%%%%%%%%%%%%%%%%%%%%%%%%%%%%%%%%%%%%%%%%%%%%%%%%%%%%%%%%%%%%%%%%%%%%%%%%%%%%%%%%%%%%%%%%%%%%%%%%%%%%%%%%%%%%%%%%%%%

\usepackage{eurosym}
\usepackage{vmargin}
\usepackage{amsmath}
\usepackage{graphics}
\usepackage{epsfig}
\usepackage{enumerate}
\usepackage{multicol}
\usepackage{subfigure}
\usepackage{fancyhdr}
\usepackage{listings}
\usepackage{framed}
\usepackage{graphicx}
\usepackage{amsmath}
\usepackage{chngpage}

%\usepackage{bigints}
\usepackage{vmargin}

% left top textwidth textheight headheight

% headsep footheight footskip

\setmargins{2.0cm}{2.5cm}{16 cm}{22cm}{0.5cm}{0cm}{1cm}{1cm}

\renewcommand{\baselinestretch}{1.3}

\setcounter{MaxMatrixCols}{10}

\begin{document}
Higher Certificate, Paper II, 2005. Question 4
\begin{framed}


4. The table below appeared in a report on the use of serologic screening of blood samples for toxoplasmosis (Toxoplasma gondii).  The data are the results of two tests, the microscopic agglutination test (MAT) and the enzyme-linked immunosorbent assay (ELISA), on blood samples from 462 pigs. 
 
  ELISA   Positive Negative Positive 67   25 MAT Negative 41 329 
 
Source:  Georgiadis M.P., Johnson W.O., Gardner I.A. and Singh R. (2003).  Correlation-adjusted estimation of sensitivity and specificity of two diagnostic tests.  Applied Statistics, 52, 63–76. 
 
 
(i) Test the hypothesis that the probability of a positive test result is the same for the two tests.  
 
\end{framed}
%%%%%%%%%%%%%%%%%%%%%%%%%%%%%%%%%%%%%%%%%%%%%%%%%%%
The test is applied to a $2 \times 2 $ contingency table, which tabulates the outcomes of two tests on a sample of n subjects, as follows. 

\begin{center}
\begin{tabular}{|c|c|c|c|}
  & Test 2 positive &  Test 2 negative & Row total   \\ \hline
Test 1 positive &  a & b & a + b  \\ \hline
Test 1 negative & c & d & c + d   \\ \hline
Column total & a + c & b + d & n   \\ \hline
\end{tabular}
\end{center}

The null hypothesis of marginal homogeneity states that the two marginal probabilities for each outcome are the same, i.e. $p_a + p_b = p_a + p_c$ and $p_c + p_d = p_b + p_d$. 

Thus the null and alternative hypotheses are         \[{\displaystyle {\begin{aligned}H_{0}&:~p_{b}=p_{c}\\H_{1}&:~p_{b}\neq p_{c}\end{aligned}}}  \]
Here pa, etc., denote the theoretical probability of occurrences in cells with the corresponding label. 

The McNemar test statistic is: 
\[ {\displaystyle \chi ^{2}={(b-c)^{2} \over b+c}.}  \]


\begin{enumerate}[(a)]
    \item McNemar's test is required because the samples are paired.
Denoting the entries in the table by , the test statistic for McNemar's test is abcd()21bcbc−−+, with approximate null distribution , the null hypothesis here being that there is no difference between the proportions (probabilities) for the MAT and ELISA tests. (Notice that McNemar's test uses the information from the "discordant" cells of the table.) 21χ
Thus the test statistic is ()2225411153.409254166−−==+. This is referred to ; the upper 5\% point is 3.84, so there is insufficient evidence to say that there is a real difference. 21χ
\newpage
\begin{framed}
(ii) Obtain approximate 95\% confidence intervals for the proportion of positive test results for 
 (a) the MAT, 
 (b) the ELISA. 
 The researchers report that the true population proportion of positive blood samples is believed to be approximately 0.069.  State, giving brief reasons, whether either or both of the approximate 95\% confidence intervals you have calculated is consistent with this value. (11) 
 \end{framed}

    \item  Approximate 95\% confidence intervals for the proportion of positive test results given by each test use the whole data. For MAT, 92462ˆ0.1991Mp==; for ELISA, 108462ˆ0.2338Ep==.
    
\begin{enumerate}[(i)]
    \item  The estimated variance of ˆMp is $\displaystyle{\frac{(0.1991)(0.8009)}{462} = 0.000345135}$, so the estimated standard deviation is 0.0186.
    \begin{itemize}
        \item Thus a 95\% confidence interval for pM is given by, approximately, $0.1991 \pm (1.96)(0.0186)$, i.e. it is $(0.163, 0.236)$.
    \end{itemize} 
\item The estimated variance of ˆEp is (0.2338)(0.7662)/462 = 0.000387744, so the estimated standard deviation is 0.0197 
\begin{itemize}
\item Thus a 95\% confidence interval for pE is given by, approximately, $0.2338 \pm (1.96)(0.0197)$, i.e. it is $(0.195, 0.272)$.
\item Neither of these intervals contains the proposed value of 0.069 – in fact, the intervals are a considerable distance away from that. 
\item So neither is consistent with this value.
\end{itemize}
%%%%%%%%%%%%%%%%%%%%%%%%%%%%%%%
\end{enumerate}

\end{enumerate}
\end{document}
