\documentclass[a4paper,12pt]{article}

%%%%%%%%%%%%%%%%%%%%%%%%%%%%%%%%%%%%%%%%%%%%%%%%%%%%%%%%%%%%%%%%%%%%%%%%%%%%%%%%%%%%%%%%%%%%%%%%%%%%%%%%%%%%%%%%%%%%%%%%%%%%%%%%%%%%%%%%%%%%%%%%%%%%%%%%%%%%%%%%%%%%%%%%%%%%%%%%%%%%%%%%%%%%%%%%%%%%%%%%%%%%%%%%%%%%%%%%%%%%%%%%%%%%%%%%%%%%%%%%%%%%%%%%%%%%

\usepackage{eurosym}
\usepackage{vmargin}
\usepackage{amsmath}
\usepackage{graphics}
\usepackage{epsfig}
\usepackage{enumerate}
\usepackage{multicol}
\usepackage{subfigure}
\usepackage{fancyhdr}
\usepackage{listings}
\usepackage{framed}
\usepackage{graphicx}
\usepackage{amsmath}
\usepackage{chngpage}

%\usepackage{bigints}
\usepackage{vmargin}

% left top textwidth textheight headheight

% headsep footheight footskip

\setmargins{2.0cm}{2.5cm}{16 cm}{22cm}{0.5cm}{0cm}{1cm}{1cm}

\renewcommand{\baselinestretch}{1.3}

\setcounter{MaxMatrixCols}{10}

\begin{document}Higher Certificate, Paper III, 2005. Question 8

%%%%%%%%%%%%%%%%%%%%%%%%%%%%%%%%%%%%%%%%%%%%%%%%%%%%%%%%%%%%%%%%%%%%%%%%
\begin{framed}
8.
An investigation was carried out by a flour manufacturer into the production line
variations of a certain baking process. Squares of puff pastry were baked in a tray
containing 7 squares across the tray and 10 squares the length of the tray.
Data were collected on the size of the finished product, just after coming out of the
oven. For each "square", the width, length, and height (mm) were measured and also,
on the basis of these measurements, the approximate volume (mm 3 ) was calculated.
The data are summarised below.
Present your conclusions about the baking process in as informative a way as possible,
including suitable plots, but avoiding formal statistical material (such as significance
tests or confidence intervals). You should consider questions such as whether there
are detectable effects due to the position on the tray of the puff pastry square, possible
relationships between the dimensions of the squares and any implications for the uses
to which the pastry squares might be put. Highlight any additional information about
the raw data that might be useful.
(20)
Summary data classified by pos-w, i.e. position across the width of the tray
Variable length, N = 10
pos-w Mean(l)
SD(l)
1
86.4
1.42984
2
87.3
1.25167
3
87.5
1.43372
4
86.7
1.70294
5
86.2
1.81353
6
84.7
1.63639
7
84.3
1.49443 Med(l)
86.5
87.0
88.0
86.5
86.0
84.5
84.0 Min(l)
85
86
84
84
83
83
82 Max(l)
89
90
89
90
89
88
87
Variable width, N = 10
pos-w Mean(w)
SD(w)
1
77.1
1.91195
2
78.1
1.44914
3
77.0
2.78887
4
76.1
1.66333
5
76.8
0.91894
6
78.3
1.41814
7
79.1
1.66333 Med(w)
77.5
77.5
76.5
76.0
77.0
78.0
79.5 Min(w)
72
77
74
73
76
77
76 Max(w)
79
81
83
79
79
81
81
Variable height, N = 10
pos-w Mean(h)
SD(h)
1
27.8
2.44040
2
27.9
2.84605
3
29.3
2.75076
4
29.2
2.44040
5
28.4
2.01108
6
30.6
2.50333
7
31.5
3.89444 Med(h)
27.5
28.0
28.0
30.0
29.0
30.0
31.5 Min(h)
24
24
27
25
24
27
27 Max(h)
31
32
34
32
31
36
38
Variable volume, N = 10
pos-w
Mean(v)
SD(v)
1
185044
15396.0
2
190217
19868.5
3
196930
13628.0
4
192254
11153.5
5
187874
11941.2
6
202796
15521.1
7
209629
22467.7 Med(v)
183344
187910
193604
194940
189535
201192
205096
Min(v)
159120
167475
178524
174174
162336
181305
183222
9
Max(v)
207669
216832
221408
204972
200970
238392
242609
\end{framed}
%%%%%%%%%%%%%%%%%%%%%%%%%%%%%%%%%%%%%%%%%%%%%%%%%%%%%%%%%%%%%%%%%%%%%%%%%
\newpage
\begin{itemize}
    \item An obvious characteristic of the data is the difference between length and width. The "squares" are in every position rectangles, and the range of length sizes in any position (1 to 7) does not even overlap with the range of width sizes. Length is lower in positions 6 and 7 across the tray than it is elsewhere, and width is highest in position 7. Both length and width measurements are about equally variable in all positions.
\item 
Height is much more variable, especially in relation to its mean size. It also increases fairly steadily from position 1 to 7 (though 5 goes against this trend), with position 7 being particularly variable, due perhaps to one or two very large values (max 38).
\item There could be a temperature gradient in the oven, related to width, which affects height, and some other trends which result in length and width not being the same although the original material was presumably squarely placed.
\item If the appearance and uniformity of the "square" product are important, some attention needs to be given to the operation of the oven.
Summary of average (mean) and range (maximum – minimum in the whole data):
Average Range
Length 86.2 8
Width 77.5 11
Height 29.2 14
\item The combined effect on volume is to produce larger values in positions 6 and 7, with 1 to 5 showing an increase followed by a decrease. High variability is noted in 7, and fairly high in 2.
Data were collected just after removal from the oven. It is quite possible that after cooling some of the characteristics measured would have settled down more. We might usefully be told how many people were involved in measuring the data, as there could have been a time effect while collecting it.
Solution continued on next page
One useful diagram is to show the mean measurements against pos-w: 765432190807060504030pos-wmean height/width/length
width
length
height
\item The above figure shows all three sets of data on the same scale. This is very useful in comparing length and width, but putting height on the same diagram hides the detail of the changes in the others because of the vertical scale. Two separate diagrams might be better.
The figure below shows an interesting comparison – volume depends quite closely on height. The numbers in brackets show positions across the width of the tray, 1 to 7. 28293031190000200000210000mean heightmean volume(1)(2)(5)(4)(3)(6)(7)\end{itemize}
\end{document}
