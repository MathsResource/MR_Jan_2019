\documentclass[a4paper,12pt]{article}

%%%%%%%%%%%%%%%%%%%%%%%%%%%%%%%%%%%%%%%%%%%%%%%%%%%%%%%%%%%%%%%%%%%%%%%%%%%%%%%%%%%%%%%%%%%%%%%%%%%%%%%%%%%%%%%%%%%%%%%%%%%%%%%%%%%%%%%%%%%%%%%%%%%%%%%%%%%%%%%%%%%%%%%%%%%%%%%%%%%%%%%%%%%%%%%%%%%%%%%%%%%%%%%%%%%%%%%%%%%%%%%%%%%%%%%%%%%%%%%%%%%%%%%%%%%%

\usepackage{eurosym}
\usepackage{vmargin}
\usepackage{amsmath}
\usepackage{graphics}
\usepackage{epsfig}
\usepackage{enumerate}
\usepackage{multicol}
\usepackage{subfigure}
\usepackage{fancyhdr}
\usepackage{listings}
\usepackage{framed}
\usepackage{graphicx}
\usepackage{amsmath}
\usepackage{chngpage}

%\usepackage{bigints}
\usepackage{vmargin}

% left top textwidth textheight headheight

% headsep footheight footskip

\setmargins{2.0cm}{2.5cm}{16 cm}{22cm}{0.5cm}{0cm}{1cm}{1cm}

\renewcommand{\baselinestretch}{1.3}

\setcounter{MaxMatrixCols}{10}

\begin{document}
\begin{enumerate}
    \item 
(i) The Poisson distribution to explain numbers of goals might be a reasonable assumption if home team scores can be regarded as random events occurring at a constant average rate throughout the season. If so, the number of home team goals in a match is Poisson with parameter (mean) equal to this constant average rate, μ say.
    \item  (ii) 634/3801.6684r==.
22221()163417781.90031379380frsfrff⎧⎫⎛⎞Σ=Σ−=−=⎨⎬⎜⎟Σ−Σ⎩⎭⎝⎠.
    \item  (iii) We take μ as 1.6684. So ()1.668400.1885PRe−===, and the expected frequency for r = 0 is 380 × 0.1885 = 71.65.
Similarly, , and the expected frequency for r = 1 is 119.51. ()1.668411.66840.3145PRe−===
Hence we have (taking the remaining expected frequencies from the question paper)
r
0
1
2
3
4
≥5
Total
Observed
81
112
101
44
28
14
380
Expected
71.65
119.51
99.72
55.46
23.13
10.51
379.98
[Note. There is a very small rounding error in the calculations of expected frequencies.]
The test statistic is
()22222(8171.65)(112119.51)(1410.51)...6.26171.65119.5110.51OEXE−−−−==+++=Σ,
which is referred to (note 4 degrees of freedom because the table has 6 cells and there is one estimated parameter). This is not significant (the 5\% point is 9.49); we cannot reject the null hypothesis, i.e. there is no evidence against the Poisson model with these data. 24χ
For the test, the expected frequencies need to be not too small (≥5 is often used as a criterion). This would not be the case if frequencies for large r were not combined.
    \item (iv) The negative binomial is commonly used where there is "over-dispersion". [It assumes that the rate (μ) is not always constant but varies (from match to match) according to a gamma distribution.]
\end{enumerate}
\end{document}
