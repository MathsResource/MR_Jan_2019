\documentclass[a4paper,12pt]{article}

%%%%%%%%%%%%%%%%%%%%%%%%%%%%%%%%%%%%%%%%%%%%%%%%%%%%%%%%%%%%%%%%%%%%%%%%%%%%%%%%%%%%%%%%%%%%%%%%%%%%%%%%%%%%%%%%%%%%%%%%%%%%%%%%%%%%%%%%%%%%%%%%%%%%%%%%%%%%%%%%%%%%%%%%%%%%%%%%%%%%%%%%%%%%%%%%%%%%%%%%%%%%%%%%%%%%%%%%%%%%%%%%%%%%%%%%%%%%%%%%%%%%%%%%%%%%

\usepackage{eurosym}
\usepackage{vmargin}
\usepackage{amsmath}
\usepackage{graphics}
\usepackage{epsfig}
\usepackage{enumerate}
\usepackage{multicol}
\usepackage{subfigure}
\usepackage{fancyhdr}
\usepackage{listings}
\usepackage{framed}
\usepackage{graphicx}
\usepackage{amsmath}
\usepackage{chngpage}

%\usepackage{bigints}
\usepackage{vmargin}

% left top textwidth textheight headheight

% headsep footheight footskip

\setmargins{2.0cm}{2.5cm}{16 cm}{22cm}{0.5cm}{0cm}{1cm}{1cm}

\renewcommand{\baselinestretch}{1.3}

\setcounter{MaxMatrixCols}{10}

\begin{document}Higher Certificate, Paper III, 2005. Question 6
%%%%%%%%%%%%%%%%%%%%%%%%%%%%%%%%%%%%%%%%%%%%%%%%%%%%%%%%%%%%%
6.
(i) What do you understand by the term simple random sampling ? Describe
conditions under which it may not be a suitable sampling procedure or where it
would be desirable to combine it with some other sampling method.
(4)
(ii) Choose three different types of non-sampling error , and briefly describe
circumstances in surveys that give rise to these errors.
(7)
(iii) Outline the main disadvantages of telephone surveys.
(3)
(iv)
A poll was conducted on the support given by the public to a new government
health policy initiative. Of 1015 people surveyed, 853 expressed support. One
year later, a similar poll of whether support was still as high yielded 780 out of
1005 people in favour. Determine whether there is any evidence of a decrease
in the proportion of people supporting the policy.
(6)
7

%%%%%%%%%%%%%%%%%%%%%%%%%%%%%%%%%%%%%%%%%%%%%%%%%%%%%%%%%%%%%%%%%%%%%%%%%%%%%%%%%%%
\begin{enumerate}
    \item  Simple random sampling, for samples of size n from a population of size N, is where every sample has the same probability of selection. (This probability is, of course, 1/.) A consequence of this is that every individual in the target population has the same probability of being selected for the sample. Nn⎛⎞⎜⎟⎝⎠
If a population is not homogenous as a whole, but can be split into groups each of which is homogenous within itself, it will be better to select randomly within each group, i.e. stratified random sampling. This allows the different groups to be studied, as well as increasing precision of overall estimates. Also, when a very large population is to be sampled using, for example, a list of names, a systematic sample can be much easier to organise and may be treated as random provided any trends or cyclical patterns in the list are avoided.
\item Errors in recording responses, due to poor training of enumerators or interviewers, and/or to carelessness or misunderstanding of subjects' answers. In a postal questionnaire, poor wording of questions may lead to respondents not answering the question intended.
(b) Transfer errors when data are taken from forms and entered into a processing system. Illegible answers could also occur on postal survey questionnaires.
(c) Non-response to postal surveys or refusal to co-operate/be interviewed. This may happen because of lack of interest in the topic being studied, objection to the wording of the questions or the approach of the interviewer, unwillingness to give time to answering, or simply being asked too often to take part in a survey.
(d) Failure to locate individuals/units chosen to take part in a survey. This may for example happen because of faulty lists, non-availability at the time an interviewer calls, premises being empty because people have moved, or different work and/or leisure habits so that individuals would need to be contacted at unusual times not planned for in the survey.
\item Telephone surveys only contact people available and willing to answer at the time of ringing, who have some interest in the topic under study, and whose numbers are not ex-directory (if a telephone directory is used as a sample frame). High rates of refusal to respond are likely from people who have been contacted frequently for such surveys. Further, in some countries by no means everyone has a telephone.
\item Let n1 = 1015, n2 = 1005 be the numbers of people questioned in years 1, 2. Then 1853ˆ0.84041015p==and 2780ˆ0.77611005p== are the estimates of the proportions in favour.
\begin{itemize}
    \item If p1, p2 are the true proportions in the population, the null hypothesis is p1 = p2 and the alternative hypothesis is p1 > p2. 
    \item We may use the Normal approximation to the binomial for these values of n and p. The test will be one-sided.
We have
()()()112121211ˆˆVar ppppppnn−−−=+
which is estimated by 0.84040.15960.77610.223910151005××+ = 0.00030505.
\item Thus the value of the test statistic is
0.84040.77610.06433.70.01750.00030505−==.
\item This is very highly significant as an observation from N(0, 1) [the upper single-tailed 0.5% point is 2.576].
\item There is strong evidence against the null hypothesis in favour of the alternative hypothesis which says that there is a decrease in the proportion of supporters.
\end{itemize}

\end{enumerate}
\end{document}
