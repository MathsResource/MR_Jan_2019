Higher Certificate, Paper I, 2005. Question 5
(i) We have Y ~ B(n, p), so ( ) ( ) ! (1 )
! !
P Y y n py p n y
y n y
− = = −
−
(for y = 0, 1, 2, …, n
and 0 < p < 1). The likelihood L is simply P(Y = y).
Hence log L = constant + y log p + (n − y)log(1− p).
log
1
d L y n y
dp p p
− ∴ = −−
which on setting equal to zero gives that the maximum
likelihood estimate is pˆ y
n
= . [Consideration of
2
2
d log L
dp
confirms that this is a
maximum.]
( ) ( ) ( ) ( )
2 2
1 1 1 Var ˆ Var 1
p p
p Y np p
n n n
−
= = − = . We may estimate p by ˆp in this and
thus obtain an estimate of the standard error of pˆ as ( ) ( ) ˆ 1 ˆ
SE ˆ
p p
p
n
−
= .
For n = 100 and y = 20, we have pˆ = 0.2 and SE( ˆ ) 0.2 0.8 0.04
100
p = × = .
(ii) P("yes") is given by P(takes drugs and coin shows "takes drugs") + P(does not
take drugs and coin shows "does not take drugs").
Hence θ = P("yes") = 0.75p + 0.25(1 – p) = 0.25 + 0.5p.
Z ~ B(n, θ ), so from part (i) we have θˆ = z/n and SE(θˆ) = θˆ (1−θˆ)/ n .
We have p = 2θ – ½, so the MLE of p is 1
2
p�� = 2θˆ − .
Thus SE( p�� ) = 2 SE(θˆ) = 2 θˆ (1−θˆ)/ n .
For n = 100 and z = 45, we have θˆ = 0.45 and so p�� = 0.4 and SE( p�� ) = 0.0995 .
(iii) Both p and the standard error are estimated to be larger by the second survey.
The larger p is plausible, as some people are likely not to admit to taking drugs when
asked directly as in the first survey. There is a much better expectation of truthful
answers in the second survey. We should not claim that the first is better just because
the SE is smaller; the first survey is likely to be biased. (Is it clear what the journalist
means by "reliable"?)
