\documentclass[a4paper,12pt]{article}

%%%%%%%%%%%%%%%%%%%%%%%%%%%%%%%%%%%%%%%%%%%%%%%%%%%%%%%%%%%%%%%%%%%%%%%%%%%%%%%%%%%%%%%%%%%%%%%%%%%%%%%%%%%%%%%%%%%%%%%%%%%%%%%%%%%%%%%%%%%%%%%%%%%%%%%%%%%%%%%%%%%%%%%%%%%%%%%%%%%%%%%%%%%%%%%%%%%%%%%%%%%%%%%%%%%%%%%%%%%%%%%%%%%%%%%%%%%%%%%%%%%%%%%%%%%%

\usepackage{eurosym}
\usepackage{vmargin}
\usepackage{amsmath}
\usepackage{graphics}
\usepackage{epsfig}
\usepackage{enumerate}
\usepackage{multicol}
\usepackage{subfigure}
\usepackage{fancyhdr}
\usepackage{listings}
\usepackage{framed}
\usepackage{graphicx}
\usepackage{amsmath}
\usepackage{chngpage}

%\usepackage{bigints}
\usepackage{vmargin}

% left top textwidth textheight headheight

% headsep footheight footskip

\setmargins{2.0cm}{2.5cm}{16 cm}{22cm}{0.5cm}{0cm}{1cm}{1cm}

\renewcommand{\baselinestretch}{1.3}

\setcounter{MaxMatrixCols}{10}

\begin{document}Higher Certificate, Paper I, 2005. Question 8

\begin{enumerate}
\item The marginal distributions of X and Y are as shown, appended to the table.
Values of Y
0 1 2
Marginal distribution
of X
–1 1/6 1/12 1/12 1/3
0 1/12 1/6 1/12 1/3
Values of X
1 1/12 1/12 1/6 1/3
Marginal distribution of Y 1/3 1/3 1/3
Hence E[X] = 0 and E[Y] = 1 (by symmetry; by noting that X and Y are both discrete
uniform; or by explicit calculation).
\[Var(X) = \sum(x – 0)2P(X = x) = {(–1)2 + 02 + 12)/3 = 2/3.\]
Var(Y) can be calculated similarly; or, since Y = X + 1, we have Var(Y) = Var(X).
\item The conditional distributions of Y for each value of X are as follows.
Y = 0 1 2
X = –1 1/2 1/4 1/4
0 1/4 1/2 1/4
1 1/4 1/4 1/2
Hence $E[Y | X = –1] = (0)(0.5) + (1)(0.25) + (2)(0.25) = 3/4$.
Similarly, $E[Y | X = 0] = 1 and E[Y | X = 1] = 5/4$.
Thus we have $E[Y | X = x] = 1+ 4x$ .
\item 
\begin{eqnarray*} 
E[XY] = (–1)(0)(1/6) + (–1)(1)(1/12) + \ldots + (1)(2)(1/6) \\ &=&  1/6.
\end{eqnarray*}

\begin{eqnarray*}
Cov(X, Y) &=& E[XY] – E[X]E[Y] \\ 
&=& 1
6 − (0)(1) \\ &=& 1/6\\
\end{eqnarray*}

\begin{eqnarray*}Corr(X, Y) \\ &=& \frac{Cov(X, Y)}{Var(X ) Var(Y  )}
\\ &=& \frac{1/6}{2/3} \\ &=& \frac{1/6}{2/3}
\end{eqnarray*}
X and Y are not independent: their correlation (or covariance) is non-zero. (In fact we
saw in part (i) that they are linearly related: Y = X + 1.)
\item ( )Z = X 3 + Y −1 3 . The values of Z and their probabilities are as shown:
Y = 0 1 2
X = –1 Z = –2; p = 1/6 Z = –1; p = 1/12 Z = 0; p = 1/12
0 Z = –1; p = 1/12 Z = 0; p = 1/6 Z =1; p = 1/12
1 Z = 0; p = 1/12 Z =1; p = 1/12 Z = 2; p = 1/6
Thus we have
Values of Z –2 –1 0 1 2
Probabilities 1/6 1/6 1/3 1/6 1/6
\end{enumerate}
\end{document}
