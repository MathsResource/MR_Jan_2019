Higher Certificate, Paper I, 2005. Question 8
(i) The marginal distributions of X and Y are as shown, appended to the table.
Values of Y
0 1 2
Marginal distribution
of X
–1 1/6 1/12 1/12 1/3
0 1/12 1/6 1/12 1/3
Values of X
1 1/12 1/12 1/6 1/3
Marginal distribution of Y 1/3 1/3 1/3
Hence E[X] = 0 and E[Y] = 1 (by symmetry; by noting that X and Y are both discrete
uniform; or by explicit calculation).
Var(X) = Σ(x – 0)2P(X = x) = {(–1)2 + 02 + 12)/3 = 2/3.
Var(Y) can be calculated similarly; or, since Y = X + 1, we have Var(Y) = Var(X).
(ii) The conditional distributions of Y for each value of X are as follows.
Y = 0 1 2
X = –1 1/2 1/4 1/4
0 1/4 1/2 1/4
1 1/4 1/4 1/2
Hence E[Y | X = –1] = (0)(½) + (1)(¼) + (2)(¼) = 3/4.
Similarly, E[Y | X = 0] = 1 and E[Y | X = 1] = 5/4.
Thus we have E[Y | X = x] = 1+ 4x .
(iii) E[XY] = (–1)(0)(1/6) + (–1)(1)(1/12) + ….. + (1)(2)(1/6) = 1/6.
∴ Cov(X, Y) = E[XY] – E[X]E[Y] = 1
6 − (0)(1) = 1/6.
∴ Corr(X, Y) = Cov( , )
Var( ) Var( )
X Y
X Y
= 1/6 1
2/3 4
= .
X and Y are not independent: their correlation (or covariance) is non-zero. (In fact we
saw in part (i) that they are linearly related: Y = X + 1.)
(iv) ( )Z = X 3 + Y −1 3 . The values of Z and their probabilities are as shown:
Y = 0 1 2
X = –1 Z = –2; p = 1/6 Z = –1; p = 1/12 Z = 0; p = 1/12
0 Z = –1; p = 1/12 Z = 0; p = 1/6 Z =1; p = 1/12
1 Z = 0; p = 1/12 Z =1; p = 1/12 Z = 2; p = 1/6
Thus we have
Values of Z –2 –1 0 1 2
Probabilities 1/6 1/6 1/3 1/6 1/6
\end{document}
