\documentclass[a4paper,12pt]{article}

%%%%%%%%%%%%%%%%%%%%%%%%%%%%%%%%%%%%%%%%%%%%%%%%%%%%%%%%%%%%%%%%%%%%%%%%%%%%%%%%%%%%%%%%%%%%%%%%%%%%%%%%%%%%%%%%%%%%%%%%%%%%%%%%%%%%%%%%%%%%%%%%%%%%%%%%%%%%%%%%%%%%%%%%%%%%%%%%%%%%%%%%%%%%%%%%%%%%%%%%%%%%%%%%%%%%%%%%%%%%%%%%%%%%%%%%%%%%%%%%%%%%%%%%%%%%

\usepackage{eurosym}
\usepackage{vmargin}
\usepackage{amsmath}
\usepackage{graphics}
\usepackage{epsfig}
\usepackage{enumerate}
\usepackage{multicol}
\usepackage{subfigure}
\usepackage{fancyhdr}
\usepackage{listings}
\usepackage{framed}
\usepackage{graphicx}
\usepackage{amsmath}
\usepackage{chngpage}

%\usepackage{bigints}
\usepackage{vmargin}

% left top textwidth textheight headheight

% headsep footheight footskip

\setmargins{2.0cm}{2.5cm}{16 cm}{22cm}{0.5cm}{0cm}{1cm}{1cm}

\renewcommand{\baselinestretch}{1.3}

\setcounter{MaxMatrixCols}{10}

\begin{document}
Higher Certificate, Paper II, 2005. Question 3
\begin{framed}

3. In a small survey of perceived health risks in the UK, each member of a random sample of 50 people was asked the question "When buying food, do you check the pack for artificial additives?".  The researchers wanted to discover whether females or males were more likely to check for artificial additives when buying food. 

 \begin{center}
 \begin{tabular}{|c|c|c|}
 & Female & Male \\
  Answer No & 18 & 17\\ 
  Answer Yes & 11 &   4 \\
 \end{tabular}
 \end{center}

 
 
(i) Test for a difference between the percentages of males and females responding "Yes" to the question about checking for artificial additives. (9) 
 


 

\end{framed}
\begin{enumerate}
\item (i) The expected frequencies on the null hypothesis of no difference between the sexes in the response are found in the usual way from the marginal totals (e.g. that for "Female, No" is $29\times35/50 = 20.3$). Thus the observed and expected frequencies are
Observed frequencies
Expected frequencies
Female
Male
Total
Female
Male
No
18
17
35
20.3
14.7
Yes
11
4
15
8.7
6.3
Total
29
21
50
\begin{itemize}
    \item All the differences between observed and expected frequencies are $\pm2.3$, becoming $\pm1.8$ if Yates' correction is used. 
        \item Thus the usual test statistic can be calculated as (using Yates' correction)
()211111.81.26720.38.714.76.3⎧⎫+++=⎨⎬⎩⎭
(or 2.07 if Yates' correction is not used).     \item This is referred to ; the upper 5\% point is 3.84, so we have no evidence of a real sex difference. 21χ
\end{itemize}

%%%%%%%%%%%%%%%%%%%%%%%%%%%%%%%%%%%%%%%%%%%%%
\newpage
\begin{framed}
(ii) Calculate an approximate 95\% confidence interval for the difference in percentages of males and females responding "Yes" to the question about checking for artificial additives.  How good do you believe the approximation to be?  (State your reason.)
 
\end{framed}
\item  $p_f - p_m$ is estimated by 1142921ˆˆ0.37930.19050.1888fp_mp−=−=−=. The estimated variance of ˆˆfp_mp− is given by
()()ˆˆ1ˆˆ10.00811840.00734260.015461ffmmfp_mpppnn−−+=+=.
\begin{itemize}
    \item Thus the approximate 95\% confidence interval is given by $0.1888 \pm (1.96\times \sqrt{0.015461})$ i.e. it is $(–0.0548, 0.4324)$ or, in percentage terms, $(–5.48\%, 43.24\%)$.
\item The Normal approximation is unlikely to be very good with these small samples, especially as the values of $\hat{p}_f$ and $\hat{p}_m$ suggest that $p_f$ and $p_m$ are some way from 0.5.
(We might note also that the confidence interval is very wide; it does not give much information, due to lack of sufficient data.)
\end{itemize}

\end{enumerate}
\end{document}
