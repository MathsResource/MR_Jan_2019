\documentclass[a4paper,12pt]{article}

%%%%%%%%%%%%%%%%%%%%%%%%%%%%%%%%%%%%%%%%%%%%%%%%%%%%%%%%%%%%%%%%%%%%%%%%%%%%%%%%%%%%%%%%%%%%%%%%%%%%%%%%%%%%%%%%%%%%%%%%%%%%%%%%%%%%%%%%%%%%%%%%%%%%%%%%%%%%%%%%%%%%%%%%%%%%%%%%%%%%%%%%%%%%%%%%%%%%%%%%%%%%%%%%%%%%%%%%%%%%%%%%%%%%%%%%%%%%%%%%%%%%%%%%%%%%

\usepackage{eurosym}
\usepackage{vmargin}
\usepackage{amsmath}
\usepackage{graphics}
\usepackage{epsfig}
\usepackage{enumerate}
\usepackage{multicol}
\usepackage{subfigure}
\usepackage{fancyhdr}
\usepackage{listings}
\usepackage{framed}
\usepackage{graphicx}
\usepackage{amsmath}
\usepackage{chngpage}

%\usepackage{bigints}
\usepackage{vmargin}

% left top textwidth textheight headheight

% headsep footheight footskip

\setmargins{2.0cm}{2.5cm}{16 cm}{22cm}{0.5cm}{0cm}{1cm}{1cm}

\renewcommand{\baselinestretch}{1.3}

\setcounter{MaxMatrixCols}{10}

\begin{document}
Higher Certificate, Paper I, 2005. Question 7
Note There are many equivalent forms of the formulae that are required to be stated.


The basic expressions are Σ(x − x)( y − y) , Σ(x − x)2 and Σ( y − y)2 . Convenient
computing expressions are Σxy − (ΣxΣy) / n and similarly for the others. [Where
appropriate, numerators and denominators of fractions could both be multiplied by n
(7) to avoid possible slight inaccuracies caused by rounding when dividing by 7.]
\begin{enumerate}
\item 
0
2
4
6
8
10
12
14
0 15 30 45 60 75 90
x
y
2 2 ( 2 2 )( 2 2 )
( )( ) /
( ) ( ) ( ) / ( ) /
r x x y y xy x y n
x x y y x x n y y n
= Σ − − = Σ − Σ Σ
Σ − Σ − Σ − Σ Σ − Σ
2 2
1773.795 (315 25.305/7 ) 635.07 0.848
(20475 315 /7 )(180.474 25.305 /7 ) 748.784
= − × = =
− −
.
This indicates a strong linear association between x and y, but nevertheless the scatter
diagram clearly suggests that the relationship is curved.
\item 
We have
2 2
1 1 sin x sin x 1 sin2 x 1 1
y a b a b a
= − + = +  −   
 
, i.e. Y 1 1 1 X
a b a
= +  −   
 
where
X and Y are as given.
So A 1
a
= and B 1 1
b a
= − .

\item 
X = sin2x 0 0.067 0.250 0.500 0.750 0.933 1
Y = 1/y 1.013 0.940 0.748 0.523 0.365 0.173 0.087
[Note. Summary statistics for these are given in the question.]
0
0.5
1
0 0.25 0.5 0.75 1
X = sin2x
Y
=1/y
\begin{itemize}
\item 
The scatter diagram indicates that the relationship between X and Y is very close to
linear, with little scatter about a straight line. 
\item Linear regression should be suitable.
\item  We have X = 3.5 / 7 = 0.500 and Y = 3.849 / 7 = 0.550.
\item  So the fitted linear regression is Y = A + BX where ( )( )
( )2
X X Y Y
B
X X
Σ − −
=
Σ −
and
A = Y − BX .
\item  Carrying out the calculations as in part (i), we get
2
1.03385 (3.5 3.849/7 ) 0.89065 0.89065
2.75 (3.5 /7) 1
B = − × = − = −
−
and hence A = 0.550 + (0.89065)(0.500) = 0.9953.
\item Thus the corresponding estimates of a and b are given by ˆ 1 1.0047
0.9953
a= = and
1 ˆ 1 1 9.556
ˆ 0.10465
b B
a
− =  +  = =  
 
.
\item The correlation coefficient for X and Y is
( )( )
( ) ( ) ( ) 2 2 2 2
0.89065 0.89065 0.9975
1 2.9136 (3.849 /7 )
X X Y Y
X X Y Y Y Y
Σ − − − − = = = −
Σ − Σ − × Σ − −
.
\item This is an even stronger indication of a linear relationship than in part (i), and we can
see from the scatter diagram that the relationship appears almost purely linear (very
little random scatter, certainly no curved component)
\end{itemize}

\end{enumerate}
\end{document}
