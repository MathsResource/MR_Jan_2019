\documentclass[a4paper,12pt]{article}
%%%%%%%%%%%%%%%%%%%%%%%%%%%%%%%%%%%%%%%%%%%%%%%%%%%%%%%%%%%%%%%%%%%%%%%%%%%%%%%%%%%%%%%%%%%%%%%%%%%%%%%%%%%%%%%%%%%%%%%%%%%%%%%%%%%%%%%%%%%%%%%%%%%%%%%%%%%%%%%%%%%%%%%%%%%%%%%%%%%%%%%%%%%%%%%%%%%%%%%%%%%%%%%%%%%%%%%%%%%%%%%%%%%%%%%%%%%%%%%%%%%%%%%%%%%%
\usepackage{eurosym}
\usepackage{vmargin}
\usepackage{amsmath}
\usepackage{graphics}
\usepackage{epsfig}
\usepackage{enumerate}
\usepackage{multicol}
\usepackage{subfigure}
\usepackage{fancyhdr}
\usepackage{listings}
\usepackage{framed}
\usepackage{graphicx}
\usepackage{amsmath}
\usepackage{chngpage}
%\usepackage{bigints}

\usepackage{vmargin}
% left top textwidth textheight headheight
% headsep footheight footskip
\setmargins{2.0cm}{2.5cm}{16 cm}{22cm}{0.5cm}{0cm}{1cm}{1cm}
\renewcommand{\baselinestretch}{1.3}

\setcounter{MaxMatrixCols}{10}
\begin{document}
\begin{enumerate}
\item  Note that Z 1
0
\[\lambda^2 xe^{-\lambda}xdx = 1\]
\item 
\[E[x] = \lambda^2  \int^{\infty}_{0}
0 x2e^{-\lambda}xdx = \lambda^2 [-1
\lambda \;x2e^{-\lambda}x]1
0 + \lambda^2  \int^{\infty}_{0}
0
1
\lambda e^{-\lambda}x \times 2xdx\]
\[= 0 + 2
\lambda
\int^{\infty}_{0}
 \lambda^2 xe^{-\lambda}xdx = 2
\lambda\]
(b)
\[E[x2] = \lambda^2  \int^{\infty}_{0}
0 x3e^{-\lambda}xdx = \lambda^2 [-1
\lambda \;x3e^{-\lambda}x]1
0 + \lambda^2  \int^{\infty}_{0}
0
1\]
\[\lambda e^{-\lambda}x \times 3x2dx
= 0 + 3\]
\[E[x] = 6
\lambda^2 \]
Therefore
\[v[x] =
6
\lambda^2 
- (
2
\lambda
)2 =
2
\lambda^2 
7\]
\item 
\begin{eqnarray*}
P(X > x) &=&
\int^{\infty}_{0}
x \lambda^2 ue^{-\lambda}udu \\ &=& [-\lambda u e\lambda u]1x
+
\int^{\infty}_{0}
x \lambda e^{-\lambda}udu
\\ &=& \lambda \;xe^{-\lambda}x + [-e^{-\lambda}u]1x
\\ &=& e^{-\lambda}x(1 + \lambda \;x):
\end{eqnarray*}
\item  \lambda = 0:01; x = 500 in(c) ; so \[P(x > 500) = e-5(1 + 5) = 0:04043\]
\item Assume X » N( 2
\lambda; 2
\lambda^2  ) i:e: N(200; 20000)
Now

\begin{eqnarray*}
P(x > 500) &=& 1 - Á( 500-200
100
p2 ) \\
&=& 1 - Á(p3 2 )\\
&=& 1 - Á(2:1213) \\
&=& 1 - 0:9835 \\ &=& 0:0165
\end{eqnarray*}
%%%%%%%%%%%%%%%%%%%%%%%%%%
\item Using the correct distribution (which is positive skewed),with \lambda = 0:01 and x = 450;

\begin{itemize}
    \item P(twin) = e-4:5(1 + 4:5) = 0:0611: 
    \item The skewness raises the right-hand tail probability
considerably
\end{itemize}
.

\end{enumerate}
\end{document}
