\documentclass[a4paper,12pt]{article}
%%%%%%%%%%%%%%%%%%%%%%%%%%%%%%%%%%%%%%%%%%%%%%%%%%%%%%%%%%%%%%%%%%%%%%%%%%%%%%%%%%%%%%%%%%%%%%%%%%%%%%%%%%%%%%%%%%%%%%%%%%%%%%%%%%%%%%%%%%%%%%%%%%%%%%%%%%%%%%%%%%%%%%%%%%%%%%%%%%%%%%%%%%%%%%%%%%%%%%%%%%%%%%%%%%%%%%%%%%%%%%%%%%%%%%%%%%%%%%%%%%%%%%%%%%%%
\usepackage{eurosym}
\usepackage{vmargin}
\usepackage{amsmath}
\usepackage{graphics}
\usepackage{epsfig}
\usepackage{enumerate}
\usepackage{multicol}
\usepackage{subfigure}
\usepackage{fancyhdr}
\usepackage{listings}
\usepackage{framed}
\usepackage{graphicx}
\usepackage{amsmath}
\usepackage{chngpage}
%\usepackage{bigints}

\usepackage{vmargin}
% left top textwidth textheight headheight
% headsep footheight footskip
\setmargins{2.0cm}{2.5cm}{16 cm}{22cm}{0.5cm}{0cm}{1cm}{1cm}
\renewcommand{\baselinestretch}{1.3}

\setcounter{MaxMatrixCols}{10}
\begin{document}
\begin{table}[ht!]
     \centering
     \begin{tabular}{|p{15cm}|}
     \hline        
The total claim amount X made in one year on a portfolio of insurance policies has probability density function 
 
 ()
2 0, 0 0 otherwise x xe x
fx
λ λλ − ≥> =  
 
 
(i) Show that 
 
 (a) ()2 EX λ = , 
 
 (b) () 2 2 Var X λ = , 
 
 (c) ()() 1x P X x e x λ λ − > = + . 
(5) 
 
\\ \hline
      \end{tabular}
    \end{table}
    


\begin{enumerate}
\item  Note that Z 1
0
\[\lambda^2 xe^{-\lambda}xdx = 1\]
\item 
\[E[x] = \lambda^2  \int^{\infty}_{0}
0 x2e^{-\lambda}xdx = \lambda^2 [-1
\lambda \;x2e^{-\lambda}x]1
0 + \lambda^2  \int^{\infty}_{0}
0
1
\lambda e^{-\lambda}x \times 2xdx\]
\[= 0 + 2
\lambda
\int^{\infty}_{0}
 \lambda^2 xe^{-\lambda}xdx = 2
\lambda\]
(b)
\[E[x2] = \lambda^2  \int^{\infty}_{0}
0 x3e^{-\lambda}xdx = \lambda^2 [-1
\lambda \;x3e^{-\lambda}x]1
0 + \lambda^2  \int^{\infty}_{0}
0
1\]
\[\lambda e^{-\lambda}x \times 3x2dx
= 0 + 3\]
\[E[x] = 6
\lambda^2 \]
Therefore
\[v[x] =
6
\lambda^2 
- (
2
\lambda
)2 =
2
\lambda^2 
7\]
%%%%%%%%%%%%%%%%%%5
\newpage


  \begin{table}[ht!]
     \centering
     \begin{tabular}{|p{15cm}|}
     \hline  
(ii) If X is measured in units of £1000, λ may be assumed to take the value 0.01.  The company has a total sum (policyholders’ premiums + reserves) of £500,000 available to meet the year’s claims.  Show that the probability that the company is ruined (i.e. P(X > 500)) is 0.040 (to 3 decimal places). (5) 
 \\ \hline
      \end{tabular}
    \end{table}
        
\item 
\begin{eqnarray*}
P(X > x) &=&
\int^{\infty}_{0}
x \lambda^2 ue^{-\lambda}udu \\ &=& [-\lambda u e\lambda u]1x
+
\int^{\infty}_{0}
x \lambda e^{-\lambda}udu
\\ &=& \lambda \;xe^{-\lambda}x + [-e^{-\lambda}u]1x
\\ &=& e^{-\lambda}x(1 + \lambda \;x):
\end{eqnarray*}
\item  \lambda = 0:01; x = 500 in(c) ; so \[P(x > 500) = e-5(1 + 5) = 0:04043\]

%%%%%%%%%%%%%%%%%%%%%%%%%%%%%%%%%%%%%%%%%%%%%%%%%%%%%%%%%
\newpage



  \begin{table}[ht!]
     \centering
     \begin{tabular}{|p{15cm}|}
     \hline  
(iii) A trainee actuary mistakenly assumes the distribution of total claim amount to be Normal with the same mean and variance as X (taking λ = 0.01).  On this assumption, find the probability of ruin, given that £500,000 is available to meet the year’s claims. (5) 
\\ \hline
      \end{tabular}
    \end{table}
\item Assume X » N( 2
\lambda; 2
\lambda^2  ) i:e: N(200; 20000)
Now

\begin{eqnarray*}
P(x > 500) &=& 1 - Á( 500-200
100
p2 ) \\
&=& 1 - Á(p3 2 )\\
&=& 1 - Á(2:1213) \\
&=& 1 - 0:9835 \\ &=& 0:0165
\end{eqnarray*}
%%%%%%%%%%%%%%%%%%%%%%%%%%
\newpage
  \begin{table}[ht!]
     \centering
     \begin{tabular}{|p{15cm}|}
     \hline 
     \noindent \textbf{part(d)}\\
     Making this mistaken assumption of Normality, the trainee calculates that £450,000 is the sum to be set aside to meet the year’s claims with a probability of ruin of less than 0.04.  What is the true probability of ruin, if only £450,000 is available to meet the year’s claims?  \\ \hline 
      \end{tabular}
    \end{table}
\item Using the correct distribution (which is positive skewed),with \lambda = 0:01 and x = 450;

\begin{itemize}
    \item P(twin) = e-4:5(1 + 4:5) = 0:0611: 
    \item The skewness raises the right-hand tail probability
considerably
\end{itemize}
.

\end{enumerate}
\end{document}
