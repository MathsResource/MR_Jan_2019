\documentclass[a4paper,12pt]{article}
%%%%%%%%%%%%%%%%%%%%%%%%%%%%%%%%%%%%%%%%%%%%%%%%%%%%%%%%%%%%%%%%%%%%%%%%%%%%%%%%%%%%%%%%%%%%%%%%%%%%%%%%%%%%%%%%%%%%%%%%%%%%%%%%%%%%%%%%%%%%%%%%%%%%%%%%%%%%%%%%%%%%%%%%%%%%%%%%%%%%%%%%%%%%%%%%%%%%%%%%%%%%%%%%%%%%%%%%%%%%%%%%%%%%%%%%%%%%%%%%%%%%%%%%%%%%
\usepackage{eurosym}
\usepackage{vmargin}
\usepackage{amsmath}
\usepackage{graphics}
\usepackage{epsfig}
\usepackage{enumerate}
\usepackage{multicol}
\usepackage{subfigure}
\usepackage{fancyhdr}
\usepackage{listings}
\usepackage{framed}
\usepackage{graphicx}
\usepackage{amsmath}
\usepackage{chngpage}
%\usepackage{bigints}

\usepackage{vmargin}
% left top textwidth textheight headheight
% headsep footheight footskip
\setmargins{2.0cm}{2.5cm}{16 cm}{22cm}{0.5cm}{0cm}{1cm}{1cm}
\renewcommand{\baselinestretch}{1.3}

\setcounter{MaxMatrixCols}{10}
\begin{document}



\begin{table}[ht!]
     \centering
     \begin{tabular}{|p{15cm}|}
     \hline        
It is required to carry out a blood test on a large number () N of persons, to check for the presence or absence of a rare characteristic.  It is assumed that the probability p of a positive result is the same for all persons, and is independent of the results of tests on other persons.
To reduce the work of testing, the blood samples of N persons are pooled into groups of size k, k  being a factor of N,  and tested together. 

\begin{itemize}
    \item If the group test is negative no further test is necessary for the k persons.  
    \item If the group test is positive each person must be tested individually and so in all (k + 1) tests are required for the group of k persons.
\end{itemize}
(i) Find the probability that the test for a pooled sample of k persons is positive.

\\ \hline
      \end{tabular}
    \end{table}
    


\begin{enumerate}[(i)]
\item \[P(\mbox{positive})= p. P(\mbox{no positive in k})= (1-p)^k,\]
(i) and so the probability of a pooled-sample positive is $1 - (1-p)^k$.
%%%%%%%%%%%%%%%%%%

  \begin{table}[ht!]
     \centering
     \begin{tabular}{|p{15cm}|}
     \hline  



Let $m$ be the number of groups, so that ${\displaystyle  m = \frac{N}{k}  }$
and let $S$ denote the total number of tests required for these people. 
Show that $S$ can be written in terms of a Binomially distributed random variable $X$ as

\[ S = m +kX, \qquad \mbox{ where } X \sim B(m,1-(1-p)^k ).\]


 \\ \hline 
      \end{tabular}
    \end{table}

    
\item S = m(one for each group) + k individual tests if the group was positive,
taken over each of the m groups = $m+kX$, where m is the number of groups
and each group has probability $1 - (1-p)^k$ of requiring k tests.
Therefore X is binomial $(m; 1 - (1-p)^k)$.

%%%%%%%%%%%%%%%%%%
%%%%%%%%%%%%%%%%%%

  \begin{table}[ht!]
     \centering
     \begin{tabular}{|p{15cm}|}
     \hline 
(iii) Hence show that
\[ E(S) = N \left[ \frac{1}{k} + 1-(1-p)^k \right] \]

\[ \operatorname{Var}(S) = Nk(1-p)^k \left[  1-(1-p)^k \right] \]

 \\ \hline 
      \end{tabular}
    \end{table}
\item $E[S] = E[m] + kE[X] = N$
k + k ¢ N
k ¢ f1 - (1-p)^kg = Nf 1
k + 1 - (1-p)^kg.






\[ 1 + k^2(1-p)^k\log_e(1-p) = 0\]
\begin{eqnarray*}
V [S] &=& k2V [X] \\ &=& k2mf1 - (1-p)^kg(1-p)^k \\ &=& Nk(1-p)^kf1 - (1-p)^kg.
\end{eqnarray*}
%%%%%%%%%%%%%%%%%%%%%%%%%%%%%%%%%%%%%%%%%%%%%%%%%%%%%%%%%%5
\newpage
  \begin{table}[ht!]
     \centering
     \begin{tabular}{|p{15cm}|}
     \hline  
(iv) Treating k as if it were continuous, show by differentiation that the value of k which gives the minimum expected number of tests for the N persons satisfies the equation
\[-\frac{N}{k^2} - N(1-p)^k ln(1-p).\]
(You may assume that this equation has a unique solution and that it gives a minimum.)
\\ \hline
      \end{tabular}
    \end{table}

%%%%%%%%%%%%%%%%%%
  \begin{table}[ht!]
     \centering
     \begin{tabular}{|p{15cm}|}
     \hline  
\noindent Part (e)\\ Given that p = 0.01 and an approximate solution of (A) is $k = 10.5$, find the value of $k$ which minimises $E(S)$ when N = 9900.
\\ \hline
      \end{tabular}
    \end{table}
\item 

%%%%%%%%%%%%%%%%%%%%%%%
\begin{eqnarray*} 
\frac{dE}{dk}
 &=&-\frac{N}{k^2}  - N \frac{d}{dk} \left[ (1-p)^k \right] \\
&=& -\frac{N}{k^2} - N(1-p)^k ln(1-p).
\end{eqnarray*}
(using the result that

\[ \frac{d(ax)}{dx} = ax \ln a)\].
\begin{itemize}
\item For minimum, set dE/dk = 0, giving $1 + k2(1-p)^k ln(1-p) = 0$.
\end{itemize}
%%%%%%%%%%%%%%%%%%%%%%%%%%%%%%%%%%
\item $1 + k^2(0.99)k ln(0.99) = 0$, so that k2 = -1
(0.99)k ln(0.99) .
Find E[S] for k = 10 and 11 (since it must be an integer).
\[E[S|k = 10] = 9900(0.1 + 1 - (0.99)^{10}) = 1936.62.\]
\[E[S|k = 11] = 9900( 1/11 + 1 - (0.99)^{11}) = 1936.15.\]
Take k = 11.
\end{enumerate}
\end{document}
