\documentclass{article}
\usepackage[utf8]{inputenc}

\usepackage{enumerate}
\usepackage{framed}

\begin{document}


\section{Introduction}
\begin{enumerate}[(i)]
\item P(positive)= p. P(no positive in k)= (1 ¡ p)k,
(i) and so the probability of a pooled-sample positive is 1 ¡ (1 ¡ p)k.
%%%%%%%%%%%%%%%%%%
\item S = m(one for each group) + k individual tests if the group was positive,
taken over each of the m groups = m+kX, where m is the number of groups
and each group has probability 1 ¡ (1 ¡ p)k of requiring k tests.
Therefore X is binomial (m; 1 ¡ (1 ¡ p)k).
4
%%%%%%%%%%%%%%%%%%
\item E[S] = E[m] + kE[X] = N
k + k ¢ N
k ¢ f1 ¡ (1 ¡ p)kg = Nf 1
k + 1 ¡ (1 ¡ p)kg.
V [S] = k2V [X] = k2mf1 ¡ (1 ¡ p)kg(1 ¡ p)k = Nk(1 ¡ p)kf1 ¡ (1 ¡ p)kg.
%%%%%%%%%%%%%%%%%%
\item dE
dk = ¡N
k2 ¡ N d
dk (1 ¡ p)k = ¡N
k2 ¡ N(1 ¡ p)k ln(1 ¡ p).
(using the result that d
dx (ax) = ax ln a).
For minimum, set dE
dk = 0, giving 1 + k2(1 ¡ p)k ln(1 ¡ p) = 0.
%%%%%%%%%%%%%%%%%%%%%%%%%%%%%%%%%%
\item 1 + k2(0:99)k ln(0:99) = 0, so that k2 = ¡1
(0:99)k ln(0:99) .
Find E[S] for k = 10 and 11 (since it must be an integer).
E[Sjk = 10] = 9900(0:1 + 1 ¡ 0:9910) = 1936:62.
E[Sjk = 11] = 9900( 1
11 + 1 ¡ 0:9911) = 1936:15.
Take k = 11.
\end{enumerate}
\end{document}