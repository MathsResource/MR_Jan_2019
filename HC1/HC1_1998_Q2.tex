\documentclass{article}
\usepackage[utf8]{inputenc}

\title{RSS_Jan_2019_HC2}
\author{kobriendublin }
\date{December 2018}

\begin{document}

\maketitle

\section{Introduction}

%%%%%%%%%%%%%%%%%%%%%%%%%%%%%%%%%%%%%%%%%%%%%%%%%%%%%%%%%%%%%%%%%%%%%%%%%%%%%%%%%%%%%%
2.(i)For a non-sufferer, X»N(7,9), so
P(x¸10) = P(z = x¡7
3 ¸ 10¡7
3 )
= 1 ¡ P(Z · 10¡7
3 = 1)
= 1 ¡ 0:8413
= 0:1587:
where Z»N(0,1)
(ii)For a sufferer, X»N(19,36), so
P(x<10) = P(Z = x¡19
6 < 10¡19
6 )
= P(Z < ¡3
2 ) = 0:0668:
where Z»N(0,1)
[Calculate as 1-P(Z<+3/2) if using tables.]
(iii)If critical level is x0, P(X¸x0jx »N(7,9))=0.05 .
In N(0,1) the upper 5% point is Z0=1.645 ; hence Z0 = x0¡7
3 = 1:645 or x0 = 7+(3£1:645) =
11:935.
(iv)P(X¸10) = P(X ¸ x0jsufferer)P(has disease) + P(X ¸ 10jnon ¡ sufferer)P(does not have disease)
= (1 ¡ 0:0668) £ 0:1 + 0:1587 £ 0:9
= 0:2362:
[use answers(i),(ii) and information that 10% if population affected.]
(v)P(diseasejx ¸ 10)=(0.9332£0.1)/(0.2362)=0.3951 .
\end{document}
