\documentclass[a4paper,12pt]{article}
%%%%%%%%%%%%%%%%%%%%%%%%%%%%%%%%%%%%%%%%%%%%%%%%%%%%%%%%%%%%%%%%%%%%%%%%%%%%%%%%%%%%%%%%%%%%%%%%%%%%%%%%%%%%%%%%%%%%%%%%%%%%%%%%%%%%%%%%%%%%%%%%%%%%%%%%%%%%%%%%%%%%%%%%%%%%%%%%%%%%%%%%%%%%%%%%%%%%%%%%%%%%%%%%%%%%%%%%%%%%%%%%%%%%%%%%%%%%%%%%%%%%%%%%%%%%
\usepackage{eurosym}
\usepackage{vmargin}
\usepackage{amsmath}
\usepackage{graphics}
\usepackage{epsfig}
\usepackage{enumerate}
\usepackage{multicol}
\usepackage{subfigure}
\usepackage{fancyhdr}
\usepackage{listings}
\usepackage{framed}
\usepackage{graphicx}
\usepackage{amsmath}
\usepackage{chngpage}
%\usepackage{bigints}

\usepackage{vmargin}
% left top textwidth textheight headheight
% headsep footheight footskip
\setmargins{2.0cm}{2.5cm}{16 cm}{22cm}{0.5cm}{0cm}{1cm}{1cm}
\renewcommand{\baselinestretch}{1.3}

\setcounter{MaxMatrixCols}{10}
\begin{document}

\begin{table}[ht!]
\centering
 \begin{tabular}{|p{15cm}|}
 \hline        
 \noindent \textbf{Part (a)}\\
\noindent
A manufacturer of colour television sets offers a one year warranty which covers free replacement if the picture tube fails and requires replacement.  He estimates the time in years until the tube fails, T, to be a random variable with probability density function
f(t) = 
5 1
. exp(−t/5), t ≥ 0
      = 0     otherwise.

\\ \hline
 \end{tabular}
\end{table}

\begin{table}[ht!]
     \centering
     \begin{tabular}{|p{15cm}|}
     \hline        
 \noindent \textbf{Part (b)}\\
\noindent       
(i) Find the probability that a randomly chosen set will require at least one replacement tube under the warranty.

\\ \hline
 \end{tabular}
\end{table}

    
\begin{enumerate}
\item 
\begin{eqnarray*}
P(¸1 \mbox{replacement})&=&1-P(0)
\\&=&1-P(T>1)
\\&=&1-
R1
1
1
5 e^{-t}=5dt
\\&=&1 - [-e^{-t}=5]1
1 
\\&=&1 - e-1=5 \\&=& 0.1813:
\end{eqnarray*}
%%%%%%%%%%%%%%%%%%%%%%%%%%%%%%%%%%%%%%%%%%%%
\item
\begin{eqnarray*}
E[\mbox{net profit}]
&=& \mbox{profit on sale} - replacement cost \times  P(\mbox{replacement})\\
&=& \times [100-70\times 0.1813] \mbox{if there is only one replacement}\\
&=& \times 87.31 .\\
\end{eqnarray*}
%%%%%%%%%%%%%%%%%%%%%%%%%%%%%%%%%%%%%%%%%%%%
\item 
8><
>:
x 10 11 12 13 14
P(X = x) 0.2 0.4 0.2 0.1 0.1 = P(Y = y)
9>=
>;
y +4000 +2000 0 -2000 -4000
Distribution of Y given by second and third rows above.
\[P(loss)=P(Y<0)=0.2 .\]

%%%%%%%%%%%%%%%%%%%%%%%%%%%%%%%%%%%%%%%%%%%%
\begin{table}[ht!]
     \centering
     \begin{tabular}{|p{15cm}|}
     \hline        
 \noindent \textbf{Part (b)}\\
\noindent (ii) If the gross profit per sale is £100 and the replacement of a tube costs £70, find the expected net profit per set sold, assuming that no set requires more than  one replacement under the warranty.
\\ \hline
 \end{tabular}
\end{table}
\item 
\begin{eqnarray*}
E[X]&=&(10\times 0.2)+(11\times 0.4)+(12\times 0.2)+(13\times 0.1)+(14\times 0.1)\\
&=& 2 + 4.4 + 2.4 + 1.3+ 1.4\\
&=&11.5\\
\end{eqnarray*}
 \begin{tabular}{|p{15cm}|}
     \hline        
 \noindent \textbf{Part (c)}\\
\noindent 
(b) The number of days required to complete a project is denoted by X, where X is assumed to have the distribution
x 10 11 12 13 14 otherwise p(X= x) 0.2 0.4 0.2 0.1 0.1 0
The return from the project is Y = £2000 (12 − X).
(i) Find the probability distribution of Y and obtain the probability of making a loss.
(ii) Find E(X), V(X), E(Y) and V(Y).

\\ \hline
 \end{tabular}
\end{table}

\begin{eqnarray*}
V[X] 
&=& E[X^2] - (E[X])^2\\
&=& (100 \times  0.2) + (121 \times  0.4) + (144 \times  0.2) + (169 \times  0.1) + (196 \times  0.1) - 11:52\\
&=& 133.7 - 132.25\\ 
&=& 1.45\\
\end{eqnarray*}

\[E[Y]=2000(12-E[X])= 1000 .\]
\[V[Y]=4\times 10^6 . V[X]=5.8\times 10^6 .\]
\end{enumerate}
\end{document}
