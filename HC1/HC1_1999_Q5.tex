\documentclass[a4paper,12pt]{article}
%%%%%%%%%%%%%%%%%%%%%%%%%%%%%%%%%%%%%%%%%%%%%%%%%%%%%%%%%%%%%%%%%%%%%%%%%%%%%%%%%%%%%%%%%%%%%%%%%%%%%%%%%%%%%%%%%%%%%%%%%%%%%%%%%%%%%%%%%%%%%%%%%%%%%%%%%%%%%%%%%%%%%%%%%%%%%%%%%%%%%%%%%%%%%%%%%%%%%%%%%%%%%%%%%%%%%%%%%%%%%%%%%%%%%%%%%%%%%%%%%%%%%%%%%%%%
\usepackage{eurosym}
\usepackage{vmargin}
\usepackage{amsmath}
\usepackage{graphics}
\usepackage{epsfig}
\usepackage{enumerate}
\usepackage{multicol}
\usepackage{subfigure}
\usepackage{fancyhdr}
\usepackage{listings}
\usepackage{framed}
\usepackage{graphicx}
\usepackage{amsmath}
\usepackage{chngpage}
%\usepackage{bigints}

\usepackage{vmargin}
% left top textwidth textheight headheight
% headsep footheight footskip
\setmargins{2.0cm}{2.5cm}{16 cm}{22cm}{0.5cm}{0cm}{1cm}{1cm}
\renewcommand{\baselinestretch}{1.3}

\setcounter{MaxMatrixCols}{10}

\begin{document}
  \begin{table}[ht!]
  \centering
  \begin{tabular}{|p{15cm}|}
  \hline
(i) The random variable X is exponentially distributed with probability density function 
 
   
  
< ∞<≤
=
−
,0,0 ,0,
)(
x xe
xf
x
λλ
 
 
  where λ > 0. 
 
  Obtain the distribution function of X, F(x) say, and sketch the graphs of f(x) and F(x). (7) 
 \\ \hline 
   \end{tabular}
 \end{table}
 



\begin{enumerate}[(a)]
 \item 
 
\[
F(X) = P(X \leq x) = \int^{x}_{0} \lambda e^{\lambda u} du
 = [-e^{-\lambda u}]^{x}_{0}  = 1-e^{-\lambda x}\] ($x \geq  0$)

%%%%%%%%%%%%%%%%%%%%%%%%%%%%5
  \begin{table}[ht!]
  \centering
  \begin{tabular}{|p{15cm}|}
  \hline 
 (ii) Given a random sample x1,...,xn from this distribution, write down the likelihood function and show that the maximum likelihood estimate of λ , ˆ λ  say, is the reciprocal of the sample mean. (6) 
\\ \hline 
   \end{tabular}
 \end{table}
%%%%%%%%%%%%%%%%%%%%%%%% 
 \item  L = ¦n i=1¸e¡¸xi = ¸n exp(¡¸
Pn
i=1
xi) = ¸ne¡n¸x ln(L) = n ln ¸ ¡ n¸x
d
\[d¸ (\ln(L)) = 1=¸ ¡ nx = 0 for \lambda¸ = 1=\lambda x d2\]
\[d¸2 (\ln(L)) = ¡n=¸2 < 0 \]for all ¸, 
so there is a maximum for $\lambda$¸

%%%%%%%%%%%%%%%%%%%%%%%%%

 
  \begin{table}[ht!]
  \centering
  \begin{tabular}{|p{15cm}|}
  \hline 
\noindent \textbf{Part (c)} \\Show that the second derivative of the log-likelihood with respect to 
λ
 is 
given by 
 
   2 2 2 ln λλ n d Ld −= . 
 
  You are given that, if n is large, 
 
   ) ln ,(~ˆ 1 2 2 −       − λ λλ d Ld N approximately. 
 
  By replacing λ by ˆ λ  in the variance of this distribution, obtain an approximate 95\% confidence interval for λ . (7) 
  
\\ \hline 
   \end{tabular}
 \end{table}
\item  Using the result just found ,$\lambda¸ \sim N(¸; ¸2=n) \sim N(¸; 1=n(x)2)$, and so approximately
\[\lambda¸
¡¸
(x
p
n)¡1 \sim N(0; 1)\], from which an approximate 95\% confidence interval for ¸ is given by
\[\lambda¸
§ 1:96=(x
p
n)\] or 1
x (1 § 1p:96
n )
\end{enumerate}
\end{document}
