\documentclass[a4paper,12pt]{article}
%%%%%%%%%%%%%%%%%%%%%%%%%%%%%%%%%%%%%%%%%%%%%%%%%%%%%%%%%%%%%%%%%%%%%%%%%%%%%%%%%%%%%%%%%%%%%%%%%%%%%%%%%%%%%%%%%%%%%%%%%%%%%%%%%%%%%%%%%%%%%%%%%%%%%%%%%%%%%%%%%%%%%%%%%%%%%%%%%%%%%%%%%%%%%%%%%%%%%%%%%%%%%%%%%%%%%%%%%%%%%%%%%%%%%%%%%%%%%%%%%%%%%%%%%%%%
\usepackage{eurosym}
\usepackage{vmargin}
\usepackage{amsmath}
\usepackage{graphics}
\usepackage{epsfig}
\usepackage{enumerate}
\usepackage{multicol}
\usepackage{subfigure}
\usepackage{fancyhdr}
\usepackage{listings}
\usepackage{framed}
\usepackage{graphicx}
\usepackage{amsmath}
\usepackage{chngpage}
%\usepackage{bigints}

\usepackage{vmargin}
% left top textwidth textheight headheight
% headsep footheight footskip
\setmargins{2.0cm}{2.5cm}{16 cm}{22cm}{0.5cm}{0cm}{1cm}{1cm}
\renewcommand{\baselinestretch}{1.3}

\setcounter{MaxMatrixCols}{10}
\begin{document}


\section{Introduction}
\begin{enumerate}
\item (i) To fit a bolt with X = 9:98, we must have Y between 10.00 and 10.18.
Y » N(10:10; 0:0016). z = Y ¡10:1
0:04 » N(0; 1).
For Y = 10:0, z = ¡0:1
0:04 = ¡2:5,
For Y = 10:18, z = +0:08
0:04 = +2:0.
3
P(z < 2:0) = 0:97725. P(z < ¡2:5) = 0:00621.
We require the difference of these, which is 0.97104.
\item  Y ¡ X » N(10:1 ¡ 10:0; 0:0016 + 0:0009) » N(0:1; 0:0025).
P(fit satisfactorily) = P(0:02 · Y ¡ X · 0:2). Corresponding z values for
0.02, 0.2 are z = 0:02¡0:1
0:05 = ¡0:08
0:05 = ¡1:60; z = 0:2¡0:1
0:05 = +2:00.
P(z < ¡1:60) = 0:05480, P(z < +2:00) = 0:97725, difference is 0.92245.
%%%%%%%%%%%%%%%%%%%%%%%%%%%%%%%%%%%%%%%%
\item  $Z » N(10:3; 0:0144)$. P(Z > 10:06) = P(z > 10:06¡10:3
0:12 ), where z » N(0; 1),

\begin{eqnarray}
i.e. &=& P(z > \frac{-0:24}{0:12} ) \\
&=& P(z > ¡2:0) \\ 
&=& 0:97725.
\end{eqnarray}
(a) Plates are independent, so required probability is (0:97725)2 = 0:95502.
%%%%%%%%%%%%%%%%%%%%%%%%%%%%%%%%%%
\item We require 10:08 · Y · 10:26 for nut and bolt to fit. Corresponding z values
are 10:08¡10:10
0:04 = ¡0:02
0:04 = ¡0:5 and 10:26¡10:10
0:04 = 0:16
0:04 = +4:0, above which we
may ignore the probability (strictly it is 0.00003). 
\begin{itemize}
\item P(z < ¡0:5) = 0:30854,
and the required probability is 1 ¡ 0:30854 = 0:69146. (strictly 0.69143).
\item Nut and bolt must fit and bolt go through the holes.
\item Given random choice,
and hence independence, this has probability 0:69146 £ 0:95502 = 0:66036
(or 0.66033).
\end{itemize}
%%%%%%%%%%%%%%%%%%%%%%%%%%%%%%%%%%%%%%%%%%%%%%%%%%
\item n = 25, $\bar{X}$ » N(10:0; 0:0009
25 ) » N(10:0; (0:006)2).
\begin{itemize}
\item The permitted deviation of $\bar{X}$ from 10.0 is only 0.01, corresponding to
z = § 0:01
0:006 = §1:667.
\item P(z > 1:667) = 0:04779 = P(z < ¡1:667).
\item Hence the probability is 2 £ 0:04779 = 0:09558 of stopping.
\end{itemize}
\end{enumerate}
\end{document}