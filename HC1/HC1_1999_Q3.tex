\documentclass[a4paper,12pt]{article}
%%%%%%%%%%%%%%%%%%%%%%%%%%%%%%%%%%%%%%%%%%%%%%%%%%%%%%%%%%%%%%%%%%%%%%%%%%%%%%%%%%%%%%%%%%%%%%%%%%%%%%%%%%%%%%%%%%%%%%%%%%%%%%%%%%%%%%%%%%%%%%%%%%%%%%%%%%%%%%%%%%%%%%%%%%%%%%%%%%%%%%%%%%%%%%%%%%%%%%%%%%%%%%%%%%%%%%%%%%%%%%%%%%%%%%%%%%%%%%%%%%%%%%%%%%%%
\usepackage{eurosym}
\usepackage{vmargin}
\usepackage{amsmath}
\usepackage{graphics}
\usepackage{epsfig}
\usepackage{enumerate}
\usepackage{multicol}
\usepackage{subfigure}
\usepackage{fancyhdr}
\usepackage{listings}
\usepackage{framed}
\usepackage{graphicx}
\usepackage{amsmath}
\usepackage{chngpage}
%\usepackage{bigints}

\usepackage{vmargin}
% left top textwidth textheight headheight
% headsep footheight footskip
\setmargins{2.0cm}{2.5cm}{16 cm}{22cm}{0.5cm}{0cm}{1cm}{1cm}
\renewcommand{\baselinestretch}{1.3}

\setcounter{MaxMatrixCols}{10}

\begin{document}
\begin{enumerate}
    \item 3 For boys’ heights, n1 = 100; X » N(160; 16); and for girls’ heights, n2 = 81; Y » N(150; 9)
\item (a) P(X > 156) = P(X¡160
4 > 156¡160
4 ) = P(Z > ¡1) where Z has the distribution
N(0,1). This is the same as P(Z¡1), which is 0.8413.
(b) P(Y > 156) = P(Y ¡150
3 > 156¡150
3 ) = P(Z > 2) = 0:0228
%%%%%%%%%%%%%%%%%%%%%%%%%%%%%%
(c) Probability=P(> 156jboy)P(boy) + P(> 156jgirl)P(girl) = 0:8412 £ 100=181 +
0:0228 £ 81=181 if selection is random from the whole population. This is 0.4750.
\item (a) Assuming that the boy’s heights are independent, the required probability is (0:8413)4
= 0:5010.
%%%%%%%%%%%%%%%%%%%%%%%%%%%%%%%%%%%
\item Mean height of boys » N(160; 16=4) » N(160; 4)
Hence P(mean > 156) = P(mean¡160
2 > 156¡160
2 ) = P(Z > ¡2) = P(Z < 2) by
symmetry=0.9772.
The assumption is likely to be reasonable except when, for example,they come from
the same family.
\item Assuming independence again, X ¡ Y » N(160 ¡ 150; 16 + 9) » N(10; 25).
P(X ¡Y > 0) = P(X¡Y ¡10
5 > 0¡10
5 ) = P(Z > ¡2) = 0:9772 Both X and Y are assumed
chosen from the year group.
%%%%%%%%%%%%%%%%%%%%%%%%%%%%%%%%%%%%%%%%%%%
\item  Mean height is n1X+n2Y
n1+n2
= W, and X » N(160; 16=100); Y » N(150; 9=81). i.e.X »
N(160; 0:16); Y » N(150; 1=9) We require P(155:5 < W < 156:5) or P(W < 156:5) ¡
P(W < 155:5)
V [W] = ( n1
n1+n2
)2V [X] + ( n1
n1+n2
)2V [Y ] = ( 100
181 )2(0:16) + ( 81
181 )2( 1
9 ) = 0:00488386 =
0:0222521 = 0:071091 and E[W] = 100£160+81£150
181 = 155:52486
Z-value corresponding to W = 155:5 is 155p:5¡155:52486
0:071091
i.e.¡0:02486
0:26663 = ¡0:0932 and for156.5
Z = 156p:5¡155:52486
0:071091
= 0:97514
0:26663 = 3:657
P(Z < ¡0:0932) = 0:4629 and P(Z < 3:657) = 0:9999 is required probability=0.9999-
0.4629=0.5370.

\end{enumerate}
\end{document}
