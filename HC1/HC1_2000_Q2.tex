\documentclass[a4paper,12pt]{article}
%%%%%%%%%%%%%%%%%%%%%%%%%%%%%%%%%%%%%%%%%%%%%%%%%%%%%%%%%%%%%%%%%%%%%%%%%%%%%%%%%%%%%%%%%%%%%%%%%%%%%%%%%%%%%%%%%%%%%%%%%%%%%%%%%%%%%%%%%%%%%%%%%%%%%%%%%%%%%%%%%%%%%%%%%%%%%%%%%%%%%%%%%%%%%%%%%%%%%%%%%%%%%%%%%%%%%%%%%%%%%%%%%%%%%%%%%%%%%%%%%%%%%%%%%%%%
\usepackage{eurosym}
\usepackage{vmargin}
\usepackage{amsmath}
\usepackage{graphics}
\usepackage{epsfig}
\usepackage{enumerate}
\usepackage{multicol}
\usepackage{subfigure}
\usepackage{fancyhdr}
\usepackage{listings}
\usepackage{framed}
\usepackage{graphicx}
\usepackage{amsmath}
\usepackage{chngpage}
%\usepackage{bigints}

\usepackage{vmargin}
% left top textwidth textheight headheight
% headsep footheight footskip
\setmargins{2.0cm}{2.5cm}{16 cm}{22cm}{0.5cm}{0cm}{1cm}{1cm}
\renewcommand{\baselinestretch}{1.3}

\setcounter{MaxMatrixCols}{10}
\begin{document}


\begin{framed}
A random variable 
$ {\displaystyle X} $
 follows the hypergeometric distribution if its probability mass function (pmf) is given by
\[ {\displaystyle p_{X}(k)=\Pr(X=k)={\frac {{\binom {K}{k}}{\binom {N-K}{n-k}}}{\binom {N}{n}}},} \]

where 
\begin{itemize}
\item ${\displaystyle N}$ 
 is the population size,
\item ${\displaystyle K}$ 
 is the number of success states in the population,
\item ${\displaystyle n}$ 
 is the number of draws (i.e. quantity drawn in each trial),
\item ${\displaystyle k}$ 
 is the number of observed successes,
\item ${\textstyle \textstyle {a \choose b}}$ 
 is a binomial coefficient.
\end{itemize}

\end{framed}
\begin{enumerate}
\item P(x irregular in sample) =
Ã
10
x
! Ã
40
5 ¡ x
!
=
Ã
50
5
!
For x=0,1,2,3,4,5
The sampling leads to the hypergeometric distribution:
1
Account : Ok Irrigular Total
Sampled 10 ¡ x x 10
Notsampled 35 + x 5 ¡ x 40
Total 45 5 50
So
\begin{eqnarray*}
P(X=0) &=& \frac{ {10 \choose 0} \times {40 \choose 5} }{ {50 \choose 5}}\\
&=& \frac{1 \times 40! \times 5!  \times 45!}{5! \times 35! \times 50|}\\
&=& \frac{ 40 \times 39 \times 38 \times 37 \times 36}{50 \times 49 \times 48 \times 47 \times 46}\\
&=& 0.3106
\end{eqnarray*}
(ii)
P(x ¸ 2) = 1 ¡ P(0) ¡ P(1):
\begin{eqnarray*}
P(X=1) &=& \frac{ {10 \choose 1} \times {40 \choose 4} }{ {50 \choose 5}}\\
&=& \frac{10 \times 40! \times 5!  \times 45!}{4! \times 36! \times 50|}\\
&=& \frac{10 \times 5 \times 40 \times 39 \times 38 \times 37}{50 \times 49 \times 48 \times 47 \times 46}\\
&=& 0.4313
\end{eqnarray*}


and P(x ¸ 2) = 0:2581:
\item P(server wins from deuce)=
P(ww) + P(wlww) + P(lwww) + p(wlwlww) + p(wllwww)
+P(lwwlww) + P(lwlwww) + P(wlwlwlww) + ¢ ¢ ¢
where the number of possible sequence doubles each deuce.this is
( 2
3 )2 + 2(2
3 )2( 1
3 £ 2
3 ) + 4(2
3 )2( 1
3 £ 2
3 )2 + 8(2
3 )2( 1
3 £ 2
3 )3 + ¢ ¢ ¢
= ( 2
3 )2[1 + ( 2
3 )2 + ( 2
3 )4 + ( 2
3 )6 + ¢ ¢ ¢]
= 9
4 [1 + 9
4 + ( 9
4 )2 + ( 9
4 )3 + ¢ ¢ ¢] = 4
9 £ 1
1¡4
9
= 4
5
P(score does not change after 2 points) = ( 2
3 £ 1
3 ) + ( 1
3 £ 2
3 ) = 4
9 :
2
P(game ends after 2 points) = ( 2
3 £ 2
3 ) + ( 1
3 £ 1
3 ) = 5
9 :
P(N = 2k) = 5
9 £ P(k ¡ 1 sequences of 2 points which do not change score)
= 5
9 £ ( 4
9 )k¡1 for k = 1; 2; 3; ¢ ¢ ¢
so that N=2,4,6,¢ ¢ ¢
writing 2k=n ,P(N = n) = 5
9 £ 9
4 £ ( 4
9 )n=2 = 5
4 £ ( 2
3 )n
\item

\begin{enumerate}
    \item P(C) = 0:3 P(V jC) = 0:8: P(L) = 0:4 P(V jL) = 0:6
    \item P(D) = 0:2 P(V jD) = 0:9 P(O) = 0:1 P(V jO) = 0
    \item P(NV jC) = 0:2 P(NV jL) = 0:4 P(NV jD) = 0:1; P(NV jO) = 1
\end{enumerate}

P(LjNV ) = P(NV jL)P(L) =
P
x=C;L;D;OP(NV jx)P(x)
= 0:4£0:4
(0:4£0:4)+(0:3£0:2)+(0:1£0:2)+(1£0:1) = 0:16
0:34 = 0:4706:

\[P(both LjNV ) = 0:4706^2 = 0:2215:\]

\end{enumerate}
\end{document}

