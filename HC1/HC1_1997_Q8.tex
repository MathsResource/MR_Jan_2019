\documentclass{article}
\usepackage[utf8]{inputenc}

\usepackage{enumerate}
\usepackage{framed}

\begin{document}


\section{Introduction}
\begin{enumerate}
\item 8. (a) (i) A binomial distribution with large n and very small p may be approximated
by a Poisson with ¹ = np. It is desirable that np should be at least 5, but in
addition n should be ¸ 20 and p · 0:1. When all these conditions are met
the approximation will be a good one.
\item  A Poisson with large mean can be approximated by N(¹; ¹). The
approximation will be good for ¹ ¸ 10, but adequate down to ¹ = 5.
(b) (i) Poisson, mean 5:
P(4) + P(5) + P(6) = e¡5( 54
4! + 55
5! + 56
6! )
= 625e¡5( 1
24 + 5
120 + 25
720 ) = 625e¡5( 1
12 + 5
144 ) = 0:49716:
%%%%%%%%%%%%%%%%%%%%%%%%%
\item  N(5; 5) with a continuity correction is required: find P(31
2 < X < 61
2 ) in
N(5; 5). Corresponding r-values are 3 1
2
¡5
p
5
= ¡0:6708 and 6 1
2
¡5
p
5
= +0:6708.
\[P(z < ¡0:6708) = P(z > +0:6708) = 0:25117,\] and so the required probability
is 1 ¡ 2 £ 0:25117 = 0:49766. The error is 0.0005, and % error
0:0005
0:49716 £ 100 = 0:1%:
Using the continuity correction with ¹ = 5, and calculating values which we
near to the mean, leads to a very good approximation.
(c) P(0) = e¡¸t = P(T > t) for the first event observed = 1 ¡ F(t). Hence
F(t) = 1 ¡ e¡¸t and g(t) = F
0(t) = ¸e¡¸t. (t ¸ 0; ¸ > 0).
[g(t) = 0 unless t ¸ 0; ¸ > 0 since neither time of events nor rate of events
occurring can be negative.] Use integration by parts.
E[T] =
Z 1
0
¸te¡¸tdt =
Z 1
0
td(¡e¡¸t) = [¡te¡¸t]1
0 +
Z 1
0
e¡¸tdt
= [¡
1
¸
e¡¸t]1
0 = 1=¸:
E[T2] =
Z 1
0
¸t2e¡¸tdt =
Z 1
0
t2d(¡e¡¸t) = [¡t2e¡¸t]1
0 +
Z 1
0
2te¡¸tdt
=
2
¸
E[T] = 2=¸2:
Hence V [T] =
2
¸2
¡ (
1
¸
)
2
= 1=¸2.
8
(d) ¸ = 5, so that E[T] = 0:2 and V [T] = 0:04. For n = 100, a sample mean
¯ T is approximately N(0:2; 0:04
100 ), and the range required is from 0.18 to 0.22,
within 10\% of 0.2. The corresponding values of r are 0p:18¡0:2
0:0004
= ¡0:02
0:02 = ¡1,
and the other = +1. P(r > 1) = 0:1587 = P(r < ¡1) and so the probability
between these values is 1 ¡ 2 £ 0:1587 = 0:6826.
\end{enumerate}
\end{document}