\documentclass[a4paper,12pt]{article}
%%%%%%%%%%%%%%%%%%%%%%%%%%%%%%%%%%%%%%%%%%%%%%%%%%%%%%%%%%%%%%%%%%%%%%%%%%%%%%%%%%%%%%%%%%%%%%%%%%%%%%%%%%%%%%%%%%%%%%%%%%%%%%%%%%%%%%%%%%%%%%%%%%%%%%%%%%%%%%%%%%%%%%%%%%%%%%%%%%%%%%%%%%%%%%%%%%%%%%%%%%%%%%%%%%%%%%%%%%%%%%%%%%%%%%%%%%%%%%%%%%%%%%%%%%%%
\usepackage{eurosym}
\usepackage{vmargin}
\usepackage{amsmath}
\usepackage{graphics}
\usepackage{epsfig}
\usepackage{enumerate}
\usepackage{multicol}
\usepackage{subfigure}
\usepackage{fancyhdr}
\usepackage{listings}
\usepackage{framed}
\usepackage{graphicx}
\usepackage{amsmath}
\usepackage{chngpage}
%\usepackage{bigints}

\usepackage{vmargin}
% left top textwidth textheight headheight
% headsep footheight footskip
\setmargins{2.0cm}{2.5cm}{16 cm}{22cm}{0.5cm}{0cm}{1cm}{1cm}
\renewcommand{\baselinestretch}{1.3}

\setcounter{MaxMatrixCols}{10}
\begin{document}
\begin{table}[ht!]
     \centering
     \begin{tabular}{|p{15cm}|}
     \hline        
 The probability that any given child in a certain family has blue eyes is 1/4, and this feature is independent of the eye colours of the other children in the family.  There are 5 children in the family.\\\medskip
\noindent \textbf{Part (a)}\\ 
State the distribution of the number of children with blue eyes and find the probability that at least one child has blue eyes.


\\ \hline
      \end{tabular}
    \end{table}
%%%%%%%%%%%%%%%%%%%%%%%%%%%%%%%%%%%%%%%%%%%%%%


\begin{enumerate}[(a)]
\item $p(B) = p = 1/4$. Family size $n = 5$. 
The distribution of X, the number with blue
eyes is binomial (n = 5; p = 1/4).
%%%%%%%%%%%%%%%%%%%%%%%%%%%%%%%%%%%%%%%%%%%%%%
  \begin{table}[ht!]
     \centering
     \begin{tabular}{|p{15cm}|}
     \hline  
\noindent \textbf{Part (b)}\\ Find the probability that at least three children have blue eyes, given that at least one child has blue eyes. \\ \hline 
      \end{tabular}
    \end{table}
%%%%%%%%%%%%%%%%%%%%%%%%%%%%%%%%%%%%%%%%%%%%%%    
\item \[P(X=0) = \left(\frac{3}{4}\right)^5 = \frac{243}{1024} \], so 
\begin{eqnarray*}
P(\mbox{at least 1 with blue eyes}) 
&=& P(X\geq 1) \\
&=& 1 - P(X=0)\\ 
&=& \frac{781}{1024}\\
&=& 0:7627.\\
\end{eqnarray*}
%%%%%%%%%%%%%%%%%%%%%%%%%%%%%%%%%%%%%%%%%%%%%%%%%%%%
  \begin{table}[ht!]
     \centering
     \begin{tabular}{|p{15cm}|}
     \hline  
\noindent \textbf{Part (c)}\\Find the probability that at least three of the children have blue eyes, given that the youngest child has blue eyes.
\\ \hline
      \end{tabular}
    \end{table}
\item  
\begin{eqnarray*}
P(at least 3 B j | at least 1 B)&=&P(r ¸ 3)\\
&=&P(r ¸ 1)\\ 
&=& P(r¸3)\\
&=& \frac{781}{1024}.
\end{eqnarray*}
%%%%%%%%%%%%%%%
\begin{eqnarray*}
P(X \in \{3,4,5\} &=&  P(X=3) + P(X=4) + P(X=5)\\
&=& {5 \choose 3} \left(\frac{1}{4}\right)^2 + \left(\frac{}{}\right)^2 + {5 \choose 3} \left(\frac{}{}\right)^2 + \left(\frac{}{}\right)^2 + {5 \choose 3} \left(\frac{}{}\right)^2 + \left(\frac{}{}\right)^2\\
&=& {5 \choose 3} \left(\frac{}{}\right)^2 + \left(\frac{}{}\right)^2 + {5 \choose 3} \left(\frac{}{}\right)^2 + \left(\frac{}{}\right)^2 + {5 \choose 3} \left(\frac{}{}\right)^2 + \left(\frac{}{}\right)^2\\
&=& \frac{1}{1024} \left[ 90 + 15 + 1 \right] \\
&=& \frac{106}{1024}
\end{eqnarray*}

So required answer is \[ \frac{106/1024}{781/1024} = \frac{106}{781} = 0.1357. \]

%%%%%%%%%%%%%%%%%%%%%%%%%%%%%%%%%%%%%%%%%%%%%%%%%%%%%%%%%%%%%%%%%%%
\newpage
  \begin{table}[ht!]
     \centering
     \begin{tabular}{|p{15cm}|}
     \hline  
(iv) Calculate the expected number of children with blue eyes,
\begin{enumerate}[(i)]
\item given that at least one child has blue eyes;
\item given that the youngest child has blue eyes.
\end{enumerate}
\\ \hline
      \end{tabular}
    \end{table}
%%%%%%%%%%%%%%%%%%%%%%%5
\item Given that a particular one - the youngest - has blue eyes means that of the
other four, at least two have blue eyes. This is found as P(2)+P(3)+P(4) in
binomial (4; 1=4): 

\begin{eqnarray*} 
P(2)+P(3)+P(4) &=& 
\left[ { 4 \choose 2}
\left( \frac{1}{4} \right) ^2 \left(\frac{3}{4}\right)^2 \right]+ 
\left[ { 4 \choose 3}
\left( \frac{1}{4} \right) ^3  \left(\frac{3}{4}\right)^1 \right]
+\left[ { 4 \choose 4}
\left( \frac{1}{4} \right) ^4  \left(\frac{3}{4}\right)^{0} \right]
\\ &=& 6 \left(\frac{1 \times 9}{4^4}\right) + 4 \left(\frac{1 \times 3}{4^4}\right) + 1 \left(\frac{1}{4^4}\right)\\
\\ &=&  \frac{6\times 9+4 \times 3+1}{256}    \\
 &=& \frac{67}{256} \\
 &=& 0.2617.
\end{eqnarray*}

%%%%%%%%%%%%%%%%%%%%%%%%%%%%%%%%%%%%%5
\item  Using binomial (5; 1=4) and excluding r = 0, the expected number is
1
1¡P(0)
X5
r=1
rP(r) =
1024
781
f1£5£(
1
4
)(
3
4
)
4
+2£10£(
1
4
)
2
(
3
4
)
3
+3£10£(
1
4
)
3
(
3
4
)
2
+
4£5£(
1
4
)
4
(
3
4
)+5£(
1
4
)
5



\begin{eqnarray*} 
P(2)+P(3)+P(4) &=& 
\left[ { 4 \choose 2}
\left( \frac{1}{4} \right) ^2 \left(\frac{3}{4}\right)^2 \right]+ 
\left[ { 4 \choose 3}
\left( \frac{1}{4} \right) ^3  \left(\frac{3}{4}\right)^1 \right]
+\left[ { 4 \choose 4}
\left( \frac{1}{4} \right) ^4  \left(\frac{3}{4}\right)^{0} \right]
\\  &=& 
+ \left( \frac{1}{4} \right) ^2  \left(\frac{3}{4}\right)^{3}
+ \left( \frac{1}{4} \right) ^3  \left(\frac{3}{4}\right)^{2}
+ \left( \frac{1}{4} \right) ^4  \left(\frac{3}{4}\right)^{1}
+ \left( \frac{1}{4} \right) ^5  \left(\frac{3}{4}\right)^{0} 
\\ &=&  \frac{(5\times 81+20\times 27+30\times 9+20\times 3+5)}{781}    \\
 &=& \frac{1280}{781} \\
 &=& 1.64.
\end{eqnarray*}

\item In binomial (4; 1=4), E[r] = np = 1.
So expected number is 1(youngest) + 1(others) = 2.
%%%%%%%%%%%%%%%%%%%%%%%%%%%%%%%%%%%%%%%%%%%%%%%%%%%%%%%%%%%
\newpage
  \begin{table}[ht!]
     \centering
     \begin{tabular}{|p{15cm}|}
     \hline     
(v) Explain why the answers to (ii) and (iii) are not the same.\\ \hline
      \end{tabular}
    \end{table}
\item Specific information about one child reduces the “subspace” in which we
have to search for the values of r concerning the others.
\end{enumerate}
\end{document}
