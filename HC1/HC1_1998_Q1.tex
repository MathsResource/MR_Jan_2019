\documentclass[a4paper,12pt]{article}
%%%%%%%%%%%%%%%%%%%%%%%%%%%%%%%%%%%%%%%%%%%%%%%%%%%%%%%%%%%%%%%%%%%%%%%%%%%%%%%%%%%%%%%%%%%%%%%%%%%%%%%%%%%%%%%%%%%%%%%%%%%%%%%%%%%%%%%%%%%%%%%%%%%%%%%%%%%%%%%%%%%%%%%%%%%%%%%%%%%%%%%%%%%%%%%%%%%%%%%%%%%%%%%%%%%%%%%%%%%%%%%%%%%%%%%%%%%%%%%%%%%%%%%%%%%%
\usepackage{eurosym}
\usepackage{vmargin}
\usepackage{amsmath}
\usepackage{graphics}
\usepackage{epsfig}
\usepackage{enumerate}
\usepackage{multicol}
\usepackage{subfigure}
\usepackage{fancyhdr}
\usepackage{listings}
\usepackage{framed}
\usepackage{graphicx}
\usepackage{amsmath}
\usepackage{chngpage}
%\usepackage{bigints}

\usepackage{vmargin}
% left top textwidth textheight headheight
% headsep footheight footskip
\setmargins{2.0cm}{2.5cm}{16 cm}{22cm}{0.5cm}{0cm}{1cm}{1cm}
\renewcommand{\baselinestretch}{1.3}

\setcounter{MaxMatrixCols}{10}
\begin{document}


\section{Introduction}
%%%%%%%%%%%%%%%%%%%%%%%%%%%%%%%%%%%%%%%%%%%%%%%%%%%%%%%%%%%%%%%%%%%%%
\begin{enumerate}

%%%%%%%%%%%%%%%%%%%%%%%%%%%%%%%%%%%
\item The probabilities of reading the three newspapers are
P(A) = 1
3 , P(B)=1
2 , P(C)=1
6 .
Let E be the event that an early is made.
P(EjA)=0.02, P(EjB)=0.01, P(EjC)=0.05 .
\[P(E) = P(EjA)P(A) + P(EjB)P(B) + P(EjC)P(C)\]
= (0:02 £ 1
3 ) + (0:01 £ 1
2 ) + (0:05 £ 1
6 )
= 1
6 (0:04 + 0:03 + 0:05)
= 0:12
6 = 0:02:
Since all entries are correct, the probability of winning is P(AjE),P(BjE),P(CjE) respectively for
readers of A, B, C.
P(AjE)P(E)=P(EjA)P(A) or P(AjE)=0:02£1=3
0:02 = 1
3 ;
\[P(BjE)=P(EjB)P(B)/P(E)=0:01£1=2\]
0:02 = 1
4 ;
\[P(CjE)=P(EjC)P(c)/P(E)=0:05£1=6
0:02 = 5\]
12 .
[Check: these must sum to 1 .
p
]

%%%%%%%%%%%%%%%%%%%%%%%%%%%%%%%%%%%
\item Since the readerships are large, we may ignore the need for a finite population correction, and
so the required probability will be \[ \left( \frac{1}{3}\right)^2 = \frac{1}{9}\]

%%%%%%%%%%%%%%%%%%%%%%%%%%%%%%%%%%%
\item Similarly, for any two different newspapers, in either order for first, second, the probability
will be: 
\begin{eqnarray*}
Prob &=& 2 \left[\left( \frac{1}{3} \times \frac{1}{4} \right) + \left( \frac{1}{3} \times \frac{5}{12} \right) + \left( \frac{1}{4} \times \frac{5}{12} \right)\right]

&=& \left( \frac{2}{12} + \frac{10}{36} + \frac{10}{48} \right) \\

&=& \left( \frac{12}{72} + \frac{20}{72} + \frac{15}{72} \right) \\

&=& \frac{47}{72} \\

%=% 0.653

\end{eqnarray}
%%%%%%%%%%%%%%%%%%%%%%%%%%%%%%%%%%%
There are 6 possible orders for one each of A, B, C, so the probability is 6£1
3£1
4£ 5
12 = 5
24 = 0:208 .
\end{enumerate}
\end{document}
