\documentclass{article}
\usepackage[utf8]{inputenc}

\title{RSS_Jan_2019_HC2}
\author{kobriendublin }
\date{December 2018}

\begin{document}

\maketitle

\section{Introduction}

3. Sample size n=10. In the first scheme, suppose r1; r2 are the numbers of defects classified
’major’ or ’minor’ respectively.
The batch is accepted only if (i)r1 = r2 = 0 or (ii)r1 = 0; r2 = 1 followed by r1 = r2 = 0 in the
second sample.
The probability is (1 ¡ p1)10(1 ¡ p2)10 + (1 ¡ p1)10 ¢ 10P2(1 ¡ p2)9(1 ¡ p1)10(1 ¡ p2)10
2
= (1 ¡ p1)10(1 ¡ p2)10f1 + 10p2(1 ¡ p2)9(1 ¡ p1)10g .
and so the probability of rejection is
1 ¡ (1 ¡ p1)10(1 ¡ p2)10f1 + 10p2(1 ¡ p2)9(1 ¡ p1)10g,
In the second scheme, accept only if r1 = 0; r2 = 0 or 1 .
The probability of this is (1 ¡ p1)20(1¡p2)20 + (1 ¡ p1)20 ¢ 20P2(1 ¡p2)19 and the probability of
rejection is therefore 1 ¡ (1 ¡ p1)20(1 ¡ p2)19(1 + 19p2).
Probabilities are:
p1 = 0:01 0:01 p1 = 0:02 0:02
p2 = 0:02 0:05 p2 = 0:02 0:05
Scheme1 0:150 0:304 0:241 0:385
Scheme2 0:231 0:398 0:372 0:509
Scheme 2 gives higher probabilities of rejection for all these values of p1; p2.
\end{document}
