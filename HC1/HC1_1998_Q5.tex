\documentclass[a4paper,12pt]{article}
%%%%%%%%%%%%%%%%%%%%%%%%%%%%%%%%%%%%%%%%%%%%%%%%%%%%%%%%%%%%%%%%%%%%%%%%%%%%%%%%%%%%%%%%%%%%%%%%%%%%%%%%%%%%%%%%%%%%%%%%%%%%%%%%%%%%%%%%%%%%%%%%%%%%%%%%%%%%%%%%%%%%%%%%%%%%%%%%%%%%%%%%%%%%%%%%%%%%%%%%%%%%%%%%%%%%%%%%%%%%%%%%%%%%%%%%%%%%%%%%%%%%%%%%%%%%
\usepackage{eurosym}
\usepackage{vmargin}
\usepackage{amsmath}
\usepackage{graphics}
\usepackage{epsfig}
\usepackage{enumerate}
\usepackage{multicol}
\usepackage{subfigure}
\usepackage{fancyhdr}
\usepackage{listings}
\usepackage{framed}
\usepackage{graphicx}
\usepackage{amsmath}
\usepackage{chngpage}
%\usepackage{bigints}

\usepackage{vmargin}
% left top textwidth textheight headheight
% headsep footheight footskip
\setmargins{2.0cm}{2.5cm}{16 cm}{22cm}{0.5cm}{0cm}{1cm}{1cm}
\renewcommand{\baselinestretch}{1.3}

\setcounter{MaxMatrixCols}{10}

\begin{document}

\section{Introduction}

\begin{enumerate}
\item E[X] =
1P
x=0
xe¡¸¸x=x!
=
1P
x=0
e¡¸¸x=(x ¡ 1)!
= ¸
1P
x=0
e¡¸¸x¡1=(x ¡ 1)! (in which the term for x = 0 is o)
= ¸
.
\item If P(X=k)=P(X=k+1), e¡¸¸k
k! = e¡¸¸k+1
(k+1)! ; i:e: ¸
k+1 = 1 so that ¸ = k + 1:
\item Since the mode has maximum probability, it is unique as in (ii) if ¸ is an integer but otherwise
satisfies P(X=m)

\begin{itemize}
    \item P(X=m¡1) > 1 and P(X=m+1)
\item P(X=m) < 1, where m is the modal value.
\item If e¡¸¸m
m! ¢ (m¡1)!
e¡¸¸m¡1 , then ¸
m > 1; i.e. m < ¸;
\item also if e¡¸¸m+1
(m+1)! ¢ m!
\item e¡¸¸m < 1, then ¸
\item m+1 < 1; i.e. ¸ < m + 1 or ¸ ¡ 1 < m ;
hence ¸ ¡ 1 < m < ¸ .
\end{itemize}

\item (a) ¸ = 1, so P(0)=e¡¸=1
e =0.3679 .
%%%%%%%%%%%%%%%%%%%%%%%%%%%%%%%%
\item P(0)+P(1)+P(2)=e¡1(1+1+1
2)=0.9197 .
\item Number of faults in 20m2 will follow Poisson with mean 4.
3
\item 

\begin{eqnarray*}
P(¸ 3) &=& 1 - P(0) ¡ P(1) ¡ P(2)\\
&=& 1 - e¡4(1 + 4 + 42 2! )\\
&=& 1 - 13e¡4\\
&=& 1 - 0:2381 \\
&=& 0:7619
\end{eqnarray*}
\item Number of rooms with ¸ 3 faults is Binomial(50,0.7619) which can be approximated as
N(50 £ 0:7619; 50 £ 0:7619 £ 0:2381) or N(38:095; 9:0704). 
%%%%%%%%%%%%%%%%%%%%
\begin{itemize}
\item The probability of being >40 is the
value corresponding to 40.5(with continuity correction) in this distribution:
\item Z = 40p:5¡38:095
9:0704
= 2:405
3:0117 = 0:7986
\item P(Z > 0:7986) = 0:2123
\item [The answer without a continuity correction would be 0.2635.]
\end{itemize}
%%%%%%%%%%%%%%%%%%%%
\end{enumerate}
\end{document}
