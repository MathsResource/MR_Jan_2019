Working with data at scale

We’ve all heard a lot about “big data,” but “big” is really a red herring. Oil companies, telecommunications companies, and other data-centric industries have had huge datasets for a long time. And as storage capacity continues to expand, today’s “big” is certainly tomorrow’s “medium” and next week’s “small.” The most meaningful definition I’ve heard: “big data” is when the size of the data itself becomes part of the problem. We’re discussing data problems ranging from gigabytes to petabytes of data. At some point, traditional techniques for working with data run out of steam.

What are we trying to do with data that’s different? According to Jeff Hammerbacher 2 (@hackingdata), we’re trying to build information platforms or dataspaces. Information platforms are similar to traditional data warehouses, but different. They expose rich APIs, and are designed for exploring and understanding the data rather than for traditional analysis and reporting. They accept all data formats, including the most messy, and their schemas evolve as the understanding of the data changes.

Most of the organizations that have built data platforms have found it necessary to go beyond the relational database model. Traditional relational database systems stop being effective at this scale. Managing sharding and replication across a horde of database servers is difficult and slow. The need to define a schema in advance conflicts with reality of multiple, unstructured data sources, in which you may not know what’s important until after you’ve analyzed the data. Relational databases are designed for consistency, to support complex transactions that can easily be rolled back if any one of a complex set of operations fails. While rock-solid consistency is crucial to many applications, it’s not really necessary for the kind of analysis we’re discussing here. Do you really care if you have 1,010 or 1,012 Twitter followers? Precision has an allure, but in most data-driven applications outside of finance, that allure is deceptive. Most data analysis is comparative: if you’re asking whether sales to Northern Europe are increasing faster than sales to Southern Europe, you aren’t concerned about the difference between 5.92 percent annual growth and 5.93 percent.

To store huge datasets effectively, we’ve seen a new breed of databases appear. These are frequently called NoSQL databases, or Non-Relational databases, though neither term is very useful. They group together fundamentally dissimilar products by telling you what they aren’t. Many of these databases are the logical descendants of Google’s BigTable and Amazon’s Dynamo, and are designed to be distributed across many nodes, to provide “eventual consistency” but not absolute consistency, and to have very flexible schema. While there are two dozen or so products available (almost all of them open source), a few leaders have established themselves:

Cassandra: Developed at Facebook, in production use at Twitter, Rackspace, Reddit, and other large sites. Cassandra is designed for high performance, reliability, and automatic replication. It has a very flexible data model. A new startup, Riptano, provides commercial support.
HBase: Part of the Apache Hadoop project, and modelled on Google’s BigTable. Suitable for extremely large databases (billions of rows, millions of columns), distributed across thousands of nodes. Along with Hadoop, commercial support is provided by Cloudera.
Storing data is only part of building a data platform, though. Data is only useful if you can do something with it, and enormous datasets present computational problems. Google popularized the MapReduce approach, which is basically a divide-and-conquer strategy for distributing an extremely large problem across an extremely large computing cluster. In the “map” stage, a programming task is divided into a number of identical subtasks, which are then distributed across many processors; the intermediate results are then combined by a single reduce task. In hindsight, MapReduce seems like an obvious solution to Google’s biggest problem, creating large searches. It’s easy to distribute a search across thousands of processors, and then combine the results into a single set of answers. What’s less obvious is that MapReduce has proven to be widely applicable to many large data problems, ranging from search to machine learning.

The most popular open source implementation of MapReduce is the Hadoop project. Yahoo’s claim that they had built the world’s largest production Hadoop application, with 10,000 cores running Linux, brought it onto center stage. Many of the key Hadoop developers have found a home at Cloudera, which provides commercial support. Amazon’s Elastic MapReduce makes it much easier to put Hadoop to work without investing in racks of Linux machines, by providing preconfigured Hadoop images for its EC2 clusters. You can allocate and de-allocate processors as needed, paying only for the time you use them.

Hadoop goes far beyond a simple MapReduce implementation (of which there are several); it’s the key component of a data platform. It incorporates HDFS, a distributed filesystem designed for the performance and reliability requirements of huge datasets; the HBase database; Hive, which lets developers explore Hadoop datasets using SQL-like queries; a high-level dataflow language called Pig; and other components. If anything can be called a one-stop information platform, Hadoop is it.

Hadoop has been instrumental in enabling “agile” data analysis. In software development, “agile practices” are associated with faster product cycles, closer interaction between developers and consumers, and testing. Traditional data analysis has been hampered by extremely long turn-around times. If you start a calculation, it might not finish for hours, or even days. But Hadoop (and particularly Elastic MapReduce) make it easy to build clusters that can perform computations on long datasets quickly. Faster computations make it easier to test different assumptions, different datasets, and different algorithms. It’s easer to consult with clients to figure out whether you’re asking the right questions, and it’s possible to pursue intriguing possibilities that you’d otherwise have to drop for lack of time.

Hadoop is essentially a batch system, but Hadoop Online Prototype (HOP) is an experimental project that enables stream processing. Hadoop processes data as it arrives, and delivers intermediate results in (near) real-time. Near real-time data analysis enables features like trending topics on sites like Twitter. These features only require soft real-time; reports on trending topics don’t require millisecond accuracy. As with the number of followers on Twitter, a “trending topics” report only needs to be current to within five minutes — or even an hour. According to Hilary Mason (@hmason), data scientist at bit.ly, it’s possible to precompute much of the calculation, then use one of the experiments in real-time MapReduce to get presentable results.

Machine learning is another essential tool for the data scientist. We now expect web and mobile applications to incorporate recommendation engines, and building a recommendation engine is a quintessential artificial intelligence problem. You don’t have to look at many modern web applications to see classification, error detection, image matching (behind Google Goggles and SnapTell) and even face detection — an ill-advised mobile application lets you take someone’s picture with a cell phone, and look up that person’s identity using photos available online. Andrew Ng’s Machine Learning course is one of the most popular courses in computer science at Stanford, with hundreds of students (this video is highly recommended).

There are many libraries available for machine learning: PyBrain in Python, Elefant, Weka in Java, and Mahout (coupled to Hadoop). Google has just announced their Prediction API, which exposes their machine learning algorithms for public use via a RESTful interface. For computer vision, the OpenCV library is a de-facto standard.

Mechanical Turk is also an important part of the toolbox. Machine learning almost always requires a “training set,” or a significant body of known data with which to develop and tune the application. The Turk is an excellent way to develop training sets. Once you’ve collected your training data (perhaps a large collection of public photos from Twitter), you can have humans classify them inexpensively — possibly sorting them into categories, possibly drawing circles around faces, cars, or whatever interests you. It’s an excellent way to classify a few thousand data points at a cost of a few cents each. Even a relatively large job only costs a few hundred dollars.

While I haven’t stressed traditional statistics, building statistical models plays an important role in any data analysis. According to Mike Driscoll (@dataspora), statistics is the “grammar of data science.” It is crucial to “making data speak coherently.” We’ve all heard the joke that eating pickles causes death, because everyone who dies has eaten pickles. That joke doesn’t work if you understand what correlation means. More to the point, it’s easy to notice that one advertisement for R in a Nutshell generated 2 percent more conversions than another. But it takes statistics to know whether this difference is significant, or just a random fluctuation. Data science isn’t just about the existence of data, or making guesses about what that data might mean; it’s about testing hypotheses and making sure that the conclusions you’re drawing from the data are valid. Statistics plays a role in everything from traditional business intelligence (BI) to understanding how Google’s ad auctions work. Statistics has become a basic skill. It isn’t superseded by newer techniques from machine learning and other disciplines; it complements them.

While there are many commercial statistical packages, the open source R language — and its comprehensive package library, CRAN — is an essential tool. Although R is an odd and quirky language, particularly to someone with a background in computer science, it comes close to providing “one stop shopping” for most statistical work. It has excellent graphics facilities; CRAN includes parsers for many kinds of data; and newer extensions extend R into distributed computing. If there’s a single tool that provides an end-to-end solution for statistics work, R is it.

Making data tell its story

A picture may or may not be worth a thousand words, but a picture is certainly worth a thousand numbers. The problem with most data analysis algorithms is that they generate a set of numbers. To understand what the numbers mean, the stories they are really telling, you need to generate a graph. Edward Tufte’s Visual Display of Quantitative Information is the classic for data visualization, and a foundational text for anyone practicing data science. But that’s not really what concerns us here. Visualization is crucial to each stage of the data scientist. According to Martin Wattenberg (@wattenberg, founder of Flowing Media), visualization is key to data conditioning: if you want to find out just how bad your data is, try plotting it. Visualization is also frequently the first step in analysis. Hilary Mason says that when she gets a new data set, she starts by making a dozen or more scatter plots, trying to get a sense of what might be interesting. Once you’ve gotten some hints at what the data might be saying, you can follow it up with more detailed analysis.

There are many packages for plotting and presenting data. GnuPlot is very effective; R incorporates a fairly comprehensive graphics package; Casey Reas’ and Ben Fry’s Processing is the state of the art, particularly if you need to create animations that show how things change over time. At IBM’s Many Eyes, many of the visualizations are full-fledged interactive applications.

Nathan Yau’s FlowingData blog is a great place to look for creative visualizations. One of my favorites is this animation of the growth of Walmart over time. And this is one place where “art” comes in: not just the aesthetics of the visualization itself, but how you understand it. Does it look like the spread of cancer throughout a body? Or the spread of a flu virus through a population? Making data tell its story isn’t just a matter of presenting results; it involves making connections, then going back to other data sources to verify them. Does a successful retail chain spread like an epidemic, and if so, does that give us new insights into how economies work? That’s not a question we could even have asked a few years ago. There was insufficient computing power, the data was all locked up in proprietary sources, and the tools for working with the data were insufficient. It’s the kind of question we now ask routinely.

Data scientists

Data science requires skills ranging from traditional computer science to mathematics to art. Describing the data science group he put together at Facebook (possibly the first data science group at a consumer-oriented web property), Jeff Hammerbacher said:

… on any given day, a team member could author a multistage processing pipeline in Python, design a hypothesis test, perform a regression analysis over data samples with R, design and implement an algorithm for some data-intensive product or service in Hadoop, or communicate the results of our analyses to other members of the organization 3

Where do you find the people this versatile? According to DJ Patil, chief scientist at LinkedIn (@dpatil), the best data scientists tend to be “hard scientists,” particularly physicists, rather than computer science majors. Physicists have a strong mathematical background, computing skills, and come from a discipline in which survival depends on getting the most from the data. They have to think about the big picture, the big problem. When you’ve just spent a lot of grant money generating data, you can’t just throw the data out if it isn’t as clean as you’d like. You have to make it tell its story. You need some creativity for when the story the data is telling isn’t what you think it’s telling.

Scientists also know how to break large problems up into smaller problems. Patil described the process of creating the group recommendation feature at LinkedIn. It would have been easy to turn this into a high-ceremony development project that would take thousands of hours of developer time, plus thousands of hours of computing time to do massive correlations across LinkedIn’s membership. But the process worked quite differently: it started out with a relatively small, simple program that looked at members’ profiles and made recommendations accordingly. Asking things like, did you go to Cornell? Then you might like to join the Cornell Alumni group. It then branched out incrementally. In addition to looking at profiles, LinkedIn’s data scientists started looking at events that members attended. Then at books members had in their libraries. The result was a valuable data product that analyzed a huge database — but it was never conceived as such. It started small, and added value iteratively. It was an agile, flexible process that built toward its goal incrementally, rather than tackling a huge mountain of data all at once.

This is the heart of what Patil calls “data jiujitsu” — using smaller auxiliary problems to solve a large, difficult problem that appears intractable. CDDB is a great example of data jiujitsu: identifying music by analyzing an audio stream directly is a very difficult problem (though not unsolvable — see midomi, for example). But the CDDB staff used data creatively to solve a much more tractable problem that gave them the same result. Computing a signature based on track lengths, and then looking up that signature in a database, is trivially simple.

Hiring trends for data science


It’s not easy to get a handle on jobs in data science. However, data from O’Reilly Research shows a steady year-over-year increase in Hadoop and Cassandra job listings, which are good proxies for the “data science” market as a whole. This graph shows the increase in Cassandra jobs, and the companies listing Cassandra positions, over time.
Entrepreneurship is another piece of the puzzle. Patil’s first flippant answer to “what kind of person are you looking for when you hire a data scientist?” was “someone you would start a company with.” That’s an important insight: we’re entering the era of products that are built on data. We don’t yet know what those products are, but we do know that the winners will be the people, and the companies, that find those products. Hilary Mason came to the same conclusion. Her job as scientist at bit.ly is really to investigate the data that bit.ly is generating, and find out how to build interesting products from it. No one in the nascent data industry is trying to build the 2012 Nissan Stanza or Office 2015; they’re all trying to find new products. In addition to being physicists, mathematicians, programmers, and artists, they’re entrepreneurs.

Data scientists combine entrepreneurship with patience, the willingness to build data products incrementally, the ability to explore, and the ability to iterate over a solution. They are inherently interdiscplinary. They can tackle all aspects of a problem, from initial data collection and data conditioning to drawing conclusions. They can think outside the box to come up with new ways to view the problem, or to work with very broadly defined problems: “here’s a lot of data, what can you make from it?”

The future belongs to the companies who figure out how to collect and use data successfully. Google, Amazon, Facebook, and LinkedIn have all tapped into their datastreams and made that the core of their success. They were the vanguard, but newer companies like bit.ly are following their path. Whether it’s mining your personal biology, building maps from the shared experience of millions of travellers, or studying the URLs that people pass to others, the next generation of successful businesses will be built around data. The part of Hal Varian’s quote that nobody remembers says it all:

The ability to take data — to be able to understand it, to process it, to extract value from it, to visualize it, to communicate it — that’s going to be a hugely important skill in the next decades.

Data is indeed the new Intel Inside.
