\documentclass{article}
\usepackage[utf8]{inputenc}
\usepackage{framed}
\usepackage{enumerate}

\begin{document}

\maketitle

\section{Introduction}

%%%%%%%%%%%%%%%%%%%%%%%%%%%%%%%%%%%%%%%%%%%%%%%%%%%%%%%%%%%%%%%%%%%%%%%%%%%%%%%%%%%%
7.Given ¹ = 1:81; ¾2 = (0:025)2. n=10.
(i)For A, ¯x=1.80 and s2=0.001977.
For B, ¯x=1.85 and s2=0.000689.
(n¡1)s2
A
¾2 = 9£0:001977
0:0252 = 28:47¤¤¤ » Â2
(9), giving very strong evidence to reject an NH that A’s
variability is the same as the laboratory standard, and to accept an AH that it is greater.
(n¡1)s2
B
¾2 = 9£0:000689
0:0252 = 9:92; n:s: as Â2
(9), so there is no statistical evidence that B’s variability
is unacceptable.
(ii)For A,p x¯¡1:81
0:001977=10
= ¡0:01
0:014 = ¡0:71 n:s: as t(9).
No evidence that A’s results are biased.
For B,p x¯¡1:81
0:000689=10
= 0:04
0:0083 = 4:82¤¤¤ as t(9).
B’s results do seem to be biased, because this value of t(9) leads us to reject the N.H. “mean=1.81”
Hence worker A produces results which are unbiased but very variable, while B is biased but precise.