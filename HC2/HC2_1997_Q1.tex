\documentclass[a4paper,12pt]{article}
%%%%%%%%%%%%%%%%%%%%%%%%%%%%%%%%%%%%%%%%%%%%%%%%%%%%%%%%%%%%%%%%%%%%%%%%%%%%%%%%%%%%%%%%%%%%%%%%%%%%%%%%%%%%%%%%%%%%%%%%%%%%%%%%%%%%%%%%%%%%%%%%%%%%%%%%%%%%%%%%%%%%%%%%%%%%%%%%%%%%%%%%%%%%%%%%%%%%%%%%%%%%%%%%%%%%%%%%%%%%%%%%%%%%%%%%%%%%%%%%%%%%%%%%%%%%
\usepackage{eurosym}
\usepackage{vmargin}
\usepackage{amsmath}
\usepackage{graphics}
\usepackage{epsfig}
\usepackage{enumerate}
\usepackage{multicol}
\usepackage{subfigure}
\usepackage{fancyhdr}
\usepackage{listings}
\usepackage{framed}
\usepackage{graphicx}
\usepackage{amsmath}
\usepackage{chngpage}
%\usepackage{bigints}

\usepackage{vmargin}
% left top textwidth textheight headheight
% headsep footheight footskip
\setmargins{2.0cm}{2.5cm}{16 cm}{22cm}{0.5cm}{0cm}{1cm}{1cm}
\renewcommand{\baselinestretch}{1.3}

\setcounter{MaxMatrixCols}{10}
\begin{document}
\begin{enumerate}

\item  For 1988, ¯x1 = 53:4, s1 = 19:7; also n = 750;
for 1990, ¯x2 = 55:3, s2 = 19:5; also n = 633.
If ¹1; ¹2 are the corresponding population means, H0 is ¹1 = ¹2 (or, strictly,
¹1 ¸ ¹2) and H1, to be tested, is ¹2 > ¹1.
V (¯x2 ¡ ¯x1) = s21
n1
+ s22
n2
= 19:72
750 + 19:52
633 = 1:11816, SE = 1:057.
As these are large samples of date we use a normal (r) test:
r = 55:3¡53:4
1:057 = 1:9
1:057 = 1:798.
\begin{itemize}
\item The form of H1 requires a one-tail test, with critical value 1.645 at 5%.
\item Hence we reject H0.
\item A 95\% confidence interval for the increase is 1:9§1:96£1:057 = 1:9§2:07,
or (-0.17; 3.97).
\item If we are certain that there must have been an increase we may prefer to
quote this result as (0; 3:97).
\end{itemize}

%%%%%%%%%%%%%%%%%
\item 

\begin{itemize}
\item For 1988, pM = 349/750 = 0:4653 and pF = 0:5347; n = 750.
\item For 1990, pM = 321/633 = 0:5071 and pF = 0:4929; n = 633.
\end{itemize}
The hypotheses H0: pM;1988 = pM;1990 and H1 : pM has changed can be
examined in a 2 £ 2 table of ‘observed’ frequencies and ‘those expected on
H0’.
OBSERVED(EXPECTED) 1988 1990 TOTAL
MALE 349(363:34) 321(306:66) 670
FEMALE 401(386:66) 312(326:34) 713
750 633 1383
Â2
(1) =
(349 ¡ 363:34)2
363:34
+ ¢ ¢ ¢ +
(312 ¡ 326:34)2
326:34
= (14:34)2f
1
363:34
+
1
306:66
+
1
386:66
+
1
326:34
g
= 205:6356 £ 0:011664 = 2:40n:s:
\begin{itemize}
    \item There is no evidence of change.
\item An alternative method is to use normal approximations for pM: N(p; p(1¡p)
n )
in each year and consider the difference. 
\item This would be needed if confidence
intervals had been required. 
\end{itemize}

\end{enumerate}
\end{document}
