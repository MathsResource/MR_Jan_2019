\documentclass[a4paper,12pt]{article}
%%%%%%%%%%%%%%%%%%%%%%%%%%%%%%%%%%%%%%%%%%%%%%%%%%%%%%%%%%%%%%%%%%%%%%%%%%%%%%%%%%%%%%%%%%%%%%%%%%%%%%%%%%%%%%%%%%%%%%%%%%%%%%%%%%%%%%%%%%%%%%%%%%%%%%%%%%%%%%%%%%%%%%%%%%%%%%%%%%%%%%%%%%%%%%%%%%%%%%%%%%%%%%%%%%%%%%%%%%%%%%%%%%%%%%%%%%%%%%%%%%%%%%%%%%%%
\usepackage{eurosym}
\usepackage{vmargin}
\usepackage{amsmath}
\usepackage{graphics}
\usepackage{epsfig}
\usepackage{enumerate}
\usepackage{multicol}
\usepackage{subfigure}
\usepackage{fancyhdr}
\usepackage{listings}
\usepackage{framed}
\usepackage{graphicx}
\usepackage{amsmath}
\usepackage{chngpage}
%\usepackage{bigints}

\usepackage{vmargin}
% left top textwidth textheight headheight
% headsep footheight footskip
\setmargins{2.0cm}{2.5cm}{16 cm}{22cm}{0.5cm}{0cm}{1cm}{1cm}
\renewcommand{\baselinestretch}{1.3}

\setcounter{MaxMatrixCols}{10}
\begin{document}

PAPER II : Statistical Methods
\begin{enumerate}
\item 1.(a) Yij
"
observation
= ¹
"
general mean
+ ®i
"
effect
due to
treatments
+ ¯ + j
"
effect of
being in
block j
+ ²ij
®i and ¯j are deviations from the general mean, due to which treatment has been given and
which block the unit(plot) is in; these are independent of one another.
f²ijg are mutually independent random residual terms, representing natural variation between
experimental units, each distributed normally with mean C and (constant) variance ¾2.
(b)Location totals: (1)31; (2)47; (3)42; (4)34. G=154. N=12.
Treatment totals: A,47; B,59; C,48.
P
y2=2060.
Total ss=2060-1542/12=83.667 .
Location ss=1
3 (312 + 472 + 422 + 342) ¡ 1542=12 = 53:667.
Treatment ss=1
4 (472 + 592 + 482) ¡ 1542=12 = 22:167.
Analysis of Variance:
SOURCE D:F: SUM OF SQUARES M:S:
Treatments 2 22:167 11:084 F(2; 6) = 8:49¤
Locations 3 53:667 17:889 F(3; 6) = 13:70¤¤
Residuals 6 7:883 1:306
TOTAL 11 83:667

\begin{itemize}
    \item The locations differed significantly, and therefore it was useful to use the randomized block scheme
with locations as blocks. We assume there is no blocks £ treatments interaction.
\item  For treatments, means are: A 11:75
C 12:00
B 14:75
. We are not told which comparisons(contrasts) among
treatments are important, but it is clear that the significance of F(2,6) must be due to the difference
between B and the other two.
\end{itemize}


\end{enumerate}
\end{document}
