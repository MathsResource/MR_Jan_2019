\documentclass[a4paper,12pt]{article}
%%%%%%%%%%%%%%%%%%%%%%%%%%%%%%%%%%%%%%%%%%%%%%%%%%%%%%%%%%%%%%%%%%%%%%%%%%%%%%%%%%%%%%%%%%%%%%%%%%%%%%%%%%%%%%%%%%%%%%%%%%%%%%%%%%%%%%%%%%%%%%%%%%%%%%%%%%%%%%%%%%%%%%%%%%%%%%%%%%%%%%%%%%%%%%%%%%%%%%%%%%%%%%%%%%%%%%%%%%%%%%%%%%%%%%%%%%%%%%%%%%%%%%%%%%%%
\usepackage{eurosym}
\usepackage{vmargin}
\usepackage{amsmath}
\usepackage{graphics}
\usepackage{epsfig}
\usepackage{enumerate}
\usepackage{multicol}
\usepackage{subfigure}
\usepackage{fancyhdr}
\usepackage{listings}
\usepackage{framed}
\usepackage{graphicx}
\usepackage{amsmath}
\usepackage{chngpage}
%\usepackage{bigints}

\usepackage{vmargin}
% left top textwidth textheight headheight
% headsep footheight footskip
\setmargins{2.0cm}{2.5cm}{16 cm}{22cm}{0.5cm}{0cm}{1cm}{1cm}
\renewcommand{\baselinestretch}{1.3}

\setcounter{MaxMatrixCols}{10}

\begin{document}
%%%%%%%%%%%%%%%%%%%%%%%%%%%%%%%%%%%%%%%%%%%%%%%%%%%%%%%%%%%%%%%%%%%%%%%%%%%%%%%%%%%%%%%%%%%%%%%%%%%%%%%%%%%%%%%%%%%%%%%%%%%%%%%%%
\begin{table}[ht!]
 
\centering
 
\begin{tabular}{|p{15cm}|}
 
\hline  

Question Text 3 


1. A journalist on a serious newspaper wants to write an article on the effects of income on spending patterns, contrasting low, middle and high income households’ expenditure.  Write briefing notes to bring out the information in the following table.  You should perform such simple calculations and/or draw such diagrams as you think suitable. (20) 
 
 
Weekly household expenditure by gross income decile group United Kingdom 1997 - 98,  £ 
 
\begin{verbatim}
 
 
Commodity or service 
 
 
Lowest ten percen t 
 
2nd decile group 
 
3rd decile group 
 
4th decile group 
 
5th decile group 
 
6th decile group 
 
7th decile group 
 
8th decile group 
 
9th decile group 
 
Highest ten percent 
 
All households 
 
Housing (net) 
 
14.30 
 
22.00 
 
27.10 
 
37.70 
 
45.70 
 
50.40 
 
61.20 
 
69.50 
 
82.10 
 
105.30 
 
51.50 
 
Fuel and Power 
 
8.70 
 
10.30 
 
10.70 
 
11.40 
 
12.50 
 
12.90 
 
13.10 
 
14.40 
 
14.70 
 
18.00 
 
12.70 
 
Food and nonalcoholic drinks 
 
22.40 
 
30.30 
 
36.40 
 
44.70 
 
49.90 
 
57.70 
 
62.50 
 
73.60 
 
80.70 
 
101.00 
 
55.90 
 
Alcoholic drink 
 
3.50 
 
4.50 
 
5.90 
 
8.50 
 
10.70 
 
14.70 
 
16.30 
 
18.20 
 
23.30 
 
27.80 
 
13.30 
 
Tobacco 
 
4.40 
 
5.10 
 
5.60 
 
6.90 
 
5.40 
 
6.80 
 
8.00 
 
7.00 
 
6.60 
 
5.50 
 
6.10 
 
Clothing and  footwear 
 
3.80 
 
7.60 
 
9.10 
 
13.20 
 
14.20 
 
17.90 
 
20.10 
 
27.70 
 
32.50 
 
48.50 
 
20.00 
 
Household goods 
 
8.50 
 
12.20 
 
15.50 
 
18.40 
 
21.60 
 
26.10 
 
31.20 
 
33.00 
 
44.20 
 
58.40 
 
26.90 
 
Household services 
 
5.50 
 
7.20 
 
8.60 
 
13.00 
 
13.60 
 
21.70 
 
16.50 
 
21.80 
 
26.20 
 
44.80 
 
17.90 
 
Personal goods and services 
 
3.80 
 
4.90 
 
6.80 
 
8.20 
 
9.60 
 
14.20 
 
14.30 
 
17.10 
 
18.70 
 
27.80 
 
12.50 
 
Motoring 
 
5.60 
 
9.80 
 
13.50 
 
29.40 
 
34.40 
 
46.50 
 
56.50 
 
70.70 
 
84.70 
 
115.20 
 
46.60 
 
Fares and other travel costs 
 
2.60 
 
3.40 
 
4.10 
 
6.50 
 
4.80 
 
8.30 
 
8.60 
 
8.50 
 
11.60 
 
22.90 
 
8.10 
 
Leisure goods 
 
4.40 
 
4.90 
 
6.90 
 
12.20 
 
13.00 
 
15.60 
 
18.60 
 
25.00 
 
24.70 
 
38.20 
 
16.30 
 
Leisure services 
 
7.90 
 
11.60 
 
16.00 
 
24.20 
 
27.60 
 
37.00 
 
40.20 
 
47.30 
 
69.10 
 
107.20 
 
38.80 
 
Miscellaneous 
 
0.30 
 
1.00 
 
0.80 
 
1.00 
 
1.80 
 
2.10 
 
2.40 
 
3.10 
 
3.70 
 
3.90 
 
2.00 
 
All expenditure groups 
 
95.6 
 
134.8 
 
167.2 
 
235.3 
 
264.8 
 
331.7 
 
369.3 
 
436.9 
 
527.7 
 
724.5 
 
328.8 
 
Source  :  Family Spending, 1997-98. 
\end{verbatim}
\\ \hline
  
\end{tabular}

\end{table}

%%%%%%%%%%%%%%%%%%%%%%%%%%%%%%%%%%%%%%%%%%%%%%%%%%%%%%%%%%%%%%%%%%%%%%%%%%%%%%%%%%%%%%%%%%%%%%%%%%%%%%%%%%%%%%%%%%%%%%%%%%%%%%%%%


\begin{enumerate}[(a)]
    \item 1 possible approaches would be:
\begin{itemize}
\item study the patterns of figures on selected rows,
\item consider these figures as proportions of total expenditure,
\item plot scatter grams to compare two rows,
\item distinguish major from trivial items,
\item distinguish luxuries from necessities.
\end{itemize}

There is not time to do regression or correlation analysis, and the calculation need to be
direction and simple. Diagrams might include pie charts, bar charts, simple plots using deciles
as horizontal axis, as well as scatter grams.
The choice of low, middle and high income groups could be made using any general definitions
known, e.g, for a candidate’s particular country;but this would introduce an extra step into the
calculations by having to combine deciles. In any case, a ”steady trend” shown by 10 points
may be visually more convincing.
\begin{enumerate}
\item  We may ignore ’Miscellaneous’, being such a small part of the total.
\item  Tobacco is fairly constant across all deciles; on further comment needed.
\item  Food is a necessity: % of thir row ¥ bottom row is
1 2 3 4 5 6 7 8 9 10 ALL
28:4 22:5 21:8 19:0 18:8 17:4 16:9 16:8 15:3 13:9 17:0
6
These %’s could be plotted against decide 1-10(on the x axis)and shows a steady decline
from 1 to 10 decile.
\item  Housing is essential but is provided in a variety of mays, so is with comment in a similar
form: % first row ¥ bottom row is
1 2 3 4 5 6 7 8 9 10 ALL
15:0 16:3 16:2 16:0 17:3 15:2 16:6 15:9 15:6 14:5 15:7
As a percentage it stays fairly constant,but an alternative plot would be a scatter gram
of actual expenditure on housing against total expenditure-even for a serious news paper
this may be less easy to understand than ”% constant but total goes up, so housing does
too”.
  \begin{table}[ht!]
     \centering
     \begin{tabular}{|p{15cm}|}
     \hline  
Question Text 2   
 \\ \hline 
      \end{tabular}
    \end{table}\item  Somewhere the total expenditure pattern over deciles needs to be shown, especially if it
is used as a basis for several different percentages.
\item  Fuel and power is best plotted on its own, in the Y-direction against decile number as x; a
jump from 1 to 2, then a steady increase, finally a jump from 9 to 10-could be a function
of size of house or of size of household, age structure etc. so some care in interpreting
the upper and lower ends should be taken. There are other categories similar to this one.
\item  Some categories could be combined, e.g. Motoring with Fare etc; although locality
obviously affects the balance (city/country) as does place of work, type of work-though,
for whatever reason, motoring on its own shows a very marked steady increase, possibly
due in part to model and age of car.

  \begin{table}[ht!]
     \centering
     \begin{tabular}{|p{15cm}|}
     \hline  
Question Text 3 
\\ \hline
      \end{tabular}
    \end{table}
        
\item  Leisure goods and services could be studied together, although in deciles 9 and 10 they
behave rather differently, probably showing what they are made up of different components
in different deciles.
\item It may be worth drawing pie charts for a selection of deciles, e.g. lowest, middle, highest,
showing the %’s of their expenditure in each of the 14 categories. But this needs a lot of
arithmetic; it also leads to rather full pie charts unless some categories are combined.
\end{enumerate}
\end{document}
