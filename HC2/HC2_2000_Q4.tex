\documentclass[a4paper,12pt]{article}
%%%%%%%%%%%%%%%%%%%%%%%%%%%%%%%%%%%%%%%%%%%%%%%%%%%%%%%%%%%%%%%%%%%%%%%%%%%%%%%%%%%%%%%%%%%%%%%%%%%%%%%%%%%%%%%%%%%%%%%%%%%%%%%%%%%%%%%%%%%%%%%%%%%%%%%%%%%%%%%%%%%%%%%%%%%%%%%%%%%%%%%%%%%%%%%%%%%%%%%%%%%%%%%%%%%%%%%%%%%%%%%%%%%%%%%%%%%%%%%%%%%%%%%%%%%%
\usepackage{eurosym}
\usepackage{vmargin}
\usepackage{amsmath}
\usepackage{graphics}
\usepackage{epsfig}
\usepackage{enumerate}
\usepackage{multicol}
\usepackage{subfigure}
\usepackage{fancyhdr}
\usepackage{listings}
\usepackage{framed}
\usepackage{graphicx}
\usepackage{amsmath}
\usepackage{chngpage}
%\usepackage{bigints}

\usepackage{vmargin}
% left top textwidth textheight headheight
% headsep footheight footskip
\setmargins{2.0cm}{2.5cm}{16 cm}{22cm}{0.5cm}{0cm}{1cm}{1cm}
\renewcommand{\baselinestretch}{1.3}

\setcounter{MaxMatrixCols}{10}
\begin{document}
%%%%%%%%%%%%%%%%%%%%%%%%%%%%%%%%%%%%%%%%%%%%%%%%%%%%%%%%%%%%%%%%%%%%%%%%%%%%%%%%%%%%%%%%%
\begin{table}[ht!]
     

\centering
     

\begin{tabular}{|p{15cm}|}
     

\hline 

 
4. A taxi company wishes to investigate whether switching to a new brand of tyre would alter fuel consumption.  Ten cars were selected at random from their fleet and driven on two separate occasions over a prescribed test course, once when fitted with the regular brand of tyre and on another occasion, without changing driver, when fitted with the new brand of tyre.  The order of the two tests was determined randomly for each car.  The petrol consumption in kilometres per litre was recorded on each occasion as follows: 


\begin{center}
\begin{tabular}{|c|c|} 
Car  &  1 2 3 4 5 6 7 8 9 10 \\ \hline 

New tyre & 4.0 4.7 6.6 6.2 4.4 4.0 4.7 4.1 4.2 4.7  \\ \hline 
Original tyre & 4.4 5.0 6.0 7.0 4.6 3.9 5.0 4.5 5.3 4.9  \\ \hline 
\end{tabular}
\end{center} 
 Test the null hypothesis that the type of tyre does not affect fuel consumption using 
 
(i) a sign test, 

(ii) a Wilcoxon signed-rank test. 
 
 
Under what circumstances would a parametric test have been more appropriate? (3) 
\\ \hline


\end{tabular}
    

\end{table}

%%%%%%%%%%%%%%%%%%%%%%%%%%%%%%%%%%%%%%%%%%%%%%%%%%%%%%%%%%%%%%%%%%%%%%%%%%%%%%%%%%%%%%%%%
\begin{enumerate}
\item A suitable N.H. is that the population distributions are the same with A.H.that
they are different. On the N.H. the number of instance where the new types will give a
higher figure is Binomial(10; p = y2).
there are 8 differences in favor of the original,and 2 for the new.
In B(10; 1
2 ), P(2 or less) = (1 + 10 + 45)(1
2 )10 = 0:0547 not significant. so the N.H. is
not rejected.

%%%%%%%%%%%%%%%%%%%%%%%
\item For a Wilcoxon test the N.H. is that population distributions are identical,and
the A.H is that they differ in location.
The actual difference(original-new) for each car are: +0.4,+0.3,-0.6,+0.8,+0.2,-0.1,+0.3+0.4,+1.1,
+0.2 and the ranks are 61
2 ; 41
2 ; 8; 9; 21
2 ; 1; 41
2 ; 61
2 ; 10; 21
2 : 

\begin{itemize}
    \item The sums of ranks are T¡ =
8 + 1 = 9; T+ = 46 the smaller of these is compared with the critical value for n=10,
which is T = 8; T¡ > 8 so the N.H. is not rejected.
\itemHowever, the result in (ii) is nearer to significance than that in(i)).
\item If we could assume that the differences for the 10 cars followed a normal distribution,
then a paired t-test could be applied.  \itemIt would, if valid, be were powerful than either of
those above.  
\end{itemize}
\end{enumerate}
\end{document}
