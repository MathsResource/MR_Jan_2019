\documentclass[a4paper,12pt]{article}
%%%%%%%%%%%%%%%%%%%%%%%%%%%%%%%%%%%%%%%%%%%%%%%%%%%%%%%%%%%%%%%%%%%%%%%%%%%%%%%%%%%%%%%%%%%%%%%%%%%%%%%%%%%%%%%%%%%%%%%%%%%%%%%%%%%%%%%%%%%%%%%%%%%%%%%%%%%%%%%%%%%%%%%%%%%%%%%%%%%%%%%%%%%%%%%%%%%%%%%%%%%%%%%%%%%%%%%%%%%%%%%%%%%%%%%%%%%%%%%%%%%%%%%%%%%%
\usepackage{eurosym}
\usepackage{vmargin}
\usepackage{amsmath}
\usepackage{graphics}
\usepackage{epsfig}
\usepackage{enumerate}
\usepackage{multicol}
\usepackage{subfigure}
\usepackage{fancyhdr}
\usepackage{listings}
\usepackage{framed}
\usepackage{graphicx}
\usepackage{amsmath}
\usepackage{chngpage}
%\usepackage{bigints}

\usepackage{vmargin}
% left top textwidth textheight headheight
% headsep footheight footskip
\setmargins{2.0cm}{2.5cm}{16 cm}{22cm}{0.5cm}{0cm}{1cm}{1cm}
\renewcommand{\baselinestretch}{1.3}

\setcounter{MaxMatrixCols}{10}

\begin{document}  
%%%%%%%%%%%%%%%%%%%%%%%%%%%%%%%%%%%%%%%%%%%%%%%%%%%%%%%%%%%%%%%%%%%%%%%%%%%%%%%%%%%%%%%%%%%%%%%%%%%%%%%%%%%%%%%%%%%%%%%%%%%%%%%%%
\begin{table}[ht!]
 
\centering
 
\begin{tabular}{|p{15cm}|}
 
\hline  


 The gender, birth weights in grams and week of pregnancy in which delivery occurred in a random sample of 24 new born babies born in a particular maternity hospital are given in the following table. 
 
 
 Gender Birth weight (grams) 
Week of delivery 
1 Male 3279 41 
2 Female 2951 37 
3 Female 2967 40 
4 Male 2886 39 
5 Female 2738 37 
6 Male 2764 35 
7 Male 3462 44 
8 Male 3386 41 
9 Female 3203 42 
10 Female 2861 36 
11 Male 2765 38 
12 Male 3199 40 
13 Female 3294 41 
14 Female 2412 35 
15 Male 3153 40 
16 Female 3473 44 
17 Male 2803 38 
18 Male 2779 36 
19 Male 2661 39 
20 Female 2614 39 
21 Male 3067 41 
22 Female 2962 40 
23 Female 2952 42 
24 Male 2596 39 
 
 
 
(i) Draw a box and whisker plot of the birth weight data and comment on the distribution.  
Note that, for the birth weight data, median = 2951.5, lower quartile = 2761.5, upper quartile = 3201.0. (6) 

\\ \hline
  
\end{tabular}

\end{table}


%%%%%%%%%%%%%%%%%%%%%%%%%%%%%%%%%%%%%%%%%%%%%%%%%%%%%%%%%%%%%%%%%%%%%%%%%%%%%%%%%%%%%%%%%%%%%%%%%%%%%%%%%%%%%%%%%%%%%%%%%%%%%%%%%
        
\begin{enumerate}
    \item For the combined data(both sexes), minium=2412, maximim=3473; quartiles are 2761.5
and 3201.0, median=2951.5.
\begin{itemize}
    \item The median is below the center of the box, so that the central half of the weights show
some tendency to have more below ”average” than above.
\item The lower whisker is somewhat
longer than the upper one, suggesting that relatively small babies can be distinctly small
while relatively large ones are not quite so far from the others.
\item The second smallest is 2596, 184 above the minimum, whereas the second largest is
3462(and the next 3386), new to the maximum. 
\item So the smallest of all may be an ”outlier”.
Some medical information would be interesting.
\end{itemize}

\begin{table}[ht!]
 
\centering
 
\begin{tabular}{|p{15cm}|}
 
\hline  
 
(ii) Calculate a 95% confidence interval for the mean of the birth weight of a new born baby stating any assumptions which you make.  
 
\\ \hline
  
\end{tabular}

\end{table}


\item  It seems not unreasonable to assume that we have a sample from an approximately
normal distribution, and or this basis the 95% confidence interval for u is ¯x§t23
p
s2=24,
where ¯x =
P
xi=24 and s2 = 1
23 (
P
x2
i ¡ (
P
x1)2=24)
P
xi = 71227:
Therefore ¯x = 2967:79 and s2 = 83213:2156; so s = 288:467 The (approximate) interval
is 2967:79 § 2:069 £ 288:467=4:899 or 2846 to 3090.

\begin{table}[ht!]
 
\centering
 
\begin{tabular}{|p{15cm}|}
 
\hline  
 
(iii) A baby is said to be a full term if it is delivered during or after the 40th week of pregnancy.  
Using an appropriate statistical test investigate whether the mean birth weight of full term babies differs between males and females.  
Question Text 3 

\\ \hline
  
\end{tabular}

\end{table}
\item Full-term babies:
Males 3279; 3462; 3386; 3199; 3153 3067: n = 6
P
xi = 19546; ¯x = 3257:67; s2 =
21885:5; s = 147:94;
Females 2967; 3203; 3294; 3473; 2962; 2952: n = 6;
P
xi = 18851; ¯x = 3141:83; s2 =
47102:2; s = 217:03.
F(5;5) = 2:15 for the ratio of variances, which is not significant so it is valid to pool them
and use s20
= 68987:6335=2 = 34493:82. so =185.725.
A two-sample t-test can be used: ¯xM¡¯xF
S0
p
1=t+1=t
that is 115.84/107.23=1.08 n.s. as t(10),
giving no evidence against the Null Hypothesis of equal mean weights. Each population
must be assumed normally distributed, with the same variance.
\end{enumerate}
\end{document}
