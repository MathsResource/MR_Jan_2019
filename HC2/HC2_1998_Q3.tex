\documentclass{article}
\usepackage[utf8]{inputenc}
\usepackage{framed}
\usepackage{enumerate}

\begin{document}

\maketitle

\section{Introduction}


3.(a)If we can assume that the lifetime distribution for the bulbs is normal with variance ¾2,
and all observations are independent of one another, then (n ¡ 1)s2=¾2 will be distributed Â2
n¡1.
Here n=10, and on H0 we take ¾2 = 1502. Then effectively we test H0 : ¾2 · 1502 against
H1 : ¾2 > 1502.
For the data,(n ¡ 1)s2 = 9 £ 35410:99. Hence Â2
(9) = 14:16, which is not significant at the 5%
level. Therefore we do not have enough evidence to reject H0 which says ¾ · 150.
(b)Since twelve randomly selected batches were used from each process we have independent estimates
of variances ¾2
1, ¾2
2. The Null Hypothesis will be ¾2
1 = ¾2
2, and AH ¾2
1 > ¾2
2.
From the data, s21
= 0:012536 and s22
= 0:003590.
Assuming that the distributions of impurity levels are normal, s21
=s22
is distributed as F(11,11).
s21
s22
= 3:49¤, significant at the 5% level so that H0 is rejected in favor of H1: there is evidence of a
reduction in process variability.