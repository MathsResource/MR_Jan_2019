\documentclass[a4paper,12pt]{article}
%%%%%%%%%%%%%%%%%%%%%%%%%%%%%%%%%%%%%%%%%%%%%%%%%%%%%%%%%%%%%%%%%%%%%%%%%%%%%%%%%%%%%%%%%%%%%%%%%%%%%%%%%%%%%%%%%%%%%%%%%%%%%%%%%%%%%%%%%%%%%%%%%%%%%%%%%%%%%%%%%%%%%%%%%%%%%%%%%%%%%%%%%%%%%%%%%%%%%%%%%%%%%%%%%%%%%%%%%%%%%%%%%%%%%%%%%%%%%%%%%%%%%%%%%%%%
\usepackage{eurosym}
\usepackage{vmargin}
\usepackage{amsmath}
\usepackage{graphics}
\usepackage{epsfig}
\usepackage{enumerate}
\usepackage{multicol}
\usepackage{subfigure}
\usepackage{fancyhdr}
\usepackage{listings}
\usepackage{framed}
\usepackage{graphicx}
\usepackage{amsmath}
\usepackage{chngpage}
%\usepackage{bigints}

\usepackage{vmargin}
% left top textwidth textheight headheight
% headsep footheight footskip
\setmargins{2.0cm}{2.5cm}{16 cm}{22cm}{0.5cm}{0cm}{1cm}{1cm}
\renewcommand{\baselinestretch}{1.3}

\setcounter{MaxMatrixCols}{10}
\begin{document}
Pair A B C D E F G H I J
Sign (Gp.2 - Gp.1) + + + ¡ ¡ + ¡ + + + 7+; 3¡
Difference +3 +8 +5 ¡1 ¡1 +25 ¡1 +3 +19 +10
Rank 41
2 7 6 2 2 10 2 41
2 9 8
11
\begin{enumerate}

\item  The number of + signs should be binomial (n = 10; p = 1=2) on the Null
Hypotheses of no difference between groups (i.e. training methods). 
\begin{itemize}
\item Using
a continuity correction, find P(r ¸ 7) in N(5; 5=2):
r = 6 1
2
¡5
p
2:5
= 1:5
1:581 = 0:949, n.s., so no evidence of difference.
\item The exact probability P(7)+P(8)+P(9)+P(10) in B(10; 1=2) = 1
210 (( 10
7 )+
( 10
8 ) + ( 10
9 ) + ( 10
10 )) = 1
210 (120 + 45 + 10 + 1) = 176
1024 = 0:172. 
\item So the
probability of the given result in a 2-tail test (A. H. “there is a difference
between groups”, direction not specified) is 0.344. 
\item  Again no evidence of any
difference.
\end{itemize}
%%%%%%%%%%%%%%%%%%%%%%%
\item  The sum of the positive ranks is 49, and of negative 6. The value 6 is
(approximately) N( 1
4n(n + 1); 1
24n(n + 1)(n + 2)), making no allowance for
the ties in the ranks (3 of -1 and 2 of +3). n = 10, the number of nonzero
differences, so n(n+1)
4 = 27:5 and 1
24n(n + 1)(n + 2) = 96:25. Using a
continuity correction, r = 6p:5¡27:5
96:25
= ¡21:0
9:81 = ¡2:14¤.
\begin{itemize}
    \item At the 5\% level, there is significant evidence against the N. H. [Using the
Wilcoxon table, the critical number is 8, and 6, being less than this, is
significant at 5%.]
\item This test uses the information on numerical sizes of differences, whereas the
sign test does not.
\item All the negative ones were very small.
\item If the differences had appeared to be normally distributed, a t-test (paired
version) would have been appropriate. 
\item This seems very unliablely, since there
is no clustering around a mean, and there are several large values.
\end{itemize}

\end{enumerate}
\end{document}