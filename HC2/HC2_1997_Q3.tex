3.
Pair A B C D E F G H I J
Sign (Gp.2 - Gp.1) + + + ¡ ¡ + ¡ + + + 7+; 3¡
Difference +3 +8 +5 ¡1 ¡1 +25 ¡1 +3 +19 +10
Rank 41
2 7 6 2 2 10 2 41
2 9 8
11
(i) The number of + signs should be binomial (n = 10; p = 1=2) on the Null
Hypotheses of no difference between groups (i.e. training methods). Using
a continuity correction, find P(r ¸ 7) in N(5; 5=2):
r = 6 1
2
¡5
p
2:5
= 1:5
1:581 = 0:949, n.s., so no evidence of difference.
The exact probability P(7)+P(8)+P(9)+P(10) in B(10; 1=2) = 1
210 (( 10
7 )+
( 10
8 ) + ( 10
9 ) + ( 10
10 )) = 1
210 (120 + 45 + 10 + 1) = 176
1024 = 0:172, and so the
probability of the given result in a 2-tail test (A. H. “there is a difference
between groups”, direction not specified) is 0.344. Again no evidence of any
difference.
(ii) The sum of the positive ranks is 49, and of negative 6. The value 6 is
(approximately) N( 1
4n(n + 1); 1
24n(n + 1)(n + 2)), making no allowance for
the ties in the ranks (3 of -1 and 2 of +3). n = 10, the number of nonzero
differences, so n(n+1)
4 = 27:5 and 1
24n(n + 1)(n + 2) = 96:25. Using a
continuity correction, r = 6p:5¡27:5
96:25
= ¡21:0
9:81 = ¡2:14¤.
At the 5% level, there is significant evidence against the N. H. [Using the
Wilcoxon table, the critical number is 8, and 6, being less than this, is
significant at 5%.]
This test uses the information on numerical sizes of differences, whereas the
sign test does not. All the negative ones were very small.
If the differences had appeared to be normally distributed, a t-test (paired
version) would have been appropriate. This seems very unliablely, since there
is no clustering around a mean, and there are several large values.
4. (A)
1 2
R 11 13 : 24
NR 7 2 : 9
18 15 33
j
More extreme tables are
10 14 : 24
8 1 : 9
18 15 33
and 9 15 : 24
9 0 : 9
18 15 33
Together, these form the “tail” of the distribution when margins are fixed.
probability are
18! 15! 24! 9!
33! 11! 7! 13! 2!
;
18! 15! 24! 9!
33! 10! 8! 14! 1!
;
18! 15! 24! 9!
33! 9! 9! 15! 0! :
i.e. 18£17£14
55£31£29 = 0:08664; 9£17
31£290 = 0:01703; 17
31£29£15 = 0:00126.
For a 2-tail test of the Null Hypothesis of no difference, the probability is
2(0:08664+0:01702+0:00126) = 0:2098. There is no significant evidence for
any difference between the two drugs.
12
(B) The Â2
(1) test also tests the N. H. that the proportions recovering are the
same on each drug. ‘Expected’ frequencies are those given by this N. H. with
the same marginal totals as the ‘Observed’.
OBSERVED (EXPECTED) Drug1 Drug2
Recovered 11(12:65) 13(11:35) 24
Not recovered 18(16:35) 13(14:65) 31
29 26 55
Â2
(1) = (1:65)2( 1
12:65 + 1
11:35 + 1
16:35 + 1
14:65 ) = 2:7225 £ 0:296578 = 0:807 n.s.
Again there is no significant evidence against the N. H.
(ii) The inference is the same in this, rather small, trial whether or not the
drop-outs are included. There were 11 drop-outs on each drug, which is a
considerable proportion of patients beginning the trial; however, no particular
reasons for drop-out are known.
