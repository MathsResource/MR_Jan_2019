\documentclass[a4paper,12pt]{article}
%%%%%%%%%%%%%%%%%%%%%%%%%%%%%%%%%%%%%%%%%%%%%%%%%%%%%%%%%%%%%%%%%%%%%%%%%%%%%%%%%%%%%%%%%%%%%%%%%%%%%%%%%%%%%%%%%%%%%%%%%%%%%%%%%%%%%%%%%%%%%%%%%%%%%%%%%%%%%%%%%%%%%%%%%%%%%%%%%%%%%%%%%%%%%%%%%%%%%%%%%%%%%%%%%%%%%%%%%%%%%%%%%%%%%%%%%%%%%%%%%%%%%%%%%%%%
\usepackage{eurosym}
\usepackage{vmargin}
\usepackage{amsmath}
\usepackage{graphics}
\usepackage{epsfig}
\usepackage{enumerate}
\usepackage{multicol}
\usepackage{subfigure}
\usepackage{fancyhdr}
\usepackage{listings}
\usepackage{framed}
\usepackage{graphicx}
\usepackage{amsmath}
\usepackage{chngpage}
%\usepackage{bigints}

\usepackage{vmargin}
% left top textwidth textheight headheight
% headsep footheight footskip
\setmargins{2.0cm}{2.5cm}{16 cm}{22cm}{0.5cm}{0cm}{1cm}{1cm}
\renewcommand{\baselinestretch}{1.3}

\setcounter{MaxMatrixCols}{10}
\begin{document}
\begin{table}[ht!]
     \centering
     \begin{tabular}{|p{15cm}|}
     \hline        
 \noindent \textbf{Part (a)}\\
Manufactured items are submitted for inspection in large batches.  Any item can have a major defect or a minor defect or both; these defects occur independently with probabilities p1 and p2 respectively.  Occurrences of defects in different items are also independent.  As part of the inspection process, a random sample of 10 items is selected from each batch.  If a major defect is found the batch is rejected and subjected to full inspection.  If the sample contains only one defect, a minor one, a second sample of 10 items is taken and if this contains any defect the batch is again rejected. 
\\ \hline
 \end{tabular}
\end{table}

\begin{table}[ht!]
     \centering
     \begin{tabular}{|p{15cm}|}
     \hline        
 \noindent \textbf{Part (a)}\\
Derive an expression for the total probability of rejection under this scheme. 
Evaluate this probability for the four cases (p1, p2 ) where p1 = 1\% and 2\% and p2=2\% and 5\%.

Compare the above with a scheme where a single sample of 20 items is taken and the batch is rejected if a major defect or more than one minor defect is found.
\\ \hline
 \end{tabular}
\end{table}
%%%%%%%%%%%%%%%%%%%%%%%%%%%%%%%%%%%%%%%%%%%%%%%
\begin{enumerate}
    \item Sample size n=10. In the first scheme, suppose $r_1$; $r_2$ are the numbers of defects classified
’major’ or ’minor’ respectively.
\item The batch is accepted only if (i)$r_1 = r_2 = 0$ or (ii)$r_1 = 0; r_2 = 1$ followed by $r_1 = r_2 = 0$ in the
second sample.
The probability is \[(1 - p_1)10(1 - p_2)10 + (1 - p_1)^{10} ¢ 10p_2(1 - p_2)^9(1 - p_1)^{10}(1 - p_2)^{10}
2\]
\[= (1 - p_1)^{10}(1 - p_2)^{10}f1 + 10p_2(1 - p_2)^{9}(1 - p_1)^{10}g .\]
and so the probability of rejection is
\[1 - (1 - p_1)^{10}(1 - p_2)^{10}(1 + 10p_2(1 - p_2)^9(1 - p_1)^{10}),\]
\item In the second scheme, accept only if $r_1 = 0$; $r_2 = 0$ or $1$ .
The probability of this is \[(1 - p_1)^{20}(1-p_2)^{20} + (1 - p_1)^{20} ¢ 20p_2(1 -p_2)^{19}\] and the probability of
rejection is therefore \[1 - (1 - p_1)^{20}(1 - p_2)^{19}(1 + 19p_2).\]
Probabilities are:
\begin{center}
\begin{tabular}{|c|c|c|c|c|}\hline
&$p_1$ = 0.01 & 0.01& $p_1$ = 0.02&  0.02 \\
&$p_2$ = 0.02 &0.05 & $p_2$ = 0.02&  0.05\\  \hline
Scheme1 & 0.150 & 0.304&  0.241&  0.385\\ \hline
Scheme2 & 0.231 & 0.398&  0.372&  0.509\\ \hline
\end{tabular}
\end{center}

Scheme 2 gives higher probabilities of rejection for all these values of $p_1$; $p_2$.
\end{enumerate}
\end{document}
