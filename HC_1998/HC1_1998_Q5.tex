\documentclass[a4paper,12pt]{article}
%%%%%%%%%%%%%%%%%%%%%%%%%%%%%%%%%%%%%%%%%%%%%%%%%%%%%%%%%%%%%%%%%%%%%%%%%%%%%%%%%%%%%%%%%%%%%%%%%%%%%%%%%%%%%%%%%%%%%%%%%%%%%%%%%%%%%%%%%%%%%%%%%%%%%%%%%%%%%%%%%%%%%%%%%%%%%%%%%%%%%%%%%%%%%%%%%%%%%%%%%%%%%%%%%%%%%%%%%%%%%%%%%%%%%%%%%%%%%%%%%%%%%%%%%%%%
\usepackage{eurosym}
\usepackage{vmargin}
\usepackage{amsmath}
\usepackage{graphics}
\usepackage{epsfig}
\usepackage{enumerate}
\usepackage{multicol}
\usepackage{subfigure}
\usepackage{fancyhdr}
\usepackage{listings}
\usepackage{framed}
\usepackage{graphicx}
\usepackage{amsmath}
\usepackage{chngpage}
%\usepackage{bigints}

\usepackage{vmargin}
% left top textwidth textheight headheight
% headsep footheight footskip
\setmargins{2.0cm}{2.5cm}{16 cm}{22cm}{0.5cm}{0cm}{1cm}{1cm}
\renewcommand{\baselinestretch}{1.3}

\setcounter{MaxMatrixCols}{10}

\begin{document}
\begin{table}[ht!]
\centering
\begin{tabular}{|p{15cm}|}
\hline        

The random variable X has a Poisson distribution with parameter λ , so that
\[P(X=x) =  \left{
\begin{cases}  \frac{e −λ }{x!}  &  x = 0,1,2,...,\\
      0  & \mbox{otherwise.} \\
\end{cases}    \]  
      \\ \hline
 \end{tabular}
\end{table} \begin{table}[ht!]
     \centering
     \begin{tabular}{|p{15cm}|}
     \hline        
 \noindent \textbf{Part (a)}\\
\noindent
Find the expected values of $X$, i.e. $E(X)$.

      \\ \hline
 \end{tabular}
\end{table} 
%%%%%%%%%%%%%%%%%%%%%%%%%%%%%%%%%%%%%%

\begin{enumerate}[(a)]
\item 
\begin{eqnarray}
E[X] &=& \frac{\sum^{\infty}_{x=0} x \cdot f(x)
\\ &=&
\frac{\sum^{\infty}_{x=0} xe^{-\lambda}x}{x!}
\\ &=&
\frac{\sum^{\infty}_{x=0}
e^{-\lambda}x}{(x - 1)!}
\\ &=& {\lambda}
\frac{\sum^{\infty}_{x=0}
e^{-\lambda}x-1}{(x - 1)!} \mbox{(in which the term for x = 0 is 0)}
\\ &=& {\lambda}
\end{eqnarray}






\begin{table}[ht!]
     \centering
     \begin{tabular}{|p{15cm}|}
     \hline        
 \noindent \textbf{Part (b)}\\
\noindent

(ii) If p(X = k) = p(X = k + l), where k is some integer, show that λ must be the integer k + 1.
\\ \hline
 \end{tabular}
\end{table} \begin{table}[ht!]
     \centering
     \begin{tabular}{|p{15cm}|}
     \hline        
 \noindent \textbf{Part (b)}\\
\noindent
(iii) If 
λ is not an integer, show that the mode, m, of the distribution satisfies the inequality λ−1< m < λ .
\\ \hline
 \end{tabular}
\end{table} 



\begin{table}[ht!]
     \centering
     \begin{tabular}{|p{15cm}|}
     \hline        
 \noindent \textbf{Part (b)}\\
\noindent (v) A new hotel requires curtains for 50 rooms.  All the curtains will be made from the material described above, and each room's curtains will use 20m2 of material.
(a) Find the probability that the curtains in any given room will contain 3 or more faults.


\\ \hline
 \end{tabular}
\end{table}



\item If P(X=k)=P(X=k+1), \[\frac{e^{-\lambda}k}{
k!} = \frac{e^{-\lambda}k+1}{(k+1)!}\] ; i:e: ${\lambda}
k+1 = 1$ so that ${\lambda} = k + 1:$
\item Since the mode has maximum probability, it is unique as in (ii) if ${\lambda}$ is an integer but otherwise
satisfies P(X=m)

\begin{itemize}
    \item $P(X=m-1) > 1$ and $P(X=m+1)$
\item $P(X=m) < 1$, where m is the modal value.
\item If \[e^{-\lambda}m
m! ¢ (m-1)!
e^{-\lambda}m-1 \], then \[{\lambda}
m > 1; i.e. m < {\lambda};\]
\item also if $e^{-\lambda}m+1$
(m+1)! ¢ m!
\item $e^{-\lambda}m < 1$, then ${\lambda}$
\item $m+1 < 1$; i.e. ${\lambda} < m + 1$ or ${\lambda} - 1 < m $;
hence ${\lambda} - 1 < m < {\lambda}$ .
\end{itemize}

\item (a) ${\lambda} = 1$, so $P(0)=e^-{\lambda}=1
e =0.3679$ .
%%%%%%%%%%%%%%%%%%%%%%%%%%%%%%%%
\begin{table}[ht!]
     \centering
     \begin{tabular}{|p{15cm}|}
     \hline        
 \noindent \textbf{Part (b)}\\
\noindent (b) Use a suitable approximation to find the probability that more than 40 of the rooms will have curtains containing 3 or more faults.
 
\\ \hline
 \end{tabular}
\end{table}\item 
\begin{eqnarray}
P(0)+P(1)+P(2)&=&e-1(1+1+1
2)\\ &=& 0.9197 .
\end{eqnarray}
\item Number of faults in 20m2 will follow Poisson with mean 4.
3
\item 

\begin{eqnarray*}
P({\lambda} 3) &=& 1 - P(0) - P(1) - P(2)\\
&=& 1 - e^{-4}(1 + 4 + 42 2! )\\
&=& 1 - 13e^{-4}\\
&=& 1 - 0.2381 \\
&=& 0.7619
\end{eqnarray*}
\item Number of rooms with ${\lambda} = 3$ faults is Binomial(50,0.7619) which can be approximated as
\[N(50 \times  0.7619; 50 \times  0.7619 \times  0.2381)\] or $N(38.095; 9.0704$). 
\begin{table}[ht!]
     \centering
     \begin{tabular}{|p{15cm}|}
     \hline        
 \noindent \textbf{Part (b)}\\
\noindent(iv) In the manufacture of curtain material, small flaws occur at random in the material at a mean rate of 1 per 5m2.  Find the probability that in a randomly selected 5m2 area of this material
(a) there are no faults,
(b) there are at most 2 faults.
\\ \hline
 \end{tabular}
\end{table}
%%%%%%%%%%%%%%%%%%%%%%%%%%%%%%%%%%
\begin{table}[ht!]
     \centering
     \begin{tabular}{|p{15cm}|}
     \hline        
 \noindent \textbf{Part (b)}\\
\noindent Question Text

\\ \hline
 \end{tabular}
\end{table} 
%%%%%%%%%%%%%%%%%%%%%%%%%%%
%%%%%%%%%%%%%%%%%%%%
\begin{itemize}
\item The probability of being >40 is the
value corresponding to 40.5(with continuity correction) in this distribution:
\item Z = 40p:5-38.095
9.0704
= 2:405
3.0117 = 0.7986
\item P(Z > 0.7986) = 0.2123
\item The answer without a continuity correction would be 0.2635.
\end{itemize}
%%%%%%%%%%%%%%%%%%%%
\end{enumerate}
\end{document}
