\documentclass[a4paper,12pt]{article}
%%%%%%%%%%%%%%%%%%%%%%%%%%%%%%%%%%%%%%%%%%%%%%%%%%%%%%%%%%%%%%%%%%%%%%%%%%%%%%%%%%%%%%%%%%%%%%%%%%%%%%%%%%%%%%%%%%%%%%%%%%%%%%%%%%%%%%%%%%%%%%%%%%%%%%%%%%%%%%%%%%%%%%%%%%%%%%%%%%%%%%%%%%%%%%%%%%%%%%%%%%%%%%%%%%%%%%%%%%%%%%%%%%%%%%%%%%%%%%%%%%%%%%%%%%%%
\usepackage{eurosym}
\usepackage{vmargin}
\usepackage{amsmath}
\usepackage{graphics}
\usepackage{epsfig}
\usepackage{enumerate}
\usepackage{multicol}
\usepackage{subfigure}
\usepackage{fancyhdr}
\usepackage{listings}
\usepackage{framed}
\usepackage{graphicx}
\usepackage{amsmath}
\usepackage{chngpage}
%\usepackage{bigints}

\usepackage{vmargin}
% left top textwidth textheight headheight
% headsep footheight footskip
\setmargins{2.0cm}{2.5cm}{16 cm}{22cm}{0.5cm}{0cm}{1cm}{1cm}
\renewcommand{\baselinestretch}{1.3}

\setcounter{MaxMatrixCols}{10}
\begin{document}

PAPER II : Statistical Methods
%%%%%%%%%%%%%%%%%%%%%%%%%%%%%%%%%%%%%%%%%%%%%%%%%%%%%%%%%%%%%%%%%%%%%%%%%%%%%%%%%%%%%%%%%%%%%%%%%%%%%%%%%%%%%%%%%%%%%
\begin{table}[ht!]
 
\centering
 
\begin{tabular}{|p{15cm}|}
 
\hline  

1. (a) State the linear fixed effects additive model used for a two-way analysis of variance.  Explain clearly what each term in the model represents and state any assumptions required for the analysis to be valid.

\\ \hline
  
\end{tabular}

\end{table}

\begin{table}[ht!]
 
\centering
 
\begin{tabular}{|p{15cm}|}
 
\hline  

(b) An experiment was performed to determine the relative effects of three different soil preparations on the first-year growth of pine seedlings.  Each soil preparation was randomly applied to one of three plots at each of four separate locations.  On each plot 20 seedlings were planted and the following table shows the average first-year growth (in cms) of the seedlings on each plot.

\begin{center}
\begin{tabular}{|c|c|c|c|c|}\hline
Soil preparation  &      1 &         2  &       3&  4 \\ \hline \hline
A: no preparation  &      9     &    16   &    12 &  10 \\ \hline
B: burning  &    12     &   18    &  16 & 13 \\ \hline
C: fertilisation  &    10    &    13  &    14 & 11  \\ \hline
\end{tabular}
\end{center}

Carry out a suitable analysis of these data.  Explain clearly your conclusions and comment upon the usefulness, or otherwise, of using a randomised block design for this experiment.

\\ \hline
  
\end{tabular}

\end{table}
%%%%%%%%%%%%%%%%%%%%%%%%%%%%%%%%%%%%%%%%%%%%%%%%%%%%%%%%%%%%%%%%%%%%%%%%%%%%%%%%%%%%%%%%%%%%%%%%%%%%%%%%%%%%%%%%%%%%%
\begin{enumerate}
\item 1.(a) Yij
"
observation
= ¹
"
general mean
+ ®i
"
effect
due to
treatments
+ ¯ + j
"
effect of
being in
block j
+ ²ij
®i and ¯j are deviations from the general mean, due to which treatment has been given and
which block the unit(plot) is in; these are independent of one another.
f²ijg are mutually independent random residual terms, representing natural variation between
experimental units, each distributed normally with mean C and (constant) variance ¾2.
(b)Location totals: (1)31; (2)47; (3)42; (4)34. G=154. N=12.
Treatment totals: A,47; B,59; C,48.
P
y2=2060.
\begin{itemize}
    \item Total ss=2060-1542/12=83.667 .
\item Location ss=1
3 (312 + 472 + 422 + 342) - 1542=12 = 53:667.
\item Treatment ss=1
4 (472 + 592 + 482) ¡ 1542=12 = 22:167.
\end{itemize}


%%%%%%%%%%%%%%%%%%%%%%%%%%%%%%%%%%%%%%%%%%%%%%%%%%%%%%%%%%
Analysis of Variance:
\begin{center}
\begin{tabular}{|c|c|c|c|c|c|}\hline
Source	&	DF	&	SS	&	MS	&	F &	 \\ \hline \hline
Treatments	&	2	&	22.167	&	11.084	&	8.436	& F(2; 6)\\ \hline
Locations	&	3	&	17.889	&	18.222	&	13.870	& F(3; 6) \\ \hline
Residuals	&	6	&	7.883	&	1.306	&		& \\ \hline
Total	&	11	&	83.667	&		&		& \\ \hline
\end{tabular}
\end{center}

%%%%%%%%%%%%%%%%%%%%%%%%%%%%%%%%%%%%%%%%%%%%%%%%%%%%%%%%%%
\begin{itemize}
    \item The locations differed significantly, and therefore it was useful to use the randomized block scheme
with locations as blocks. We assume there is no blocks $\times$ treatments interaction.
\item  For treatments, means are: A 11:75
C 12:00
B 14:75
. 
\item We are not told which comparisons(contrasts) among
treatments are important, but it is clear that the significance of $F(2,6)$ must be due to the difference
between B and the other two.
\end{itemize}


\end{enumerate}
\end{document}
