\documentclass{article}
\usepackage[utf8]{inputenc}
\usepackage{enumerate}
\author{kobriendublin }
\date{December 2018}

\begin{document}

%- Higher Certificate, Module 5, 2009. Question 2
\section{Introduction}
\begin{enumerate}[(i)]
%%%%%%%%%%%%%%%%%%%%%%%%%%%%%%%%%%%%%%%%%%%%%%
\item


(i) ()()1122()212122kkdmtktktdt−−−⎛⎞=−×−−=−⎜⎟⎝⎠
()0()tdmtEXkdt=∴==.
()()()222222()21122122kkdmtkktkkdt −−−⎛⎞=−×−−−=+−⎜⎟⎝⎠
()()2220()2tdmtEXkkdt=∴== .
()()()()222Var22XEXEXkkkk∴=−=+−=.

%%%%%%%%%%%%%%%%%%%%%%%%%%%%%%%%%%%%%%%%%%%%%%
\item ()()120014xttxmtefxdxxedx⎛⎞−−∞∞⎜⎟⎝⎠==∫∫
112200111422xtxteexdxtt∞⎛⎞⎛⎞−−−−⎜⎟⎜⎟⎝⎠⎝⎠∞⎧⎫⎡⎤⎪⎪⎢⎥⎪⎪⎢⎥=−⎨⎬⎛⎞⎛⎞⎢⎥⎪⎪−−−−⎜⎟⎜⎟⎢⎥⎪⎪⎝⎠⎝⎠⎣⎦⎩⎭∫
()12222011112441122xtettt∞⎛⎞−−⎜⎟⎝⎠−⎡⎤⎢⎥⎢⎥=−=×=−⎢⎥⎛⎞⎛⎞−−⎢⎥⎜⎟⎜⎟⎝⎠⎝⎠⎣⎦, as required.

%%%%%%%%%%%%%%%%%%%%%%%%%%%%%%%%%%%%%%%%%%%%%%
\item ()()()1122121,2,...,;iYmttint−=−=< .
By the convolution theorem,
()()()2112,innVYimtmtt−===−Π and this is the mgf of 2χ.n

Therefore, by the 1:1 correspondence between mgfs and distributions, 2χ.nV∼

%%%%%%%%%%%%%%%%%%%%%%%%%%%%%%%%%%%%%%%%%%%%%%
\item For n = 300, we have 2300χ.V∼
By part (i), E(V) = 300 and Var(V) = 2 × 300 = 600.

By the central limit theorem, since V is the sum of a large number of random variables (independent identically distributed, finite variance), V has a Normal distribution, V ~ N(300, 600), approximately.
()()3103003100.40820.6584600PV−⎛⎞∴≤≈Φ=Φ=⎜⎟⎝⎠.
\end{enumerate}
\end{document}