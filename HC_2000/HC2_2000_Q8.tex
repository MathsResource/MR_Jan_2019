


%%%%%%%%%%%%%%%%%%%%%%%%%%%%%%%%%%%%%%%%%%%%%%%%%%%%%%%%%%%%%%%%%%%%%%%%%%%%%%%%%%%%%%%%%
\begin{table}[ht!]
     

\centering
     

\begin{tabular}{|p{15cm}|}
     

\hline 


\begin{framed}
\begin{verbatim}
Gross value added at 1995 basic prices: by industry  Index numbers  
Indices 1995 = 100  Weight per 1000 in 1995 1989 1990 1991 1992 1993 1994 1995 1996 1997 Manufacturing           Food; beverages & tobacco 28.7 95.6 97.2 97.0 98.7 99.0 101.5 100.0 101.0 103.4 Textiles & textile products 10.1 112.9 111.8 101.0 101.9 101.3 103.6 100.0 99.6 96.2 Leather & leather products   1.4 116.2 112.1 97.1 94.5 97.8 97.3 100.0 99.9 103.7 Wood & wood products   2.9 115.2 111.6 99.4 98.2 100.4 108.2 100.0 97.6 95.1 Pulp, paper & paper products; publishing, printing 26.3 94.2 96.4 92.0 93.1 96.1 98.5 100.0 98.0 98.3 Coke, petroleum products & nuclear fuel   4.7 81.1 77.4 83.5 88.5 88.9 89.7 100.0 91.8 92.2 Chemicals, chemical products & man-made fibres 24.1 83.6 83.5 85.8 88.5 90.4 95.1 100.0 100.7 101.7 Rubber & plastic products 10.5 85.8 88.2 83.1 85.1 88.9 98.0 100.0 98.8 98.4 Other non-metallic mineral products   7.9 116.7 109.4 99.1 94.7 99.1 102.8 100.0 96.5 99.0 Basic metals & fabricated metal products 24.6 112.1 111.2 101.0 96.0 95.0 97.3 100.0 99.7 101.3 Machinery & equipment not elsewhere classified 19.2 110.1 110.6 98.8 94.8 94.7 99.9 100.0 98.0 95.8 Electrical & optical equipment 26.9 80.0 80.8 77.6 78.9 83.2 93.3 100.0 104.0 105.3 Transport equipment 20.5 111.2 108.8 101.8 99.8 98.1 100.7 100.0 105.7 110.4 Manufacturing not elsewhere classified   7.6 111.1 112.5 98.5 98.0 99.4 102.4 100.0 100.2 101.6            Total manufacturing 215.7 97.9 97.7 92.8 92.8 94.1 98.5 100.0 100.4 101.4 
 
\end{verbatim}
\end{framed}
 
The table above is reproduced from Table 2.4 of United Kingdom National Accounts 1998.  
It shows indices of gross value added in the 14 sections of the manufacturing sector in United Kingdom industries from 1989 to 1997.  In 1995, manufacturing accounted for 21.57% of total gross value added in United Kingdom industries. 

 
Write an article for a serious financial or economic weekly newspaper based on this table.  Your article should incorporate such diagrams and such statistics calculated from the table as you think appropriate. (20) 
 
\\ \hline


\end{tabular}
    

\end{table}

%%%%%%%%%%%%%%%%%%%%%%%%%%%%%%%%%%%%%%%%%%%%%%%%%%%%%%%%%%%%%%%%%%%%%%%%%%%%%%%%%%%%%%%%%

\newpage

This
would influence index movements.
One approach is to compare trends, and illustrate these on a graph with time on the
horizontal axis. The extent to which individuals reflected the general trend (bottom
row) should be commented on. with 14 individual rows, there is a lot of information,
and perhaps the main components as given by 1995 weight are enough to illustrate.
An alternative is to recalculate indices relative to the overall index for each year, e.g.
Food 1989:95.6/97.9=97.7(see table below).
1989 1990 1991 1992 1993 1994 1995 1996 1997
Food:etc 97:7 99:5 104:5 106:4 105:2 103.0 100.0 100:6 102.0
Pulp; paper; etc 96:2 98:7 99:1 100:3 102:1 100.0 100.0 97:6 96:9
chemical 85:4 85:5 92:5 95:4 96:1 96:5 100.0 100:3 100:3
Basic Metals; etc 114:5 113:8 108:8 103:4 101.0 98:8 100.0 99:3 99:9
Electrical 81:7 82:7 83:6 85.0 88:4 94:7 100.0 103:6 103:8
Transport 113:7 111:4 109:7 107:5 104:3 102:2 100.0 105:3 108:9
Thus relative to the picture for total manufacturing, both chemicals and electrical have
steadily increased their value added over time,metals have steadily reduced, food rose
until 1993 and has since dropped back, pulp and paper rose and fell again while transport
fell and then rose again.
Transport was never below the level for total.
although most other categories formed small parts of the total (by 1995 weight things),
some points stand out; coke etc. is nearly always well below. Total except for 1995;
textiles show a steady fall,rubber and plastic a steady rise.
Higher value added will often reflect greater efficiency, but will also be affected by the
15
factors mentioned above.
