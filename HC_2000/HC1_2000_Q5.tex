\documentclass[a4paper,12pt]{article}
%%%%%%%%%%%%%%%%%%%%%%%%%%%%%%%%%%%%%%%%%%%%%%%%%%%%%%%%%%%%%%%%%%%%%%%%%%%%%%%%%%%%%%%%%%%%%%%%%%%%%%%%%%%%%%%%%%%%%%%%%%%%%%%%%%%%%%%%%%%%%%%%%%%%%%%%%%%%%%%%%%%%%%%%%%%%%%%%%%%%%%%%%%%%%%%%%%%%%%%%%%%%%%%%%%%%%%%%%%%%%%%%%%%%%%%%%%%%%%%%%%%%%%%%%%%%
\usepackage{eurosym}
\usepackage{vmargin}
\usepackage{amsmath}
\usepackage{graphics}
\usepackage{epsfig}
\usepackage{enumerate}
\usepackage{multicol}
\usepackage{subfigure}
\usepackage{fancyhdr}
\usepackage{listings}
\usepackage{framed}
\usepackage{graphicx}
\usepackage{amsmath}
\usepackage{chngpage}
%\usepackage{bigints}

\usepackage{vmargin}
% left top textwidth textheight headheight
% headsep footheight footS_{k}ip
\setmargins{2.0cm}{2.5cm}{16 cm}{22cm}{0.5cm}{0cm}{1cm}{1cm}
\renewcommand{\baselinestretch}{1.3}

\setcounter{MaxMatrixCols}{10}

\begin{document}

\begin{table}[ht!]
     \centering
     \begin{tabular}{|p{15cm}|}
     \hline        \large
5. It is desired to carry out a blood test on a large number (N) of persons, to check for the presence or absence of a rare characteristic. 

\begin{itemize}
    \item It is assumed that the probability p of a positive result is the same for all persons and is independent of the results of tests on other persons. 
 
\item To reduce the work of testing, the blood samples of N persons are pooled into m groups of size $k$, where $N = mk$.  The $k$ samples in each group are first tested together.  

\item If the group test is negative, no further test is necessary for the persons in that group.  If the group test is positive each person is tested individually and so in all ($k + 1$) tests are required for the group of $k$ persons. 
\end{itemize} 
 
(i) Find the probability that the test for any particular pooled sample of $k$ persons is positive. 
 
\\ \hline
      \end{tabular}
    \end{table}
    
  \begin{table}[ht!]
     \centering
     \begin{tabular}{|p{15cm}|}
     \hline   Let $kS$ denote the total number of tests required for the N people when initially tested in groups of $k$.  Explain why $S_k$ can be written in terms of a binomial random variable $X$, as 
 
\[S_k = m + kX, \mbox{ where } X \sim \mbox{Bin}\left(m, 1-(1-p)^k\right)\]
\\ \hline
      \end{tabular}
    \end{table}
    
  \begin{table}[ht!]
     \centering
     \begin{tabular}{|p{15cm}|}
     \hline \large  Hence show that 

\[ E(S_k) =  N \left( \frac{1}{k} + 1-(1-p)^k\right)  \] , 
 
\[ \operatorname{Var}(S_k) =  Nk (1-p)^k \left( \frac{1}{k} + 1-(1-p)^k\right)  \]
\\ \hline
      \end{tabular}
    \end{table}
    

\item 

\begin{eqnarray*}
P(\mbox{group test: Positive}) &=& 1 - P(\mbox{grouptest : negative})&=&
&=& 1 - P(\mbox{ all }k \mbox{ individuals test negative})&=&
&=& 1 - (1 - P)^k\\
\end{eqnarray*}

\item The N persons form m independent groups ,each group having a group test: also
if the group test is positive there are k individual tests, so this happens with probability
$(1 - (1 - p)^k)$
Hence the number of test is $S_{k} = m + k \times x$,where $x$ is the number out of the m
groups where individual tests have to be made; so x is $Binomial(m; 1 - (1 - p)^k)$.

%%%%%%%%%%%%%%%%%%%%%%%%%%%%%%
\item The mean and variance of Binomial(n; ¼) are n¼, n¼(1 - ¼). hence
$E[x] = m(1 - (1 - p)^{k})$; 
$V [x] = m(1 - p)^{k}(1 - (1 - p)^{k})
$
since N=mk

\begin{eqnarray*}
E[S_{k}] &=& m + kE[x] \\&=& N
k + N(1 - (1 - p)^{k}) 
\\&=& N[ \frac{1}{
k} + 1 - (1 - p)^{k}]
\end{eqnarray*}

\begin{eqnarray*}
V [S_{k}] &=& k^2V [x] \\&=& Nk(1 - p)^{k}[1 - (1 - p)^{k}]
\end{eqnarray*}

  \begin{table}[ht!]
     \centering
     \begin{tabular}{|p{15cm}|}
     \hline   Assuming that p = 0.01, show by calculation that $E(S_{10}) > E(S_{11})$ and $E(S_{11}) > E(S_{12})$.
\\ \hline
      \end{tabular}
    \end{table}
    

%%%%%%%%%%%%%%%%%%%%%%
\item When P=0.01
\begin{eqnarray*}
E(S_{10}) - E(S_{11}) &=& N \left( \frac{1}{10} -  \frac{1}{11} - (0.99^{10}) + (0.99^{11}) \right)
\\&=& N\left( \frac{1}{10} - \frac{1}{11} - 0.01 \times (0.99^{10})\right) \\&=& 4.7089 \times 10^{-5}N
\end{eqnarray*}
\begin{eqnarray*}
E(S_{12}) - E(S_{11}) &=& N \left( \frac{1}{12} - \frac{1}{11} - (0.99^{12}) + (0.99^{11}) \right)\\&=& N \left( \frac{1}{12} - \frac{1}{11} + 0.01 \times 0.9911 \right) \\&=&
 1.3776 \times 10^{-3} N\\
\end{eqnarray*}
Both of these are positive, so $E(S_{11})$ is less than $E(S_{10})$ and $E(S_{12})$.
ALTERNATIVELY by directly calculation as below.
6
\item When p=0.05,
\begin{itemize}
\item $E[S_{4}] = N[\frac{5}{4} - 0.95^4] = 0.435494$N
\item $E[S_{5}] = N[\frac{6}{5} - 0.95^5] = 0.426219$N
\item $E[S_{6}] = N[\frac{7}{6} - 0.95^6] = 0.431575$N
\item $E[S_{7}] = N[\frac{8}{7} - 0.95^7] = 0.444520$N
\end{itemize}

k=5 minimizes $E[S_{k}]$ in this range.
%%%%%%%%%%%%%%%%%%%%%%%%
\newpage
  \begin{table}[ht!]
     \centering
     \begin{tabular}{|p{15cm}|}
     \hline  \large Find the value of $k$ for which $E(S_{k})$ is minimised when p = 0.05. [Note: the optimum value of $k$ lies between 4 and 7 inclusive.] 
\\ \hline
      \end{tabular}
    \end{table}
    
  \begin{table}[ht!]
     \centering
     \begin{tabular}{|p{15cm}|}
     \hline   Compare your results for parts (iv) and (v) with those based on k = 1 (i.e. no pooling), and comment briefly. 
\\ \hline
      \end{tabular}
    \end{table}
    
%%%%%%%%%%%%%%%%%%%%%%%%%%%%%%%%%%%%%%%%%%%%%%%%%%%%%
    
\item When k=1
\[E[s] = N(2 - (1 - p)) = 1:01N for P = 0:01\]
= 1:05N for P = 0:05
\begin{itemize}
    \item this suggests that when P is very small the total number of tests $S_{k}$ can be very substantially
reduced. 
\item Even for $P=0.05$ it is (more than) halved. 
\item Also the group size may
be larger the smaller P is: the above results suggest optima of k=11 or 5 for $p=0.01$ or
$0.05$. 
\item As P increase there is less scope for economy of testing.
\end{itemize}


\end{enumerate}
\end{document}
