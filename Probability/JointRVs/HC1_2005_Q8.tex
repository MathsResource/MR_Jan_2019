\documentclass[a4paper,12pt]{article}

%%%%%%%%%%%%%%%%%%%%%%%%%%%%%%%%%%%%%%%%%%%%%%%%%%%%%%%%%%%%%%%%%%%%%%%%%%%%%%%%%%%%%%%%%%%%%%%%%%%%%%%%%%%%%%%%%%%%%%%%%%%%%%%%%%%%%%%%%%%%%%%%%%%%%%%%%%%%%%%%%%%%%%%%%%%%%%%%%%%%%%%%%%%%%%%%%%%%%%%%%%%%%%%%%%%%%%%%%%%%%%%%%%%%%%%%%%%%%%%%%%%%%%%%%%%%

\usepackage{eurosym}
\usepackage{vmargin}
\usepackage{amsmath}
\usepackage{graphics}
\usepackage{epsfig}
\usepackage{enumerate}
\usepackage{multicol}
\usepackage{subfigure}
\usepackage{fancyhdr}
\usepackage{listings}
\usepackage{framed}
\usepackage{graphicx}
\usepackage{amsmath}
\usepackage{chngpage}

%\usepackage{bigints}
\usepackage{vmargin}

% left top textwidth textheight headheight

% headsep footheight footskip

\setmargins{2.0cm}{2.5cm}{16 cm}{22cm}{0.5cm}{0cm}{1cm}{1cm}

\renewcommand{\baselinestretch}{1.3}

\setcounter{MaxMatrixCols}{10}

\begin{document}
	
	%% Higher Certificate, Paper I, 2005. Question 8
	
	%%%%%%%%%%%%%%%%%%%%%%%%%%%%%%%%%%%%%%%%%%%%%%%%%%%%%%%%%%%%%%%%%%
	\large 
	\noindent The joint probability mass function of $X$ and $Y$ is tabulated below.
	\begin{center}
		\begin{tabular}{|c|c|c|c|}\hline 
			& Y= 0    & Y= 1    & Y= 2        \\ \hline \hline 
			X= -1 & 1/6  & 1/12 & 1/12  \\ \hline 
			X=  0 & 1/12 & 1/6  & 1/12  \\ \hline 
			X=  1 & 1/12 & 1/12 & 1/6  \\ \hline 
			
		\end{tabular}
	\end{center}
	
	\begin{framed}
		\noindent \textbf{Part (a)}\\ Obtain the marginal distributions of $X$ and $Y$, and hence calculate $E(X)$, $E(Y)$,
		$\operatorname{Var}(X)$ and $\operatorname{Var}(Y)$.
	\end{framed}
	\noindent \textbf{(a)}\\
	\large
	The marginal distributions of $X$ and $Y$ are as shown, appended to the table.
	\begin{center}
		\begin{tabular}{|c|c|c|c||c|}\hline 
			& Y= 0    & Y= 1    & Y= 2    &    \\ \hline \hline 
			X= -1 & 1/6  & 1/12 & 1/12 & 1/3 \\ \hline 
			X=  0 & 1/12 & 1/6  & 1/12 & 1/3 \\ \hline 
			X=  1 & 1/12 & 1/12 & 1/6  & 1/3 \\ \hline \hline 
			& 1/3 & 1/3 & 1/3 & \\\hline 
		\end{tabular}
	\end{center}
	\medskip
	%Marginal distribution of Y 
	Hence $E(X) = 0$ and $E(Y) = 1$ (by symmetry; by noting that $X$ and $Y$ are both discrete
	uniform; or by explicit calculation).
	\begin{itemize}
		\item $\operatorname{Var}(X)$ can be calculated as:
		\[\operatorname{Var}(X) = \sum(x - 0)^2P(X = x) = \frac{{(-1-0)^2}}{3} + \frac{{(0-0)^2}}{3} + \frac{{(1-0)^2}}{3} = 2/3.\]
		
		\item $\operatorname{Var}(Y)$ can be calculated similarly:
		\[\operatorname{Var}(Y) = \sum(y - 1)^2P(X = x) = \frac{{(0-1)^2}}{3} + \frac{{(1-1)^2}}{3} + \frac{{(2-1)^2}}{3} = 2/3.\]
	\end{itemize}
	
	Equivalently, since $Y = X + 1$, we have $\operatorname{Var}(Y) = \operatorname{Var}(X)$.
	%%%%%%%%%%%%%%%%%%%%%%%%%%%%%%%%%%%%%%%%%%%%%%%%%%%%%%%%%%%
	%----------------------------------------------------------------%
	\newpage
	\large
	\begin{framed}
		\noindent \textbf{Part (b)}\\ Obtain the conditional distribution of Y for each possible value of $X$, and hence
		show that $E(Y | X = x)$ is a linear function of x.
	\end{framed}
	%----------------------------------------------------------------%
	
	\noindent \textbf{(b)}\\
	\large
	The conditional distributions of $Y$ for each value of $X$ are as follows.
	\begin{center}
		\begin{tabular}{|c|c|c|c|}\hline 
			& Y= 0    & Y= 1    & Y= 2        \\ \hline \hline 
			X= -1 & 1/2 &  1/4 & 1/4 \\ \hline 
			X = 0 & 1/4 & 1/2 & 1/4 \\ \hline 
			X = 1 & 1/4 & 1/4 & 1/2 \\ \hline 
		\end{tabular}
	\end{center}
	\begin{itemize}
		\item 
		Hence \begin{eqnarray*}E[Y | X = -1] &=& (0)(0.5) + (1)(0.25) + (2)(0.25)\\ &=& 0.75\\
		\end{eqnarray*}
		\item 
		Similarly, 
		\begin{eqnarray*}E[Y | X = 0] &=& (0)(0.25) + (1)(0.50) + (2)(0.25)\\ &=& 1.00\\
		\end{eqnarray*}
		\begin{eqnarray*}E[Y | X = 1] &=& (0)(0.25) + (1)(0.25) + (2)(0.50)\\ &=& 1.25\\
		\end{eqnarray*}
		%$E[Y | X = 0] = 1$ and $E[Y | X = 1] = 5/4$.
		\item Thus we have $E[Y | X = x] = 1 + \frac{x}{4} $ .
	\end{itemize}
	
	
	
	%%%%%%%%%%%%%%%%%%%%%%%%%%%%%%%%%%%%%%%%%%%%%%%%%%%%%%%%%%%
	\newpage
	\large
	\begin{framed}
		\noindent \textbf{Part (c)}\\ Find $E(XY)$ and deduce $\operatorname{Cov}(X, Y)$ and $\operatorname{Corr}(X, Y)$. Are $X$ and $Y$ independent?
	\end{framed}
	\noindent \textbf{(c)}\\
	\large
	\begin{center}
		\begin{tabular}{|c|c|c|c|}\hline 
			& Y= 0    & Y= 1    & Y= 2        \\ \hline \hline 
			X= -1 & 1/6  & 1/12 & 1/12  \\ \hline 
			X=  0 & 1/12 & 1/6  & 1/12  \\ \hline 
			X=  1 & 1/12 & 1/12 & 1/6  \\ \hline 
			
		\end{tabular}
	\end{center}
	\begin{eqnarray*} 
		E(XY) &=& \left[(-1)(0)(1/6)\right] \; + \; \left[(-1)(1)(1/12)\right] \; + \; \left[(-1)(2)(1/12)\right] + 
		\\ & & \left[(0)(0)(1/12)\right] \; + \; \left[(0)(1)(1/6)\right] \; + \;  \left[(0)(2)(1/12)\right] + 
		\\ & &\left[(1)(0)(1/12)\right] \; + \; \left[(1)(1)(1/12)\right] \; + \; \left[(1)(2)(1/6)\right]
		\\ & & \\
		&=& \left[\; 0\;\right] \; + \; \left[\;-1/12\;\right] \; + \; \left[\;-2/12\;\right] + 
		\\ & &  \left[\; 0\;\right] \; + \; \left[\; 0\;\right] \; + \;  \left[\; 0\;\right] + 
		\\ & &\left[\; 0\;\right] \; + \; \left[\;1/12\;\right] \; + \; \left[\;2/6\;\right]
		\\ &=&  1/6.\\
	\end{eqnarray*}
	\begin{eqnarray*}
		\operatorname{Cov}(X, Y) &=& E(XY) - E(X)E(Y) \\ 
		&=& \frac{1}{6} - (0)(1) \\ &=& 1/6\\
	\end{eqnarray*}
	
	\begin{eqnarray*}\operatorname{Corr}(X, Y) &=& \frac{\operatorname{Cov}(X, Y)}{\sqrt{ \operatorname{Var}(X) \operatorname{Var}(Y)}}
		\\ &=& \frac{1/6}{2/3} \\ &=& \frac{2/12}{8/12} 
		\\ &=& \frac{1}{4}
	\end{eqnarray*}
	$X$ and $Y$ are not independent: their correlation (or covariance) is non-zero.\\ \medskip (In fact we
	saw in part (b) that they are linearly related: $Y = X + 1$.)
	
	%%%%%%%%%%%%%%%%%%%%%%%%%%%%%%%%%%%%%%%%%%%%%%%%%%%%%%%%%%%
	\newpage
	\begin{framed}
		\noindent \textbf{Part (d)}\\ Find the probability distribution of $Z = X^3 + (Y - 1)^3$.
	\end{framed}
	\noindent \textbf{(d)}\\
	\large $Z = X^3 + (Y - 1)^3$ . The values of $Z$ and their probabilities are as shown:
	\begin{center}
		\begin{tabular}{|c||c|c|c|} \hline
			& Y = 0 & Y = 1 & Y = 2 \\ \hline \hline
			
			X = -1  &  Z = -2; p = 1/6  & Z = -1; p = 1/12 & Z = 0; p = 1/12  \\ \hline
			X = 0  &  Z = -1; p = 1/12 & Z = 0; p = 1/6   & Z =1; p = 1/12
			\\ \hline
			X = 1  & Z = 0; p = 1/12 & Z =1; p = 1/12 & Z = 2; p = 1/6 \\ \hline
		\end{tabular}
	\end{center}
	Thus we have
	
	\begin{center}
		\begin{tabular}{|c|c|c|c|c|c|} \hline
			Values of Z & -2&  -1&  0 & 1 & 2\\  \hline
			Probabilities &  1/6&  1/6&  1/3&  1/6&  1/6\\ \hline
		\end{tabular}
	\end{center}
	%%%%%%%%%%%%%%%%%%%%%%%%%%%%%%%%%%%%%%%
	
	
\end{document}
