\documentclass[a4paper,12pt]{article}
%%%%%%%%%%%%%%%%%%%%%%%%%%%%%%%%%%%%%%%%%%%%%%%%%%%%%%%%%%%%%%%%%%%%%%%%%%%%%%%%%%%%%%%%%%%%%%%%%%%%%%%%%%%%%%%%%%%%%%%%%%%%%%%%%%%%%%%%%%%%%%%%%%%%%%%%%%%%%%%%%%%%%%%%%%%%%%%%%%%%%%%%%%%%%%%%%%%%%%%%%%%%%%%%%%%%%%%%%%%%%%%%%%%%%%%%%%%%%%%%%%%%%%%%%%%%
  \usepackage{eurosym}
\usepackage{vmargin}
\usepackage{amsmath}
\usepackage{graphics}
\usepackage{epsfig}
\usepackage{enumerate}
\usepackage{multicol}
\usepackage{subfigure}
\usepackage{fancyhdr}
\usepackage{listings}
\usepackage{framed}
\usepackage{graphicx}
\usepackage{amsmath}
\usepackage{chngpage}
%\usepackage{bigints}

\usepackage{vmargin}
% left top textwidth textheight headheight
% headsep footheight footskip
\setmargins{2.0cm}{2.5cm}{16 cm}{22cm}{0.5cm}{0cm}{1cm}{1cm}
\renewcommand{\baselinestretch}{1.3}

\setcounter{MaxMatrixCols}{10}
\begin{document}
%%%%%%%%%%%%%%%%%%%%%%%%%%%%%%%%%%%%%%%%%%%%%%%%%%%%%%%%%%%%%%%%%%%%%%%%%%%%%%%%%%%%%%%%%%%%%%%%%%


Higher Certificate, Paper II, 2001. Question 5
%%%%%%%%%%%%%%%%%%%%%%%%%%%%%%%%%%%%%%%%%%%%%%%%%%%%%%%%%%%%%%%%%%%%%%%%%%%%%%%%%%%%%%%%%%%%%%%%%%%%%%%%%%%%%%%%%%%%%%%%%%%%%%%%%%%%%%%%%%% 
\begin{table}[ht!]
 
\centering
 
\begin{tabular}{|p{15cm}|}
 
\hline  

5. (a) A food tasting experiment was conducted in which a panel of judges were asked to taste two new mixtures, A and B, for a fruit flavoured soft drink and indicate which they preferred.  The results were as follows. 
 
Judge 1 2 3 4 5 6 7 8 9 10 11 12 Preference A B B A B B B A B B B A 
 
Carry out a suitable analysis of these data to investigate which mixture would be more popular and comment on your results. (8)
\\ \hline
  
\end{tabular}

\end{table} 


\begin{table}[ht!]
 
\centering
 
\begin{tabular}{|p{15cm}|}
 
\hline  

 (b) In a psychological experiment to investigate the effects of stress on the ability to perform simple tasks, 90 volunteers were asked to perform a simple puzzle assembly task under normal conditions and under conditions of stress.  Each subject was given three minutes to complete the task and on each occasion it was recorded whether or not they were successful.  The order of the conditions under which each subject performed the task was determined at random.  The results of the experiment are given in the following table. 
 
Normal conditions  Successful Unsuccessful Successful 52 9 Under stress Unsuccessful 20 9 
 
  (i) Apply McNemar's test to the above results. 
(8) 
 
  (ii) A conventional 2×2 chi-squared test of the above results, without using Yates' correction, gives a test statistic of 3.26.  How does any difference in the outcome of the two tests arise? (4) 


\\ \hline
  
\end{tabular}

\end{table} 

\begin{enumerate}[(a)]
\item  A suitable null hypothesis is that each judge is equally likely to choose either;
hence N, the number preferred, is binomial with parameters 12 and p = ½. We have
nA = 4 and nB = 8.
( ) ( ) { }
4 12
12
0
4 in 12, 1 12 1 1 1 12 66 220 495
2 r ! 12 ! 2 2
P N B
= r r
≤   =   = + + + +   −       \sigma
794 0.194
4096
= = .
This result is not "unlikely", and the null hypothesis cannot be rejected. We have no
conclusive evidence to say which may be more popular.
(b) (i) McNemar's test deals with paired data, as these are.
H0: there is no association between stress and success,
H1: there is such an association.
The test uses the two terms in the right-to-left diagonal, giving test statistic
( )2 9 20
4.172
9 20
−
=
+
Comparing with 2
1 χ , this is significant at the 5% level. So there is evidence
against the null hypothesis, in favour of the existence of some association.
(ii) The standard 2×2 test has given a non-significant result ( 2 )
1 χ < 3.84 . It
does not use the matched nature of the data, so it has less power than
McNemar's test to look for association. In this example, McNemar's result is
based on the proportions of successful individuals in the two samples "stress"
and "no stress".

\end{enumerate}
\end{document}
