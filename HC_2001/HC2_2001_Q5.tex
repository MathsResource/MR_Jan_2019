\documentclass[a4paper,12pt]{article}
%%%%%%%%%%%%%%%%%%%%%%%%%%%%%%%%%%%%%%%%%%%%%%%%%%%%%%%%%%%%%%%%%%%%%%%%%%%%%%%%%%%%%%%%%%%%%%%%%%%%%%%%%%%%%%%%%%%%%%%%%%%%%%%%%%%%%%%%%%%%%%%%%%%%%%%%%%%%%%%%%%%%%%%%%%%%%%%%%%%%%%%%%%%%%%%%%%%%%%%%%%%%%%%%%%%%%%%%%%%%%%%%%%%%%%%%%%%%%%%%%%%%%%%%%%%%
  \usepackage{eurosym}
\usepackage{vmargin}
\usepackage{amsmath}
\usepackage{graphics}
\usepackage{epsfig}
\usepackage{enumerate}
\usepackage{multicol}
\usepackage{subfigure}
\usepackage{fancyhdr}
\usepackage{listings}
\usepackage{framed}
\usepackage{graphicx}
\usepackage{amsmath}
\usepackage{chngpage}
%\usepackage{bigints}

\usepackage{vmargin}
% left top textwidth textheight headheight
% headsep footheight footskip
\setmargins{2.0cm}{2.5cm}{16 cm}{22cm}{0.5cm}{0cm}{1cm}{1cm}
\renewcommand{\baselinestretch}{1.3}

\setcounter{MaxMatrixCols}{10}
\begin{document}
%%%%%%%%%%%%%%%%%%%%%%%%%%%%%%%%%%%%%%%%%%%%%%%%%%%%%%%%%%%%%%%%%%%%%%%%%%%%%%%%%%%%%%%%%%%%%%%%%%


Higher Certificate, Paper II, 2001. Question 5
\begin{enumerate}[(a)]
\item  A suitable null hypothesis is that each judge is equally likely to choose either;
hence N, the number preferred, is binomial with parameters 12 and p = ½. We have
nA = 4 and nB = 8.
( ) ( ) { }
4 12
12
0
4 in 12, 1 12 1 1 1 12 66 220 495
2 r ! 12 ! 2 2
P N B
= r r
≤   =   = + + + +   −       \sigma
794 0.194
4096
= = .
This result is not "unlikely", and the null hypothesis cannot be rejected. We have no
conclusive evidence to say which may be more popular.
(b) (i) McNemar's test deals with paired data, as these are.
H0: there is no association between stress and success,
H1: there is such an association.
The test uses the two terms in the right-to-left diagonal, giving test statistic
( )2 9 20
4.172
9 20
−
=
+
Comparing with 2
1 χ , this is significant at the 5% level. So there is evidence
against the null hypothesis, in favour of the existence of some association.
(ii) The standard 2×2 test has given a non-significant result ( 2 )
1 χ < 3.84 . It
does not use the matched nature of the data, so it has less power than
McNemar's test to look for association. In this example, McNemar's result is
based on the proportions of successful individuals in the two samples "stress"
and "no stress".

\end{enumerate}
\end{document}
