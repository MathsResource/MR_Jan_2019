\documentclass[a4paper,12pt]{article}
%%%%%%%%%%%%%%%%%%%%%%%%%%%%%%%%%%%%%%%%%%%%%%%%%%%%%%%%%%%%%%%%%%%%%%%%%%%%%%%%%%%%%%%%%%%%%%%%%%%%%%%%%%%%%%%%%%%%%%%%%%%%%%%%%%%%%%%%%%%%%%%%%%%%%%%%%%%%%%%%%%%%%%%%%%%%%%%%%%%%%%%%%%%%%%%%%%%%%%%%%%%%%%%%%%%%%%%%%%%%%%%%%%%%%%%%%%%%%%%%%%%%%%%%%%%%
\usepackage{eurosym}
\usepackage{vmargin}
\usepackage{amsmath}
\usepackage{graphics}
\usepackage{epsfig}
\usepackage{enumerate}
\usepackage{multicol}
\usepackage{subfigure}
\usepackage{fancyhdr}
\usepackage{listings}
\usepackage{framed}
\usepackage{graphicx}
\usepackage{amsmath}
\usepackage{chngpage}
%\usepackage{bigints}

\usepackage{vmargin}
% left top textwidth textheight headheight
% headsep footheight footskip
\setmargins{2.0cm}{2.5cm}{16 cm}{22cm}{0.5cm}{0cm}{1cm}{1cm}
\renewcommand{\baselinestretch}{1.3}

\setcounter{MaxMatrixCols}{10}
\begin{document}
Higher Certificate, Paper I, 2001. Question 2
(a) Fix the position of M1 (suppose him to be the host).
M1 Label the positions clockwise.
(1)
(6) (2) M2, M3, W1, W2, W3 may be arranged in 5! =
120 ways.
(5) (3)
(4)
(i) M2, M3 must occupy (3) and (5); W1, W2, W3 may occupy the other
places in 3! ways, making 2×3! arrangements. The probability is then
12 1 .
120 10
=
(ii) M2, M3 must occupy (2) and (6) or (2) and (3) or (5) and (6). In each
case, M2, M3 can be placed in two orders, making 6 positions altogether for
the three men. The women may again fill the remaining places in 3! ways.
The probability is 6 6 3
120 10
× = .
(iii) EITHER 1 1 3 3
10 10 5
− − = , because this is the only other arrangement
possible besides (i) and (ii);
OR by having M2 in (2), M3 in (4) or (5); M2 in (6), M3 in (3) or (4); M2 in
(3), M3 in (4); M2 in (4), M3 in (5); or any of these with M2, M3 interchanged,
giving 12 positionings of the men. There are again 3! orders for the women,
so the probability is 6 12 3
120 5
× = .
(b) (i) Event D is "has disease", T is "tests positive".
( | ) ( ) ( | ) ( ) ( | ) ( )
( ) ( ) ( | ) ( ) ( | ) ( )
P D T P D T P T D P D P T D P D
PT PT P T D P D P T D P D
= ∩ = =
+
1 0
1 0 2 0 (1 )(1 )
p p
pp p p
=
+ − −
.
(ii) 0.95 0.005 0.00475 0.0872
(0.95 0.005) (0.05 0.995) 0.0545
× = =
× + ×
.
The error rates in the clinical tests are large compared to the chance of having
the disease, so the calculated probability is very small.
