\documentclass[a4paper,12pt]{article}
%%%%%%%%%%%%%%%%%%%%%%%%%%%%%%%%%%%%%%%%%%%%%%%%%%%%%%%%%%%%%%%%%%%%%%%%%%%%%%%%%%%%%%%%%%%%%%%%%%%%%%%%%%%%%%%%%%%%%%%%%%%%%%%%%%%%%%%%%%%%%%%%%%%%%%%%%%%%%%%%%%%%%%%%%%%%%%%%%%%%%%%%%%%%%%%%%%%%%%%%%%%%%%%%%%%%%%%%%%%%%%%%%%%%%%%%%%%%%%%%%%%%%%%%%%%%
  \usepackage{eurosym}
\usepackage{vmargin}
\usepackage{amsmath}
\usepackage{graphics}
\usepackage{epsfig}
\usepackage{enumerate}
\usepackage{multicol}
\usepackage{subfigure}
\usepackage{fancyhdr}
\usepackage{listings}
\usepackage{framed}
\usepackage{graphicx}
\usepackage{amsmath}
\usepackage{chngpage}
%\usepackage{bigints}

\usepackage{vmargin}
% left top textwidth textheight headheight
% headsep footheight footskip
\setmargins{2.0cm}{2.5cm}{16 cm}{22cm}{0.5cm}{0cm}{1cm}{1cm}
\renewcommand{\baselinestretch}{1.3}

\setcounter{MaxMatrixCols}{10}
\begin{document}
%%%%%%%%%%%%%%%%%%%%%%%%%%%%%%%%%%%%%%%%%%%%%%%%%%%
Higher Certificate, Paper III, 2001. Question 5
%%%%%%%%%%%%%%%%%%%%%%%%%%%%%%%%%%%%%%%%%%%%%%%%%%%%%%%%%%%%%%%%%%%%%%%%%%%%%%%%%%%%%%%%%

\begin{table}[ht!]
     


\centering
     


\begin{tabular}{|p{15cm}|}
     


\hline 

5. A biologist is studying the proportion of foetuses found dead in rat litters that each contain exactly five rats.  
Data are available from 169 litters and a summary of the observations is as follows. 

\begin{center} 
\begin{tabular}{|ccccccc|}
Number of foetal & &&&&&&
deaths per litter&  0 & 1 & 2 & 3 & 4 & 5\\ 
Number of litters&  75&  44 & 36&  7&  4&  3 \\
\end{tabular}
\end{center} 
 
(i) Using n and p to denote the number of Bernoulli trials involved and the probability of a "success" respectively, write down expressions 
for the mean and variance of the number of successes assuming that it follows a binomial distribution.  
 
(ii) Taking the death of an individual foetus as a "success", obtain an estimate of p from the data, and hence calculate the mean 
and variance of a binomial distribution for the number of successes.  
(iii) Again taking the binomial distribution as a model for these data, calculate the expected numbers of litters having $0, 1, 2, \geq 3 $ foetuses found dead. 
 Draw a suitable diagram displaying the observed and expected numbers of litters having 0, 1, 2, ≥3 foetuses found dead, and test 
the goodness of the fit of the expected frequencies to the observed frequencies.  
 
(iv) What assumptions about the individual foetuses within a litter, and the probabilities of death across litters, are implied by
 the model used in (ii)?  Comment on how reasonable these assumptions are for these data.  The comments should involve the 
sample variance s2 = 1.3273 (N.B. this is not the variance of the binomial distribution).  
 
(v) How would the analysis be changed if "success" was taken to be an individual foetus living rather than dying? 
 [There is no need to carry out further calculations, but the required changes should be stated clearly, giving reasons where appropriate.]  
 
 
\\ \hline



\end{tabular}
    


\end{table}


%%%%%%%%%%%%%%%%%%%%%%%%%%%%%%%%%%%%%%%%%%%%%%%%%%%%%%%%%%%%%%%%%%%%%%%%%%%%%%%%%%%%%%%%%

\begin{enumerate}[(a)]
\item  Mean = np; variance = np(1 − p) .
(ii) Mean (0 75) (1 44) (2 36) (3 7) (4 4) (5 3) 168
169 169
× + × + × + × + × + ×
= =
= 0.9941 = npˆ .
Hence ˆ 0.9941 0.1988, ˆ (1 ˆ ) 0.7965
5
p = = np − p = .
(iii) ( ) ( ) () ( ) 5 4 p 0 = 1− p = 0.33014; p 1 = 5 p 1− p = 0.40959
( ) ( ) ( ) p 2 =10 p2 1− p 3 = 0.20326; p ≥ 3 = 0.05701.
Multiplying these by 169 gives the expected frequencies.
0
25
50
75
0 1 2 >=3
Number of deaths
Number of Litters
Observed
Expected
Frequencies 0 1 2 ≥3 TOTAL
Observed 75 44 36 14 169
Expected 55.79 69.22 34.35 9.63 168.99
( )2
2 6.615 9.189 0.079 1.983 17.87
OBS EXP
X
EXP
−
=\sigma = + + + = which is highly
significant as an observation from 22
χ (2 degrees of freedom since an estimated value
of p is used).
The null hypothesis of a binomial model is rejected.
(iv) The value of p is assumed the same for each individual in each litter, all
independently of one another. The calculated variance s2 = 1.3273 is larger than
npˆ (1− pˆ ) = 0.7965 . The data observed have more 0s than expected on this model,
and fewer 1s; also more of the higher number 3, 4, 5. There is evidence of
"overdispersion", the independence and constancy assumptions breaking down.
(v) Mean would be 4.006, new p = old(1 − p), variance the same, histograms the
mirror images of old ones. χ2 same, so inferences same.

\end{enumerate}
\end{document}
