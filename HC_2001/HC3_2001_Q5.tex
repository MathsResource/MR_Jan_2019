\documentclass[a4paper,12pt]{article}
%%%%%%%%%%%%%%%%%%%%%%%%%%%%%%%%%%%%%%%%%%%%%%%%%%%%%%%%%%%%%%%%%%%%%%%%%%%%%%%%%%%%%%%%%%%%%%%%%%%%%%%%%%%%%%%%%%%%%%%%%%%%%%%%%%%%%%%%%%%%%%%%%%%%%%%%%%%%%%%%%%%%%%%%%%%%%%%%%%%%%%%%%%%%%%%%%%%%%%%%%%%%%%%%%%%%%%%%%%%%%%%%%%%%%%%%%%%%%%%%%%%%%%%%%%%%
  \usepackage{eurosym}
\usepackage{vmargin}
\usepackage{amsmath}
\usepackage{graphics}
\usepackage{epsfig}
\usepackage{enumerate}
\usepackage{multicol}
\usepackage{subfigure}
\usepackage{fancyhdr}
\usepackage{listings}
\usepackage{framed}
\usepackage{graphicx}
\usepackage{amsmath}
\usepackage{chngpage}
%\usepackage{bigints}

\usepackage{vmargin}
% left top textwidth textheight headheight
% headsep footheight footskip
\setmargins{2.0cm}{2.5cm}{16 cm}{22cm}{0.5cm}{0cm}{1cm}{1cm}
\renewcommand{\baselinestretch}{1.3}

\setcounter{MaxMatrixCols}{10}
\begin{document}
%%%%%%%%%%%%%%%%%%%%%%%%%%%%%%%%%%%%%%%%%%%%%%%%%%%
Higher Certificate, Paper III, 2001. Question 5
\begin{enumerate}[(a)]
\item  Mean = np; variance = np(1 − p) .
(ii) Mean (0 75) (1 44) (2 36) (3 7) (4 4) (5 3) 168
169 169
× + × + × + × + × + ×
= =
= 0.9941 = npˆ .
Hence ˆ 0.9941 0.1988, ˆ (1 ˆ ) 0.7965
5
p = = np − p = .
(iii) ( ) ( ) () ( ) 5 4 p 0 = 1− p = 0.33014; p 1 = 5 p 1− p = 0.40959
( ) ( ) ( ) p 2 =10 p2 1− p 3 = 0.20326; p ≥ 3 = 0.05701.
Multiplying these by 169 gives the expected frequencies.
0
25
50
75
0 1 2 >=3
Number of deaths
Number of Litters
Observed
Expected
Frequencies 0 1 2 ≥3 TOTAL
Observed 75 44 36 14 169
Expected 55.79 69.22 34.35 9.63 168.99
( )2
2 6.615 9.189 0.079 1.983 17.87
OBS EXP
X
EXP
−
=\sigma = + + + = which is highly
significant as an observation from 22
χ (2 degrees of freedom since an estimated value
of p is used).
The null hypothesis of a binomial model is rejected.
(iv) The value of p is assumed the same for each individual in each litter, all
independently of one another. The calculated variance s2 = 1.3273 is larger than
npˆ (1− pˆ ) = 0.7965 . The data observed have more 0s than expected on this model,
and fewer 1s; also more of the higher number 3, 4, 5. There is evidence of
"overdispersion", the independence and constancy assumptions breaking down.
(v) Mean would be 4.006, new p = old(1 − p), variance the same, histograms the
mirror images of old ones. χ2 same, so inferences same.

\end{enumerate}
\end{document}