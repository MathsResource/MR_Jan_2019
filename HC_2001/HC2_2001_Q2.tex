\documentclass[a4paper,12pt]{article}
%%%%%%%%%%%%%%%%%%%%%%%%%%%%%%%%%%%%%%%%%%%%%%%%%%%%%%%%%%%%%%%%%%%%%%%%%%%%%%%%%%%%%%%%%%%%%%%%%%%%%%%%%%%%%%%%%%%%%%%%%%%%%%%%%%%%%%%%%%%%%%%%%%%%%%%%%%%%%%%%%%%%%%%%%%%%%%%%%%%%%%%%%%%%%%%%%%%%%%%%%%%%%%%%%%%%%%%%%%%%%%%%%%%%%%%%%%%%%%%%%%%%%%%%%%%%
  \usepackage{eurosym}
\usepackage{vmargin}
\usepackage{amsmath}
\usepackage{graphics}
\usepackage{epsfig}
\usepackage{enumerate}
\usepackage{multicol}
\usepackage{subfigure}
\usepackage{fancyhdr}
\usepackage{listings}
\usepackage{framed}
\usepackage{graphicx}
\usepackage{amsmath}
\usepackage{chngpage}
%\usepackage{bigints}

\usepackage{vmargin}
% left top textwidth textheight headheight
% headsep footheight footskip
\setmargins{2.0cm}{2.5cm}{16 cm}{22cm}{0.5cm}{0cm}{1cm}{1cm}
\renewcommand{\baselinestretch}{1.3}

\setcounter{MaxMatrixCols}{10}
\begin{document}
%%%%%%%%%%%%%%%%%%%%%%%%%%%%%%%%%%%%%%%%%%%%%%%%%%%
Higher Certificate, Paper II, 2001. Question 2
\begin{enumerate}[(a)]
\item  If all dice are thrown "fairly" (and all are "fair" in construction), and throws
are independent, then the conditions for a binomial distribution are satisfied; p = 1/6
since there are six equally possible results, and n = 5, the "sample size" (number
thrown) each time. The number of "success" (sixes) is counted (R).
\item ( )
5 1 5 5
6 6
r r
P R r
r
−      = =    
    
for r = 0, 1, 2, 3, 4, 5.
Expected frequencies are these probabilities × 200.
( ) ( ) ( )
5 4 2 3 0 5 0.40188, 1 5 1 5 0.40188, 2 10 1 5 0.16075
6 66 6 6
P =   = P =    = P =     =
        
( ) ( ) ( )
3 2 4 5 3 10 1 5 0.03215, 4 5 1 5 0.00322, 5 1 0.00013
6 6 6 6 6
P =     = P =     = P =   =
\begin{itemize}         
\item Compare these expected values with those observed, in the form of frequencies, using
a χ2 test. \item Group 3, 4, 5 together to prevent very small expected frequencies.
r 0 1 2 3 4 5 Total
Observed 90 78 26 4 1 1 200
combine these to give
6
Expected 80.38 80.38 32.15 7.10 (200.01)
( ) ( ) ( ) ( ) 2 2 2 2
2 90 80.38 78 80.38 26 32.15 6 7.10
2.569
80.38 80.38 32.15 7.10
X
− − − −
= + + + = which is
not significant as an observation from 2
3 χ (3 degrees of freedom since we are given
the value of p).
The null hypothesis is that R is binomial with n = 5, p = 1/6, and there is no evidence
that the observed frequencies depart seriously from those expected on this hypothesis.
We may assume the binomial model explains the data, and therefore the conditions
for a binomial do apply.
\end{enumerate}
\end{document}
