\documentclass[a4paper,12pt]{article}
%%%%%%%%%%%%%%%%%%%%%%%%%%%%%%%%%%%%%%%%%%%%%%%%%%%%%%%%%%%%%%%%%%%%%%%%%%%%%%%%%%%%%%%%%%%%%%%%%%%%%%%%%%%%%%%%%%%%%%%%%%%%%%%%%%%%%%%%%%%%%%%%%%%%%%%%%%%%%%%%%%%%%%%%%%%%%%%%%%%%%%%%%%%%%%%%%%%%%%%%%%%%%%%%%%%%%%%%%%%%%%%%%%%%%%%%%%%%%%%%%%%%%%%%%%%%
  \usepackage{eurosym}
\usepackage{vmargin}
\usepackage{amsmath}
\usepackage{graphics}
\usepackage{epsfig}
\usepackage{enumerate}
\usepackage{multicol}
\usepackage{subfigure}
\usepackage{fancyhdr}
\usepackage{listings}
\usepackage{framed}
\usepackage{graphicx}
\usepackage{amsmath}
\usepackage{chngpage}
%\usepackage{bigints}

\usepackage{vmargin}
% left top textwidth textheight headheight
% headsep footheight footskip
\setmargins{2.0cm}{2.5cm}{16 cm}{22cm}{0.5cm}{0cm}{1cm}{1cm}
\renewcommand{\baselinestretch}{1.3}

\setcounter{MaxMatrixCols}{10}
\begin{document}
%%%%%%%%%%%%%%%%%%%%%%%%%%%%%%%%%%%%%%%%%%%%%%%%%%%


Higher Certificate, Paper III, 2001. Question 6
\begin{enumerate}[(a)]
\item  ( ) ( ) 0 0 0 0
0 0 0 0
1 1 1 1 S t = P T ≥ t = − t \lambda e^{−\lambda} tdt = − −e^{−\lambda} t t = − −e^{−\lambda}t +  = e^{−\lambda} t ∫     .
ALTERNATIVELY, ( )
t0
f t dt ∞ ∫ may be calculated directly.
(ii)
12
1
12, i 6028
i
n t
=
= \sigma = .
( ) 12
12
1
ti
i
i
L ft e\lambda \lambda −
=
=Π = \sigma .
ln 12ln i L ≡ l = \lambda −\lambda\sigmat , so 12
i
dl t
d\lambda \lambda
= −\sigma
and this is 0 when ˆ 12 12 1.9907 10 3
6028 i t
\lambda = = = × − \sigma .
( ) 2
2 2 2
2
Var ˆ 1 and 12
E
d l
d l d
d
\lambda
\lambda \lambda
\lambda
≈ =−
 
−  
 
, so this is
2
7 12 3.302 10
\lambda − = × .
(iii) Measured in years, 6028 , so ˆ 0.7266
365 i \sigmat = \lambda = .
Setting ( ) 0.7266
0 t =1 in (i), S 1 = e− = 0.484.

\end{enumerate}
\end{document}
