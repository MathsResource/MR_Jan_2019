\documentclass[a4paper,12pt]{article}
%%%%%%%%%%%%%%%%%%%%%%%%%%%%%%%%%%%%%%%%%%%%%%%%%%%%%%%%%%%%%%%%%%%%%%%%%%%%%%%%%%%%%%%%%%%%%%%%%%%%%%%%%%%%%%%%%%%%%%%%%%%%%%%%%%%%%%%%%%%%%%%%%%%%%%%%%%%%%%%%%%%%%%%%%%%%%%%%%%%%%%%%%%%%%%%%%%%%%%%%%%%%%%%%%%%%%%%%%%%%%%%%%%%%%%%%%%%%%%%%%%%%%%%%%%%%
\usepackage{eurosym}
\usepackage{vmargin}
\usepackage{amsmath}
\usepackage{graphics}
\usepackage{epsfig}
\usepackage{enumerate}
\usepackage{multicol}
\usepackage{subfigure}
\usepackage{fancyhdr}
\usepackage{listings}
\usepackage{multicol}
\usepackage{framed}
\usepackage{graphicx}
\usepackage{amsmath}
\usepackage{chngpage}
%\usepackage{bigints}

\usepackage{vmargin}
% left top textwidth textheight headheight
% headsep footheight footskip
\setmargins{2.0cm}{2.5cm}{16 cm}{22cm}{0.5cm}{0cm}{1cm}{1cm}
\renewcommand{\baselinestretch}{1.3}
%- Higher Certificate, Paper I, 2001. Question 3
\setcounter{MaxMatrixCols}{10}
\begin{document}
\begin{enumerate}
\item    
\begin{multicols}{2}
\begin{eqnarray*}
P( S \geq 2300) &=& 1- P (S \leq 2300)\\
&=& 1-\Phi(1)\\
&=& 1- 0.8413 \\
&=& 0.1587 \\
\end{eqnarray*}
\begin{framed}
\noindent \textbf{Z Score for $S = 2300$}
\[z_{2300}  = \frac{2300 - 2000}{ 300}  = \frac{300}{300} = 1\]
\end{framed}
\begin{eqnarray*}
P( H \geq 2300) &=& 1- P (H \leq 2300)\\
&=& 1-\Phi(-1.6)\\
&=& \Phi(1.6) \\
&=& 0.9452 \\
\end{eqnarray*}
\begin{framed}
\noindent \textbf{Z Score for $H = 2300$}
\[z_{2300}  = \frac{2300 - 2500}{125}  = -\frac{200}{125} = -1.6\]
\end{framed}
\end{multicols}


\item  $X \sim N(2000,300^2)$

$P(S \geq 2300) = 1- P(S \leq 2300)$

\begin{framed}
\noindent \textbf{Z Score for $S = 2300$}

\[z_{2300}  = \frac{2300 - 2000}{ 300}  = \frac{300}{300} = 1\]
\end{framed}

%-----------%
$P(S >H) = P(S-H>0)$ where (S-H)  is $N(-500, (300^2) + (125^2)$

\[P(S >H) = \Phi\left( \frac{-500}{325} \right) = \phi (-1.5385) = 0.0620\]

%-----------%
%%%%%%%%%%%%%%%%%%%%%%%%%%%%%%%5
\item The lifetime X is S with probability 0.6 and H with probability 0.4.
Hence \[E[X ] = 0.6×2000 + 0.4× 2500 = 2200 \mbox{hrs} .\]
\begin{eqnarray*}
P( X > 2600) &=& P(X > 2600 | S)P(S ) + P( X > 2600 | H )P(H )\\
 &=& 600 0.6 100 0.4
300 125\\
\end{eqnarray*}


\begin{framed}
\noindent \textbf{Z Score for $H = 2300$}
\[z_{2300}  = \frac{2300 - 2500}{125}  = -\frac{200}{125} = -1.6\]
\end{framed}

\begin{eqnarray*} 
P(X \geq 2600) &=& \left(0.6 \times \Phi(-2)\right) + \left(0.4 \times \Phi(-0.8)\right)\\
&=& \left(0.6 \times 0.02275) + \left(0.4 \times 0.2119\right)\\
 &=& 0.01365 + 0.08476 \\ 
 &=& 0.09841.
\end{eqnarray*}

(using the appropriate tail areas from Normal tables)


%%%%%%%%%%%%%%%%%%%%%%%%%%%%%%%%%%%%%%%%%%%%%%%%%%



\begin{eqnarray*}
P(X > 2600) &=& P(X \geq 2600|S)P(S) + P(X > 2600|H)P(H) \\
 &=& \Phi \left( - \frac{600}{300}\right) + \Phi \left( - \frac{600}{300}\right) \\
 &=& 0.01363 + 0.08476 \\
 &=& 0.09841 \\
\end{eqnarray*}

   
%%%%%%%%%%%%%%%%%%%%%%%%%%%%%%%
(ii) ( ) ( ) ( )
( )
2600 | 0.08476 | 2600 0.8613
2600 0.09841
P X H P H
P H
P X
>
> = = =
>
.

\item X will have mean 2200. It is not Normally distributed but we way
apply the Central Limit Theorem if we know its variance. In large samples we
may take X as approximately N(2200, 346.82), so that $P(\bar{X}\geq 2300)$ = 
  
\[\bar{X} \sim n(\mu = 2200, \sigma^2 = (346.8)^2\]
               
 \[Z_{2300}  =   \frac{2300-2200}{       \left(\frac{346.8}{\sqrt{100}} \right)} = 2.8835\]



 %%%%%%%%%%%%%%%%%%%%%%%%%%%%%%%%%%%%%%%%%%
\begin{eqnarray*} 
P(\bar{X}\geq 2300) &=& 1- \Phi(2.8835)\\  &=& 1- 0.9980 \\&=& 0.02
\end{eqnarray*}

\end{enumerate}

%%%%%%%%%%%%%%%%%%%%%%%%%%%%%%%%%%%%%%%%%%
  

                                              

\end{document}
