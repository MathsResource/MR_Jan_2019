\documentclass[a4paper,12pt]{article}
%%%%%%%%%%%%%%%%%%%%%%%%%%%%%%%%%%%%%%%%%%%%%%%%%%%%%%%%%%%%%%%%%%%%%%%%%%%%%%%%%%%%%%%%%%%%%%%%%%%%%%%%%%%%%%%%%%%%%%%%%%%%%%%%%%%%%%%%%%%%%%%%%%%%%%%%%%%%%%%%%%%%%%%%%%%%%%%%%%%%%%%%%%%%%%%%%%%%%%%%%%%%%%%%%%%%%%%%%%%%%%%%%%%%%%%%%%%%%%%%%%%%%%%%%%%%
  \usepackage{eurosym}
\usepackage{vmargin}
\usepackage{amsmath}
\usepackage{graphics}
\usepackage{epsfig}
\usepackage{enumerate}
\usepackage{multicol}
\usepackage{subfigure}
\usepackage{fancyhdr}
\usepackage{listings}
\usepackage{framed}
\usepackage{graphicx}
\usepackage{amsmath}
\usepackage{chngpage}
%\usepackage{bigints}

\usepackage{vmargin}
% left top textwidth textheight headheight
% headsep footheight footskip
\setmargins{2.0cm}{2.5cm}{16 cm}{22cm}{0.5cm}{0cm}{1cm}{1cm}
\renewcommand{\baselinestretch}{1.3}

\setcounter{MaxMatrixCols}{10}
\begin{document}
%%%%%%%%%%%%%%%%%%%%%%%%%%%%%%%%%%%%%%%%%%%%%%%%%%%%%%%%%%%%%%%%%%%%%%%%%%%%%%%%%%%%%%%%%%%%%%%%%%

Higher Certificate, Paper II, 2001. Question 4
\begin{enumerate}[(a)]
\item  We need to assume that the lifetimes follow a Normal distribution, i.e. the data
are independent, identically distributed with variance \sigma 2 .
Then ( ) 2
2
2 1
1
~ χn
n S
\sigma −
−
.
The estimated variance s2 = 401.143 with 7 d.f.
The null hypothesis is \sigma 2 = 625 , so the test statistic is 7 401.143 4.493
625
× = .
Comparing with 2
7 χ , this is not significant, so the null hypothesis (strictly \sigma 2 ≤ 625 )
is not rejected.
\item The null hypothesis is 2 2
1 2 \sigma =\sigma where 2
i \sigma
is the variance on process i, and the
alternative hypothesis will be 2 2
1 2 \sigma >\sigma . (Again the null hypothesis is, strictly,
2 2
1 2 \sigma ≤\sigma .)
The ratio 1 2
2
1
2 1, 1
2
~ n n
S F
S − −. Estimates each have 9 d.f. 2 2
1 2 s = 384.667, s = 166.622 .
Test statistic is 384.667 2.309
166.622
= , not significant on F9,9 since the one-tail 5% point is
3.18.
Although there is a reduction, on this evidence it is not significant and the null
hypothesis cannot therefore be rejected. We must assume the variance has not been
reduced. Again the Normality of both sets of data must be assumed.

\end{enumerate}
\end{document}
