\documentclass[a4paper,12pt]{article}
%%%%%%%%%%%%%%%%%%%%%%%%%%%%%%%%%%%%%%%%%%%%%%%%%%%%%%%%%%%%%%%%%%%%%%%%%%%%%%%%%%%%%%%%%%%%%%%%%%%%%%%%%%%%%%%%%%%%%%%%%%%%%%%%%%%%%%%%%%%%%%%%%%%%%%%%%%%%%%%%%%%%%%%%%%%%%%%%%%%%%%%%%%%%%%%%%%%%%%%%%%%%%%%%%%%%%%%%%%%%%%%%%%%%%%%%%%%%%%%%%%%%%%%%%%%%
  \usepackage{eurosym}
\usepackage{vmargin}
\usepackage{amsmath}
\usepackage{graphics}
\usepackage{epsfig}
\usepackage{enumerate}
\usepackage{multicol}
\usepackage{subfigure}
\usepackage{fancyhdr}
\usepackage{listings}
\usepackage{framed}
\usepackage{graphicx}
\usepackage{amsmath}
\usepackage{chngpage}
%\usepackage{bigints}

\usepackage{vmargin}
% left top textwidth textheight headheight
% headsep footheight footskip
\setmargins{2.0cm}{2.5cm}{16 cm}{22cm}{0.5cm}{0cm}{1cm}{1cm}
\renewcommand{\baselinestretch}{1.3}

\setcounter{MaxMatrixCols}{10}
\begin{document}
%%%%%%%%%%%%%%%%%%%%%%%%%%%%%%%%%%%%%%%%%%%%%%%%%%%%%%%%%%%%%%%%%%%%%%%%%%%%%%%%%%%%%%%%%%%%%%%%%%

Higher Certificate, Paper II, 2001. Question 4

%%%%%%%%%%%%%%%%%%%%%%%%%%%%%%%%%%%%%%%%%%%%%%%%%%%%%%%%%%%%%%%%%%%%%%%%%%%%%%%%%%%%%%%%%%%%%%%%%%%%%%%%%%%%%%%%%%%%%%%%%%%%%%%%%%%%%%%%%%% 
\begin{table}[ht!]
 
\centering
 
\begin{tabular}{|p{15cm}|}
 
\hline  

4. (i) A manufacturer of candles claims to be able to control the variability in the length of life of the candles so that the standard deviation σ  (in minutes) is no greater than 25.  Wishing to check this claim, a wholesaler takes a random sample of 8 candles from one day's large output and tests them in the laboratory, giving the following results in minutes. 
 
  \[ 725   741   706   711   735   697   745   752 \]
 
Test the hypothesis H0: 
σ = 25 against the alternative σ > 25.  Explain your results and state any assumptions you made. (10) 


\\ \hline
  
\end{tabular}

\end{table}




 
%%%%%%%%%%%%%%%%%%%%%%%%%%%%%%%%%%%%%%%%%%%%%%%%%%%%%%%%%%%%%%%%%%%%%%%%%%%%%%%%%%%%%%%%%%%%%%%%%%%%%%%%%%%%%%%%%%%%%%%%%%%%%%%%%%%%%%%%%%%
\begin{enumerate}[(a)]
\item  We need to assume that the lifetimes follow a Normal distribution, i.e. the data
are independent, identically distributed with variance \sigma 2 .
Then ( ) 2
2
2 1
1
~ χn
n S
\sigma −
−
.
The estimated variance s2 = 401.143 with 7 d.f.
The null hypothesis is \sigma 2 = 625 , so the test statistic is 7 401.143 4.493
625
× = .
Comparing with 2
7 χ , this is not significant, so the null hypothesis (strictly \sigma 2 ≤ 625 )
is not rejected.



\newpage

\begin{table}[ht!]
 
\centering
 
\begin{tabular}{|p{15cm}|}
 
\hline  

(ii) The candle manufacturer is considering whether making slight adjustments to the manufacturing process will reduce the variability in the length of life of the candles produced.  Before making a decision, an experiment was conducted in which a number of candles were manufactured using each process and then tested in the laboratory, with the following results. 
 
   Process 1 724  743  705  711  736  699  745  752  740  705 
 
   Process 2 725  740  715  732  720  702  740  741  738  725 
 
Using an appropriate statistical test, investigate whether the manufacturer has been successful in reducing the variability in the lifetime of the candles using process 2.  Explain your conclusions, stating any assumptions that you made. (10)


\\ \hline
  
\end{tabular}

\end{table} 



\item The null hypothesis is 2 2
1 2 \sigma =\sigma where 2
i \sigma
is the variance on process i, and the
alternative hypothesis will be 2 2
1 2 \sigma >\sigma . (Again the null hypothesis is, strictly,
2 2
1 2 \sigma ≤\sigma .)
The ratio 1 2
2
1
2 1, 1
2
~ n n
S F
S − −. Estimates each have 9 d.f. 2 2
1 2 s = 384.667, s = 166.622 .
Test statistic is 384.667 2.309
166.622
= , not significant on F9,9 since the one-tail 5% point is
3.18.
Although there is a reduction, on this evidence it is not significant and the null
hypothesis cannot therefore be rejected. We must assume the variance has not been
reduced. Again the Normality of both sets of data must be assumed.

\end{enumerate}
\end{document}
