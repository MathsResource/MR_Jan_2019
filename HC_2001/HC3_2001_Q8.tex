\documentclass[a4paper,12pt]{article}
%%%%%%%%%%%%%%%%%%%%%%%%%%%%%%%%%%%%%%%%%%%%%%%%%%%%%%%%%%%%%%%%%%%%%%%%%%%%%%%%%%%%%%%%%%%%%%%%%%%%%%%%%%%%%%%%%%%%%%%%%%%%%%%%%%%%%%%%%%%%%%%%%%%%%%%%%%%%%%%%%%%%%%%%%%%%%%%%%%%%%%%%%%%%%%%%%%%%%%%%%%%%%%%%%%%%%%%%%%%%%%%%%%%%%%%%%%%%%%%%%%%%%%%%%%%%
  \usepackage{eurosym}
\usepackage{vmargin}
\usepackage{amsmath}
\usepackage{graphics}
\usepackage{epsfig}
\usepackage{enumerate}
\usepackage{multicol}
\usepackage{subfigure}
\usepackage{fancyhdr}
\usepackage{listings}
\usepackage{framed}
\usepackage{graphicx}
\usepackage{amsmath}
\usepackage{chngpage}
%\usepackage{bigints}

\usepackage{vmargin}
% left top textwidth textheight headheight
% headsep footheight footskip
\setmargins{2.0cm}{2.5cm}{16 cm}{22cm}{0.5cm}{0cm}{1cm}{1cm}
\renewcommand{\baselinestretch}{1.3}

\setcounter{MaxMatrixCols}{10}
\begin{document}
%%%%%%%%%%%%%%%%%%%%%%%%%%%%%%%%%%%%%%%%%%%%%%%%%%%
Higher Certificate, Paper III, 2001. Question 8
\begin{enumerate}[(a)]
\item The data should be Normally distributed about the respective treatment means;
and the variances of all the observations should be the same. The constant variance
condition is more important. For these data, the variances within the different
treatments are so different that this condition cannot be assumed. There is an obvious
relation between mean and variance, which a transformation may be able to correct
for.
\item  x : rate 3, variance = 13.49.
ln x : rate 2, mean = 4.827, variance = 0.1212.
1/ x : rate 1, variance = 0.0004721;
rate 4, mean = 0.00147.
The logarithmic transformation should be used, since it achieves approximately the
same variance for each rate and there is no evidence of a mean–variance relation.
\item 
SOURCE DF SS MS
Rates 3 21.153 7.051 F3,8 = 45.6
Residual 8 1.236 0.1545 = s2
TOTAL 11 22.389
Comparing 45.6 with F3,8, there is strong evidence of a difference among the means
for the various rates.
\item Rate 3: mean x = 5.716 , pooled variance from ANOVA = 0.1545 (8df).
Two-tailed 5% point of t8 is 2.571. r = 3. Limits are
2
x t s
r
\pm  .
This gives 5.716 2.571 0.1545 5.716 0.583
3
\pm  = \pm  i.e. 5.133 to 6.299 .
e5.133 =169.5; e6.299 = 544.0 [Note : e5.716 = 303.7 ]

\end{enumerate}
\end{document}