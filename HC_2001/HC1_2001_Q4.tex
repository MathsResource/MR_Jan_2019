\documentclass[a4paper,12pt]{article}
%%%%%%%%%%%%%%%%%%%%%%%%%%%%%%%%%%%%%%%%%%%%%%%%%%%%%%%%%%%%%%%%%%%%%%%%%%%%%%%%%%%%%%%%%%%%%%%%%%%%%%%%%%%%%%%%%%%%%%%%%%%%%%%%%%%%%%%%%%%%%%%%%%%%%%%%%%%%%%%%%%%%%%%%%%%%%%%%%%%%%%%%%%%%%%%%%%%%%%%%%%%%%%%%%%%%%%%%%%%%%%%%%%%%%%%%%%%%%%%%%%%%%%%%%%%%
  \usepackage{eurosym}
\usepackage{vmargin}
\usepackage{amsmath}
\usepackage{graphics}
\usepackage{epsfig}
\usepackage{enumerate}
\usepackage{multicol}
\usepackage{subfigure}
\usepackage{fancyhdr}
\usepackage{listings}
\usepackage{framed}
\usepackage{graphicx}
\usepackage{amsmath}
\usepackage{chngpage}
%\usepackage{bigints}

\usepackage{vmargin}
% left top textwidth textheight headheight
% headsep footheight footskip
\setmargins{2.0cm}{2.5cm}{16 cm}{22cm}{0.5cm}{0cm}{1cm}{1cm}
\renewcommand{\baselinestretch}{1.3}

\setcounter{MaxMatrixCols}{10}
\begin{document}
Higher Certificate, Paper I, 2001. Question 4

\begin{table}[ht!]
\centering
\begin{tabular}{|p{15cm}|}
\hline
\noindent 
4. The random variable X follows the binomial B(n, p) distribution with probability
mass function
\[f(x,n,p)=\Pr(xk;n,p)=\Pr(X=x)={\binom {n}{x}}p^{x}(1-p)^{n-x}}\]
,
where q = 1 − p. Show that $E(X) = np$ and Var(X) = npq.
\\ \hline
\end{tabular}
\end{table}
%%%%%%%%%%%%%%%%%%%%%%%%%%%%%%%%%%%%%%%%
\begin{framed}
The probability of getting exactly k successes in n trials is given by the probability mass function: 
\[ {\displaystyle f(k,n,p)=\Pr(k;n,p)=\Pr(X=k)={\binom {n}{k}}p^{k}(1-p)^{n-k}}  \]
for k = 0, 1, 2, ..., n, where 
\[{\displaystyle {\binom {n}{k}}={\frac {n!}{k!(n-k)!}}} \]
\end{framed}
%%%%%%%%%%%%%%%%%%%%%%%%%%%%%%%%%%%%%%%%


\begin{enumerate}[(a)]
\item An easy method is to consider X as $\sumX_i$ , where $X_i$ are a set of n Bernoulli variables
with ( 1) , ( 0) (1 ). i i P X = = p P X = = − p
\begin{itemize}
\item Then E[ ] , so [ ] i X = p E X = np .
\item Also E 2 , so Var ( ) 2 and Var ( ) ( 2 ) . i i X  = p X = p − p X = n p − p = npq
\item ALTERNATIVELY: [ ] ( ) ( ) 0 1

\end{itemize}

\begin{table}[ht!]
\centering
\begin{tabular}{|p{15cm}|}
\hline
\noindent 
A mathematics class in a school is divided into set A with 12 students and set B with 25 students. Both groups are given a test consisting of 16 short questions.
For any student in set A, the score (that is, the number of correct answers) is distributed as B(16, 0.75); for any student in set B, the score is distributed as
B(16, 0.5). All students answer independently.
(i) Find the probability that
(a) a given set A student gets all 16 questions right,

(b) at least one student in set A gets all 16 questions right.

\\ \hline
\end{tabular}
\end{table}
%==========================================================================================%
\begin{eqnarray*}
E[X] 
&=& \sum^{n}_{x=0} x {n \choose x} p^x (1-p)^{n-x} \\
&=& \sum^{n}_{x=1} \left[ x {n \choose x} \right] p^x (1-p)^{n-x} \\
&=& \sum^{n}_{x=1} \left[  \frac{x \;\times\; n!}{(n-x)! x!} \;\times\; \right] p^x (1-p)^{n-x} \\
&=& \sum^{n}_{x=1} \left[  \frac{ n!}{(n-x)! (x-1)!} \;\times\; \right] p^x (1-p)^{n-x} \\
&=& np \sum^{n}_{x=1}  {n-1 \choose x-1} p^{x-1} (1-p)^{(n-1)-(x-1)} \\
&=& np
\end{eqnarray*}



\begin{framed}
\[\operatorname{Var}(X) = E[X^2] - (E[X])^2\] \smallskip
\[E[X^2] = E[X(X-1)] + E[X]\] \small skip
\[\operatorname{Var}(X) = E[X(X-1)] + E[X] - (E[X])^2\]
\end{framed}


\begin{eqnarray*}
E[X(X-1)] 
&=& \sum^{n}_{x=0} x(x-1) {n \choose x} p^x (1-p)^{n-x} \\
&=& \sum^{n}_{x=2} x(x-1) {n \choose x} p^x (1-p)^{n-x} \\
&=& n(n-1)p^2 \sum^{n}_{x=2} {n-2 \choose x-2} p^{x-2} (1-p)^{(n-2)-(x-2)} \\
&=& n(n-1)p^2 \\
\end{eqnarray*}


\begin{framed}
Sum of all probabilities for $(X-2)$ is equal to 1
\[ \sum^{n}_{x=2} {n-2 \choose x-2} p^{x-2} (1-p)^{(n-2)-(x-2)} = 1\]
\end{framed}


Hence 
\begin{eqnarray*}
\operatorname{Var}(X) &=& (n(n-1)p^2) + np + n^2p^2\\
&=& np - np^2 \\
&=&np(1-p)\\
&=&npq\\
\end{eqnarray*}


PGFs and MGFs can also be used.


\[0.75^{16} \approx 0.01000226 \approx 0.0100\]
\begin{eqnarray*}
Var (X) &=& n(n −1) p2 + np − n2 p2 \\
&=& np − np2\\ 
&=& np(1-p) \\
&=& npq \\
\end{eqnarray*}
PGFs or MGFs could also be used.
\item  $0.75^{16} \approx 0.0100226 = 0.0100$ approx.
%%%%%%%%%%%%%%%%%%%%%%%%%%%%%%%%%%%%%%%%%%%%%%%%%%%%%%%%%%%%%%%%%%%
\newpage
\begin{table}[ht!]
\centering
\begin{tabular}{|p{15cm}|}
\hline
\noindent (ii) Use an appropriate approximation to find the probability that a given set B
student scores more than a given set A student.



\\ \hline
\end{tabular}
\end{table}


\item 1 − P(no one gets all 16 right), probability is { }1− 1− 0.7516 12
= 1 − {0.9899774}12 = 0.1139.
\item $P(B − A > 0)$ can be studied using a Normal approximation to the difference
B − A, i.e. N(16{0.5 − 0.75}, 16{(0.5 \times 0.5) + (0.75 \times 0.25)}), i.e. N(−4,7) .
\begin{itemize}
\item The probability is found as 1
2
P B − A > 
 
using a continuity correction since B − A
takes discrete values.
\item Hence it is  − ≈
   
\[  \Phi\left( \frac{0.5 -(-4)}{7}\right) = \Phi\left( \frac{4.7}{7}\right) = \Phi() \approx \]

\item Note: this would be $0.0653$ without the continuity correction.
\end{itemize}
%%%%%%%%%%%%%%%%%%%%%%%%%%%%%%%%%%%%%%%%%%%%%%%%%%%%%%%%%%%%%%%%%
\newpage

\begin{table}[ht!]
\centering
\begin{tabular}{|p{15cm}|}
\hline
\noindent (iii) Let X and Y denote the mean scores of students in set A and set B
respectively. Write down $E(X)$ and $E(Y)$ , and show that
$Var(X ) =1/ 4$ and $Var(Y ) = 4/ 25$.
\\ \hline
\end{tabular}
\end{table}
\item 
\begin{eqnarray*}
E X  &=& E[X ] \\
&=& np \\
&=& 16\times 0.75 \\
&=& 12 
\end{eqnarray*}

in set A .

\begin{itemize}
\item Similarly, E Y  =16×0.5 = 8 i$n$set B.
\item There are 12 students in A and 25 in B, so that
\item $Var (X)  = 16 \times 0.75 \times 0.25 = 3$
  \end{itemize}

1
12 4
X = × × = in set A

Var ( ) 16 0.5 0.5 4
25 25
Y = × × = in set B.

\end{enumerate}
\end{document}
