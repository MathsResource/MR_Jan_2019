\documentclass[a4paper,12pt]{article}
%%%%%%%%%%%%%%%%%%%%%%%%%%%%%%%%%%%%%%%%%%%%%%%%%%%%%%%%%%%%%%%%%%%%%%%%%%%%%%%%%%%%%%%%%%%%%%%%%%%%%%%%%%%%%%%%%%%%%%%%%%%%%%%%%%%%%%%%%%%%%%%%%%%%%%%%%%%%%%%%%%%%%%%%%%%%%%%%%%%%%%%%%%%%%%%%%%%%%%%%%%%%%%%%%%%%%%%%%%%%%%%%%%%%%%%%%%%%%%%%%%%%%%%%%%%%
  \usepackage{eurosym}
\usepackage{vmargin}
\usepackage{amsmath}
\usepackage{graphics}
\usepackage{epsfig}
\usepackage{enumerate}
\usepackage{multicol}
\usepackage{subfigure}
\usepackage{fancyhdr}
\usepackage{listings}
\usepackage{framed}
\usepackage{graphicx}
\usepackage{amsmath}
\usepackage{chngpage}
%\usepackage{bigints}

\usepackage{vmargin}
% left top textwidth textheight headheight
% headsep footheight footskip
\setmargins{2.0cm}{2.5cm}{16 cm}{22cm}{0.5cm}{0cm}{1cm}{1cm}
\renewcommand{\baselinestretch}{1.3}

\setcounter{MaxMatrixCols}{10}
\begin{document}
%%%%%%%%%%%%%%%%%%%%%%%%%%%%%%%%%%%%%%%%%%%%%%%%%%%

Higher Certificate, Paper II, 2001. Question 3
\begin{enumerate}[(a)]
\item  Both data sets fairly symmetrical, but not clustered round mean.
Cholesterol Level
200 220 240 260 280 300 320
Placebo (11obs)
Drug (15obs)
Placebo on average somewhat higher than Drug.
\item 
Rank 1 2 3 4 5 6 7 8 9 10 11 12 13
Obs. 225 227 230 233 240 242 246 250 251 255 257 262 263
Trt. D D D D D P P D P P D D D
14 15 16 17 18 19 20 21 22 23 24 25 26
266 270 271 271 275 280 281 282 282 285 292 294 299
P P D D D D P D D P P P P
(________) (_________)
A Mann-Whitney U test may be applied.
Sum of ranks of P = 179. ( ) ( ) P
11 15 1 11 12 179
2
U = × + × −
= 52
The 5% one-sided critical value is 44 for n1=11, n2=15.
Therefore on these data there is no evidence for claiming that the drug reduces blood
pressure.
\item If we assume the data to be Normally distributed, with the same \sigma 2 in each
distribution, a t test can be applied.
Placebo: ( )x = 271.00, s2 = 20.489 2
Drug: ( )x = 256.53, s2 = 20.908 2
clearly s2
P, s2
D can be pooled to give:
s2 = ( ) ( ) ( )
2 2
2 10 20.489 14 20.908
429.9172 20.734
24
× + ×
= = .
0 D P 1 D P H :\mu =\mu , H :\mu < \mu
( )
P D
24
P D
271.00 256.53 14.47 1.758
1 1 8.23 20.734
11 15
t x x
SE x x
= − = − = =
−
+
which is greater than the one-tail 5% value which is 1.711.
Hence there is evidence to claim a reduction using the drug.
\item The assumption of Normality increases the power of the t test compared with
Mann-Whitney which makes no distributional assumption.
\end{enumerate}
\end{document}
