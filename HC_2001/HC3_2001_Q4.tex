\documentclass[a4paper,12pt]{article}
%%%%%%%%%%%%%%%%%%%%%%%%%%%%%%%%%%%%%%%%%%%%%%%%%%%%%%%%%%%%%%%%%%%%%%%%%%%%%%%%%%%%%%%%%%%%%%%%%%%%%%%%%%%%%%%%%%%%%%%%%%%%%%%%%%%%%%%%%%%%%%%%%%%%%%%%%%%%%%%%%%%%%%%%%%%%%%%%%%%%%%%%%%%%%%%%%%%%%%%%%%%%%%%%%%%%%%%%%%%%%%%%%%%%%%%%%%%%%%%%%%%%%%%%%%%%
  \usepackage{eurosym}
\usepackage{vmargin}
\usepackage{amsmath}
\usepackage{graphics}
\usepackage{epsfig}
\usepackage{enumerate}
\usepackage{multicol}
\usepackage{subfigure}
\usepackage{fancyhdr}
\usepackage{listings}
\usepackage{framed}
\usepackage{graphicx}
\usepackage{amsmath}
\usepackage{chngpage}
%\usepackage{bigints}

\usepackage{vmargin}
% left top textwidth textheight headheight
% headsep footheight footskip
\setmargins{2.0cm}{2.5cm}{16 cm}{22cm}{0.5cm}{0cm}{1cm}{1cm}
\renewcommand{\baselinestretch}{1.3}

\setcounter{MaxMatrixCols}{10}
\begin{document}
%%%%%%%%%%%%%%%%%%%%%%%%%%%%%%%%%%%%%%%%%%%%%%%%%%%

Higher Certificate, Paper III, 2001. Question 4
%%%%%%%%%%%%%%%%%%%%%%%%%%%%%%%%%%%%%%%%%%%%%%%%%%%%%%%%%%%%%%%%%%%%%%%%%%%%%%%%%%%%%%%%%

\begin{table}[ht!]
     


\centering
     


\begin{tabular}{|p{15cm}|}
     


\hline 

4. Describe the circumstances in which simple exponential smoothing may be used to provide forecasts 
of future values of a time series t X .  
 (i) If () ˆ ,1 x t denotes the one-step-ahead forecast at time t where 
 
  ( ) ( ) ( ) ( ) 23 123ˆ ,1 1 1 1 ... t t t t x t x x x x α α α α α α α −−− = + − + − + − +
 
 show that ( ) ( ) { } () ˆ ˆ ˆ ,1 1,1 1,1 . t x t x x t x t α = − − + −   

\begin{framed}
The raw data sequence is often represented by 
${\displaystyle \{x_{t}\}}$ 
 beginning at time 
$ {\displaystyle t=0}$ 
, and the output of the exponential smoothing algorithm is commonly written as 
$ {\displaystyle \{s_{t}\}} $
, which may be regarded as a best estimate of what the next value of 
$ {\displaystyle x} $
 will be. When the sequence of observations begins at time 
${\displaystyle t=0} $
, the simplest form of exponential smoothing is given by the formulas:[1] 
\[ {\displaystyle {\begin{aligned}s_{0}&=x_{0}\\s_{t}&=\alpha x_{t}+(1-\alpha )s_{t-1},\ t>0\end{aligned}}} \]
 
where 
$ {\displaystyle \alpha } $
 is the smoothing factor, and 
$ {\displaystyle 0<\alpha <1} $. 
\end{framed}
 
(ii) The following data give the weekly sales of a commodity over a 12 week period where no special sales promotions or higher 
than usual levels of advertising took place. 
 
Week 1 2 3 4 5 6 7 8 9 10 11 12 
Sales 5105 5618 5514 5423 6044 5790 5437 5848 6253 5835 5740 6063 
 
(a) Plot the data as a time series. 

 
(b) Use simple exponential smoothing with α = 0.25 and the first week's sales as the forecast for the second week's sales to forecast
 one week ahead at each time point. 
 
(c) Plot the set of forecasts on the same graph as the original series and comment on their adequacy. 
 
(d) Suppose a similar set of forecasts had been calculated for some other value of $\alpha$ .  How would you compare the adequacy of the two sets of forecasts?  
 
 

\\ \hline



\end{tabular}
    


\end{table}


%%%%%%%%%%%%%%%%%%%%%%%%%%%%%%%%%%%%%%%%%%%%%%%%%%%%%%%%%%%%%%%%%%%%%%%%%%%%%%%%%%%%%%%%%

If using a model based on polynomial trends, long-past observations still have some
influence on forecasts. Weighted averages of observations are useful, most recent
receiving largest weights. In (i), the forecast x (tˆ,1) uses weights (1 )j \alpha  −\alpha  which
decay exponentially to 0. The current observation, at j = 0, receives most weight. If
\alpha  = 1, the forecast is simply the current observation. This forecasting method only
uses at each step one previous value of x, one time-unit past, is quick and easy, and is
based on a statistical model where the first difference t t 1 x x − − is MA(1) in the errors.
\begin{enumerate}[(a)]
\item  Using ( ) ( ) 2
1 2 ˆ ,1 1 (1 ) ... t t t x t \alpha  x \alpha  \alpha  x \alpha  \alpha  x − − = + − + − +
we have
( ) ( ) ( )2
1 2 3 ˆ 1,1 1 1 ... t t t x t \alpha  x \alpha  \alpha  x \alpha  \alpha  x − − − − = + − + − + .
( ) ( ) ( ) ( ) ( ) 2 3
1 2 3 1 ˆ 1,1 1 1 1 ... t t t \alpha  x t \alpha  \alpha  x \alpha  \alpha  x \alpha  \alpha  x − − − ∴ − − = − + − + − +
which gives
ˆ ( ,1) (1 ) ˆ ( 1,1) t x t =\alpha  x + −\alpha  x t −
{ ˆ ( 1,1)} ˆ ( 1,1) t=\alpha  x − x t − + x t − .
(ii) (a)
5100
5300
5500
5700
5900
6100
6300
0 2 4 6 8 10 12 14
WEEK
WEEKLY SALES
Sales
Forecast using
alpha = 0.25
(b) With \alpha  = 0.25, the forecast for week 3 sales is ¼(5618 − 5105) + 5105,
i.e. 5233; then for week 4 the forecast is ¼(5514 − 5233) + 5233, etc. Hence:
week 2 3 4 5 6 7 8 9
5105 5233 5303 5333 5511 5581 5545 5621
week 10 11 12 13
5779 5793 5780 5850
(c) Forecasts do not seem adequate as they frequently underestimate sales
by a large amount. Perhaps using a larger value of \alpha , to reduce the weight of
past history, would improve this model.
(d) Considering ˆ ( 1,1) tx − x t − as an error, and comparing its average size
on two models, would be a guide: we could use a root-mean-square
{ ( )}
1
ˆ 1,1 2 2
11
t x − x t − 
 
 
 
\sigma for weeks 2 to 12.
\end{enumerate}
\end{document}
