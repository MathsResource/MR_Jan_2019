\documentclass[a4paper,12pt]{article}
%%%%%%%%%%%%%%%%%%%%%%%%%%%%%%%%%%%%%%%%%%%%%%%%%%%%%%%%%%%%%%%%%%%%%%%%%%%%%%%%%%%%%%%%%%%%%%%%%%%%%%%%%%%%%%%%%%%%%%%%%%%%%%%%%%%%%%%%%%%%%%%%%%%%%%%%%%%%%%%%%%%%%%%%%%%%%%%%%%%%%%%%%%%%%%%%%%%%%%%%%%%%%%%%%%%%%%%%%%%%%%%%%%%%%%%%%%%%%%%%%%%%%%%%%%%%
  \usepackage{eurosym}
\usepackage{vmargin}
\usepackage{amsmath}
\usepackage{graphics}
\usepackage{epsfig}
\usepackage{enumerate}
\usepackage{multicol}
\usepackage{subfigure}
\usepackage{fancyhdr}
\usepackage{listings}
\usepackage{framed}
\usepackage{graphicx}
\usepackage{amsmath}
\usepackage{chngpage}
%\usepackage{bigints}

\usepackage{vmargin}
% left top textwidth textheight headheight
% headsep footheight footskip
\setmargins{2.0cm}{2.5cm}{16 cm}{22cm}{0.5cm}{0cm}{1cm}{1cm}
\renewcommand{\baselinestretch}{1.3}

\setcounter{MaxMatrixCols}{10}
\begin{document}
%%%%%%%%%%%%%%%%%%%%%%%%%%%%%%%%%%%%%%%%%%%%%%%%%%%

Higher Certificate, Paper III, 2001. Question 1
\begin{enumerate}[(a)]
\item  Using the differences, test the null hypothesis "mean difference = 0", assuming
Normality of the distribution of differences.
12.3, 2 (24.3176)2 ; so test statistic is 12.3 0 1.60
24.3176 / 10 d d = − s = − − = − , which is not
significant as an observation from t9.
(ii) Correction term = 48312 / 20 =1166928.05 .
SS for weeks 1 (24772 23542 ) correction 1167684.50 1166928.05
10
= + − = −
= 756.45
SS for patients 1 (2582 ... 6132 ) correction = 1238009.50 1166928.05
2
= ++ − −
=71081.45
Analysis of Variance
ITEM DF SS MS
Patients 9 71081.45 7897.94 F9,9 = 26.71 sig at 0.1%
Weeks 1 756.45 756.45 F1,9 = 2.56 not significant
Residual 9 2661.05 295.67
TOTAL 19 74498.95
(iii)
Ranking of |diff| 4 5 6 8 9 24 25 26 40 56
(1) (2) (3) (4) (5) (6) (7) (8) (9) (10)
Sign + + + − − − + − − −
Sum of + ranks is S+ = 13; S− = 42. Tables show that for n = 10 and at the 5% level
in a two-tail test, the smaller of S+ and S− should be 8 or less for significance.
(iv) The null hypothesis for (i) and (ii) is as stated in (i). The alternative
hypothesis is "mean difference ≠ 0". Normality of the data would be required in (ii),
not just of the differences. A dot-plot would in either case cast serious doubt on this
assumption. The Wilcoxon test does not require any distributional assumption, only
that the + and − rankings are randomly placed in the set. In each case we must not
reject the null hypothesis because we do not have any statistically significant test
results.
(v) 2
9 1,9 t = F .


\end{enumerate}
\end{document}