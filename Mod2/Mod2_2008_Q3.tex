\documentclass{article}
\usepackage[utf8]{inputenc}
\usepackage{framed}
\usepackage{enumerate}
\title{RSS_Jan_2019_Mod2}
\author{kobriendublin }
\date{December 2018}

\begin{document}



%- Higher Certificate, Module 2, 2008. Question 3

\begin{enumerate}
    \item 
X ~ N(0, 1), Y ~ N(0, 1); X and Y are independent
(i)\[ P(3X > 4Y + 2) = P(3X − 4Y > 2),
= P(V > 2),\] where \[V = 3X − 4Y ~ N(0, 3^2 + 4^2 = 25).\]
\[P(V > 2) = P(Z > 205− = 0.4\] [where Z ~ N(0, 1)] ) 

\[P(V > 2)  = 1 − \Phi(0.4) = 0.3446.\]
Since X and Y are independent, 
\[P(X \leq x, Y \leq x) = P(X \leq x)\times P(Y \leq x) = [\Phi(x)]2.\]
%%%%%%%%%%%%%%%%%%%%%%%%%%%%%%%%%%
    \item \[max(X, Y) \leq w ⇔ (X \leq w)∩(Y \leq w)\],
so \[P(max(X, Y) \leq w) = [\Phi(w)]^2\] from above.
%%%%%%%%%%%%%%%%%%%%%%%%%%%%%%%%%%
    \item Q1 satisfies ()411=QFW, so ()2114Q\Phi=.
()()111,and10.52QQ− ∴\Phi==\Phi=.

Similarly, ()()33330.86642WFQQ=⇒\Phi==, so = 1.108 using linear interpolation in the Society's Statistical tables for use in examinations [1.11 was allowed as the (3 s.f.) answer in the examination, being the nearest tabular entry]. $\Phi()13 = 0.86Q−=$
%%%%%%%%%%%%%%%%%%%%%%%%%%%%%%%%%%
    \item $P(W outside (Q1, Q3)) = 0.5$, so $N \sim B(100, 0.5)$ which we approximate by $N(50, 25)$. 
    Hence
\[P(N ≥ 58) \approx 1 − (
5.115505.57\Phi\]


\[−=⎟⎠⎞⎜⎝⎛−\Phi = 1 − 0.9332 = 0.0668.\]
\end{enumerate}
\end{document}