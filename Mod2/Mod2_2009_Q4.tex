\documentclass[a4paper,12pt]{article}
%%%%%%%%%%%%%%%%%%%%%%%%%%%%%%%%%%%%%%%%%%%%%%%%%%%%%%%%%%%%%%%%%%%%%%%%%%%%%%%%%%%%%%%%%%%%%%%%%%%%%%%%%%%%%%%%%%%%%%%%%%%%%%%%%%%%%%%%%%%%%%%%%%%%%%%%%%%%%%%%%%%%%%%%%%%%%%%%%%%%%%%%%%%%%%%%%%%%%%%%%%%%%%%%%%%%%%%%%%%%%%%%%%%%%%%%%%%%%%%%%%%%%%%%%%%%
\usepackage{eurosym}
\usepackage{vmargin}
\usepackage{amsmath}
\usepackage{graphics}
\usepackage{epsfig}
\usepackage{enumerate}
\usepackage{multicol}
\usepackage{subfigure}
\usepackage{fancyhdr}
\usepackage{listings}
\usepackage{framed}
\usepackage{graphicx}
\usepackage{amsmath}
\usepackage{chngpage}
%\usepackage{bigints}

\usepackage{vmargin}
% left top textwidth textheight headheight
% headsep footheight footskip
\setmargins{2.0cm}{2.5cm}{16 cm}{22cm}{0.5cm}{0cm}{1cm}{1cm}
\renewcommand{\baselinestretch}{1.3}

\setcounter{MaxMatrixCols}{10}
\begin{document}

%- Higher Certificate, Module 2, 2009. Question 4



\begin{enumerate}
    \item (),222xxXduuxFxx\theta\theta\theta\theta\theta\theta\theta\theta−−+⎡⎤===−≤⎢⎥⎣⎦∫ .
∴ P(X > x) = 1 − = 1 − ()xFX,22xxx\theta\theta\theta\theta\theta\theta+−=−≤≤ .
    \item As X and Y are independent and with the same distribution, we have for Z = max(X, Y)
()()()()()()2.XPZzPXzYzPXzPYzFz≤=≤∩≤=≤≤=⎡⎤⎡⎣⎦⎣ .
2(),2ZzFzz\theta\theta\theta\theta+⎛⎞∴=−≤⎜⎟⎝⎠ , using the first result of part (i).

Differentiating, we have that the pdf of Z is
()()2,2Zzfzz\theta\theta\theta\theta+=−≤ .
()()2322246zzzzEZdz\theta\theta\theta\theta\theta \theta\theta\theta\theta−−+⎡⎤∴==+⎢⎥⎣⎦∫ .
%%%%%%%%%%%%%%%%%%%%%%%%%%%%%%%%%%%%%%%%%%%
    \item  Arguing similarly to part (ii),
()()()()()()2.PWwPXwYwPXwPYwPXw>=>∩>>>=>⎡⎤⎡⎣⎦⎣ =.
∴()2,2wPWww\theta\theta\theta\theta−⎛⎞>=−≤≤⎜⎟⎝⎠ , using the second result of part (i).
Differentiating (and reversing the sign), we thus have that the pdf of W is
()()2,2Wwfww\theta\theta\theta\theta−=−≤ .
()()2322246wwwwEWdw\theta\theta\theta\theta\theta \theta\theta\theta\theta−−−⎡⎤∴==−⎢⎥⎣⎦∫
.
[This could also be argued by symmetry from the result for E(Z) based on the underlying uniform distribution.]
%%%%%%%%%%%%%%%%%%%%%%%%%%%%%%%%%%%%%%%%%%%
    \item  E[k(Z − W)] = \thetak[⅓ + ⅓] = 2\thetak/3. This equals \theta if k = 3/2.
\end{enumerate}

\end{document}
