Question 3
Let X represent cycling time without delays: X ~ N(15, 1).
(i) ()()17151720.97721PX−⎛⎞≤=Φ=Φ=⎜⎟⎝⎠.
[Φ denotes the cdf of the standard Normal distribution as usual.]
(ii) Adding in the delay times, also Normally distributed [N(0.7, 0.09)], and letting T denote the total time:
(a) T ~ N(15.7, 1.09), so ()()1715.7171.2450.89341.09PT−⎛⎞≤=Φ=Φ=⎜⎟⎝⎠;
(b) T ~ N(16.4, 1.18), so ()()1716.4170.5520.70961.18PT−⎛⎞≤=Φ=Φ=⎜⎟⎝⎠;
(c) T ~ N(17.1, 1.27), so ()()1717.1170.08870.46461.27PT−⎛⎞≤=Φ=Φ−=⎜⎟⎝⎠.
(iii) The number of delays is distributed as B(3, ½). Hence the situations in (i), (ii)(a), (ii)(b) and (ii)(c) arise with probabilities 1/8, 3/8, 3/8 and 1/8 respectively, so the (unconditional) mean of the total journey time is 1331128.4()1515.716.417.116.0588888ET=×+×+×+×== minutes.
(iv) Mean time 1.502516.05,10TN⎛⎞⎜⎟⎝⎠∼. ()()1716.05172.4510.99290.15025PT−⎛⎞≤=Φ=Φ=⎜⎟⎝⎠.

[Φ denotes the cdf of the standard Normal distribution as usual.]
[Note: this would be 0.0653 without the continuity correction.]
(iii) []160.7512 in set EXEXnp⎡⎤===×=⎣⎦ .
Similarly, 160.58EY⎡⎤=×=⎣⎦ in set B.
There are 12 students in A and 25 in B, so that ()160.750.251Var 124X××== in set A ()160.50.54Var 2525Y××= in set B.