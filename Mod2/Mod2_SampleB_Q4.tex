Question 4
Probability mass function: ()()1,0,1,2,...,01xfxppxp=−=< .
(i)
x
5
4
3
2
1
0
4/27
2/9
1/3
P(X = x)
(ii) 0()(1)xxEXxp
∞=Σ. This can be found in several ways; the method shown here (using geometric series) is perhaps not the most efficient, but should be easy to understand. We have
{}230(1)0(1)2(1)3(1)...xxxpppppp∞=−=+−+−+−+Σ
23233{0(1)(1
))...ppp
...} 231(1)(1)11...1(1)1(1)1(1)1(1)pppppp ppppp⎧⎫⎧⎫−−−−=+++==⎨⎬⎨−−−−−−−−⎩⎭⎩⎭ .
Similarly, ()22220(1)(1)()1xxppEXxppp∞=−+−=−=Σ.
Thus Var(X) = E(X2) – {E(X)}2 = 2222(1)(1)11ppp ppp⎛⎞−+−−−=−⎜⎟⎝⎠ .
[The probability generating function or moment generating function could also be used – this work is in Module 5.]
Solution continued on next page
(iii) ()()()()(11[geometric series]111xrxrxppPXxpppp∞=−≥=−==−−−Σ (for x = 0, 1, 2, …). We now use ()()()PABPABPB∩= and take the event A as "X ≥ l + m" and the event B as "X ≥ l", so that A∩B = A. Thus ()()()()(111lmmlpPXlmXlpPXmp+−≥+≥==−=≥− .
This is the "lack of memory" property of a geometric distribution.
(iv) By independence, ()()()()(11zzPZzPXzPYzpθ≥=≥≥=−−.
()()()()(){}()(){}111111zzPZzPZzPZzppθθ+==≥−≥+=−−−−−
()(){}()()(){}(111111zz ppppppθθθθθθ=−−−++−=−−+−, for z = 0, 1, … .
This is a geometric distribution as given at the start of the question with p replaced by p + θ – pθ. Hence, from part (ii), 1()ppEZppθθθθ−−+=+−, Var(Z) = ()21ppppθθθθ−−++−.