\documentclass[a4paper,12pt]{article}
%%%%%%%%%%%%%%%%%%%%%%%%%%%%%%%%%%%%%%%%%%%%%%%%%%%%%%%%%%%%%%%%%%%%%%%%%%%%%%%%%%%%%%%%%%%%%%%%%%%%%%%%%%%%%%%%%%%%%%%%%%%%%%%%%%%%%%%%%%%%%%%%%%%%%%%%%%%%%%%%%%%%%%%%%%%%%%%%%%%%%%%%%%%%%%%%%%%%%%%%%%%%%%%%%%%%%%%%%%%%%%%%%%%%%%%%%%%%%%%%%%%%%%%%%%%%
\usepackage{eurosym}
\usepackage{vmargin}
\usepackage{amsmath}
\usepackage{graphics}
\usepackage{epsfig}
\usepackage{enumerate}
\usepackage{multicol}
\usepackage{subfigure}
\usepackage{fancyhdr}
\usepackage{listings}
\usepackage{framed}
\usepackage{graphicx}
\usepackage{amsmath}
\usepackage{chngpage}
%\usepackage{bigints}

\usepackage{vmargin}
% left top textwidth textheight headheight
% headsep footheight footskip
\setmargins{2.0cm}{2.5cm}{16 cm}{22cm}{0.5cm}{0cm}{1cm}{1cm}
\renewcommand{\baselinestretch}{1.3}

\setcounter{MaxMatrixCols}{10}
\begin{document}

Higher Certificate, Module 2, 2010. Question 1

\section{Introduction}

\begin{enumerate}
    \item
$E(X) = 2p$ $Var(X) = 2p(1 – p)$ $P(X = 2) = p2$
P(X = 0 | X < 2) = ()()()()202021PXXPXPXp=∩<⎡⎤=⎣⎦=<− = ()221111pppp−−=−+
P(X = 1 | X < 2) = ()()()()212121PXXPXPXp=∩<⎡⎤=⎣⎦=<− = ()221211ppppp−=−+

\smallskip
[Alternatively, $P(X = 1 | X < 2)$ may be obtained as 1 − P(X = 0 | X < 2).]
    \item Each X is the number of successes in two Bernoulli trials with probability p of success. 
    \item So, noting that the Xs are independent and p is the same for all of them, Y is the number of successes in 200 such trials and we have $Y \sim B(200, p)$.
    \item With p = 2/3, Y ~ N400400,39⎛⎜⎝⎠, approximately.
()()140.5(4003)140111075203PY.⎛⎞−>≈−Φ=−Φ⎜⎟⎝⎠
(where $\Phi$ denotes the standard Normal cdf as usual)
\[= 1 − 0.859 = 0.141.\]
%%%%%%%%%%%%%%%%%%%%%%%%%%%%%%
\item With p = 0.02, Y ~ Poisson(4), approximately.
\begin{eqnarray}
P(Y > 2) &=& 1 – P(Y \leq 2)\\
&=&  1 – 0.2381 from tables; or by use of 1−244142e−⎛⎞++⎜⎟⎝⎠\\
&=&  0.762 \mbox{approximately}.
\end{eqnarray}
    \item Now $p = 0.98$.
Consider 200 − Y ~ B(200, 0.02) ~ Poisson(4) approximately.
\[P(Y ≤ 197) = P(200 − Y ≥ 3)\]
\[≈ 1 – P(Poisson(4) ≤ 2)\] = 0.762 approximately (as in (c)).
\end{enumerate}
\end{document}
