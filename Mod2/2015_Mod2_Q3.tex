
3. Given a model involving a parameter , suppose that the likelihood obtained from a
set of data is given by L( ) but no simple expression can be found for the maximum
likelihood estimate
ˆ
 .
Describe the Newton-Raphson method for finding the value of
ˆ

numerically.
(4)
1 2 , , , X X Xn
is a random sample from a population with probability density
function
 
2 3
3 3
3
( ) exp for 0, x x f x x
 
  
where  > 0 is an unknown parameter.
(i) Show that
3 3 E X( )  .
(4)
(ii) Find

ˆ
,
the maximum likelihood estimator of .
(6)
(iii) Using an asymptotic result, find the approximate distribution of

ˆ
when n is
large.
(3)
(iv) Hence calculate an approximate 95% confidence interval for  when n = 200
and
3
1600. Xi 
(3)
4
