\documentclass[a4paper,12pt]{article}
%%%%%%%%%%%%%%%%%%%%%%%%%%%%%%%%%%%%%%%%%%%%%%%%%%%%%%%%%%%%%%%%%%%%%%%%%%%%%%%%%%%%%%%%%%%%%%%%%%%%%%%%%%%%%%%%%%%%%%%%%%%%%%%%%%%%%%%%%%%%%%%%%%%%%%%%%%%%%%%%%%%%%%%%%%%%%%%%%%%%%%%%%%%%%%%%%%%%%%%%%%%%%%%%%%%%%%%%%%%%%%%%%%%%%%%%%%%%%%%%%%%%%%%%%%%%
  \usepackage{eurosym}
\usepackage{vmargin}
\usepackage{amsmath}
\usepackage{graphics}
\usepackage{epsfig}
\usepackage{enumerate}
\usepackage{multicol}
\usepackage{subfigure}
\usepackage{fancyhdr}
\usepackage{listings}
\usepackage{framed}
\usepackage{graphicx}
\usepackage{amsmath}
\usepackage{chngpage}
%\usepackage{bigints}

\usepackage{vmargin}
% left top textwidth textheight headheight
% headsep footheight footskip
\setmargins{2.0cm}{2.5cm}{16 cm}{22cm}{0.5cm}{0cm}{1cm}{1cm}
\renewcommand{\baselinestretch}{1.3}

\setcounter{MaxMatrixCols}{10}

\section{Higher Certificate, Module 2, 2008. Question 4 }
\begin{enumerate}
\begin{framed}
Given a model involving a parameter , suppose that the likelihood obtained from a
set of data is given by L( ) but no simple expression can be found for the maximum
likelihood estimate
ˆ
 .
\end{framed}
%%%%%%%%%%%%%%%%%%%%%%%%%%%%%%%%%%
\item 
Describe the Newton-Raphson method for finding the value of
ˆ

numerically.
(4)
1 2 , , , X X Xn
is a random sample from a population with probability density
function
 
2 3
3 3
3
( ) exp for 0, x x f x x
 
  
where  > 0 is an unknown parameter.
(i) Show that
3 3 E X( )  .
(4)
(ii) Find

ˆ
,
the maximum likelihood estimator of .


\item Using an asymptotic result, find the approximate distribution of

ˆ
when n is
large.


\item Hence calculate an approximate 95% confidence interval for  when n = 200
and
3
1600. Xi 
(3)
\end{enumerate}
\end{document}
