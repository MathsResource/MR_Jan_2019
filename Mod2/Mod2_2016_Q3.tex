\documentclass[a4paper,12pt]{article}
%%%%%%%%%%%%%%%%%%%%%%%%%%%%%%%%%%%%%%%%%%%%%%%%%%%%%%%%%%%%%%%%%%%%%%%%%%%%%%%%%%%%%%%%%%%%%%%%%%%%%%%%%%%%%%%%%%%%%%%%%%%%%%%%%%%%%%%%%%%%%%%%%%%%%%%%%%%%%%%%%%%%%%%%%%%%%%%%%%%%%%%%%%%%%%%%%%%%%%%%%%%%%%%%%%%%%%%%%%%%%%%%%%%%%%%%%%%%%%%%%%%%%%%%%%%%
\usepackage{eurosym}
\usepackage{vmargin}
\usepackage{amsmath}
\usepackage{graphics}
\usepackage{epsfig}
\usepackage{enumerate}
\usepackage{multicol}
\usepackage{subfigure}
\usepackage{fancyhdr}
\usepackage{listings}
\usepackage{framed}
\usepackage{graphicx}
\usepackage{amsmath}
\usepackage{chngpage}
%\usepackage{bigints}

\usepackage{vmargin}
% left top textwidth textheight headheight
% headsep footheight footskip
\setmargins{2.0cm}{2.5cm}{16 cm}{22cm}{0.5cm}{0cm}{1cm}{1cm}
\renewcommand{\baselinestretch}{1.3}

\setcounter{MaxMatrixCols}{10}
\begin{document}

\begin{enumerate}
    \item Question 3 (a) Let L be the volume of the large bottle in litres.
Let S be the volume of the small bottle in litres.
\[L \sim N(1.5, 0.012)\]
\[S \sim N(0.5, 0.0082)\]
\item Take a random sample of 1 large bottle and 3 small bottles.
$P(L > 3S) = P(L -3S > 0)$
Let Y = L -3S.
\[E(L -S) = E(L) -3E(S) = 1:5 - (3 \times 0.5) = 0\]
[1]
\[Var(L-S) = Var(L)+9\times Var(S) = 0:012+9\times 0:0082 = 0:000676\]
6
\[sd(L -3S) = 0:026\]
NB The variance is not needed here, but some may not realise
that. It is not necessary to calculate, or even mention, the vari-
ance.
\[P(Y > 0) = P Z > \left(\frac{0-0}{0:026}\right)\]

\begin{eqnarray}
P(Y > 0) &=& P(Z > 0)\\
&=& 1 -0:5\\
&=& 0:5 \\
\end{eqnarray}

%%%%%%%%%%%%%%%%%%%%5
\item  Let T = 10L + 3S.


\begin{eqnarray*}
E(10L + 3S) &=& 10E(L) + 3E(S) \\
&=& 10 \times 1:5 + 3 \times  0:5 \\
&=& 16:5 \\
\end{eqnarray}


%%%%%%%%%%%%%%%%
\begin{eqnarray*}
Var(10L + 3S) &=& 102 Var(L) + 32 Var(S) \\
&=&  100 \times  0:012 + 9 \times 0:0082\\
&=&  0:01 + 0:000576\\
&=& 0:010576 \\
\end{eqnarray}

T is a linear combination of normal r.v.'s so T is also normal [1]
$T \sim N(16:5; 0:010576)$ [1]
(b) $n = 50$, $\mu = 30g$ and $\sigma = 5g$.
\item  As n is large the distribution of the sample mean tends to nor-
mality by the Central Limit Theorem. [1]
\[E( X) = \mu = 30g\][1]

\[Var( X) = \frac{\sigma^2}{n} = \frac{25}{50} = \frac{1}{2}\]

\[X \sim N(30; 0:5)\]
(ii)
\[P( X< 29) = P(Z < \frac{29 - 30}{\sqrt{2}}\]

p
0:5

[1]
\begin{eqnarray*}
P( X< 29) &=& P(Z < -p2)\\
&=&P(Z < -:41) [\\ 
&=& P(Z > 1:41)\\
&=& 1 -P(Z < 1:41)\\
&=& 1 -0:9207\\
&=& 0:0793 \\
\end{eqnarray}
%%%%%%%%%%%%%%%%%%%%%%%%%%%%%%%%5
\item  Sample of size n.
P
0
@Z <
29 �� 30
p5
n
1
A = 0:05 [1]
)
29 �� 30
p5
n
< ��1:645 [1]

p
n
` [1]
��
p

$\sqrt{n}> 8.225$
The minimum value of n is 68. [1]
\end{enumerate}
\end{document}
