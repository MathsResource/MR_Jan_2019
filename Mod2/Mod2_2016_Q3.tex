\documentclass{article}
\usepackage[utf8]{inputenc}
\usepackage{framed}
\usepackage{enumerate}
\title{RSS_Jan_2019_Mod2}
\author{kobriendublin }
\date{December 2018}

\begin{document}

\maketitle

\section{Introduction}

\begin{enumerate}
    \item Question 3 (a) Let L be the volume of the large bottle in litres.
Let S be the volume of the small bottle in litres.
L  N(1:5; 0:012)
S  N(0:5; 0:0082)
(i) Take a random sample of 1 large bottle and 3 small bottles.
P(L > 3S) = P(L �� 3S > 0) [1]
Let Y = L �� 3S.
E(L �� 3S) = E(L) �� 3E(S) = 1:5 �� 3  0:5 = 0
[1]
Var(L��3S) = Var(L)+9Var(S) = 0:012+90:0082 = 0:000676
6
sd(L �� 3S) = 0:026
NB The variance is not needed here, but some may not realise
that. It is not necessary to calculate, or even mention, the vari-
ance.
P(Y > 0) = P

Z >
0 �� 0
0:026

= P(Z > 0)
= 1 �� 0:5
= 0:5 [1]
(ii) Let T = 10L + 3S.
E(10L + 3S) = 10E(L) + 3E(S) [1]
= 10  1:5 + 3  0:5
= 16:5 [1]
Var(10L + 3S) = 102 Var(L) + 32 Var(S) [1]
= 100  0:012 + 9  0:0082
= 0:01 + 0:000576
= 0:010576 [1]
T is a linear combination of normal r.v.'s so T is also normal [1]
T  N(16:5; 0:010576) [1]
(b) n = 50,  = 30g and  = 5g.
(i) As n is large the distribution of the sample mean tends to nor-
mality by the Central Limit Theorem. [1]
E( X
) =  = 30g
[1]
Var( X
) =
2
n
=
25
50
=
1
2
[1]
X
 N(30; 0:5)
(ii)
P( X
< 29) = P

Z <
29 �� 30
p
0:5

[1]
= P(Z < ��
p
2)
= P(Z < ��1:41) [1]
= P(Z > 1:41)
= 1 �� P(Z < 1:41)
= 1 �� 0:9207
= 0:0793 [1]
(iii) Sample of size n.
P
0
@Z <
29 �� 30
p5
n
1
A = 0:05 [1]
)
29 �� 30
p5
n
< ��1:645 [1]
��1 < ��1:645 
5
p
n
` [1]
��
p
n < ��1:634  5
p
n > 8:225 [1]
The minimum value of n is 68. [1]