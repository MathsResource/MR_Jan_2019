Question 4
An easy method is to consider X as , where iXΣ are a set of n Bernoulli variables with Then E(1), (0)(1).iiPXpPXp====−[][], so iXpEXn ==.
Also E()()()22, so Var and Var.ii XpXppXnppnpq⎡⎤==−=−⎣⎦
ALTERNATIVELY: []()(01!1!!xnxnnxnxxxnnpqEXxpqxxnx−−==⎛⎞==⎜⎟−−⎝⎠ΣΣ
= ()()111111nnxxxnnppqnpx−−−−=−⎛⎞=⎜⎟−⎝⎠Σ.
Similarly, ()()[][]()2VarX=1+EXXEXEX−−⎡⎤⎣⎦, and we have
()()()02111nnxnxxnxxxnnEXXxxpqxxpqxx−−==⎛⎞⎛⎞−=−=−⎡⎤⎜⎟⎜⎟⎣⎦⎝⎠⎝⎠ΣΣ
()()()()222222112nnxxxnnnppqnnpx−−−−=−⎛⎞=−=−⎜⎟−⎝⎠Σ ,
and hence . ()()2222VarX1nnpnpnpnpnpnp=−+−=−=
[The probability generating function or moment generating function could also be used – this work is in Module 5.]
(i) (a) 160.750.01002260.0100 approx.≈=
(b) 1 − P(no one gets all 16 right), probability is {}1216110.75−−
= 1 − {0.9899774}12 = 0.1139.
Solution continued on next page
(ii) P(B − A > 0) can be studied using a Normal approximation to the difference B − A, i.e. []N(160.5)(160.75),(160.50.5)(160.750.25)×−×××+××, i.e. N(–4, 7).
The probability is then found using this approximation and a continuity correction (since B − A takes discrete values) as ()0.5PBA−>.
Hence it is found as
P(N(–4, 7) > 0.5) = 1 – P(N(–4, 7) < 0.5)
= ()()0.544.51111.70080.044577−−⎛⎞⎛⎞−Φ=−Φ=−Φ≈⎜⎟⎜⎟⎝⎠⎝⎠.