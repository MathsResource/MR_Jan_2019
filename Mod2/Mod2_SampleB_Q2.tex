Question 2
()N160,16H∼
(i) P(156 < H ≤ 164) = 15616016416044PZ−⎛ ⎞⎟ [where ]
= ()()()()(){}1PZ−<≤1=Φ1−Φ−1=Φ1−1−Φ1 by symmetry, i.e. ()()21Φ− = 0.6826. [Φ denotes the cdf of the standard Normal distribution as usual.]
()()()1682120.0228PHPZ>=>=−Φ=.
(ii) (a) The relevant portion of the Normal distribution of H is that beginning at 168, which corresponds to the value Z = 2. The median value m within this portion has (120.0228 probability above it, i.e. 0.0114, so = 1 − 0.0114 = 0.9886, corresponding to Z = 2.277. The corresponding value of H is ()mΦZμσ+ which is 160 + (4 × 2.277) = 169.1 cm.
(b) H = 170 corresponds to ()102.5 ; we have 2.50.99384Z==Φ=, so . Conditional on H > 168, the probability is ()12.50.0062−Φ=()()170168PHPH>> = 0.00620.0228 = 0.272.
(iii) Mean height of 25 members 21.352N169.5,25⎛⎞∼⎜⎟⎝⎠.
P(mean > 170) = ()()170169.50.55111.8491.352/51.352PZ⎛⎞−×⎛⎞>=−Φ=−Φ⎜⎟⎜⎟⎜⎟⎝⎠⎝⎠
= 1. 0.96770.0323−=


\end{document}