\documentclass{article}
\usepackage[utf8]{inputenc}
\usepackage{framed}
\usepackage{enumerate}
\title{RSS_Jan_2019_Mod2}
\author{kobriendublin }
\date{December 2018}

\begin{document}

\maketitle

\section{Introduction}
Higher Certificate, Module 2, 2010. Question 3
\begin{enumerate}
    \item
(),0,tTftetλλλ− =>>
(i) The cdf is ()001ttxxT Ftedxeeλλλ−−⎡⎤==−=−⎣⎦∫ for t > 0.
[Also for t ≤ 0.] ()0TFt=
0 tF(t)1.00.80.60.40.20.0CDF of T
[Note. The graph should of course be a smooth curve. It may not be shown as such, due to the limits of electronic reproduction.]
(ii) ()()()()()11baaTTPaTbFbFaeeee λλλ −−−<≤=−=−−−=− .
(iii) ()011PTeλ−<≤=−.
()()2121PTeeeeλλλλ−−−−<≤=−=−.
So we have (){}121eeeλλ−−−−=− which gives 2eλ= so λ = log 2 = 0.693.
Solution continued on next page
(iv) For t > c > 0, we have
()()()()()()andttccPTtTcPTtePTtTcePTcPTceλλλ−−−−>>>>>====>>.
Hence the conditional pdf of T given that T > c is ()()()1tctcdeedtλλλ−−−−−= (for t > c).
Arguing similarly for the random variable T – c, we first find, for t > 0,
()()()()()()andtctcPTctTcPTtcePTctTcePTcPTceλλλ−+−−−>>>+−>>====>>.
Hence the conditional pdf of T – c given that T > c is teλλ−, for t > 0.
Thus the conditional distribution of T – c given that T > c is the same as the unconditional distribution of T, for any constant c > 0.
