\documentclass[a4paper,12pt]{article}
%%%%%%%%%%%%%%%%%%%%%%%%%%%%%%%%%%%%%%%%%%%%%%%%%%%%%%%%%%%%%%%%%%%%%%%%%%%%%%%%%%%%%%%%%%%%%%%%%%%%%%%%%%%%%%%%%%%%%%%%%%%%%%%%%%%%%%%%%%%%%%%%%%%%%%%%%%%%%%%%%%%%%%%%%%%%%%%%%%%%%%%%%%%%%%%%%%%%%%%%%%%%%%%%%%%%%%%%%%%%%%%%%%%%%%%%%%%%%%%%%%%%%%%%%%%%
\usepackage{eurosym}
\usepackage{vmargin}
\usepackage{amsmath}
\usepackage{graphics}
\usepackage{epsfig}
\usepackage{enumerate}
\usepackage{multicol}
\usepackage{subfigure}
\usepackage{fancyhdr}
\usepackage{listings}
\usepackage{framed}
\usepackage{graphicx}
\usepackage{amsmath}
\usepackage{chngpage}
%\usepackage{bigints}

\usepackage{vmargin}
% left top textwidth textheight headheight
% headsep footheight footskip
\setmargins{2.0cm}{2.5cm}{16 cm}{22cm}{0.5cm}{0cm}{1cm}{1cm}
\renewcommand{\baselinestretch}{1.3}

\setcounter{MaxMatrixCols}{10}
\begin{document}

\maketitle

\section{Introduction}
Higher Certificate, Module 2, 2010. Question 3

\[(),0,tTftetλλλ− =>>\]
\begin{enumerate}


%%%%%%%%%%%%%%%%%%%%%%%%%%%%%%%%%%%%%
\item   The cdf is ()001ttxxT Ftedxeeλλλ−−⎡⎤==−=−⎣⎦∫ for t > 0.
[Also for t ≤ 0.] ()0TFt=
0 tF(t)1.00.80.60.40.20.0CDF of T
[Note. The graph should of course be a smooth curve. It may not be shown as such, due to the limits of electronic reproduction.]
%%%%%%%%%%%%%%%%%%%%%%%%%%%%%%%%%%%%%
\item   ()()()()()11baaTTPaTbFbFaeeee λλλ −−−<≤=−=−−−=− .
%%%%%%%%%%%%%%%%%%%%%%%%%%%%%%%%%%%%%
\item  ()011PTeλ−<≤=−.
()()2121PTeeeeλλλλ−−−−<≤=−=−.
So we have (){}121eeeλλ−−−−=− which gives 2eλ= so 
\[λ = log 2 = 0.693.\]
%%%%%%%%%%%%%%%%%%%%%%%%%%%%%%%%%%%%%
\item  For t > c > 0, we have
\[()()()()()()andttccPTtTcPTtePTtTcePTcPTceλλλ−−−−>>>>>====>>.\]
Hence the conditional pdf of T given that T > c is ()()()1tctcdeedtλλλ−−−−−= (for t > c).
\begin{itemize}
\item Arguing similarly for the random variable T – c, we first find, for t > 0,
()()()()()()andtctcPTctTcPTtcePTctTcePTcPTceλλλ−+−−−>>>+−>>====>>.
\item Hence the conditional pdf of $T – c$ given that T > c is teλλ−, for t > 0.
\item Thus the conditional distribution of T – c given that T > c is the same as the unconditional distribution of T, for any constant c > 0.
\end{itemize}

\end{document}
