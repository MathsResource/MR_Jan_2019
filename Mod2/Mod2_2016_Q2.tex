\documentclass[a4paper,12pt]{article}
%%%%%%%%%%%%%%%%%%%%%%%%%%%%%%%%%%%%%%%%%%%%%%%%%%%%%%%%%%%%%%%%%%%%%%%%%%%%%%%%%%%%%%%%%%%%%%%%%%%%%%%%%%%%%%%%%%%%%%%%%%%%%%%%%%%%%%%%%%%%%%%%%%%%%%%%%%%%%%%%%%%%%%%%%%%%%%%%%%%%%%%%%%%%%%%%%%%%%%%%%%%%%%%%%%%%%%%%%%%%%%%%%%%%%%%%%%%%%%%%%%%%%%%%%%%%
\usepackage{eurosym}
\usepackage{vmargin}
\usepackage{amsmath}
\usepackage{graphics}
\usepackage{epsfig}
\usepackage{enumerate}
\usepackage{multicol}
\usepackage{subfigure}
\usepackage{fancyhdr}
\usepackage{listings}
\usepackage{framed}
\usepackage{graphicx}
\usepackage{amsmath}
\usepackage{amssymb}
\usepackage{chngpage}
%\usepackage{bigints}

\usepackage{vmargin}
% left top textwidth textheight headheight
% headsep footheight footskip
\setmargins{2.0cm}{2.5cm}{16 cm}{22cm}{0.5cm}{0cm}{1cm}{1cm}
\renewcommand{\baselinestretch}{1.3}

\setcounter{MaxMatrixCols}{10}
\begin{document}


\section{Introduction}

\begin{enumerate}
    \item Question 2 (i) Let X = number of failures before the first success.
PX(x) = (1-p)xp [1]
for $x = \{0; 1; 2; 3; 4; : : :\}$ [1]
1X

\begin{eqnarray*}
\sum^{\infty}_{\sum^{\infty}_{x=0}}
(1-p)xp &=& p
1X
\sum^{\infty}_{\sum^{\infty}_{x=0}}
(1-p)x \\
&=& p
1
1 �� (1-p)
sum of GP \\
&=& p 
1
p \\
&=& 1 \\


\end{eqnarray}


\item 

\begin{eqnarray}
E(X) &=&
1X
\sum^{\infty}_{\sum^{\infty}_{x=0}}
xPX(x) \\
&=&
1X
\sum^{\infty}_{\sum^{\infty}_{x=0}}
x(1-p)xp \\ 

\end{eqnarray}
Using the fact that:
d
dp
1X

\begin{eqnarray}
\sum^{\infty}_{\sum^{\infty}_{x=0}}
(1-p)x = ��
1X
\sum^{\infty}_{\sum^{\infty}_{x=0}}
x(1-p)x��1\\
&=&
d
dp1p\\
&=&
1
p2 [1]
\end{eqnarray}
Hence
\begin{eqnarray}
E(X) &=& p(1-p)
1X
\sum^{\infty}_{\sum^{\infty}_{x=0}}
x(1-p)x��1
&=& p(1-p) 
1
p2
&=&
(1-p)
[1]
\end{eqnarray}

p
NB Other correct solutions would be given credit.

%%%%%%%%%%%%%%%%%%%%%%%%%
\item  (a) When p = 0:01,
P(X = 80) = 0:99790:01 [1]
= 0:00452 [1]
(b) EITHER

\begin{eqnarray}
P(X > 80) &=& 1-p(X \leq 80) [1]\\
&=& 1 ��
X80
\sum^{\infty}_{\sum^{\infty}_{x=0}}
x(1-p)xp [1]\\
&=& 1-p \left(\frac{1 -(1-p)^{80}}{1 -(1-p)}\right)\\
&=& 1 -:0.01( \left(\frac{1 -(1-p)^{80}}{1 -(1-p)}\right):\\
&=& 1 -0:5525\\
&=& 0:4475 [1]\\
\end{eqnarray}

OR As it will take more than 80 attempts if and only if the rst 80
are all failures [2]
(Either get 2 marks or 0)
\[P(X > 80) = 0:9980 \]
(either get 2 marks or 0)
= 0:4475: [1]

%%%%%%%%%%%%%%%%%%%%%%%%%%%%%%%%%%%%%%%%%%
\item  Let Y = total number of failures before the second success. The prob-
ability mass function is
PY (y) =
 
y + 1
1
!
(1-p)yp2 = (y + 1)(1-p)yp2 with $y = \{0; 1; 2; : : : \}$
The experiment to obtain Y can be considered as running the first
experiment until the first success and then repeating an independent
experiment to obtain a further success. [1]
Hence
\begin{eqnarray}
E(Y ) &=& E(X1) + E(X2) \\
&=& (1-p)
p
+
(1-p)
p \\
&=& 2
(1-p)
p
\end{eqnarray}
\end{enumerate}
\end{document}

