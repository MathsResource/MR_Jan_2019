5. The independent random variables X1, X2 and X3 each have the uniform distribution on
the interval 0 to 1. Let W denote the median of X1, X2 and X3, and V denote the
maximum of X1, X2 and X3.
(i) The joint probability density function of two order statistics, X(i) and X(j), of a
sample of n independent and identically distributed random variables, each
with probability density function f (x) and distribution function F (x), is given
by
1 1
() ( ) () ( ) ()
( ) () ( )
! (, ) () ( ) () ( 1)!( 1)!( )!
1 ( ) ( ) ( ).
[ ][ ]
[ ]
i j i
ij i j i
n j
j ij
n gx x Fx Fx Fx i ji n j
Fx f x f x
  

    
 
Use this expression to show that the joint probability density function of W and
V is
gwv w w v   , 6 , 0 1.   
(3)
(ii) Show that, for non-negative integers k and m,
  6
( 2)( 3)
k m EWV
k km     .
Hence or otherwise, find E(W), Var(W), E(V), Var(V) and Cov(W, V).
(12)
(iii) In a certain experiment, the same unknown quantity  is to be measured three
times. One researcher intends to estimate  by the median of the three
measurements while another researcher prefers to estimate  by the maximum
of the three measurements. Assuming that the measurements are 1
1 2   X – ,
1
2 2   X – and 1
3 2   X – , find the expected value and variance of the
difference between these two estimators.
(5)
