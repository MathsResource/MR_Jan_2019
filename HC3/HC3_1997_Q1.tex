III. STATISTICAL APPLICATIONS & PRACTICE
1. (i) Missing entries are (¤ indicates cannot be calculated):
Moving Average ¤, ¤, 108.250, 141.913, ¤, ¤.
Difference ¤, ¤, 14.90000, -7.4250, 0.7500, 13.8875, ¤, ¤.
(ii) see next sheet for graph.
(iii) Seasonal effects:
Quarter 1 2 3 4
¡7:4250 0:4375 14:9000 ¡5:7625
¡5:2375 0:7500 16:7125 ¡10:6250
¡7:3875 1:3000 8:0500 ¡3:0375
¡8:1750 4:9875 13:8875 ¡6:8000
MEAN ¡7:05625 1:86875 13:3875 ¡6:55625 : 1:64375
Correction ¡0:41094 ¡0:41094 ¡0:41094 ¡0:41094 (¼ 0)
SEASONAL ¡7:4672 1:4578 12:9766 ¡6:9672
(iv) Since the data are given to 1 decimal place, 7.5 should be added to each Q1
item, 7.0 to each Q4 item, 1.5 subtracted from each Q2 item and 13.0 from
each Q3 item, to “deseasonalise”.
(v)
1997 : Q1 Q2 Q3 Q4
50 + 6t : 176 182 188 194
Seasonalised: 168:5 183:5 201:0 187:0
This assumes same general trend and seasonal effects continue.
2. (i) The calculations of sums of squares and products are heavily influenced by
the two point (3:48; 6:05) and (3:49; 6:29). These lead to
P
(x ¡ ¯x)(y ¡ ¯y)
being negative, and hence the slope is negative.
(ii) Removing these two points gives completely different summary line for the
remainder. The new values of sums etc. are:
P
x = 100:04 ¡ 3:48 ¡ 3:49 =
93:07;
P
y = 117:12 ¡ 6:05 ¡ 6:29 = 104:78;
P
x2 = 436:9760 ¡ 3:482 ¡
3:492 = 412:6855;
P
xy = 508:0134¡(3:48£6:05)¡(3:49£6:29) = 465:0073.
P
(x ¡ ¯x)(y ¡ ¯y) = 465:0073 ¡ 1
21(104:78 £ 93:07) = 0:63232.
P
(x ¡ ¯x)2 =
412:6855 ¡ 93:072=21 = 0:20812. ˆb = 3:038.
(iii) A regression line has to go through the mean (¯x; ¯y) of all the data. If there
are two (or more) parts to the population or set of data, as here, then it
18
does not explain the data well. The sub-populations have to be separated.
In this case there are only two points that are well away from the rest. All
the remaining 21 have their x values (log surface temperature) between 4.2
and 4.6 approximately. For temperatures in this range therefore we can use
the regression line with slope +3:038 as a summary of the relationship. We
do not have enough information to propose relationships outside this range
of x-values (which correspond to y
0s between about 4.3 and 5.6).
