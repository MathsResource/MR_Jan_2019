3. A set of data may be fitted by a statistical model, e.g. a linear regression yi =
a+bxi+ei or an experimental design model such as yij = ¹+ti+bj +eij for
randomized complete blocks. The terms feig or feijg are generally assumed
N(0; ¾2), independently of one another. After the parameters a; b or ¹, ftig,
19
fbjg have been estimated the fitted values ˆyi = ˆa+ˆbxi or ˆyij = ˆ¹+ˆti+ˆbj can
be found. Then the differences yi ¡ ˆyi or yij ¡ ˆyij , observed minus expected
(fitted), are the residuals. These residuals should be from the same N(0; ¾2)
population. Their sizes should bear no systematic relationship to the sizes
of the corresponding ˆyi, ˆyij , or (for example) to xi if x represents time in a
set of time-series data or if x is any variable on a quantitative scale such as
a level of fertilized application. Clearly they should cluster around 0 and be
symmetrical. We may examine several of these properties in diagrams. A
useful one is to plot the residuals ei against corresponding fitted values yi.
If the wrong model has been fitted, e.g. a linear regression which should be
a curve, the residuals will show a regular patten, e.g.,
If the values of ¾2 is not constant, but increases as y increases, a ‘fan’ shape
may appear.
A skew distribution, rather than normal, will have all the largest residuals
on the same side of 0:
20
There may be outliers in the data, which will show up as isolated large values
(positive or negative) of ei:
However, this does not always happen (cf. Qu 2 where the two “odd” points
can be fitted quite well by the first line with negative slope). Normal probability
plotting can also be used to check the assumption of normality. The
residuals, ordered by size, are plotted against the expected values of normal
order statistics. Noticeable non-linearity is a warning that the assumption
may be valid.
4. 2 £ 2 factorial experiment in 5 replicates, completely randomized.
TOTALS.
Time: H L
1210 73:88 68:93 : 142:81
1240 71:25 71:03 : 142:28
145:13 139:96 285:09
(i) Correction term G2=N = 285:092=20 = 4063:8154.
S.S. for Times = 1
10(145:132 + 139:962) ¡ G2
N = 1:33645
S.S. for Temperatures = 1
10(142:812 + 142:282) ¡ G2
N = 0:01405
S.S. for all “treatments” = 1
5(73:882+68:932+71:252+71:032)¡G2
N = 2:46914.
Corrected total S.S. = 4067:00 ¡ 4063:8154 = 3:1846.
Analysis of Variance.
D:F: S:S: M:S:
Temperatures 1 0:01405 0:0141
Times 1 1:33645 1:3365
Interaction 1 1:11864 1:1186 F(1;16) = 25:03¤¤¤
3 2:46914
Residual 16 0:71546 0:0447 = s2:
TOTAL 19 3:18460
Since there is a very strong interaction of time with temperature, main effects
21
should not be quoted.
(ii) Means are:
Time: H L
Temperature 1210 14:78 14:79
1240 14:25 14:21
A graph shows the results clearly:
The standard error of a single mean is
p
s2=5 = 0:095. Hence at 12400 C,
time has no effect, while at 12100 C time H gives a thicker layer.
(iii) Report should make the point that neither time nor temperature alone
determines the thickness of the layer; also for a thicker layer we should
use the lower temperature and longer time, while the lower temperature
and shorter time gives a relatively thin layer. At the higher temperature,
with either time, the thickness of the layer is between these other two, and
apparently not affected by time.
