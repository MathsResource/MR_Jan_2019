\documentclass[a4paper,12pt]{article}
%%%%%%%%%%%%%%%%%%%%%%%%%%%%%%%%%%%%%%%%%%%%%%%%%%%%%%%%%%%%%%%%%%%%%%%%%%%%%%%%%%%%%%%%%%%%%%%%%%%%%%%%%%%%%%%%%%%%%%%%%%%%%%%%%%%%%%%%%%%%%%%%%%%%%%%%%%%%%%%%%%%%%%%%%%%%%%%%%%%%%%%%%%%%%%%%%%%%%%%%%%%%%%%%%%%%%%%%%%%%%%%%%%%%%%%%%%%%%%%%%%%%%%%%%%%%
\usepackage{eurosym}
\usepackage{vmargin}
\usepackage{amsmath}
\usepackage{graphics}
\usepackage{epsfig}
\usepackage{enumerate}
\usepackage{multicol}
\usepackage{subfigure}
\usepackage{fancyhdr}
\usepackage{listings}
\usepackage{framed}
\usepackage{graphicx}
\usepackage{amsmath}
\usepackage{chngpage}
%\usepackage{bigints}

\usepackage{vmargin}
% left top textwidth textheight headheight
% headsep footheight footskip
\setmargins{2.0cm}{2.5cm}{16 cm}{22cm}{0.5cm}{0cm}{1cm}{1cm}
\renewcommand{\baselinestretch}{1.3}

\setcounter{MaxMatrixCols}{10}
\begin{document}
\begin{enumerate}
\item We need to find out whether there are systematic trends along the rows, and
/ or whether one row is likely to do better than the other.
\begin{itemize}
\item We also want to know whether all the grow-bags came from the same source,
contain the same compost mixture, are the same size, have equally good
drainage, the same thickness of wall so that temperature is likely to be the
same.
\item Reasons for blocking would be: difference between rows, trend along rows,
different sorts of bag.
\end{itemize}
%%%%%%%%%%%%%%%%%%%%%%%%%%%%%%%%%%
\item  The experimental unit is a bag of 4 plants. We would analyse the total (or
mean) yield of plants per bag. If any plants died, we would need to adjust
for this, so it should be recorded.
22
\item  If there is no known or suspected systematic variation revealed in the answers
to (i), a completely randomized design may be used, with a fully random
choice of 16 bags for each of the four nutrient solutions. 
\begin{itemize}
\item This could be
achieved by using a random number table, reading digits in pairs, discarding
pairs 00, 65 - 99, taking the first 16 positions for treatment A, the next 16
for B, the next 16 for C and the others for D, 01 - 64 represent the two rows
with 32 bags in each.
\item If the answers to (i) indicate likely differences in the positions, make up 16
blocks each of which is as homogeneous as possible. Number the bags 1,2,3,4
in each block and permute these numbers at random to determine the order
in which the 4 nutrients will be allocated to bags.
\end{itemize}
\item For the completely randomized design, the analysis is:
Source of Variation D.F.
Nutrients 3
Residual 60
TOTAL 63
Using blocks of 4 in a randomized complete block design gives:
Source of Variation D.F.
Blocks 15
Nutrients 3
Residual 45
TOTAL 63

\end{enumerate}
\end{document}
