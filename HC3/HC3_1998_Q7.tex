\documentclass[a4paper,12pt]{article}
%%%%%%%%%%%%%%%%%%%%%%%%%%%%%%%%%%%%%%%%%%%%%%%%%%%%%%%%%%%%%%%%%%%%%%%%%%%%%%%%%%%%%%%%%%%%%%%%%%%%%%%%%%%%%%%%%%%%%%%%%%%%%%%%%%%%%%%%%%%%%%%%%%%%%%%%%%%%%%%%%%%%%%%%%%%%%%%%%%%%%%%%%%%%%%%%%%%%%%%%%%%%%%%%%%%%%%%%%%%%%%%%%%%%%%%%%%%%%%%%%%%%%%%%%%%%
\usepackage{eurosym}
\usepackage{vmargin}
\usepackage{amsmath}
\usepackage{graphics}
\usepackage{epsfig}
\usepackage{enumerate}
\usepackage{multicol}
\usepackage{subfigure}
\usepackage{fancyhdr}
\usepackage{listings}
\usepackage{framed}
\usepackage{graphicx}
\usepackage{amsmath}
\usepackage{chngpage}
%\usepackage{bigints}

\usepackage{vmargin}
% left top textwidth textheight headheight
% headsep footheight footskip
\setmargins{2.0cm}{2.5cm}{16 cm}{22cm}{0.5cm}{0cm}{1cm}{1cm}
\renewcommand{\baselinestretch}{1.3}

\setcounter{MaxMatrixCols}{10}
\begin{document}7.(i)P(T ¸ t0) =
R1
t0
¸e¡¸tdt = [¡e¡¸t]1t
0 = e¡¸t0 ,
(ii)For 12 observations, L =
Q12
i=1
¸e¡¸ti , so the log likelihood is (lnL) = n ln ¸ ¡ ¸
P12
i=1
ti =
12 ln ¸ ¡ 6028¸.
d
dt (lnL) = 12
¸ ¡ 6028 = 0 when ˆ¸ = 12
6028 = 0:001991.
(iii) d2
d¸2 (lnL) = ¡12
¸2 , and V ar(ˆ¸) ¼ ¡1=(¡12
¸2 ) = ¸2=12.
(iv)The likelihood function is now
L = ¸12e¡6028¸ ¢ e¡¸(641+234+87) (using (i))
= ¸12e¡6990¸
.
The same analysis now gives ˆ¸ = 12
6990 = 0:001717.
\end{enumerate}
\end{document}
