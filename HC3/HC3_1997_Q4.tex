\documentclass[a4paper,12pt]{article}
%%%%%%%%%%%%%%%%%%%%%%%%%%%%%%%%%%%%%%%%%%%%%%%%%%%%%%%%%%%%%%%%%%%%%%%%%%%%%%%%%%%%%%%%%%%%%%%%%%%%%%%%%%%%%%%%%%%%%%%%%%%%%%%%%%%%%%%%%%%%%%%%%%%%%%%%%%%%%%%%%%%%%%%%%%%%%%%%%%%%%%%%%%%%%%%%%%%%%%%%%%%%%%%%%%%%%%%%%%%%%%%%%%%%%%%%%%%%%%%%%%%%%%%%%%%%
\usepackage{eurosym}
\usepackage{vmargin}
\usepackage{amsmath}
\usepackage{graphics}
\usepackage{epsfig}
\usepackage{enumerate}
\usepackage{multicol}
\usepackage{subfigure}
\usepackage{fancyhdr}
\usepackage{listings}
\usepackage{framed}
\usepackage{graphicx}
\usepackage{amsmath}
\usepackage{chngpage}
%\usepackage{bigints}

\usepackage{vmargin}
% left top textwidth textheight headheight
% headsep footheight footskip
\setmargins{2.0cm}{2.5cm}{16 cm}{22cm}{0.5cm}{0cm}{1cm}{1cm}
\renewcommand{\baselinestretch}{1.3}

\setcounter{MaxMatrixCols}{10}
\begin{document}
4. 2 £ 2 factorial experiment in 5 replicates, completely randomized.
TOTALS.
Time: H L
1210 73:88 68:93 : 142:81
1240 71:25 71:03 : 142:28
145:13 139:96 285:09
\begin{enumerate}
\item  Correction term G2=N = 285:092=20 = 4063:8154.
S.S. for Times = 1
10(145:132 + 139:962) ¡ G2
N = 1:33645
S.S. for Temperatures = 1
10(142:812 + 142:282) ¡ G2
N = 0:01405
S.S. for all “treatments” = 1
5(73:882+68:932+71:252+71:032)¡G2
N = 2:46914.
Corrected total S.S. = 4067:00 ¡ 4063:8154 = 3:1846.
Analysis of Variance.
D:F: S:S: M:S:
Temperatures 1 0:01405 0:0141
Times 1 1:33645 1:3365
Interaction 1 1:11864 1:1186 F(1;16) = 25:03¤¤¤
3 2:46914
Residual 16 0:71546 0:0447 = s2:
TOTAL 19 3:18460
Since there is a very strong interaction of time with temperature, main effects
should not be quoted.
\item  Means are:
Time: H L
Temperature 1210 14:78 14:79
1240 14:25 14:21
\begin{itemize}
    \item A graph shows the results clearly:
\item The standard error of a single mean is
p
s2=5 = 0:095. \item Hence at 12400 C,
time has no effect, while at 12100 C time H gives a thicker layer.
\end{itemize}

\item  Report should make the point that neither time nor temperature alone
determines the thickness of the layer; also for a thicker layer we should
use the lower temperature and longer time, while the lower temperature
and shorter time gives a relatively thin layer. At the higher temperature,
with either time, the thickness of the layer is between these other two, and
apparently not affected by time.
\end{enumerate}
\end{document}