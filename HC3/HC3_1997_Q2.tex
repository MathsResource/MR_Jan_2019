\documentclass[a4paper,12pt]{article}
%%%%%%%%%%%%%%%%%%%%%%%%%%%%%%%%%%%%%%%%%%%%%%%%%%%%%%%%%%%%%%%%%%%%%%%%%%%%%%%%%%%%%%%%%%%%%%%%%%%%%%%%%%%%%%%%%%%%%%%%%%%%%%%%%%%%%%%%%%%%%%%%%%%%%%%%%%%%%%%%%%%%%%%%%%%%%%%%%%%%%%%%%%%%%%%%%%%%%%%%%%%%%%%%%%%%%%%%%%%%%%%%%%%%%%%%%%%%%%%%%%%%%%%%%%%%
\usepackage{eurosym}
\usepackage{vmargin}
\usepackage{amsmath}
\usepackage{graphics}
\usepackage{epsfig}
\usepackage{enumerate}
\usepackage{multicol}
\usepackage{subfigure}
\usepackage{fancyhdr}
\usepackage{listings}
\usepackage{framed}
\usepackage{graphicx}
\usepackage{amsmath}
\usepackage{chngpage}
%\usepackage{bigints}

\usepackage{vmargin}
% left top textwidth textheight headheight
% headsep footheight footskip
\setmargins{2.0cm}{2.5cm}{16 cm}{22cm}{0.5cm}{0cm}{1cm}{1cm}
\renewcommand{\baselinestretch}{1.3}

\setcounter{MaxMatrixCols}{10}
\begin{document}
\begin{enumerate}
    \item The calculations of sums of squares and products are heavily influenced by
the two point (3:48; 6:05) and (3:49; 6:29). These lead to
P
(x ¡ ¯x)(y ¡ ¯y)
being negative, and hence the slope is negative.
%%%%%%%%%%%%%%%%%%%%%%%%%%%
\item Removing these two points gives completely different summary line for the
remainder. The new values of sums etc. are:
P
x = 100:04 ¡ 3:48 ¡ 3:49 =
93:07;
P
y = 117:12 ¡ 6:05 ¡ 6:29 = 104:78;
P
x2 = 436:9760 ¡ 3:482 ¡
3:492 = 412:6855;
P
xy = 508:0134¡(3:48£6:05)¡(3:49£6:29) = 465:0073.
P
(x ¡ ¯x)(y ¡ ¯y) = 465:0073 ¡ 1
21(104:78 £ 93:07) = 0:63232.
P
(x ¡ ¯x)2 =
412:6855 ¡ 93:072=21 = 0:20812. ˆb = 3:038.
\item A regression line has to go through the mean (¯x; ¯y) of all the data. If there
are two (or more) parts to the population or set of data, as here, then it
18
does not explain the data well. 
\begin{itemize}
    \item The sub-populations have to be separated.
\item In this case there are only two points that are well away from the rest. All
the remaining 21 have their x values (log surface temperature) between 4.2
and 4.6 approximately.
\item For temperatures in this range therefore we can use
the regression line with slope +3:038 as a summary of the relationship. 
\item We
do not have enough information to propose relationships outside this range
of x-values (which correspond to y
0s between about 4.3 and 5.6).
\end{itemize}

\end{enumerate}
\end{document}