\documentclass[a4paper,12pt]{article}
%%%%%%%%%%%%%%%%%%%%%%%%%%%%%%%%%%%%%%%%%%%%%%%%%%%%%%%%%%%%%%%%%%%%%%%%%%%%%%%%%%%%%%%%%%%%%%%%%%%%%%%%%%%%%%%%%%%%%%%%%%%%%%%%%%%%%%%%%%%%%%%%%%%%%%%%%%%%%%%%%%%%%%%%%%%%%%%%%%%%%%%%%%%%%%%%%%%%%%%%%%%%%%%%%%%%%%%%%%%%%%%%%%%%%%%%%%%%%%%%%%%%%%%%%%%%
\usepackage{eurosym}
\usepackage{vmargin}
\usepackage{amsmath}
\usepackage{graphics}
\usepackage{epsfig}
\usepackage{enumerate}
\usepackage{multicol}
\usepackage{subfigure}
\usepackage{fancyhdr}
\usepackage{listings}
\usepackage{framed}
\usepackage{graphicx}
\usepackage{amsmath}
\usepackage{chngpage}
%\usepackage{bigints}

\usepackage{vmargin}
% left top textwidth textheight headheight
% headsep footheight footskip
\setmargins{2.0cm}{2.5cm}{16 cm}{22cm}{0.5cm}{0cm}{1cm}{1cm}
\renewcommand{\baselinestretch}{1.3}

\setcounter{MaxMatrixCols}{10}

\begin{document}
\begin{table}[ht!]
 \centering
 \begin{tabular}{|p{15cm}|}
 \hline  
In a study of whether two forms of iron (Fe2+ and Fe3+) are retained in different amounts within the body, 18 mice were randomly selected from a group of 36 to receive Fe2+ at a 0.3 millimolar concentration as an addition to their food and the remaining 18 mice received Fe3+ at the same concentration as an addition to their food.  The percentage of each type of iron retained was measured for each mouse. The data are listed in the table below. 
 
Fe3+ Fe2+ 2.25 4.71 3.93 5.43 5.08 6.38 5.82 6.74 5.84 8.32 6.89 9.04 8.50 9.56 8.56 10.01 9.44 10.08 10.52 10.62 13.46 13.80 13.57 14.35 14.76 14.90 18.41 15.25 26.96 17.32 27.56 19.87 32.82 31.60 39.13 37.25 
 
On a single diagram, draw boxplots for the two samples.  Using this diagram, comment on the following
 


\\ \hline
  \end{tabular}
\end{table}

\begin{table}[ht!]
 \centering
 \begin{tabular}{|p{15cm}|}
 \hline  
 (i) whether there appears to be a difference in the general level of retention of the two forms of iron 
 \\ \hline
  \end{tabular}
\end{table}


\begin{table}[ht!]
 \centering
 \begin{tabular}{|p{15cm}|}
 \hline  
 
 (ii) whether the distributions of values appear symmetrical. 
(10) 
 
The investigators decided to perform a nonparametric test rather than the standard t test for comparison of means.  Explain why this is an appropriate choice. (3) 
 
Carry out the appropriate test and state clearly any conclusions you reach. 
(7) \\ \hline
  \end{tabular}
\end{table}

5 F3+
e :min=2.25, lower quartile=5.84, median=9.98, upper quartile=18.41, max=39.13.
F2+
e :min=4.71, lower quartile=8.32, median=10.35, upper quartile=15.25, max=37.25.
(i) There does not seen to be a large difference in retention as measure by the middle parts
of the data sets (between quartiles).
(ii) The distributions do not appear at all symmetrical, and F2+
e has two suspect outliers, well
above the other 16 figures. Because of the lack of symmetry and the extreme values in the
14
data, a mann-whitney test will be more satisfactory than a t-test; the two distributions
seem roughly similar shapes. labelling F3+
e A and F2+
e B, the joint ranking of the 36
mice is in the order:
AABABAABBABAABABBBABAABBABBBABAABABA
U = number of times an A precedes a B
= 18 + 18 + 17 + 16 + 16 + 14 + 13 + 13 + 12 + 9 + 8 + 8 + 6 + 3 + 2 + 2 + 1 + 0 = 176
The rank-sum statistic T = U+1
2£18£19 = 176+171 = 347, and for these sample sizes is
approximately normal with mean 1
2 £18£37 = 333 and variance 1
12 £18£18£37 = 999
Therefore Z = 347¡333
sqrt999 » N(0; 1) approximately, Z = 14
31:4 = 0:44 This provides no
evidence against a Null Hypothesis of the same median values of iron retention.The data
therefore provide no evidence of difference.
CANDIDATE’S NUMBER
Worksheet for fertilizer deliveries,for use with Question 4:
15
Deliveries M.AV. Y-M.AV. Seasonal Predicted Residual
1990 2 49.5 * * -14.94 * *
3 94.2 * * -0.3475 * *
4 62.6 69.6750 -7.0750 -2.1675 67.5075 -4.9075
1991 1 72.7 66.3875 6.3125 17.455 83.8425 -11.1425
2 48.9 60.7250 -11.8250 -14.94 45.7850 3.1150
3 68.5 58.8875 9.6125 -0.3475 58.5400 9.9600
4 43.0 58.4125 -15.4125 -2.1675 56.2450 -13.2450
1992 1 77.6 55.6750 21.9250 17.455 73.1300 4.4700
2 40.2 53.400 -13.2000 -14.94 38.4600 1.7400
3 55.3 53.8250 1.4750 -0.3475 53.4775 1.8225
4 38.0 54.6125 -16.6125 -2.1675 52.4450 -14.4450
1993 1 86.0 55.1125 30.8875 17.455 72.5675 13.4325
2 38.1 64.3875 -26.2875 -14.94 49.4475 -11.3475
3 61.4 72.2125 -10.8125 -0.3475 71.8650 -10.4660
4 106.1 72.9750 33.1250 -2.1675 70.8075 35.2925
1994 1 80.5 73.0500 7.4500 17.455 90.5050 -10.0050
2 49.7 63.6625 -13.9625 -14.94 48.7225 0.9775
3 50.4 54.3750 -3.9750 -0.3475 54.0275 -3.6275
4 42.0 52.1375 -10.1375 -2.1675 49.9700 -7.9700
1995 1 70.3 51.6375 18.6625 17.455 69.0925 1.2075
2 42.0 53.4625 -11.4625 -14.94 38.5225 3.4775
16
Calculation of seasonal effects
Q1 Q2 Q3 Q4
6:3125 ¡11:8250 9:6125 ¡7:0750
21:9250 ¡13:2000 1:4750 ¡15:4125
30:8875 ¡26:2875 ¡10:8125 ¡16:6125
7:4500 ¡13:9625 ¡3:9750 33:1250
18:6625 ¡11:4625 ¡0:0750 ¡10:1375
¤ ¤ 0:6625
17:0475 ¡15:3475 ¡0:755 ¡2:575 ¡1:63
17:455 ¡14:94 ¡0:3475 ¡2:1675

different from one another.
19
20
\end{enumerate}

\end{document}
