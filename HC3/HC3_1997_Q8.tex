\documentclass[a4paper,12pt]{article}
%%%%%%%%%%%%%%%%%%%%%%%%%%%%%%%%%%%%%%%%%%%%%%%%%%%%%%%%%%%%%%%%%%%%%%%%%%%%%%%%%%%%%%%%%%%%%%%%%%%%%%%%%%%%%%%%%%%%%%%%%%%%%%%%%%%%%%%%%%%%%%%%%%%%%%%%%%%%%%%%%%%%%%%%%%%%%%%%%%%%%%%%%%%%%%%%%%%%%%%%%%%%%%%%%%%%%%%%%%%%%%%%%%%%%%%%%%%%%%%%%%%%%%%%%%%%
\usepackage{eurosym}
\usepackage{vmargin}
\usepackage{amsmath}
\usepackage{graphics}
\usepackage{epsfig}
\usepackage{enumerate}
\usepackage{multicol}
\usepackage{subfigure}
\usepackage{fancyhdr}
\usepackage{listings}
\usepackage{framed}
\usepackage{graphicx}
\usepackage{amsmath}
\usepackage{chngpage}
%\usepackage{bigints}

\usepackage{vmargin}
% left top textwidth textheight headheight
% headsep footheight footskip
\setmargins{2.0cm}{2.5cm}{16 cm}{22cm}{0.5cm}{0cm}{1cm}{1cm}
\renewcommand{\baselinestretch}{1.3}

\setcounter{MaxMatrixCols}{10}
\begin{document}
\begin{table}[ht!]
 \centering
 \begin{tabular}{|p{15cm}|}
 \hline  
In a pilot study of anaesthesia in cardiac surgery twenty two patients were randomly assigned to three treatment groups. The table shows the red cell folate levels in the three groups after 24 hours of treatment.
Group A (n = 8)
Group B ( n = 9 )
Group C ( n = 5 ) 243 206 241 251 210 258 275 226 270 291 249 923 347 255 328 354 273 380 285 392 295 309 Sum 2533 2308 2020 Sum of squares 826145 602898 1157058
(i) Make an informative plot of the data. One of the data points seems to be unusual compared with the rest. State which one it is and why you regard it as unusual.


\\ \hline
  \end{tabular}
\end{table}

\begin{table}[ht!]
 \centering
 \begin{tabular}{|p{15cm}|}
 \hline  
(ii) A one-way analysis of all the data performed by a computer package is summarised in the analysis of variance table below. Some values have been left out of the table. Complete the table and state the conclusions that you feel are justified by this analysis.
Source of variation Sum of squares
Degrees of freedom
Mean square
Variance ratio Between treatments * 2 * * Residual * 19 19797 Total 446405 21\\ \hline
  \end{tabular}
\end{table}
\begin{table}[ht!]
 \centering
 \begin{tabular}{|p{15cm}|}
 \hline  
(iii) After some discussion with the surgical team who conducted the study you decide to leave out the observation which is unusual because it is not possible to determine whether it is a genuine observation or a mistake in recording the data. Re-analyse the data and draw up an analysis of variance table similar to the one above. State the conclusions which are justified by your second analysis.
(iv) How would you explain any difference between your conclusions from the two analyses?
\\ \hline
  \end{tabular}
\end{table}

%%%%%%%%%%%%%%%%%%%%%%%%%%%%%%%%%%%%%%%%%%%%%%%%%%%%%%%
\begin{itemize}
\item  A dot - plot for each set of data, on the same scale, is useful.
\item 923 for C seems highly unlikely. This level may be physically impossible, to
judge from all the other observations. Or it may be a recording error for 293
(or even 329).
\item  Residual S.S. = 19 £ 19797 = 376143; hence treatment S.S. = 70262 and
M.S. = 35131. The variance ratio is then 35131
19797 = 1:77.
The analysis, by itself, suggests that there are no significant treatment differences,
and also that the standard deviation of an observation is very large
(
p
19797 = 140:7).
\item Revised sums and S.S. are:
A B C TOTAL
Sum 2533 2308 1097 5938
n 8 9 4 21
Sum of squares 826145 602898 305129 1734172
Treatments SS = 25332
8 + 23082
9 + 10972
4 ¡ 59382
21 = 1694737¡1679040 = 15697.
25
and Total SS = 1734172 ¡ 59382=21 = 55132.
Source of variation DF Sum of Squares M:S:
Treatments 2 15697 7849 F(2;18) = 3:58¤
Residual 18 39435 2191
TOTAL 20 55132
F is just significant at 5%. The estimated variance of an observation is 2191,
S.D. = 46.8. Means are: A, 316.6; B, 256.4; C, 274.3. Var[¯xA ¡ ¯xB] =
s2( 1
8 + 1
9 ) = 517:32, S.D. = 22.7, t(18) = 60:2
22:7 = 2:65¤, so A and B appear
to differ. Var[¯xA ¡ ¯xC] = s2( 1
8 + 1
4 ) = 821:63, S.D. = 28.7, t(18) = 42:3
28:7 , not
significant. A and C do not differ; nor will B and C.
\item The one doubtful observation greatly increased the variance estimate.
\end{document}
