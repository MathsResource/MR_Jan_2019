\documentclass[a4paper,12pt]{article}
%%%%%%%%%%%%%%%%%%%%%%%%%%%%%%%%%%%%%%%%%%%%%%%%%%%%%%%%%%%%%%%%%%%%%%%%%%%%%%%%%%%%%%%%%%%%%%%%%%%%%%%%%%%%%%%%%%%%%%%%%%%%%%%%%%%%%%%%%%%%%%%%%%%%%%%%%%%%%%%%%%%%%%%%%%%%%%%%%%%%%%%%%%%%%%%%%%%%%%%%%%%%%%%%%%%%%%%%%%%%%%%%%%%%%%%%%%%%%%%%%%%%%%%%%%%%
\usepackage{eurosym}
\usepackage{vmargin}
\usepackage{amsmath}
\usepackage{graphics}
\usepackage{epsfig}
\usepackage{enumerate}
\usepackage{multicol}
\usepackage{subfigure}
\usepackage{fancyhdr}
\usepackage{listings}
\usepackage{framed}
\usepackage{graphicx}
\usepackage{amsmath}
\usepackage{chngpage}
%\usepackage{bigints}

\usepackage{vmargin}
% left top textwidth textheight headheight
% headsep footheight footskip
\setmargins{2.0cm}{2.5cm}{16 cm}{22cm}{0.5cm}{0cm}{1cm}{1cm}
\renewcommand{\baselinestretch}{1.3}

\setcounter{MaxMatrixCols}{10}
\begin{document}
\begin{enumerate}
\item
16
(ii) P
y = 424:8 sxy = 17005:66 ¡ (424:8 £ P 393:9)=10 = 272:788
x = 393:9 sxx = 15840:23 ¡ 393:92=10 = 324:509
Hence the slope
ˆb
=
272:788
324:509
= 0:8406
The fitted line is y¡¯y =ˆb(x¡¯x) or y¡42:48 = 0:8406(x¡39:39) i.e. y = 0:8406x+9:3681
or y = 9:37 + 0:84x
\item The proportion of total variation (sum of square), that is explained by the relation
of observation to formula is to 0.843.[(229:31=272:18) ¼ 0:8425] this is reasonably
good.
\item The calculation is based on the theoretical model y = ®+¯x+",where var(") =
¾2 is estimated by s2 = (2:315)2 = 5:36 It has 8 d.f. var(ˆb) = ¾2=sxx estimated as
5:36
324:509 = 0:016517, so SˆE = 0:1285
17
A 95% interval for ¯ is :
ˆb
§ 2:306 £ 0:1285 = 0:8406 § 0:2963 i:e:(0:544 to 1:137)
[t(8;5%) = 2:306]. A 95% interval for ® is :
ˆa § 2:306
s
s2(
1
10
+
¯x2
sxx
) = 9:368 § 2:306 £
p
5:36 £ 4:8813 = 9:368 § 11:795
This give(-2.427 to +21.16)
Note that this is very imprecisely determined(and is a little use since there are no data
near to zero to confirm whether a linear relation still holds)
\end{enumerate}
\end{document}
