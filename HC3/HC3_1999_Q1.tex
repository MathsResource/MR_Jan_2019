\documentclass[a4paper,12pt]{article}
%%%%%%%%%%%%%%%%%%%%%%%%%%%%%%%%%%%%%%%%%%%%%%%%%%%%%%%%%%%%%%%%%%%%%%%%%%%%%%%%%%%%%%%%%%%%%%%%%%%%%%%%%%%%%%%%%%%%%%%%%%%%%%%%%%%%%%%%%%%%%%%%%%%%%%%%%%%%%%%%%%%%%%%%%%%%%%%%%%%%%%%%%%%%%%%%%%%%%%%%%%%%%%%%%%%%%%%%%%%%%%%%%%%%%%%%%%%%%%%%%%%%%%%%%%%%
\usepackage{eurosym}
\usepackage{vmargin}
\usepackage{amsmath}
\usepackage{graphics}
\usepackage{epsfig}
\usepackage{enumerate}
\usepackage{multicol}
\usepackage{subfigure}
\usepackage{fancyhdr}
\usepackage{listings}
\usepackage{framed}
\usepackage{graphicx}
\usepackage{amsmath}
\usepackage{chngpage}
%\usepackage{bigints}

\usepackage{vmargin}
% left top textwidth textheight headheight
% headsep footheight footskip
\setmargins{2.0cm}{2.5cm}{16 cm}{22cm}{0.5cm}{0cm}{1cm}{1cm}
\renewcommand{\baselinestretch}{1.3}

\setcounter{MaxMatrixCols}{10}

\begin{document}
\section{Introduction}
\begin{enumerate}
    \item 
    
    
    Paper III
Statistical Applications and Practice
1 PH totals are 56.4; 70.6; 49.7; 31.4; G=208.1. Between PH’s ss = 1
5(56:42 +70:662 +49:72 +
34:12) ¡ 1
20208:12 = 158:99
12
Analysis of Variance
SOURCE OF DEGREES SUM OF MEAN
V ARIATION OF FREEDOM SQUARES SQUARE
Between PH0s 3 158:99 52:997 F(3;16) = 43:30
Residual 16 19:58 1:224
Total 19 178:57
The analysis shows that the bulk of the variability in the data is due to the difference
between the activities at the different PH values. Each observation is assumed to be normally,
distributed about its mean, with the same variance for all. It is assumed that the model:
observation = mean for given PH + random residual yij = mi + eij i = 1; 2; 3; 4; explain
the data.
The filled live passes through (5,11.4) and (9,7.6), as shown. Although the analysis shows a significant
linear component in the regression, the plot clearly indicates the need for a curve. (The
deviations from-linearity s.s.is 158.99-91.97=67.02, which is also significantly large). There is
no point in studying PH9, nor PH7 because the maximum does not seen to be near there.
The maximum will be in the region of 5, probably on the lower side, so values such as 4.25
by steps of 0.25 to 5.25 (omitting 5), or 4.5 by steps of 0.25 to 5.25 including 5.0 in the same
experiment, would be appropriate.

\end{enumerate}
\end{document}
