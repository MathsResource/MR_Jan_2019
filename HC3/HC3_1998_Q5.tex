5.(a)Writing p for price, q for volume, o for January 1995 and l for January 1997, the Paosehe
Price index for 1997 is
(i)
P
Pp1q1
p0q1
= 2:45£167:3+6:43£777:3+2:66£165:2+15:15£3193:7
2:98£167:3+4:40£777:3+3:85£165:2+11:41£3193:7
= 54231:911
40994:811 = 1:3229
.
The index is thus 132.29.
(ii) 15:15
11:41 =1.3278, i.e. 132.78%.
The weight(q) for D is so large that the price change in D dominates the index.
(b)(i) (ii)
Share values and predictions by exponential smoothing
(ii)Exponential smoothing relates forecasts F at successive times t and actual figures as:
Ft = ®xt¡1 + (1 ¡ ®)Ft¡1
.
Calculations for ® = 0:4 are on the next page.
14
(iii)If ® = 0:9, predictions will track actual prices(x) much more closely.
or ® = 0:4, Ft = 0:4 £ xt¡1 + 0:6Ft¡1, for t=2,¢ ¢ ¢,11.
Day 1 2 3 4 5 6 7 8 9 10 11
x 3:85 3:56 3:65 3:54 3:71 3:43 3:28 3:43 3:35 3:45 ¡
F (3:85) 3:85 3:73 3:70 3:64 3:67 3:57 3:45 3:44 3:41 3:42
6.(i)Given that ¹ = 0:52 and ¾ = 23:43, the cumulative frequencies in a normal distribution are
given by 520 ©(Z);
for -67.5, ©(¡67:5¡0:52
23:43 ) = ©(¡2:903) = 0:00185; CF = 0:962;
for -52.5, ©(¡52:5¡0:52
23:43 ) = ©(¡2:263) = 0:01182; CF = 6:146.
Because of the very small expected frequency in the first group we shall combine it with the
second, to give Obs.=7, Exp.=6.146. Continuing down the table, the cumulative frequency to -7.5
is 6.146+21.063+57.512+105.632=190.353.
For +7.5, ©(7:5¡0:52
23:43 ) = ©(0:298) = 0:61715; CF = 320:918.
Now 320.918-190.353=130.565, which is the frequency in (-7.5,+7.5).
For +37.5, ©(37:5¡0:52
23:43 ) = ©(1:578) = 0:94272. Hence the frequency in (22.5,37.5)=490.214-
320.918-108.573=60.723 .
Now check that 490.214+(22.873+5.789+1.106)=519.982=: n within acceptable rounding error.
For testing normality the last two groups are combined: O=3, E=6.895.
There are 9 groups, two parameters were estimated, so Â2 has 6 d f.
Â2
(6) = (7¡6:146)2
6:146 + (26¡21:063)2
21:063 + (54¡57:512)2
57:512 + (90¡105:632)2
105:632 + (147¡130:565)2
130:565 + (102¡108:573)2
108:573
+(66¡60:723)2
60:723 + (25¡22:873)2
22:873 + (3¡6:895)2
6:895 = 9:20 n:s:
This provides no evidence against the fit to a normal distribution.
(ii)Unless there are enough observations to combine into several groups, the Â2 is a very poor
approximation, and also the pattern in the data is difficult to detect. Power against an alternative
of non-normality is very low.
(iii)There will be 30 residuals, which should be i.i.d. N(0; ¾2). If it is only normality that is
tested, and not only of the other assumptions made in analysis of a randomized block, a normal
probability plot is suitable. The residuals are ranked in order, from largest negative to largest positive.
Normal probability paper allows these to be plotted against the order-statistics for a normal
distribution with a sample of 30 items; the ith observed value is plotted against ©¡1(i=31). This
should give roughly a straight line. Further information comes from identifying which blocks and
treatments give the largest residuals, positive or negative. If, for example, one treatment seems to
have mostly large residuals it may be indicating that variances differ from one treatment to another.
