\documentclass[a4paper,12pt]{article}
%%%%%%%%%%%%%%%%%%%%%%%%%%%%%%%%%%%%%%%%%%%%%%%%%%%%%%%%%%%%%%%%%%%%%%%%%%%%%%%%%%%%%%%%%%%%%%%%%%%%%%%%%%%%%%%%%%%%%%%%%%%%%%%%%%%%%%%%%%%%%%%%%%%%%%%%%%%%%%%%%%%%%%%%%%%%%%%%%%%%%%%%%%%%%%%%%%%%%%%%%%%%%%%%%%%%%%%%%%%%%%%%%%%%%%%%%%%%%%%%%%%%%%%%%%%%
\usepackage{eurosym}
\usepackage{vmargin}
\usepackage{amsmath}
\usepackage{graphics}
\usepackage{epsfig}
\usepackage{enumerate}
\usepackage{multicol}
\usepackage{subfigure}
\usepackage{fancyhdr}
\usepackage{listings}
\usepackage{framed}
\usepackage{graphicx}
\usepackage{amsmath}
\usepackage{chngpage}
%\usepackage{bigints}

\usepackage{vmargin}
% left top textwidth textheight headheight
% headsep footheight footskip
\setmargins{2.0cm}{2.5cm}{16 cm}{22cm}{0.5cm}{0cm}{1cm}{1cm}
\renewcommand{\baselinestretch}{1.3}

\setcounter{MaxMatrixCols}{10}
\begin{document}5.(a)Writing p for price, q for volume, o for January 1995 and l for January 1997, the Paosehe
Price index for 1997 is
(i)
P
Pp1q1
p0q1
= 2:45£167:3+6:43£777:3+2:66£165:2+15:15£3193:7
2:98£167:3+4:40£777:3+3:85£165:2+11:41£3193:7
= 54231:911
40994:811 = 1:3229
.
The index is thus 132.29.
(ii) 15:15
11:41 =1.3278, i.e. 132.78%.
The weight(q) for D is so large that the price change in D dominates the index.
(b)(i) (ii)
Share values and predictions by exponential smoothing
(ii)Exponential smoothing relates forecasts F at successive times t and actual figures as:
Ft = ®xt¡1 + (1 ¡ ®)Ft¡1
.
Calculations for ® = 0:4 are on the next page.
14
(iii)If ® = 0:9, predictions will track actual prices(x) much more closely.
or ® = 0:4, Ft = 0:4 £ xt¡1 + 0:6Ft¡1, for t=2,¢ ¢ ¢,11.
Day 1 2 3 4 5 6 7 8 9 10 11
x 3:85 3:56 3:65 3:54 3:71 3:43 3:28 3:43 3:35 3:45 ¡
F (3:85) 3:85 3:73 3:70 3:64 3:67 3:57 3:45 3:44 3:41 3:42

\end{enumerate}
\end{document}
