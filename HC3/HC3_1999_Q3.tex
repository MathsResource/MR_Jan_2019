\documentclass[a4paper,12pt]{article}
%%%%%%%%%%%%%%%%%%%%%%%%%%%%%%%%%%%%%%%%%%%%%%%%%%%%%%%%%%%%%%%%%%%%%%%%%%%%%%%%%%%%%%%%%%%%%%%%%%%%%%%%%%%%%%%%%%%%%%%%%%%%%%%%%%%%%%%%%%%%%%%%%%%%%%%%%%%%%%%%%%%%%%%%%%%%%%%%%%%%%%%%%%%%%%%%%%%%%%%%%%%%%%%%%%%%%%%%%%%%%%%%%%%%%%%%%%%%%%%%%%%%%%%%%%%%
\usepackage{eurosym}
\usepackage{vmargin}
\usepackage{amsmath}
\usepackage{graphics}
\usepackage{epsfig}
\usepackage{enumerate}
\usepackage{multicol}
\usepackage{subfigure}
\usepackage{fancyhdr}
\usepackage{listings}
\usepackage{framed}
\usepackage{graphicx}
\usepackage{amsmath}
\usepackage{chngpage}
%\usepackage{bigints}

\usepackage{vmargin}
% left top textwidth textheight headheight
% headsep footheight footskip
\setmargins{2.0cm}{2.5cm}{16 cm}{22cm}{0.5cm}{0cm}{1cm}{1cm}
\renewcommand{\baselinestretch}{1.3}

\setcounter{MaxMatrixCols}{10}

\begin{document}
\section{Introduction}
\begin{enumerate}
    \item (i) (i)(iii)(v) See next sheet.
    \item n = 15; ¯y = 1000=15 = 66:67; ¯x = 1231=15 = 82:07;
P
y2 = 88268;
P
xy =
77068;
P
x2 = 103411 Therefore ˆy = 378:81 ¡ 3:714x.
ˆb
= (1706:5 ¡
12310:00
15
) ¥ (103411 ¡
1231
15
) = ¡
4987:67
2386:93
= ¡2:094
ˆa = 66:667 + 2:094 £ 82:007 = 238:51 Therefore ˆy = 238:51 ¡ 2:094x. Line A:
    \item ˆy = 378:81 ¡ 3:714x Line B:
    \item Country C is out of step with the others, and is very influential in determining the fitted
line. In fact neither line fits the lower literacy rate values at all well, although line B does
rather better than A for the upper values.The variability in y as x decreases suggests
that a linear regression assuming constant variance is not lively to do well; a weighted
regression might be considered if suitable data is variable.
\end{enumerate}

\end{document}
